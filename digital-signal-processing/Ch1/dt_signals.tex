\documentclass{report}
% PACKAGES
\usepackage{adjustbox}
\usepackage{amsmath}
\usepackage{amssymb}
\usepackage{bodegraph}
\usepackage{bbm}
\usepackage{circledsteps}
\usepackage{circuitikz}
\usepackage{enumerate}
\usepackage{mathtools}
\usepackage{nicematrix}
\usepackage{pgfplots}
\usepackage{polynom}
\usepackage{qtree}
\usepackage{rotating}
\usepackage[usestackEOL]{stackengine}
\usepackage[subpreambles=true]{standalone}
\usepackage{steinmetz}
\usepackage{subcaption}
\usepackage{tabularray}
\usepackage{tcolorbox}
\usepackage{tikz}
\usepackage{xcolor}

\usepackage[colorlinks=true,linkcolor=blue,urlcolor=black,bookmarksopen=true]{hyperref}
\usepackage{bookmark}

\usetikzlibrary{shapes.arrows}
\usetikzlibrary{shapes.misc}
\usetikzlibrary{backgrounds}
\tikzset{cross/.style={cross out, draw=black, minimum size=2*(#1-\pgflinewidth), inner sep=0pt, outer sep=0pt},
%default radius will be 1pt. 
cross/.default={1pt}}

\renewcommand{\Re}{\operatorname{Re}}
\renewcommand{\Im}{\operatorname{Im}}

\usepackage{pifont}
\newcommand{\cmark}{\text{\ding{51}}}
\newcommand{\xmark}{\text{\ding{55}}}

\newcommand{\circconv}[1]{\text{ \small\Circled{#1} }}

\newcommand{\tikzmark}[3][]{\tikz[remember picture,baseline] \node [anchor=base,#1](#2) {$#3$};}

\pgfplotsset{compat=1.18}
\pgfplotsset{
    dirac/.style={
        mark=triangle*,
        mark options={scale=1.5},
        ycomb,
        scatter,
        visualization depends on={y/abs(y)-1 \as \sign},
        scatter/@pre marker code/.code={\scope[rotate=90*\sign,yshift=-2pt]}
    }
}

\usepackage[letterpaper, portrait, margin=1.25in]{geometry}
\usepackage[font=bf]{caption}
\hbadness = 10000
\hfuzz=2pt

\newtheorem{theorem}{Theorem}[chapter]

\tikzset{every tree node/.style={anchor=north,align=center}}
\usetikzlibrary{decorations.markings}
\usetikzlibrary{arrows}
\definecolor{darkgreen}{rgb}{0.133,0.545,0.133}
\tikzstyle{n}= [circle, fill=blue, minimum size=4pt,inner sep=0pt, outer sep=0pt]

\patchcmd{\thebibliography}{\chapter*}{\section*}{}{}
\setcounter{secnumdepth}{5}

%%%%%%%%%%% EXAMPLE ENVIRONMENT %%%%%%%%%%
\usepackage{calc}
\usepackage{tabto}
\usepackage[framemethod=tikz]{mdframed} % for colored backgrounds
\newcommand{\halmos}{} % makes a box at the end

\newlength{\framedinnerleftmargin}
\newlength{\framedinnertopmargin}
\newlength{\framedreversedinnerleftmargin}
\setlength{\framedinnerleftmargin}{\widthof{Theoreme 10.10.10}+2em}
\setlength{\framedreversedinnerleftmargin}{\widthof{Theoreme 10.10.10}+1em}
\setlength{\framedinnertopmargin}{1em}
 
% first argument: label in upper left corner,
% second argument: background color
\newenvironment{boxedtext}[2]{\begin{mdframed}[%
hidealllines=true,%
backgroundcolor=#2,%
innertopmargin=\framedinnertopmargin,%
innerleftmargin=\framedinnerleftmargin,%
innerrightmargin=1em%
]%
\tabto{-\framedreversedinnerleftmargin}\textbf{#1}\tabto*{0em}%
}% begin code
{\hskip 0pt\\\hspace*{\fill}\halmos{}\end{mdframed}\vspace{1em}} % end code
 
\newenvironment{summary}[0]{\begin{center}\begin{minipage}[c]{\summarywidth}\begin{spacing}{0.9}\footnotesize} % begin code
{\end{spacing}\end{minipage}\end{center}} % end code

\newcounter{example}
 
% optional! if you want it to start at zero
% with every new chapter/section/etc.
\numberwithin{example}{section}
 
\newenvironment{example}[0]
{\refstepcounter{example}\vspace{1em plus 1em}\begin{boxedtext}{Example \theexample.}{blue!7}}%\setlength{\parskip}{0em}}
{\end{boxedtext}\vspace{-1em plus 1em}}
 
\newenvironment{example*}[0]
{\vspace{1em plus 1em}\begin{boxedtext}{Example.}{blue!7}}
{\end{boxedtext}\vspace{-1em plus 1em}}

\newmdenv[
  topline=false,
  bottomline=false,
  rightline=false,
  skipabove=\topsep,
  skipbelow=\topsep,
  linecolor=purple,
  frametitle={\noindent\textcolor{purple}{\textbf{SOLUTION }}},
  endinnercode={$\hfill\textcolor{purple}{\blacksquare}$}
]{solution}

%%%%%%%%%%%%%% COLOR BOXED %%%%%%%%%%%%%%%
% Syntax: \colorboxed[<color model>]{<color specification>}{<math formula>}
\newcommand*{\colorboxed}{}
\def\colorboxed#1#{%
  \colorboxedAux{#1}%
}
\newcommand*{\colorboxedAux}[3]{%
  % #1: optional argument for color model
  % #2: color specification
  % #3: formula
  \begingroup
    \colorlet{cb@saved}{.}%
    \color#1{#2}%
    \boxed{%
      \color{cb@saved}%
      #3%
    }%
  \endgroup
}
%%%%%%%%%%%%%%%%%%%%%%%%%%%%%%%%%%%%%%%%%

\begin{document}
 
\chapter{Discrete-Time Signals}
\emph{Digital signal processing} (DSP) usually refers to the analysis and modification of digital signals (which are generated 
by digitizing real-world analog signals). While digital signals generally refer to discrete-time, discrete-valued signals, 
technological advances in computing allows us to store data with increasingly greater precision. The most common floating-point formats 
are single-precision (32-bit) and double-precision (64-bit).
\\ \\ 
Because of the large precision of current computer number formats, the term digital signal processing effectively can be  
applied to \emph{discrete-time signals} (which can have infinite precision). In fact, the modern usage of the term \emph{digital signal processing} tends to be  
interchangeable with the outdated term \emph{discrete-time signal processing}.
\\ \\
In the context of DSP, the distinction between discrete-time signals and digital signals only becomes prevalent when the rounding errors become non-neglibile. 

\begin{tcolorbox}[width=\textwidth,colback={white}, sharp corners]
    For the remainder of this text, we will only be looking into signals with the following characteristics:
    \begin{itemize}
        \item single-channel, one-dimensional
        \item discrete-time (DT)
        \item deterministic
    \end{itemize}
\end{tcolorbox}

\section{Discrete-Time Signal Representation}
Since a sampler samples a continuous-time (CT) signal every $T_s$ seconds, the relationship between discrete-time (DT) signals (also called discrete signals) and CT signals is 
\begin{align}
    x[n] = x(t=nT_s).
\end{align}
Instead of dealing with the points of time $t$ of a CT signal, the indices $n$ are considered for a DT signal. $x[n]$ then is essentially a sequence, which is a special form of a function 
with an integer domain, where $x[N]$ represents the $N^{th}$ term of the sequence. In discrete time, $n$ can be referred to as ``time,'' with units of number of samples.
\\ \\
Other than piecewise notation, DT signals can be written in bracket notation, where the value corresponding to $x[0]$ is underlined, and any values with indices outside of the brackets are implicitly zero. 
\begin{align}
    x[n] = \left\{x[-N], ..., x[-1],\underline{x[0]}, x[1], ..., x[M]\right\}
\end{align}

\begin{example}
    Given the plot of $x[n]$ below, write an expression for $x[n]$. \\ \\
    \begin{tikzpicture}
        \begin{axis}[
            axis x line=center, axis y line=center,
            ymin=0, ymax=4.5, ytick={0,...,4}, ylabel={$x[n]$},
            xmin=-2.5, xmax=2.5, xtick={-2,...,2}, xlabel={$n$},
            width=7cm, height=4cm]
            \addplot+ [
                ycomb,
            ] coordinates {(-1,3) (0,2) (2,4)};
        \end{axis}
    \end{tikzpicture}
\end{example}
\begin{solution}
    The expressions for $x[n]$ in both bracket notation and piecewise notation are provided:
    \begin{align*}
        x[n] = \left\{3,\underline{2},0,4\right\} = 
        \begin{cases}
            3, & n=-1 \\
            2, & n=0 \\
            4, & n=2 \\
            0, & \text{otherwise}
        \end{cases}
    \end{align*}
\end{solution}

\section{Signal Transformations}
Similar to the CT case, a DT signal can be modified by apply an \emph{affine transformation on the independent variable} such that $x[n] \mapsto x[an+b]$.

\subsection{Time Shifting}
A DT signal that is time-shifted by $N$ units can be expressed as 
\begin{align}
    y[n] = x[n-N].
\end{align}
If $N>0$, then $y[n]$ is said to be delayed by $N$ units relative to $x[n]$. If $N<0$, then $y[n]$ is advanced by $N$ units.

\subsection{Time Scaling}
A CT signal that is time-scaled by some integer factor $a$ can be expressed as 
\begin{align}
    y[n] = x[an].
\end{align}
If $|a|>1$, then $y[n]$ is a temporally compressed (or \emph{downsampled}) version of $x[n]$, which keeps every $a^{th}$ sample of $x[n]$. 
If $|a|<1$, then $y(t)$ is a temporally expanded (or \emph{upsampled}) version of $x[n]$, which stuffs $a-1$ zeros between each sample of $x[n]$.

\subsection{Time Reversal}
A time reversal is a reflection of some signal over the vertical axis and can be expressed as 
\begin{align}
    y[n]=x[-n].
\end{align}

\subsection{Combined Transformation}
Let 
\begin{align}
    y[n] = x[an-b] = x\left[a\left(n-\frac{b}{a}\right)\right].
\end{align}
\begin{tcolorbox}[width=\textwidth,colback={white}, sharp corners]
Since $b/a$ is not guaranteed to be an integer, there is only one surefire approach to transforming the signal $x[n]$ to $y[n]$.
\begin{enumerate}
    \itemsep0em
    \item Shift by $b$ units.
    \item Time scale by $|a|$. Then reflect if $a<0$.
\end{enumerate}
\end{tcolorbox}

\subsection{Amplitude Transformation}
Some applications may call for vertically scaling the amplitude of a DT signal or vertically shifting the signal. The amplitude transformation 
\begin{align}
    y[n] = Ax[n]+B
\end{align}
scales the signal $x[n]$ by factor $A$ and then vertically shifts the signal by $B$ units. 

\section{Discrete-Time Sequences}
\subsection{Discrete-Time Exponential and Sinusoidal Signals}
\emph{Complex exponential sequences} take on the form  
\begin{align}
    x[n] = Ae^{j\Omega n}=A\operatorname{exp}(j\Omega n),
\end{align}
with some complex scalar $A = |A|e^{j\theta}$ and \emph{discrete-time angular frequency} $\Omega$. In fact, by sampling the CT complex exponential function every $T_s$ seconds, 
the DT complex exponential sequence can also be written as 
\begin{align}
    x[n] = x(t=nT_s) = A\operatorname{exp}(j\omega_0 (nT_s)) = A\operatorname{exp}(j\Omega n),
\end{align}
where $\Omega = \omega_0 T_s$ [rad/samp]. Unlike the CT form, the DT complex exponential sequence may or may not be periodic; in order for the DT sequence to be periodic, 
$2\pi/\Omega$ must be a rational number. In fact, if the DT signal is periodic, then 
\begin{align}
    \frac{2\pi}{\Omega} = \frac{N_0}{k},
\end{align}
where $N_0$ is the \emph{discrete-time fundamental period}, and $k$ is the smallest integer such that $N_0$ is an integer. If given the CT complex exponential signal with period $T_0$ and the sampling period $T_s$, 
the test for periodicity can be done using the following relation:
\begin{align}
    \frac{2\pi}{\Omega} = \frac{T_0}{T_s}.
\end{align}
If the DT complex exponential sequence is periodic, then $\Omega \triangleq \Omega_0 \in [0,\pi]$ is the \emph{discrete-time fundamental angular frequency}. 
\\ \\
The real and imaginary parts of the complex exponential sequence can be isolated to obtain \emph{discrete-time sinusoidal sequences}:
\begin{align}
    \Re(Ae^{j\Omega n}) = A \cos(\Omega n) = |A| \cos(\Omega n + \theta) \\
    \Im(Ae^{j\Omega n}) = A \sin(\Omega n) = |A| \sin(\Omega n + \theta)
\end{align}
Similar to the DT complex exponential sequence, the DT sinusoidal sequences are not necessarily periodic and can only be periodic if and only if
\begin{align}
    \frac{2\pi}{\Omega} = \frac{T_0}{T_s} = \frac{N_0}{k}
\end{align}
yields a rational number. The reciprocal of this ratio is called the \emph{fundamental normalized frequency}.

\subsection{Discrete-Time Counterparts of Singularity Signals}
Since the concept of continuity does not apply in discrete time, sequences are limited to finite amplitudes, and singularity points do not exist. 
For this reason, we will collectively refer to the DT counterparts of CT singularity signals as \emph{discretized singularity signals}, which 
emphasizes the fact that the CT singularity signals are being sampled and quantized to a finite amplitude.
\\ \\
The \emph{unit impulse sequence}, also called the \emph{Kronecker delta sequence} or the \emph{unit sample sequence}, is defined as 
\begin{align}
    \delta[n-N] = \delta_{n,N} =
    \begin{cases} 
        1, & n=N \\
        0, & \text{otherwise}
    \end{cases}, \\
    \text{with } \sum_{n=-\infty}^{+\infty} \delta[n-N] = 1.
\end{align}
Interestingly, when time-scaled by a factor $a$,
\begin{align}
    \delta[an] = \delta[n].
\end{align}
The impulse sequence also has a \emph{sampling property}, also called \emph{sifting property}:
\begin{align}
    \sum_{n=-\infty}^{+\infty} x[n]\delta[n-N]&= x[N], \\
    x[n]\delta[n-N] &= x[N]\delta[n-N].
\end{align} \\
The cumulative summation of the unit impulse sequence is the \emph{unit step sequence}, also called the \emph{Heaviside step sequence}, and is defined as 
\begin{align}
    u[n-N] = 
    \begin{cases} 
        0, & n<N \\
        1, & n\geq N
    \end{cases}.
\end{align}
Similar to the CT case, the unit step sequence can be treated as an ``on switch'', where at time $n=N$, the signal being multiplied gets turned on. 
\\ \\
The cumulative summation of the unit step sequence is the \emph{unit ramp sequence}, which is defined as
\begin{align}
    r[n-N] = \operatorname{ramp}[n-N] &= (n-N)u[n-N] \\
    & =
    \begin{cases} 
        0, & n<N \\
        n-N, & n\geq N
    \end{cases}.
\end{align}
Notice that the ramp sequence can be written as a product of the linear sequence and the unit step sequence. Using the ``on switch'' concept, the linear sequence is suppressed 
for all times before $N$ and is activated at $n=N$.
\\ \\
The \emph{rectangular sequence} is defined as 
\begin{align}
    u[n-N_1] - u[n-N_2]
    &=
    \begin{cases} 
        1, & N_1 \leq n < N_2 \\
        0, & \text{otherwise}
    \end{cases}.
\end{align}
Unlike the CT case, here the DT rectangular sequence includes $n=N_1$ but excludes $n=N_2$.
\\ \\
The \emph{signum sequence} is defined as 
\begin{align}
    \operatorname{sgn}[n] &= u[n]-u[-n] \\ 
    &= \left\{...,-1,-1,\underline{0},1,1,...\right\} \\
    &=
    \begin{cases} 
        -1, & n<0 \\
        0, & n=0 \\
        1, & n>0
    \end{cases}.
\end{align}
\\ \\
The \emph{geometric sequence} is defined as 
\begin{align}
    a^n u[n]
    &=
    \begin{cases} 
        a^n, & n\geq 0 \\
        0, & \text{otherwise}
    \end{cases}.
\end{align}
Here, the value $a$ can be real or complex. Note that when $a=e^{j\Omega}$, the geometric sequence essentially becomes the 
one-sided complex exponential sequence.
\\ \\ 
The stem plots of all DT signals described in this section can be found in Table \ref{sequences}. As with the CT case, these signals 
can also be linearly combined.

\begin{center}
    \begin{table}
    \small
    \centering
    \caption{Common DT sequences.}
    \label{sequences}
    \resizebox{\textwidth}{!}{%
    \begin{tabular}{ c|c|c }
    Sequence & Expression & Stem plot \\[0.1cm]
    \hline 
    Unit impulse & $\delta[n-N] = \mathbbm{1}_{\{n=N\}}[n]$ & 
    \adjustbox{valign=m}{\resizebox{0.25\textwidth}{!}{
        \begin{tikzpicture}
            \begin{axis}[
                axis x line=center, axis y line=center,
                ymin=0, ymax=1.75, ytick={0,1}, ylabel={$\delta[n]$},
                xmin=-5, xmax=5, xtick={-4,...,4}, xlabel={$n$},
                width=6cm, height=3cm]
                \addplot+ [
                    ycomb,
                ] coordinates {(-4,0) (-3,0) (-2,0) (-1,0) (0,1) (1,0) (2,0) (3,0) (4,0)};
            \end{axis}
        \end{tikzpicture}}} \\[1cm]
    Unit step & $u[n-N] = \mathbbm{1}_{\{n\geq N\}}[n]$ & 
    \adjustbox{valign=m}{\resizebox{0.25\textwidth}{!}{
        \begin{tikzpicture}
            \begin{axis}[
                axis x line=center, axis y line=center,
                ymin=0, ymax=1.75, ytick={0,1}, ylabel={$u[n]$},
                xmin=-5, xmax=5, xtick={-4,...,4}, xlabel={$n$},
                width=6cm, height=3cm]
                \addplot+ [
                    ycomb,
                ] coordinates {(-4,0) (-3,0) (-2,0) (-1,0) (0,1) (1,1) (2,1) (3,1) (4,1)};
            \end{axis}
        \end{tikzpicture}}} \\[1cm]
    Unit ramp & $r[n-N] = \operatorname{ramp}[n-N] = (n-N)u[n-N]$ & 
    \adjustbox{valign=m}{\resizebox{0.25\textwidth}{!}{
        \begin{tikzpicture}
            \begin{axis}[
                axis x line=center, axis y line=center,
                ymin=0, ymax=5, ytick={0,2,4}, ylabel={$\operatorname{ramp}[n]$},
                xmin=-5, xmax=5, xtick={-4,...,4}, xlabel={$n$},
                width=6cm, height=3cm]
                \addplot+ [
                    ycomb,
                ] coordinates {(-4,0) (-3,0) (-2,0) (-1,0) (0,0) (1,1) (2,2) (3,3) (4,4)};
            \end{axis}
        \end{tikzpicture}}} \\[1cm]
    Geometric & $a^n u[n] = 
    \begin{cases} 
        a^n, & n\geq 0 \\
        0, & \text{otherwise}
    \end{cases}$ & 
    \adjustbox{valign=m}{\resizebox{0.25\textwidth}{!}{
        \begin{tikzpicture}
            \begin{axis}[
                axis x line=center, axis y line=center,
                ymin=0, ymax=1.75, ytick={0,1}, ylabel={$(1/2)^n u[n]$},
                xmin=-5, xmax=5, xtick={-4,...,4}, xlabel={$n$},
                width=6cm, height=3cm]
            \addplot+ [
                ycomb,
            ] coordinates {(-4,0) (-3,0) (-2,0) (-1,0) (0,1) (1,0.5) (2,0.25) (3,0.125) (4,0.0625)};
            \end{axis}
        \end{tikzpicture}}} \\[1cm]
    Rectangular & $u[n-N_1] - u[n-N_2] = \mathbbm{1}_{[N_1,N_2)}[n]$ & 
    \adjustbox{valign=m}{\resizebox{0.25\textwidth}{!}{
        \begin{tikzpicture}
            \begin{axis}[
                axis x line=center, axis y line=center,
                ymin=0, ymax=1.75, ytick={0,1}, ylabel={$u[n]-u[n-3]$},
                xmin=-5, xmax=5, xtick={-4,...,4}, xlabel={$n$},
                width=6cm, height=3cm]
            \addplot+ [
                ycomb,
            ] coordinates {(-4,0) (-3,0) (-2,0) (-1,0) (0,1) (1,1) (2,1) (3,0) (4,0)};
            \end{axis}
        \end{tikzpicture}}} \\[1cm]
    Signum & $\operatorname{sgn}[n] = u[n]-u[-n]$ & 
    \adjustbox{valign=m}{\resizebox{0.25\textwidth}{!}{
        \begin{tikzpicture}
            \begin{axis}[
                axis x line=center, axis y line=center,
                ymin=-1.75, ymax=1.75, ytick={-1,0,1}, ylabel={$\text{sgn}[n]$},
                xmin=-5, xmax=5, xtick={0,...,4}, xlabel={$n$},
                width=6cm, height=3.75cm]
            \addplot+ [
                ycomb,
            ] coordinates {(-4,-1) (-3,-1) (-2,-1) (-1,-1) (0,0) (1,1) (2,1) (3,1) (4,1)};
            \end{axis}
        \end{tikzpicture}}} \\
    \hline
    \end{tabular}
    }
    \end{table}
\end{center}

\section{Properties of Discrete-Time Signals}
\subsection{Causality}
Unlike the CT case, DT signals can either be causal, noncausal, or anticausal, depending on the application.
\begin{tcolorbox}[width=\textwidth,colback={white}, sharp corners]
    \begin{itemize}
        \item Causal signals: $x[n]=0$ for $n<0$
        \item Noncausal signals: $x[n]\neq 0$ for $n<0$
        \item Anticausal signals: $x[n]=0$ for $n>0$
    \end{itemize}
\end{tcolorbox}
DT signals must be causal if \emph{real-time processing} is performed; however, the option to do \emph{offline processing} gives a pathway for noncausal signals 
(and anticausal signals by extension). 

\subsection{Symmetry}
A DT signal could have even symmetry, odd symmetry, or no symmetry.
\begin{tcolorbox}[width=\textwidth,colback={white}, sharp corners]
    \begin{itemize}
        \item Even signals: $x[n]=x[-n]$
        \item Odd signals: $x[n]=-x[-n]$, or alternatively $x[-n]=-x[n]$
    \end{itemize}
\end{tcolorbox}
While a DT signal might not have any symmetry, any signal can be expressed as a sum of two component signals: one with even symmetry 
and the other with odd symmetry. Sometimes it is easier to analyze each individual component rather than analyze the whole signal on its own. For 
\begin{align}
    x[n] = x_e[n] + x_o[n],
\end{align}
the even and odd components are given by
\begin{align}
    x_e[n] &= \frac{1}{2}\left(x[n]+x[-n]\right) \\
    x_o[n] &= \frac{1}{2}\left(x[n]-x[-n]\right)
\end{align}

\subsection{Periodicity}
A DT signal $x[n]$ is said to be \emph{periodic} if $x[n]=x[n+mN_0]$ for all integers $m$ and time $n$ with fundamental period $N_0$. 
Note that $\Omega_0=2\pi/N_0$ is the fundamental angular frequency.

\begin{tcolorbox}[width=\textwidth,colback={white}, sharp corners]
    Unlike the CT case, the sum of $M$ periodic DT signals is always periodic with 
    \begin{itemize}
        \item $\Omega_0=\mathbf{GCD}(\Omega_1,\Omega_2,...,\Omega_M)$, where $\Omega_k$ is the fundamental angular frequency of the $k^{th}$ signal, and 
        $N_0 = 2\pi/\Omega_0 = \mathbf{LCM}(N_1,N_2,...,N_M)$
        \item $\forall m\in\left\{\frac{\Omega_1}{\Omega_0}, \frac{\Omega_2}{\Omega_0}, ..., \frac{\Omega_M}{\Omega_0}\right\}$ are integers, 
        with each $m$ called the $m^{th}$ harmonic (or mode) of fundamental angular frequency $\Omega_0$
    \end{itemize}
\end{tcolorbox}


\subsection{Signal Power and Energy}
The total energy of a DT signal $x[n]$ is given by 
\begin{align}
    E = \sum_{n=-\infty}^{+\infty} |x[n]|^2,
\end{align}
whereas the average power of a DT signal is given by 
\begin{align}
    P_{av} = \lim_{N\rightarrow\infty}\frac{1}{2N+1}\sum_{n=-N}^{N} |x[n]|^2.
\end{align}
Interestingly, all finite-duration DT signals are energy signals.
\\ \\
Furthermore, if a DT signal $x[n]$ is periodic with fundamental period $N_0$, then
\begin{align}
    P_{av} = \frac{1}{N_0}\sum_{n=0}^{N_0-1}|x[n]|^2.
\end{align}
It can then be said that all periodic signals are power signals. 


\begin{example}
    Fully describe what type of signal $x[n] = \exp(j\frac{5\pi}{31}n) + \cos(\frac{2\pi}{7}n+\frac{1}{6})$ is.
\end{example}
\begin{solution}
    First let $x_1[n] = \exp(j\frac{5\pi}{31}n)$. Then
    \begin{align*}
        \frac{2\pi}{\Omega_1} = \frac{2\pi}{5\pi/31} = \frac{62}{5} = \frac{N_1}{k}
    \end{align*}
    is a rational number, and $x_1[n]$ thus is periodic with $N_1 = 62$.
    \\ \\
    Now let $x_2[n] = \cos(\frac{2\pi}{7}n+\frac{1}{6})$. Then
    \begin{align*}
        \frac{2\pi}{\Omega_2} = \frac{2\pi}{2\pi/7} = 7 = \frac{N_2}{k}
    \end{align*}
    is a rational number, and $x_2[n]$ thus is periodic with $N_2 = 7$. 
    \\ \\
    Therefore, $x[n] = x_1[n] + x_2[n]$ is periodic with $N_0 = \mathbf{LCM}(62,7) = 434$. Since $x[n]$ is periodic, it is also a power signal. 
    \\ \\
    Checking for symmetry, we find there is none since
    \begin{align*}
        x[-n] &= \exp(-j5\pi n/31) + \cos(2\pi n/7 - 1/6) \neq x[n] \\
        -x[-n] &= -\exp(-j5\pi n/31) - \cos(2\pi n/7 - 1/6) \neq x[n]
    \end{align*}
    Therefore, $x[n]$ is:
    \begin{itemize}
        \item a discrete-time signal,
        \item deterministic, 
        \item noncausal, 
        \item asymmetric,
        \item periodic,
        \item and a power signal.
    \end{itemize}
\end{solution}

\end{document}