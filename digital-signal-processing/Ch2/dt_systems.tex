\documentclass{report}
% PACKAGES
\usepackage{adjustbox}
\usepackage{amsmath}
\usepackage{amssymb}
\usepackage{bodegraph}
\usepackage{bbm}
\usepackage{circledsteps}
\usepackage{circuitikz}
\usepackage{enumerate}
\usepackage{mathtools}
\usepackage{nicematrix}
\usepackage{pdfpages}
\usepackage{pgfplots}
\usepackage{polynom}
\usepackage{qtree}
\usepackage{rotating}
\usepackage[usestackEOL]{stackengine}
\usepackage[subpreambles=true]{standalone}
\usepackage{steinmetz}
\usepackage{subcaption}
\usepackage{tabularray}
\usepackage{tcolorbox}
\usepackage{tikz}
\usepackage{xcolor}

\usepackage[colorlinks=true,linkcolor=blue,urlcolor=black,bookmarksopen=true]{hyperref}
\usepackage{bookmark}

\usetikzlibrary{shapes.arrows}
\usetikzlibrary{shapes.misc}
\usetikzlibrary{backgrounds}
\tikzset{cross/.style={cross out, draw=black, minimum size=2*(#1-\pgflinewidth), inner sep=0pt, outer sep=0pt},
%default radius will be 1pt. 
cross/.default={1pt}}

\renewcommand{\Re}{\operatorname{Re}}
\renewcommand{\Im}{\operatorname{Im}}

\usepackage[OT2,T1]{fontenc}
\DeclareSymbolFont{cyrletters}{OT2}{wncyr}{m}{n}
\DeclareMathSymbol{\Sha}{\mathalpha}{cyrletters}{"58}

\usepackage{pifont}
\newcommand{\cmark}{\text{\ding{51}}}
\newcommand{\xmark}{\text{\ding{55}}}

\newcommand{\circconv}[1]{\text{ \small\Circled{#1} }}

\newcommand{\tikzmark}[3][]{\tikz[remember picture,baseline] \node [anchor=base,#1](#2) {$#3$};}

\pgfplotsset{compat=1.18}
\pgfplotsset{
    dirac/.style={
        mark=triangle*,
        mark options={scale=1.5},
        ycomb,
        scatter,
        visualization depends on={y/abs(y)-1 \as \sign},
        scatter/@pre marker code/.code={\scope[rotate=90*\sign,yshift=-2pt]}
    }
}

\usepackage[letterpaper, portrait, margin=1.25in]{geometry}
\usepackage[font=bf]{caption}
\hbadness = 10000
\hfuzz=2pt

\newtheorem{theorem}{Theorem}[chapter]

\tikzset{every tree node/.style={anchor=north,align=center}}
\usetikzlibrary{decorations.markings}
\usetikzlibrary{arrows}
\definecolor{darkgreen}{rgb}{0.133,0.545,0.133}
\tikzstyle{n}= [circle, fill=blue, minimum size=4pt,inner sep=0pt, outer sep=0pt]

\patchcmd{\thebibliography}{\chapter*}{\section*}{}{}
\setcounter{secnumdepth}{5}

%%%%%%%%%%% EXAMPLE ENVIRONMENT %%%%%%%%%%
\usepackage{calc}
\usepackage{tabto}
\usepackage[framemethod=tikz]{mdframed} % for colored backgrounds
\newcommand{\halmos}{} % makes a box at the end

\newlength{\framedinnerleftmargin}
\newlength{\framedinnertopmargin}
\newlength{\framedreversedinnerleftmargin}
\setlength{\framedinnerleftmargin}{\widthof{Theoreme 10.10.10}+2em}
\setlength{\framedreversedinnerleftmargin}{\widthof{Theoreme 10.10.10}+1em}
\setlength{\framedinnertopmargin}{1em}
 
% first argument: label in upper left corner,
% second argument: background color
\newenvironment{boxedtext}[2]{\begin{mdframed}[%
hidealllines=true,%
backgroundcolor=#2,%
innertopmargin=\framedinnertopmargin,%
innerleftmargin=\framedinnerleftmargin,%
innerrightmargin=1em%
]%
\tabto{-\framedreversedinnerleftmargin}\textbf{#1}\tabto*{0em}%
}% begin code
{\hskip 0pt\\\hspace*{\fill}\halmos{}\end{mdframed}\vspace{1em}} % end code
 
\newenvironment{summary}[0]{\begin{center}\begin{minipage}[c]{\summarywidth}\begin{spacing}{0.9}\footnotesize} % begin code
{\end{spacing}\end{minipage}\end{center}} % end code

\newcounter{example}
 
% optional! if you want it to start at zero
% with every new chapter/section/etc.
\numberwithin{example}{section}
 
\newenvironment{example}[0]
{\refstepcounter{example}\vspace{1em plus 1em}\begin{boxedtext}{Example \theexample.}{blue!7}}%\setlength{\parskip}{0em}}
{\end{boxedtext}\vspace{-1em plus 1em}}
 
\newenvironment{example*}[0]
{\vspace{1em plus 1em}\begin{boxedtext}{Example.}{blue!7}}
{\end{boxedtext}\vspace{-1em plus 1em}}

\newmdenv[
  topline=false,
  bottomline=false,
  rightline=false,
  skipabove=\topsep,
  skipbelow=\topsep,
  linecolor=purple,
  frametitle={\noindent\textcolor{purple}{\textbf{SOLUTION }}},
  endinnercode={$\hfill\textcolor{purple}{\blacksquare}$}
]{solution}

%%%%%%%%%%%%%% COLOR BOXED %%%%%%%%%%%%%%%
% Syntax: \colorboxed[<color model>]{<color specification>}{<math formula>}
\newcommand*{\colorboxed}{}
\def\colorboxed#1#{%
  \colorboxedAux{#1}%
}
\newcommand*{\colorboxedAux}[3]{%
  % #1: optional argument for color model
  % #2: color specification
  % #3: formula
  \begingroup
    \colorlet{cb@saved}{.}%
    \color#1{#2}%
    \boxed{%
      \color{cb@saved}%
      #3%
    }%
  \endgroup
}
%%%%%%%%%%%%%%%%%%%%%%%%%%%%%%%%%%%%%%%%%

\begin{document}
\setcounter{chapter}{1}
\setcounter{page}{9}
\chapter{Discrete-Time Systems}
Similar to the continuous case, a discrete-time system essentially transforms an input DT signal into an output DT signal and can be 
modeled as an operator that maps an input sequence to an output sequence.

\begin{center}
    \begin{tikzpicture}
        \node [] (input) at (-3,0){$x[n]$};
        \node [] (output) at (+3,0){$y[n]$};
        \node [draw,
            fill=yellow!50, 
            minimum width=2cm, 
            minimum height=1.2cm
        ] (system) at (0,0){DT System};
        \draw [-latex, line width=1.5pt] (input) -- (system);
        \draw [-latex, line width=1.5pt] (system) -- (output);
    \end{tikzpicture}
\end{center}

An affine transformation on the independent variable is essentially a system consisting of a delay block and 
either an upsampler or downsampler in the real-time case. 
\begin{center}
    \begin{tikzpicture}
        \draw [dashed] (-1.5,-1.25) rectangle (+4,1.5);
        \node[] at (1.25,1.1) {DT System};
        \node [] (input) at (-3,0){$x[n]$};
        \node [] (output) at (+6.5,0){$y[n]=x[n/M-N]$};
        \node [draw,
            fill=green!50,
            minimum width=1.2cm,
            minimum height=1.2cm
        ] (delay) at (+0,0){$\mathbf{D}$};
        \node [draw,
            fill=blue!30,
            minimum width=1.2cm,
            minimum height=1.2cm
        ] (samp) at (+3,0){$\mathbf{\uparrow M}$};
        \draw [-latex, line width=1.5pt] (input) -- (delay);
        \draw [-latex, line width=1.5pt] (delay) -- (samp);
        \draw [-latex, line width=1.5pt] (samp) -- (output);
    \end{tikzpicture}
\end{center}
\begin{center}
    \begin{tikzpicture}
        \draw [dashed] (-1.5,-1.25) rectangle (+4,1.5);
        \node[] at (1.25,1.1) {DT System};
        \node [] (input) at (-3,0){$x[n]$};
        \node [] (output) at (+6.5,0){$y[n]=x[Mn-N]$};
        \node [draw,
            fill=green!50,
            minimum width=1.2cm,
            minimum height=1.2cm
        ] (delay) at (+0,0){$\mathbf{D}$};
        \node [draw,
            fill=blue!30,
            minimum width=1.2cm,
            minimum height=1.2cm
        ] (samp) at (+3,0){$\mathbf{\downarrow M}$};
        \draw [-latex, line width=1.5pt] (input) -- (delay);
        \draw [-latex, line width=1.5pt] (delay) -- (samp);
        \draw [-latex, line width=1.5pt] (samp) -- (output);
    \end{tikzpicture}
\end{center}

Similarly, an amplitude transformation 
$y[n]=Ax[n]+b$ can be modeled as a system which inherently has a multiplier and a bias $b$:
\begin{center}
    \begin{tikzpicture}
        \draw [dashed] (-1.5,-1.25) rectangle (+4,2);
        \node[] at (1.25,-0.85) {DT System};
        \node [] (input) at (-3,0){$x[n]$};
        \node [] (scale) at (0,+1.5){$A$};
        \node [] (bias) at (+3,+1.5){$b$};
        \node [] (output) at (+6.5,0){$y[n]=Ax[n]+b$};
        \node[draw,
            circle,
            minimum size=0.4cm,
            fill=orange!50
        ] (mult) at (+0,0){\LARGE $\times$};
        \node[draw,
            circle,
            minimum size=0.4cm,
            fill=red!30
        ] (sum) at (+3,0){\LARGE +};
        \draw [-latex, line width=1.5pt] (input) -- (mult);
        \draw [-latex, line width=1.5pt] (scale) -- (mult);
        \draw [-latex, line width=1.5pt] (mult) -- (sum);
        \draw [-latex, line width=1.5pt] (bias) -- (sum);
        \draw [-latex, line width=1.5pt] (sum) -- (output);
    \end{tikzpicture}
\end{center}

This chapter will cover one-dimensional SISO discrete-time systems and their properties as well as explore 
the discrete equivalent of LTI systems. Some references will refer to them as \emph{linear shift-invariant (LSI) systems} 
to emphasize that samples are being manipulated -- doing so builds a foundation to digital image processing 
as each pixel of an image signal represents a discrete (spatial) sample of two dimensions 
rather than a discrete (temporal) sample of one dimension.
\\ \\
However, this text will continue to use the term LTI systems to provide comparative and contextual analysis between 
continuous-time and discrete-time systems -- doing so also draws attention to the fact that $n$ here is simply an instance of time.

\section{Properties of Discrete-Time Systems}
\subsection{Linear vs Nonlinear}
A DT system is said to be \emph{linear} if it follows the \emph{superposition principle}, as depicted in the block diagram below with constants $c_1,c_2$, input signal 
addends $x_1[n], x_2[n]$, and output signal addends $y_1[n],y_2[n]$:
\begin{center}
    \begin{tikzpicture}
        \node [] (input) at (-3,0){$c_1x_1[n]+c_2x_2[n]$};
        \node [] (output) at (+3,0){$c_1y_1[n]+c_2y_2[n]$};
        \node [draw,
            fill=yellow!50, 
            minimum width=2cm, 
            minimum height=1.2cm
        ] (system) at (0,0){$\Psi[\cdot]$};
        \draw [-latex, line width=1.5pt] (input) -- (system);
        \draw [-latex, line width=1.5pt] (system) -- (output);
    \end{tikzpicture}
\end{center}
In practice, if the superposition principle is met, then a sum of $N$ input addends will generate a sum of $N$ output addends. \\ \\
The superposition principle can be broken down into two properties: additivity and scalability. 
A DT system has \emph{additive property} if for $y_1[n]=\Psi[x_1[n]]$ and $y_2[n]=\Psi[x_2[n]]$, the system-generated output from 
the summed inputs is the sum of the outputs $\Psi[x_1[n]+x_2[n]]=y_1[n]+y_2[n]$. That is:
\begin{center}
    \begin{tikzpicture}
        \node [] (input) at (-3,0){$x_1[n]+x_2[n]$};
        \node [] (output) at (+3,0){$y_1[n]+y_2[n]$};
        \node [draw,
            fill=yellow!50, 
            minimum width=2cm, 
            minimum height=1.2cm
        ] (system) at (0,0){$\Psi[\cdot]$};
        \draw [-latex, line width=1.5pt] (input) -- (system);
        \draw [-latex, line width=1.5pt] (system) -- (output);
    \end{tikzpicture}
\end{center}
A DT system has \emph{scaling property} if for some constant $c$, the system-generated output for a scaled input is 
an appropriately scaled output $\Psi[c\cdot x[n]]=c\cdot y[n]$. That is:
\begin{center}
    \begin{tikzpicture}
        \node [] (input) at (-3,0){$c\cdot x[n]$};
        \node [] (output) at (+3,0){$c\cdot y[n]$};
        \node [draw,
            fill=yellow!50, 
            minimum width=2cm, 
            minimum height=1.2cm
        ] (system) at (0,0){$\Psi[\cdot]$};
        \draw [-latex, line width=1.5pt] (input) -- (system);
        \draw [-latex, line width=1.5pt] (system) -- (output);
    \end{tikzpicture}
\end{center}
If a system has both scaling and additive properties, then the superposition principle is met, and the system is linear. If the superposition principle is not met, 
then the system is said to be \emph{nonlinear}.

\subsection{Time-Invariant vs Time-Variant}
A DT system is said to be \emph{time-invariant} if a shift in the input signal results in a corresponding shift in the output signal such that $\Psi[x[n-N]]=y[n-N]$. That is:
\begin{center}
    \begin{tikzpicture}
        \node [] (input) at (-3,0){$x[n-N]$};
        \node [] (output) at (+3,0){$y[n-N]$};
        \node [draw,
            fill=yellow!50, 
            minimum width=2cm, 
            minimum height=1.2cm
        ] (system) at (0,0){$\Psi[\cdot]$};
        \draw [-latex, line width=1.5pt] (input) -- (system);
        \draw [-latex, line width=1.5pt] (system) -- (output);
    \end{tikzpicture}
\end{center}
Some resources refer to this property for DT signals as ``shift-invariant'' to highlight the sampled nature of the signals.
\\ \\
A method to check for time-invariance is to consider the following two block diagrams, with the ``D''-block representing a shift of $N$ samples:
\begin{center}
    \begin{tikzpicture}
        \node [] (input) at (-3,0){$x_1[n]$};
        \node [] (output) at (+3,0){$y_1[n]$};
        \node [] (delayed) at (+9,0){$y_1[n-N]$};
        \node [draw,
            fill=yellow!50, 
            minimum width=2cm, 
            minimum height=1.2cm
        ] (system) at (0,0){$\Psi[\cdot]$};
        \node [draw,
            fill=green!50,
            minimum width=1.2cm,
            minimum height=1.2cm
        ] (delay) at (5.8,0){$\mathbf{D}$};
        \draw [-latex, line width=1.5pt] (input) -- (system);
        \draw [-latex, line width=1.5pt] (system) -- (output);
        \draw [-latex, line width=1.5pt] (output) -- (delay);
        \draw [-latex, line width=1.5pt] (delay) -- (delayed);
    \end{tikzpicture}
    \\[0.5cm]
    \begin{tikzpicture}
        \node [] (input) at (-3,0){$x_1[n]$};
        \node [] (delayed) at (+2,0){$x_2[n]=x_1[n-N]$};
        \node [] (output) at (+8,0){$y_2[n]$};
        \node [draw,
            fill=green!50,
            minimum width=1.2cm,
            minimum height=1.2cm
        ] (delay) at (-1,0){$\mathbf{D}$};
        \node [draw,
            fill=yellow!50, 
            minimum width=2cm, 
            minimum height=1.2cm
        ] (system) at (+5.5,0){$\Psi[\cdot]$};
        \draw [-latex, line width=1.5pt] (input) -- (delay);
        \draw [-latex, line width=1.5pt] (delay) -- (delayed);
        \draw [-latex, line width=1.5pt] (delayed) -- (system);
        \draw [-latex, line width=1.5pt] (system) -- (output);
    \end{tikzpicture}
\end{center}
From the outputs above, if $y_2[n]=y_1[n-N]$, then the DT system is time-invariant. Otherwise, if the outputs are not equal, 
then the DT system is \emph{time-variant}, in which $\Psi[x[n-N],N]=y[n-N]$ results in an additional dependence on $N$.
\\ \\
DT systems that are both linear and time-invariant are called linear time-invariant (LTI) systems.

\subsection{Dynamic vs Memoryless}
A DT system is said to be \emph{memoryless} or \emph{static} if the output $y[n]$ at time $n$ depends only on the input $x[n]$ at time $n$. The only DT LTI system 
that is also memoryless has the form $y[n]=ax[n]$ for $a$ is a constant. \\ \\
Otherwise, a DT system whose output additionally depends on past and/or future values of the input is a \emph{dynamic} system; that is, the output at $n$ 
depends on the input at $n-N$ for any $N\neq 0$. 

\subsection{Causal vs Noncausal}
\emph{Causal} systems in discrete-time are DT systems whose output at time $n$ only depend on the past and present values of the input at time $n-N$ for $N>0$ and time $n$, respectively. 
Otherwise, a DT system that anticipates a future value of the input at time $n+N$ before generating an output at time $n$ is said to be \emph{noncausal}.
\\ \\
Essentially, noncausal systems see the future, which is impractical for real-time processing but can be done offline. Because of this, while causality is important in the continuous-time case, 
the ability to handle offline processing opens opportunities for working with noncausal signals and systems. This is crucial later when considering both the unilateral and bilateral $z$-transforms.
\\ \\
One way to determine causality is by using an impulse signal $\delta[n]$ as input and observing its output $h[n]$, also called the \emph{impulse response}. If $h[n]=0$ for all $n<0$, 
then the DT system is causal; otherwise it is noncausal. That is, if the impulse response is causal, then the system is causal.

\begin{center}
    \begin{tikzpicture}
        \node [] (input) at (-3,0){$\delta[n]$};
        \node [] (output) at (+3,0){$h[n]$};
        \node [draw,
            fill=yellow!50, 
            minimum width=2cm, 
            minimum height=1.2cm
        ] (system) at (0,0){$\Psi[\cdot]$};
        \draw [-latex, line width=1.5pt] (input) -- (system);
        \draw [-latex, line width=1.5pt] (system) -- (output);
    \end{tikzpicture}
\end{center}

\subsection{BIBO Stable vs Unstable}
A DT signal $x[n]$ is said to be \emph{bounded} if there exists some constant $C$ such that 
\begin{align}
    |x[n]| \leq C, \text{ for } \forall n.
\end{align}
A DT system with a \emph{bounded input, bounded output} (BIBO, for short) is said to be \emph{BIBO stable}; that is, every bounded input signal results in a bounded output signal. Otherwise, if a DT system 
produces an unbounded output for a bounded input, it is said to be \emph{unstable}.
\\ \\
For LTI systems, we can particularly look at the impulse response $h[n]$. A DT LTI system is BIBO stable if and only if $h[n]$ is \emph{absolutely summable} such that
\begin{align}
    \sum_{n=-\infty}^{+\infty} |h[n]| = C, \text{ for $C$ is finite.}
\end{align}

\subsection{Invertible vs Non-Invertible}
A DT system is said to be \emph{invertible} if it generates unique output signals for every unique input signal. That is, an invertible system has a one-to-one mapping between 
inputs and outputs. If a DT system has a many-to-one mapping between inputs and outputs, then it is said to be \emph{non-invertible}. 
\\ \\
Alternatively, an invertible system has an inverse system $\Psi^{-1}[\cdot]$ that maps outputs back to the inputs of the forward system $\Psi[\cdot]$ -- it may be shifted but the shape of the waveform is maintained. 
\begin{center}
    \begin{tikzpicture}
        \node [] (input) at (-3,0){$x[n]$};
        \node [] (output) at (+6,0){$x[n]$};
        \node [draw,
            fill=yellow!50, 
            minimum width=2cm, 
            minimum height=1.2cm
        ] (fwd) at (0,0){$\Psi[\cdot]$};
        \node [draw,
            fill=blue!30, 
            minimum width=2cm, 
            minimum height=1.2cm
        ] (inv) at (3,0){$\Psi^{-1}[\cdot]$};
        \draw [-latex, line width=1.5pt] (input) -- (fwd);
        \draw [-latex, line width=1.5pt] (fwd) -- (inv);
        \draw [-latex, line width=1.5pt] (inv) -- (output);
    \end{tikzpicture}
\end{center}
\begin{center}
    \begin{tikzpicture}
        \node [] (input) at (-3,0){$y[n]$};
        \node [] (output) at (+6,0){$y[n]$};
        \node [draw,
            fill=blue!30, 
            minimum width=2cm, 
            minimum height=1.2cm
        ] (inv) at (0,0){$\Psi^{-1}[\cdot]$};
        \node [draw,
            fill=yellow!50, 
            minimum width=2cm, 
            minimum height=1.2cm
        ] (fwd) at (3,0){$\Psi[\cdot]$};
        \draw [-latex, line width=1.5pt] (input) -- (inv);
        \draw [-latex, line width=1.5pt] (inv) -- (fwd);
        \draw [-latex, line width=1.5pt] (fwd) -- (output);
    \end{tikzpicture}
\end{center}
While a DT system can be invertible, it does not mean it is implementable. For this very reason, when inverting LTI systems, generally we are also interested in the properties of the 
inverse system; that is, if the original system is BIBO stable, we would want the inverse system to be BIBO stable as well. Depending on whether DT systems are implemented in real-time 
or offline, causality may or may not be of importance for the inverse system. 

\begin{example}
    Fully describe the system characterized by the difference equation 
    \begin{align*}
        y[n] + 2y[n-1] = 3x[n] + nx[n-1].
    \end{align*}
\end{example}
\begin{solution}
    First, assume $cx[\cdot]\rightarrow cy[\cdot]$ for scalability such that
    \begin{align*}
        (cy[n]) + 2(cy[n-1]) = 3(cx[n]) + n(cx[n-1]).
    \end{align*}
    The constant $c$ can be factored such that 
    \begin{align*}
        cy[n] + 2cy[n-1] &= 3cx[n] + ncx[n-1] \\
        \Longrightarrow c\cdot(y[n] + 2y[n-1] &= 3x[n] + nx[n-1]).
    \end{align*}
    Therefore, the system is scalable. Checking for additivity, consider the following sum:
    \begin{center}
        \begin{tabular}{ cc }
            & $y_1[n] + 2y_1[n-1] = 3x_1[n] + nx_1[n-1]$ \\[0.1cm]
            + & $y_2[n] + 2y_2[n-1] = 3x_2[n] + nx_2[n-1]$ \\[0.1cm]
        \end{tabular}
        \begin{tabular}{ c }
            \hline 
            $(y_1[n]+y_2[n]) + 2(y_1[n-1]+y_2[n-1]) = 3(x_1[n]+x_2[n]) + n(x_1[n-1]+x_2[n-1])$
        \end{tabular}
    \end{center} 
    The system is also additive. Thus the system is linear.
    \\ \\
    Now we check for shift invariance. First, we feed an input and delay the output:
    \begin{align*}
        y_1[n] + 2y_1[n-1] &= 3x_1[n] + nx_1[n-1] \\
        y_1[n-N] + 2y_1[(n-N)-1] &= 3x_1[n-N] + (n-N)x_1[(n-N)-1]
    \end{align*}
    If instead we delay the input first and then feed it into the system, we get
    \begin{align*}
        x_2[n] &= x_1[n-N] \\
        y_2[n] + 2y_2[n-1] &= 3x_2[n] + nx_2[n-1] = 3x_1[n-N] + nx_1[n-N-1]
    \end{align*}
    Note that $y_1[n-N] + 2y_1[(n-N)-1] \neq y_2[n] + 2y_2[n-1]$ due to the linear term $n$. Therefore, the system is time-variant. 
    \\ \\
    Checking for causality, first we input an impulse sequence to attain the impulse response:
    \begin{align*}
        h[n] + 2h[n-1] &= 3\delta[n] + n\delta[n-1] \\
        \Longrightarrow h[n] + 2h[n-1] &= 3\delta[n] + 1\delta[n-1]
    \end{align*}
    This can be simplified to 
    \begin{align*}
        h[n] + 2h[n-1] &= C_n
    \end{align*}
    where
    \begin{align*}
        C_n &= \begin{cases}
            3, & n = 0 \\
            1, & n = 1 \\
            0, & otherwise
        \end{cases}
    \end{align*}
    We can rearrange the recursive equation to be  
    \begin{align*}
        h[n] = -2h[n-1] + C_n, 
    \end{align*}
    which shows that the impulse response is dependent on its past values (in addition to the present constant value $C_n$) and not future values. Therefore the system is causal. 
    \\ \\
    As seen in the difference equation, $y[n]$ depends on past values of $x[n]$ as well as $y[n]$ (in addition to present values of $x[n]$). Therefore, the system is dynamic.
    \\ \\
    Due to the term $nx[n-1]$, the output is not guaranteed to be bounded, regardless of the input signal. As $n\rightarrow\infty$, the term also approaches infinity and thus the output approaches infinity. 
    Therefore, the system is not BIBO stable.
    \\ \\
    Lastly, to check for invertibility, we solve for $x[n]$. First define $y[m] + 2y[m-1] = 3x[m] + mx[m-1]$. Then 
    \begin{align*}
        n=m-1 \Longrightarrow m=n+1.
    \end{align*}
    Substituting this relationship in, we get 
    \begin{align*}
        y[n+1] + 2y[(n+1)-1] &= 3x[n+1] + (n+1)x[(n+1)-1] \\
        \Longrightarrow x[n] + \frac{3}{n+1}\,x[n+1] &= \frac{1}{n+1}\,y[n+1] + \frac{2}{n+1}\,y[n].
    \end{align*}
    The system is thus invertible.
    \\ \\
    The DT system can be described as 
    \begin{itemize}
        \item linear,
        \item time-variant,
        \item causal,
        \item unstable,
        \item dynamic,
        \item and invertible.
    \end{itemize}
\end{solution}

\section{DT LTI Systems}
By now, we know that an \emph{DT LTI system} is a DT system that is both linear and time-invariant. LTI systems are particularly useful for their predictable nature. 
As long as we know the system response to a few select input signals, we can accurately predict the output for all input signals.
\\ \\
The option to perform either real-time processing or offline processing is what makes DT LTI systems different from CT LTI systems. 
When modeling CT real systems using LTI systems, one would need to choose an LTI model that is also dynamic, causal, and stable. However, 
when working with DT LTI systems, one would choose an LTI system that is at least dynamic and stable, with the additional causality condition 
given for real-time processing applications and not necessarily for offline processing applications.

\subsection{LTI System Response to Discretized Singularity Signals}
As prefaced before, the \emph{impulse response} $h[n]$ of a DT system is the system response to an inputted impulse signal $\delta[n]$, given zero initial conditions. 
The impulse response of a DT LTI system is depicted in the following block diagram:
\begin{center}
    \begin{tikzpicture}
        \node [] (input) at (-3,0){$\delta[n]$};
        \node [] (output) at (+3,0){$h[n]$};
        \node [draw,
            fill=yellow!50, 
            minimum width=2cm, 
            minimum height=1.2cm
        ] (system) at (0,0){DT LTI};
        \draw [-latex, line width=1.5pt] (input) -- (system);
        \draw [-latex, line width=1.5pt] (system) -- (output);
    \end{tikzpicture}
\end{center}
For LTI systems, the impulse response plays an important role. An LTI system can be characterized by its impulse response such that any output signal can be predicted by 
performing an operation called \emph{linear convolution} between the input signal and the impulse response. The block diagram of a DT LTI system can be depicted as:
\begin{center}
    \begin{tikzpicture}
        \node [] (input) at (-3,0){$x[n]$};
        \node [] (output) at (+3,0){$y[n]$};
        \node [draw,
            fill=yellow!50, 
            minimum width=2cm, 
            minimum height=1.2cm
        ] (system) at (0,0){$h[n]$};
        \draw [-latex, line width=1.5pt] (input) -- (system);
        \draw [-latex, line width=1.5pt] (system) -- (output);
    \end{tikzpicture}
\end{center}
\begin{tcolorbox}[width=\textwidth,colback={white}, sharp corners]
    Linear convolution is an operation defined by the \emph{convolution sum} in the discrete-time case:
    \begin{align}
        x[n] * h[n] &= \sum_{k=-\infty}^{+\infty} x[k]h[n-k] \\
        &= \sum_{k=-\infty}^{+\infty} x[n-k]h[k].
    \end{align}
\end{tcolorbox}
Here, we specify the more direct term ``linear convolution''. This is to avoid any confusion when circular convolution 
is covered in a later chapter.
\\ \\
The \emph{step response} $y_{step}[n]$ of a DT system is the system response to an inputted unit step signal $u[n]$, given zero initial conditions. 
The step response of a DT LTI system is depicted in the following block diagram:
\begin{center}
    \begin{tikzpicture}
        \node [] (input) at (-3,0){$u[n]$};
        \node [] (output) at (+3,0){$y_{step}[n]$};
        \node [draw,
            fill=yellow!50, 
            minimum width=2cm, 
            minimum height=1.2cm
        ] (system) at (0,0){DT LTI};
        \draw [-latex, line width=1.5pt] (input) -- (system);
        \draw [-latex, line width=1.5pt] (system) -- (output);
    \end{tikzpicture}
\end{center}
While not as prevalent in DSP applications, the step response appears in primarily digital control systems, where 
we are interested in how well a system ``tracks'' a step input.
\\ \\
Interestingly, just as $u[n]$ is the cumulative sum of $\delta[n]$, the step response $y_{step}[n]$ is the cumulative sum of the impulse response $h[n]$. That is,
\begin{align}
    h[n] &= y_{step}[n] - y_{step}[n-1], \\
    y_{step}[n] &= \sum_{k=-\infty}^{n} h[k].
\end{align}

\subsection{LTI System Response to DT Exponential and Sinusoidal Signals}
Using the convolution sum, we can generalize the LTI system response to a DT complex exponential signal $e^{j\Omega n}$.
\begin{align*}
    x[n] * h[n] &= \sum_{k=-\infty}^{+\infty} x[n-k]h[k] \\
    \Longrightarrow e^{j\Omega n} * h[n] &= \sum_{k=-\infty}^{+\infty} e^{j\Omega (n-k)}h[n] \\
    &= e^{j\Omega n} \underbrace{\sum_{k=-\infty}^{+\infty} h[k]e^{-j\Omega k} }_\textrm{$H(e^{j\Omega})$}.
\end{align*}
Therefore, the LTI system response to a DT complex exponential can be depicted as: 
\begin{center}
    \begin{tikzpicture}
        \node [] (input) at (-3,0){$e^{j\omega n}$};
        \node [] (output) at (+3,0){$H(e^{j\Omega})e^{j\Omega n}$};
        \node [draw,
            fill=yellow!50, 
            minimum width=2cm, 
            minimum height=1.2cm
        ] (system) at (0,0){$h[n]$};
        \draw [-latex, line width=1.5pt] (input) -- (system);
        \draw [-latex, line width=1.5pt] (system) -- (output);
    \end{tikzpicture}
\end{center}
Here, $H(e^{j\Omega})$ is called the \emph{discrete-time frequency response} of the system. Note that $H(e^{j\Omega})$ is independent of time $n$, 
is a complex function of $j\Omega$, and is defined for an everlasting complex exponential. 
\\ \\
Since sinusoids form the real and imaginary parts of a complex exponential, the DT LTI sinusoidal response can also be depicted: 
\begin{center}
    \begin{tikzpicture}
        \node [] (input) at (-3,0){$A\cos(\Omega n+\phi)$};
        \node [] (output) at (+4.5,0){$A|H(e^{j\Omega})|\cos\big(\Omega n+\phase{H(e^{j\Omega})}+\phi\big)$};
        \node [draw,
            fill=yellow!50, 
            minimum width=2cm, 
            minimum height=1.2cm
        ] (system) at (0,0){$h[n]$};
        \draw [-latex, line width=1.5pt] (input) -- (system);
        \draw [-latex, line width=1.5pt] (system) -- (output);
    \end{tikzpicture}
\end{center}
In a later chapter, the relationship between the DT impulse reponse and the DT frequency response will be explored. There exists some operator in which the impulse response can be 
mapped to the frequency response and vice versa.

\subsection{Eigensequences of DT LTI Systems}
An \emph{eigensequence} of a DT LTI system is some function $x[n]$ such that the system response is $\lambda x[n]$ for some scalar $\lambda$ (independent of $n$). That is,  
\begin{center}
    \begin{tikzpicture}
        \node [] (input) at (-3,0){$x[n]$};
        \node [] (output) at (+3,0){$\lambda x[n]$};
        \node [draw,
            fill=yellow!50, 
            minimum width=2cm, 
            minimum height=1.2cm
        ] (system) at (0,0){DT LTI};
        \draw [-latex, line width=1.5pt] (input) -- (system);
        \draw [-latex, line width=1.5pt] (system) -- (output);
    \end{tikzpicture}
\end{center}
For DT LTI systems characterized by $h[n]$, we can use the convolution sum to determine if a sequence 
is an eigensequence of the system. 
\\ \\
Previously, we saw that for a DT complex exponential input $e^{j\Omega n}$, 
we get a scaled output $H(e^{j\Omega})e^{j\Omega n}$. Therefore, the complex exponential is an eigensequence of the system. 
In fact, this particular eigensequence yields the DT frequency response function $H(e^{j\Omega})$, which is crucial for discrete-time Fourier transforms 
in a later chapter.

\begin{example}
    Determine if $z^n$ is an eigensequence of DT LTI systems, for $z$ is a complex variable.
\end{example}
\begin{solution}
    \begin{align*}
        z^n * h[n] &= \sum_{k=-\infty}^{+\infty} z^{n-k}h[k] = z^n \underbrace{\sum_{k=-\infty}^{+\infty} h[k]z^{-k} }_\textrm{$H(z)$}
    \end{align*}
    Since $H(z)$ is a constant for a given $z$ and is scaling $z^n$, the function $z^n$ is an eigensequence of DT LTI systems. 
    In fact, it yields the discrete-time transfer function $H(z)$, which is crucial for $z$-transforms in a later chapter.
    \\ \\ 
    \begin{center}
    \begin{tikzpicture}
        \node [] (input) at (-3,0){$z^n$};
        \node [] (output) at (+3,0){$H(z)z^n$};
        \node [draw,
            fill=yellow!50, 
            minimum width=2cm, 
            minimum height=1.2cm
        ] (system) at (0,0){DT LTI};
        \draw [-latex, line width=1.5pt] (input) -- (system);
        \draw [-latex, line width=1.5pt] (system) -- (output);
    \end{tikzpicture}
    \end{center}
\end{solution}

\begin{example}
    Determine if $(1/2)^n u[n]$ is an eigensequence of DT LTI systems.
\end{example}
\begin{solution}
    \begin{align*}
        (1/2)^n u[n] * h[n] &= \sum_{k=-\infty}^{+\infty} (1/2)^{n-k}u[n-k]h[k] = (1/2)^n \sum_{k=-\infty}^{n} h[n](1/2)^{-k}
    \end{align*}
    Since the sum is dependent on time $n$, it is not a constant. Therefore $(1/2)^n u[n]$ is not an eigensequence of DT LTI systems.
\end{solution}

\subsection{Linear Constant-Coefficient Difference Equations (LCCDE)}
DT LTI systems can be characterized by a class of difference equations called 
\emph{linear constant-coefficient difference equations} (LCCDE):
\begin{align}
    \sum_{k=0}^{N} a_k y[n-k] = \sum_{k=0}^{M} b_k x[n-k].
\end{align}
Difference equations can be isolated by their parts: the autoregressive (AR) and the moving average (MA) terms. 
\\ \\
The \emph{autoregressive} (AR) term is given by 
\begin{align}
    \sum_{k=0}^{N} a_k y[n-k] = x[n],
\end{align}
whereas the \emph{moving average} (MA) term is given by
\begin{align}
    y[n] = \sum_{k=0}^{M} b_k x[n-k].
\end{align}
Systems that are described by AR are called autoregressive systems, whereas systems characterized by MA are called moving average systems. Systems that 
are both AR and MA (i.e., following the general LCCDE) are called autoregressive and moving average (ARMA) systems.



\end{document}