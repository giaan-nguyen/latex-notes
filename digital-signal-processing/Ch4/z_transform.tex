\documentclass{report}
% PACKAGES
\usepackage{adjustbox}
\usepackage{amsmath}
\usepackage{amssymb}
\usepackage{bodegraph}
\usepackage{bbm}
\usepackage{circledsteps}
\usepackage{circuitikz}
\usepackage{enumerate}
\usepackage{mathtools}
\usepackage{nicematrix}
\usepackage{pgfplots}
\usepackage{polynom}
\usepackage{qtree}
\usepackage{rotating}
\usepackage[usestackEOL]{stackengine}
\usepackage[subpreambles=true]{standalone}
\usepackage{steinmetz}
\usepackage{subcaption}
\usepackage{tabularray}
\usepackage{tcolorbox}
\usepackage{tikz}
\usepackage{xcolor}

\usepackage[colorlinks=true,linkcolor=blue,urlcolor=black,bookmarksopen=true]{hyperref}
\usepackage{bookmark}

\usetikzlibrary{shapes.arrows}
\usetikzlibrary{shapes.misc}
\usetikzlibrary{backgrounds}
\tikzset{cross/.style={cross out, draw=black, minimum size=2*(#1-\pgflinewidth), inner sep=0pt, outer sep=0pt},
%default radius will be 1pt. 
cross/.default={1pt}}

\renewcommand{\Re}{\operatorname{Re}}
\renewcommand{\Im}{\operatorname{Im}}

\usepackage{pifont}
\newcommand{\cmark}{\text{\ding{51}}}
\newcommand{\xmark}{\text{\ding{55}}}

\newcommand{\circconv}[1]{\text{ \small\Circled{#1} }}

\newcommand{\tikzmark}[3][]{\tikz[remember picture,baseline] \node [anchor=base,#1](#2) {$#3$};}

\pgfplotsset{compat=1.18}
\pgfplotsset{
    dirac/.style={
        mark=triangle*,
        mark options={scale=1.5},
        ycomb,
        scatter,
        visualization depends on={y/abs(y)-1 \as \sign},
        scatter/@pre marker code/.code={\scope[rotate=90*\sign,yshift=-2pt]}
    }
}

\usepackage[letterpaper, portrait, margin=1.25in]{geometry}
\usepackage[font=bf]{caption}
\hbadness = 10000
\hfuzz=2pt

\newtheorem{theorem}{Theorem}[chapter]

\tikzset{every tree node/.style={anchor=north,align=center}}
\usetikzlibrary{decorations.markings}
\usetikzlibrary{arrows}
\definecolor{darkgreen}{rgb}{0.133,0.545,0.133}
\tikzstyle{n}= [circle, fill=blue, minimum size=4pt,inner sep=0pt, outer sep=0pt]

\patchcmd{\thebibliography}{\chapter*}{\section*}{}{}
\setcounter{secnumdepth}{5}

%%%%%%%%%%% EXAMPLE ENVIRONMENT %%%%%%%%%%
\usepackage{calc}
\usepackage{tabto}
\usepackage[framemethod=tikz]{mdframed} % for colored backgrounds
\newcommand{\halmos}{} % makes a box at the end

\newlength{\framedinnerleftmargin}
\newlength{\framedinnertopmargin}
\newlength{\framedreversedinnerleftmargin}
\setlength{\framedinnerleftmargin}{\widthof{Theoreme 10.10.10}+2em}
\setlength{\framedreversedinnerleftmargin}{\widthof{Theoreme 10.10.10}+1em}
\setlength{\framedinnertopmargin}{1em}
 
% first argument: label in upper left corner,
% second argument: background color
\newenvironment{boxedtext}[2]{\begin{mdframed}[%
hidealllines=true,%
backgroundcolor=#2,%
innertopmargin=\framedinnertopmargin,%
innerleftmargin=\framedinnerleftmargin,%
innerrightmargin=1em%
]%
\tabto{-\framedreversedinnerleftmargin}\textbf{#1}\tabto*{0em}%
}% begin code
{\hskip 0pt\\\hspace*{\fill}\halmos{}\end{mdframed}\vspace{1em}} % end code
 
\newenvironment{summary}[0]{\begin{center}\begin{minipage}[c]{\summarywidth}\begin{spacing}{0.9}\footnotesize} % begin code
{\end{spacing}\end{minipage}\end{center}} % end code

\newcounter{example}
 
% optional! if you want it to start at zero
% with every new chapter/section/etc.
\numberwithin{example}{section}
 
\newenvironment{example}[0]
{\refstepcounter{example}\vspace{1em plus 1em}\begin{boxedtext}{Example \theexample.}{blue!7}}%\setlength{\parskip}{0em}}
{\end{boxedtext}\vspace{-1em plus 1em}}
 
\newenvironment{example*}[0]
{\vspace{1em plus 1em}\begin{boxedtext}{Example.}{blue!7}}
{\end{boxedtext}\vspace{-1em plus 1em}}

\newmdenv[
  topline=false,
  bottomline=false,
  rightline=false,
  skipabove=\topsep,
  skipbelow=\topsep,
  linecolor=purple,
  frametitle={\noindent\textcolor{purple}{\textbf{SOLUTION }}},
  endinnercode={$\hfill\textcolor{purple}{\blacksquare}$}
]{solution}

%%%%%%%%%%%%%% COLOR BOXED %%%%%%%%%%%%%%%
% Syntax: \colorboxed[<color model>]{<color specification>}{<math formula>}
\newcommand*{\colorboxed}{}
\def\colorboxed#1#{%
  \colorboxedAux{#1}%
}
\newcommand*{\colorboxedAux}[3]{%
  % #1: optional argument for color model
  % #2: color specification
  % #3: formula
  \begingroup
    \colorlet{cb@saved}{.}%
    \color#1{#2}%
    \boxed{%
      \color{cb@saved}%
      #3%
    }%
  \endgroup
}
%%%%%%%%%%%%%%%%%%%%%%%%%%%%%%%%%%%%%%%%%

\begin{document}
\setcounter{chapter}{3}
\setcounter{page}{30}
\chapter{z--Transform}
As with the CT case, DT LTI systems can be characterized by their impulse responses $h[n]$, and 
the outputs of such systems are given by the linear convolution of the input signal and the impulse response. 
An example of such a system is the affine transformation on the independent variable, whose impulse response is 
\begin{align}
    h[n]=\delta[an-b]=\delta\left[n-\frac{b}{a}\right].
\end{align}
\\ \\
As some DT LTI systems are represented by a class of difference equations called LCCDEs, it is easier to find 
the system response to such systems by mapping them to a domain that uses algebraic equations, similar to how the 
Laplace tranform works in the CT case.
\\ \\
Unlike the CT case, because DT systems can either be implemented in real-time or offline, causality is not as 
crucial depending on the application. Therefore, the bilateral (two-sided) $z$--transform is of interest.

\section{Bilateral z--Transform}
\begin{tcolorbox}[width=\textwidth,colback={white}, sharp corners]
    The \emph{bilateral $z$--transform} of a DT signal $x[n]$ is defined as  
    \begin{align}
        X(z) = \mathcal{Z}[x[n]] = \sum_{n=-\infty}^{+\infty} x[n] z^{-n},
    \end{align}
    where $z$ is a complex variable with some \emph{region of convergence} (ROC), $R$. 
\end{tcolorbox}
\noindent The $z$--transform essentially is an operator that maps a signal defined in the \emph{time domain} to another signal 
defined in the \emph{$z$-domain}. $x[n]$ and $X(z)$ constitute a \emph{$z$--transform pair}. This relationship can be written as 
\begin{align}
    x[n] \iff X(z) \text{ for ROC } R,
\end{align}
where $R$ is the ROC, defined as the values for $z$ for which the bilateral $z$-domain summation is absolutely summable 
for a given signal $x[n]$ for all $n$. Because of this relationship, there exists an \emph{inverse $z$--transform} defined by 
the contour integral
\begin{align}
    x[n] = \mathcal{Z}^{-1}[X(z)] = \frac{1}{j2\pi}\oint\nolimits_{\Gamma} X(z) z^{n-1} \,dz,
\end{align}
where $\Gamma$ is a counterclockwise closed circular contour centered at the origin and entirely in the ROC.
Direct computation of the inverse contour integral however is often more cumbersome than referencing a table of $z$--transform pairs 
or finding other computational methods.
\\ \\
Additionally, for signal $x[n]$ to be transformable, the bilateral $z$--transform summation must absolutely converge such that 
\begin{align}
    \sum_{n=-\infty}^{+\infty} |x[n] z^{-n}| < \infty
\end{align}
The \emph{region of convergence} (ROC) can then be further defined as the values for $z$ for which the bilateral $z$--transform summation absolutely 
converges for a given signal $x[n]$ for all $n$. 
\\ \\
Table \ref{z_prop} lists the properties of the bilateral $z$--transform.
\begin{table}[hbt!]
    \centering
    \caption{Properties of the bilateral z--transform.}
    \label{z_prop}
    \begin{tabular}{|c|c|c|c|}
        \hline
        Property & $x[n]$ & $X(z)=\mathcal{Z}[x[n]]$ & ROC: $R$ \\[0.15cm]
        \hline
        & & & \\
        Superposition & $K_1x_1[n]+K_2x_2[n]$ & $K_1X_1(z)+K_2X_2(z)$ & At least $R_1\cap R_2$ \\[0.5cm]
        Time shift & $x[n-N]$ & $z^{-N}X(z)$ & \shortstack{$R$, except for possible \\ addition/deletion of 0} \\[0.5cm]
        Time reversal & $x[-n]$ & $X\left(\dfrac{1}{z}\right)$ & $R^{-1}$ \\[0.5cm]
        Complex modulation & $a^n x[n]$ & $X\left(\dfrac{z}{a}\right)$ & $|a|\cdot R$ \\[0.5cm]
        Finite difference & $x[n]-x[n-1]$ & $(1-z^{-1})X(z)$ & At least $R\cap(|z|>0)$ \\[0.5cm]
        Accumulation & $\displaystyle\sum_{k=-\infty}^{n} x[k]$ & $\dfrac{1}{1-z^{-1}} X(z)$ & At least $R\cap(|z|>1)$ \\[0.5cm]
        $z$-domain derivative & $n\cdot x[n]$ & $-z\,\dfrac{dX(z)}{dz} = -z\cdot X'(z)$ & $R$ \\[0.5cm]
        Upsampling & $x[n/M],\text{ } M\in\mathbb{Z}^+$ & $X(z^M)$ & $R^{1/M}$ \\[0.5cm]
        Downsampling & $x[Mn],\text{ } M\in\mathbb{Z}^+$ & $\dfrac{1}{M}\displaystyle\sum_{k=0}^{M-1}X(e^{-j2\pi k/M}z^{1/M})$ & $R^M$ \\[0.5cm]
        Convolution & $x_1[n]*x_2[n]$ & $X_1(z)X_2(z)$ & At least $R_1\cap R_2$ \\[0.5cm]
        \hline
    \end{tabular}
\end{table}

\begin{example}
    Find the (bilateral) $z$--transform of $x[n]=n\cdot u[n]$. 
\end{example}
\begin{solution}
    First, find the $z$--transform of $u[n]$:
    \begin{align*}
        \mathcal{Z}[u[n]] = \sum_{n=-\infty}^{+\infty} u[n] z^{-n} = \sum_{n=0}^{\infty} z^{-n} = \sum_{n=0}^{\infty} (z^{-1})^n = \frac{1}{1-z^{-1}}.
    \end{align*}
    Testing for absolute convergence, we get 
    \begin{align*}
        \sum_{n=-\infty}^{+\infty} \left|u[n] z^{-n}\right| = \sum_{n=0}^{\infty} \left|(z^{-1})^n\right| = \sum_{n=0}^{\infty} \left|z^{-1}\right|^n = \frac{1}{1-\left|z^{-1}\right|} < \infty,
    \end{align*}
    which can only hold true when $|z^{-1}| < 1$, or equivalently $|z| > 1$. Therefore,
    \begin{align*}
        u[n] \iff \frac{1}{1-z^{-1}} \text{ for } |z| > 1.
    \end{align*}
    Using the $z$--domain derivative property, we get 
    \begin{align*}
        X(z) = \mathcal{Z}[x[n]] &= \mathcal{Z}[n\cdot u[n]] = -z\cdot \frac{d}{dz}\left[\frac{1}{1-z^{-1}}\right] \\
        &= -z\cdot \frac{-1}{(z-1)^2} = \frac{z}{(z-1)^2} = \frac{z^{-1}}{(1-z^{-1})^2}.
    \end{align*}
    Since the ROC remains unchanged under this property, the $z$--transform of $x[n]$ is given by
    \begin{align*}
        x[n] = n\cdot u[n] \iff X(z) = \frac{z^{-1}}{(1-z^{-1})^2} \text{ for } |z| > 1.
    \end{align*}
\end{solution}

\begin{example}
    Find the (bilateral) $z$--transform of $x[n]=-a^n u[-n-1]$. 
\end{example}
\begin{solution}
    Using the $z$--transform summation:
    \begin{align*}
        X(z) = \mathcal{Z}[x[n]] &= \sum_{n=-\infty}^{+\infty} -a^n u[-n-1] z^{-n} = \sum_{n=-\infty}^{-1} -a^n z^{-n} = \sum_{n=-\infty}^{-1} -(az^{-1})^n \\
        &= \sum_{m=1}^{\infty} -(az^{-1})^{-m} = -\sum_{m=1}^{\infty} (a^{-1}z)^m = -\frac{a^{-1}z}{1-a^{-1}z} = \frac{1}{1-az^{-1}}.
    \end{align*}
    Testing for absolute convergence:
    \begin{align*}
        \sum_{n=-\infty}^{+\infty} \left|-a^n u[-n-1] z^{-n}\right| = \sum_{m=1}^{\infty} \left|a^{-1}z\right|^m = \frac{\left|a^{-1}z\right|}{1-\left|a^{-1}z\right|} < \infty,
    \end{align*}
    which can only hold true when $\left|a^{-1}z\right| < 1$, or equivalently $|z|<|a|$. Therefore,
    \begin{align*}
        x[n] = -a^n u[-n-1] \iff X(z) = \frac{1}{1-az^{-1}} \text{ for } |z| < |a|.
    \end{align*}
\end{solution}

Table \ref{z_pairs} lists some common bilateral $z$--transform pairs. Note that the $z$--transform is typically written in terms of $z^{-N}$ for $N\geq 0$ in order to exploit the time shift property.
\begin{table}[hbt!]
    \centering
    \caption{Bilateral z--transform pairs.}
    \label{z_pairs}
    \begin{tabular}{|c|c|c|}
        \hline
        $x[n]$ & $X(z)=\mathcal{Z}[x[n]]$ & ROC \\[0.15cm]
        \hline
        & & \\[0.1cm]
        $\delta[n]$ & $1$ & All $z$ \\[0.5cm]
        $\delta[n-N],\, N>0$ & $z^{-N}$ & All $z$ except $z=0$ \\[0.5cm]
        $\delta[n+N],\, N>0$ & $z^N$ & All $z$ except $z=\infty$ \\[0.5cm]
        $u[n]$ & $\dfrac{1}{1-z^{-1}}$ & $|z|>1$ \\[0.5cm]
        $-u[-n-1]$ & $\dfrac{1}{1-z^{-1}}$ & $|z|<1$ \\[0.5cm]
        $n\cdot u[n]$ & $\dfrac{z^{-1}}{(1-z^{-1})^2}$ & $|z|>1$ \\[0.5cm]
        $-n\cdot u[-n-1]$ & $\dfrac{z^{-1}}{(1-z^{-1})^2}$ & $|z|<1$ \\[0.5cm]
        $a^n u[n]$ & $\dfrac{1}{1-az^{-1}}$ & $|z|>|a|$ \\[0.5cm]
        $-a^n u[-n-1]$ & $\dfrac{1}{1-az^{-1}}$ & $|z|<|a|$ \\[0.5cm]
        $na^n u[n]$ & $\dfrac{az^{-1}}{(1-az^{-1})^2}$ & $|z|>|a|$ \\[0.5cm]
        $-na^n u[-n-1]$ & $\dfrac{az^{-1}}{(1-az^{-1})^2}$ & $|z|<|a|$ \\[0.5cm]
        $\dfrac{\prod_{k=1}^{m-1}(n+k)}{(m-1)!}a^n u[n]$ & $\dfrac{1}{(1-az^{-1})^m}$ & $|z|>|a|$ \\[0.5cm]
        $\cos(\Omega_0 n)u[n]$ & $\dfrac{1-(\cos\Omega_0)z^{-1}}{1-(2\cos\Omega_0)z^{-1}+z^{-2}}$ & $|z|>1$ \\[0.5cm]
        $\sin(\Omega_0 n)u[n]$ & $\dfrac{(\sin\Omega_0)z^{-1}}{1-(2\cos\Omega_0)z^{-1}+z^{-2}}$ & $|z|>1$ \\[0.5cm]
        $a^n\cos(\Omega_0 n)u[n]$ & $\dfrac{1-(a\cos\Omega_0)z^{-1}}{1-(2a\cos\Omega_0)z^{-1}+a^2z^{-2}}$ & $|z|>|a|$ \\[0.5cm]
        $a^n\sin(\Omega_0 n)u[n]$ & $\dfrac{(a\sin\Omega_0)z^{-1}}{1-(2a\cos\Omega_0)z^{-1}+a^2z^{-2}}$ & $|z|>|a|$ \\[0.5cm]
        \hline
    \end{tabular}
\end{table}

\begin{example}
    Suppose a DT signal $x[n]= u[n]$ is inputted into a DT LTI system characterized by $h[n]=(0.1)^nu[n]$. Find the (bilateral) $z$--transform of 
    the output signal $y[n]$.
\end{example}
\begin{solution}
    Using the table of $z$--transform pairs, it follows that
    \begin{align*}
        x[n] = u[n] &\iff X(z) = \frac{1}{1-z^{-1}} \text{ for } |z| > 1, \\
        h[n] = (0.1)^nu[n] &\iff H(z) = \frac{1}{1-0.1z^{-1}} \text{ for } |z| > 0.1.
    \end{align*}
    Then by the convolution property, 
    \begin{align*}
        Y(z) = X(z)H(z) = \frac{1}{1-z^{-1}}\cdot\frac{1}{1-0.1z^{-1}} = \frac{1}{1-1.1z^{-1}+0.1z^{-2}}.
    \end{align*}
    Since $(|z|>1)\cap(|z|>0.1)=(|z|>1)$, it follows that the ROC at least contains $|z|>1$.
\end{solution}

The region of convergence (ROC) can be visualized as regions of the $z$--plane for which $X(z)$ is finite. They often are 
depicted geometrically as open disks, annuli (rings), exteriors of circles, or even nearly the entire $z$--plane. Interestingly enough, 
depending on the type of signal in terms of causality and duration, these region types typically can be predicted. 
See Table \ref{roc} for reference.
\\ \\
Lastly, note that while the DT signal $x[n]$ is discrete in time, its $z$--transform $X(z)$ is actually continuous in $z$.

\begin{table}[hbt!]
    \centering
    \caption{Characteristic families of DT signals and their corresponding ROCs.}
    \label{roc}
    \resizebox{\textwidth}{!}{%
    \begin{tabular}{ c|c|c }
    Signal type & Stem plot example & ROC in the $z$--plane \\[0.1cm]
    \hline 
    \shortstack{Finite and \\causal} 
    & \adjustbox{valign=m}{\resizebox{0.25\textwidth}{!}{
        \begin{tikzpicture}
            \begin{axis}[
                axis x line=center, axis y line=center,
                ymin=0, ymax=1.5, ytick={0}, ylabel={$x[n]$},
                xmin=-6, xmax=6, xtick={0}, xlabel={},
                width=6cm, height=3cm]
                \addplot+ [
                    ycomb,
                ] coordinates {(-4,0) (-3,0) (-2,0) (-1,0) (0,1) (1,0.8) (2,0.64) (3,0.512) (4,0) (5,0)};
            \end{axis}
        \end{tikzpicture}}}
    & \adjustbox{valign=m}{\resizebox{0.175\textwidth}{!}{
        \begin{tikzpicture}
            \begin{axis}[
                axis x line=center, axis y line=center,
                ymin=-2, ymax=2, ytick={0}, ylabel={$\Im(z)$},
                xmin=-2, xmax=2, xtick={0}, xlabel={$\Re(z)$},
                width=6cm, height=6cm]
                \fill[red!5, opacity=0.85] (-5,-5) rectangle ++ (5,5);
                \addplot+ [
                    ycomb,
                ] coordinates {(0,0)};
            \end{axis}
        \end{tikzpicture}}} \shortstack{Entire $z$--plane\\except $z=0$} \\[1.3cm]
    \shortstack{Finite and \\anticausal} 
    & \adjustbox{valign=m}{\resizebox{0.25\textwidth}{!}{
        \begin{tikzpicture}
            \begin{axis}[
                axis x line=center, axis y line=center,
                ymin=0, ymax=1.5, ytick={0}, ylabel={$x[n]$},
                xmin=-6, xmax=6, xtick={0}, xlabel={},
                width=6cm, height=3cm]
                \addplot+ [
                    ycomb,
                ] coordinates {(5,0) (4,0) (3,0) (2,0) (1,0) (0,1) (-1,0.8) (-2,0.64) (-3,0.512) (-4,0) (-5,0)};
            \end{axis}
        \end{tikzpicture}}}
    & \adjustbox{valign=m}{\resizebox{0.175\textwidth}{!}{
        \begin{tikzpicture}
            \begin{axis}[
                axis x line=center, axis y line=center,
                ymin=-2, ymax=2, ytick={0}, ylabel={$\Im(z)$},
                xmin=-2, xmax=2, xtick={0}, xlabel={$\Re(z)$},
                width=6cm, height=6cm]
                \fill[red!5, opacity=0.85] (-5,-5) rectangle ++ (5,5);
            \end{axis}
        \end{tikzpicture}}} \shortstack{Entire $z$--plane\\except $z=\infty$} \\[1.3cm]
    \shortstack{Finite and \\noncausal (two-sided)} 
    & \adjustbox{valign=m}{\resizebox{0.25\textwidth}{!}{
        \begin{tikzpicture}
            \begin{axis}[
                axis x line=center, axis y line=center,
                ymin=0, ymax=1.5, ytick={0}, ylabel={$x[n]$},
                xmin=-6, xmax=6, xtick={0}, xlabel={},
                width=6cm, height=3cm]
                \addplot+ [
                    ycomb,
                ] coordinates {(-5,0) (-4,0) (-3,0) (-2,0.36) (-1,0.6) (0,1) (1,0.8) (2,0.64) (3,0.512) (4,0) (5,0)};
            \end{axis}
        \end{tikzpicture}}}
    & \adjustbox{valign=m}{\resizebox{0.175\textwidth}{!}{
        \begin{tikzpicture}
            \begin{axis}[
                axis x line=center, axis y line=center,
                ymin=-2, ymax=2, ytick={0}, ylabel={$\Im(z)$},
                xmin=-2, xmax=2, xtick={0}, xlabel={$\Re(z)$},
                width=6cm, height=6cm]
                \fill[red!5, opacity=0.85] (-5,-5) rectangle ++ (5,5);
                \addplot+ [
                    ycomb,
                ] coordinates {(0,0)};
            \end{axis}
        \end{tikzpicture}}} \shortstack{Entire $z$--plane\\except $z=0$\\and $z=\infty$} \\[1.3cm]
    \shortstack{Infinite and \\causal} 
    & \adjustbox{valign=m}{\resizebox{0.25\textwidth}{!}{
        \begin{tikzpicture}
            \begin{axis}[
                axis x line=center, axis y line=center,
                ymin=0, ymax=1.5, ytick={0}, ylabel={$x[n]$},
                xmin=-6, xmax=6, xtick={0}, xlabel={},
                width=6cm, height=3cm]
                \addplot+ [
                    ycomb,
                ] coordinates {(-4,0) (-3,0) (-2,0) (-1,0) (0,1) (1,0.8) (2,0.64) (3,0.512)};
                \node[] at (4.5,0.4096) {\Large $\cdots$};
            \end{axis}
        \end{tikzpicture}}}
    & \adjustbox{valign=m}{\resizebox{0.175\textwidth}{!}{
        \begin{tikzpicture}
            \begin{axis}[
                axis x line=center, axis y line=center,
                ymin=-2, ymax=2, ytick={0}, ylabel={$\Im(z)$},
                xmin=-2, xmax=2, xtick={0}, xlabel={$\Re(z)$},
                width=6cm, height=6cm]
                \fill[red!5, opacity=0.85] (-5,-5) rectangle ++ (5,5); 
                \filldraw[red!60, fill=white, very thick, opacity=0.9] (0,0) circle[radius=1];
                \draw [-latex, very thick, color=blue] (0,0) -- (0.8,0.6);
                \draw (0.3,0.5) node {$r$};
            \end{axis}
        \end{tikzpicture}}} $\;\;\;\;\;\;|z|>r\;\;\;\;\;\;\,$ \\[1.3cm]
    \shortstack{Infinite and \\anticausal} 
    & \adjustbox{valign=m}{\resizebox{0.25\textwidth}{!}{
        \begin{tikzpicture}
            \begin{axis}[
                axis x line=center, axis y line=center,
                ymin=0, ymax=1.5, ytick={0}, ylabel={$x[n]$},
                xmin=-6, xmax=6, xtick={0}, xlabel={},
                width=6cm, height=3cm]
                \addplot+ [
                    ycomb,
                ] coordinates {(5,0) (4,0) (3,0) (2,0) (1,0) (0,1) (-1,0.8) (-2,0.64) (-3,0.512)};
                \node[] at (-4.5,0.4096) {\Large $\cdots$};
            \end{axis}
        \end{tikzpicture}}}
    & \adjustbox{valign=m}{\resizebox{0.175\textwidth}{!}{
        \begin{tikzpicture}
            \begin{axis}[
                axis x line=center, axis y line=center,
                ymin=-2, ymax=2, ytick={0}, ylabel={$\Im(z)$},
                xmin=-2, xmax=2, xtick={0}, xlabel={$\Re(z)$},
                width=6cm, height=6cm]
                \filldraw[red!60, fill=red!5, very thick, opacity=0.9] (0,0) circle[radius=1];
                \draw [-latex, very thick, color=blue] (0,0) -- (0.8,0.6);
                \draw (0.3,0.5) node {$r$};
            \end{axis}
        \end{tikzpicture}}} $\;\;\;\;\;\;|z|<r\;\;\;\;\;\;\,$ \\[1.3cm]
    \shortstack{Infinite and \\noncausal (two-sided)}
    & \adjustbox{valign=m}{\resizebox{0.25\textwidth}{!}{
        \begin{tikzpicture}
            \begin{axis}[
                axis x line=center, axis y line=center,
                ymin=0, ymax=1.5, ytick={0}, ylabel={$x[n]$},
                xmin=-6, xmax=6, xtick={0}, xlabel={},
                width=6cm, height=3cm]
                \addplot+ [
                    ycomb,
                ] coordinates {(-2,0.36) (-1,0.6) (0,1) (1,0.8) (2,0.64) (3,0.512)};
                \node[] at (4.5,0.4096) {\Large $\cdots$};
                \node[] at (-4,0.216) {\Large $\cdots$};
            \end{axis}
        \end{tikzpicture}}}
    & \adjustbox{valign=m}{\resizebox{0.175\textwidth}{!}{
        \begin{tikzpicture}
            \begin{axis}[
                axis x line=center, axis y line=center,
                ymin=-2, ymax=2, ytick={0}, ylabel={$\Im(z)$},
                xmin=-2, xmax=2, xtick={0}, xlabel={$\Re(z)$},
                width=6cm, height=6cm]
                \filldraw[red!60, fill=red!5, very thick, opacity=0.9] (0,0) circle[radius=1.2];
                \filldraw[red!60, fill=white, very thick, opacity=0.85] (0,0) circle[radius=0.8];
                \draw [-latex, very thick, color=blue] (0,0) -- (0.96,0.72);
                \draw (0.3,0.5) node {$r_2$};
                \draw [-latex, very thick, color=blue] (0,0) -- (0.64,-0.48);
                \draw (0.3,-0.5) node {$r_1$};
            \end{axis}
        \end{tikzpicture}}} $\;\,r_1<|z|<r_2\;\,$ \\[1.3cm]
    \hline
    \end{tabular}
    }
\end{table}

\pagebreak
\section{Unilateral z--Transform}
In the event that only \emph{causal} signals and systems are of interest, then the unilateral $z$--transform is sufficient enough for analysis.
\begin{tcolorbox}[width=\textwidth,colback={white}, sharp corners]
    The \emph{unilateral $z$--transform} of a DT signal $x[n]$ is defined as  
    \begin{align}
        X^+(z) = \mathcal{Z}^+[x[n]] = \sum_{n=0}^{\infty} x[n] z^{-n}.
    \end{align}
\end{tcolorbox}

Table \ref{u_prop} lists the properties of the unilateral z--transform. Note that the properties are largely the same as that of the bilateral z--transform, 
with the exception of the time shifts. Additionally, the initial and final value theorems hold true for causal signals.
\\ \\ 
Similar to the CT case with the unilateral Laplace transform, if DT signal $x[n]$ is causal, then there exists a unique unilateral $z$--transform $X^+(z)$, with its 
unilateral $z$--transform pair denoted as
\begin{align}
    \text{causal }x[n] \iff X^+(z).
\end{align}
In fact, when $x[n]$ is causal, the closed form expression for its unilateral and bilateral $z$--transforms are the same. 
\\ \\ 
Note that the ROC is not mentioned for the unilateral form. This is because the unilateral $z$--transform works under the assumption that the signals and systems of interest are causal; only then 
is when the pair is unique. In such a specific case, there are other methods of analysis without need of finding the ROC.
\\ \\
However, in the event that an anticausal or noncausal element (signal or system) is introduced, the bilateral $z$--transform is more suitable. For the rest of this text, 
the bilateral $z$--transform will be referred to as simply the $z$--transform. The unilateral $z$--transform will still be referred to by its full nomenclature.

\begin{table}[hbt!]
    \centering
    \caption{Properties of the unilateral z--transform (for causal signals).}
    \label{u_prop}
    \begin{tabular}{|c|c|c|}
        \hline
        Property & $x[n]$ & $X^+(z)=\mathcal{Z}^+[x[n]]$ \\[0.15cm]
        \hline
        & & \\
        Superposition & $K_1x_1[n]+K_2x_2[n]$ & $K_1X_1^+(z)+K_2X_2^+(z)$ \\[0.5cm]
        Time delay of causal signal & $x[n-N]u[n-N],\text{ } N>0$ & $z^{-N}X^+(z)$ \\[0.5cm]
        Time delay of noncausal signal & $x[n-N]u[n],\text{ } N>0$ & $z^{-N}\left[X^+(z)+\displaystyle\sum_{n=1}^{N}x[-n]z^n\right]$ \\[0.5cm]
        Time advance & $x[n+N]u[n],\text{ } N>0$ & $z^N\left[X^+(z)-\displaystyle\sum_{n=0}^{N-1}x[n]z^{-n}\right]$ \\[0.5cm]
        Time reversal & $x[-n]$ & $X^+\left(\dfrac{1}{z}\right)$ \\[0.5cm]
        Complex modulation & $a^n x[n]$ & $X^+\left(\dfrac{z}{a}\right)$ \\[0.5cm]
        Finite difference & $x[n]-x[n-1]$ & $(1-z^{-1})X^+(z)$ \\[0.5cm]
        Accumulation & $\displaystyle\sum_{k=0}^{n} x[k]$ & $\dfrac{1}{1-z^{-1}} X^+(z)$ \\[0.5cm]
        $z$-domain derivative & $n\cdot x[n]$ & $-z\,\dfrac{dX^+(z)}{dz}$ \\[0.5cm]
        Upsampling & $x[n/M],\text{ } M\in\mathbb{Z}^+$ & $X^+(z^M)$ \\[0.5cm]
        Downsampling & $x[Mn],\text{ } M\in\mathbb{Z}^+$ & $\dfrac{1}{M}\displaystyle\sum_{k=0}^{M-1}X^+(e^{-j2\pi k/M}z^{1/M})$ \\[0.5cm]
        Convolution & $x_1[n]*x_2[n]$ & $X_1^+(z)X_2^+(z)$ \\[0.5cm]
        Initial value theorem & $x[0]$ & $\displaystyle\lim_{z\to\infty} X^+(z)$ \\[0.5cm]
        Final value theorem & $\displaystyle\lim_{n\to\infty}x[n]$ & $\displaystyle\lim_{z\to 1} \,(z-1)X^+(z)$ \\[0.5cm]
        \hline
    \end{tabular}
\end{table}

\begin{example}
    Determine the final value $x[\infty]$ of $x[n]=(0.3)^{n-1}(n-1)u[n-1]$.
\end{example}
\begin{solution}
    Note that the signal is causal. Using the table of $z$--transform pairs and the time delay property for the unilateral $z$--transform, 
    \begin{align*}
        X^+(z) &= z^{-1} \mathcal{Z}[(0.3)^n n\cdot u[n]] = z^{-1}\frac{0.3z^{-1}}{(1-0.3z^{-1})^2} = \frac{0.3z^{-2}}{(1-0.3z^{-1})^2}.
    \end{align*}
    Then by the final value theorem,
    \begin{align*}
        x[\infty] = \lim_{z\to 1} (z-1)X^+(z) = \lim_{z\to 1} (z-1)\frac{0.3z^{-2}}{(1-0.3z^{-1})^2} = 0\times \frac{0.3}{(0.7)^2} = 0.
    \end{align*}
\end{solution}

\section{Inverse z--Transform}
While the inverse $z$--transform could be computed via contour integration, the process is too cumbersome. Typically, a table of $z$--transform pairs is 
used, albeit some manipulation is required. Therefore, we will explore different methods of finding the inverse $z$--transform.

\subsection{Inverse z--Transform by Power Series Expansion}
Recall that the $z$--transform of a DT signal $x[n]$ is given by 
\begin{align}
    X(z) = \mathcal{Z}[x[n]] = \sum_{n=-\infty}^{+\infty} x[n] z^{-n}.
\end{align}
If some $z$-domain function $X(z)$ can be written in such a format, then by inspection we can easily list the coefficients and express the signal $x[n]$ 
as a sequence. Typically, this method is reserved for finite signals, though if only the first few terms of a one-sided signal is needed, then power series expansion 
is sufficient enough.

\begin{tcolorbox}[width=\textwidth,colback={white}, sharp corners]
    There are four ROC cases, each with a different task at hand.
    \begin{itemize}
        \item (ROC: nearly all $z$): $x[n]$ is \emph{finite}. Simply read the coefficients.
        \item (ROC: $|z|>|a|$): $x[n]$ is \emph{infinite} and \emph{causal}. Perform polynomial (in $z^{-1}$) long division with 1 as the leading term of the divisor. Continue until enough coefficients are needed.
        \item (ROC: $|z|<|a|$): $x[n]$ is \emph{infinite} and \emph{anticausal}. Perform polynomial (in $z^{-1}$) long division with the lowest power $z^{-N}$ as the leading term of the divisor. Continue until enough coefficients are needed.
        \item (ROC: $|a_1|<|z|<|a_2|$): $x[n]$ is \emph{infinite} and \emph{two-sided}. Power series expansion is not a useful method; partial fraction expansion is needed.
    \end{itemize}
\end{tcolorbox}

\begin{example}
    Find the inverse $z$--transform of 
    \begin{align*}
        X(z) = \frac{7z^2+3z+6}{z^3}, \text{ for } z\neq 0.
    \end{align*}
\end{example}
\begin{solution}
    Since the ROC is nearly the entire $z$-plane, the signal $x[n]$ must be finite. We just need to simplify and list the coefficients.
    \begin{align*}
        X(z) = \frac{7z^2+3z+6}{z^3} &= 7z^{-1} + 3z^{-2} + 6z^{-3} \\
        \Longrightarrow x[n] &= \left\{\underbar{0},7,3,6\right\} 
    \end{align*}
\end{solution}

\begin{example}
    Find the first few terms of $x[n]$, given that its $z$--transform is  
    \begin{align*}
        X(z) = \frac{z-1}{z-2}, \text{ for } |z|>2.
    \end{align*}
\end{example}
\begin{solution}
    From the ROC alone, the signal $x[n]$ must be infinite and causal. Rewriting $X(z)$ as a polynomial in $z^{-1}$,
    \begin{align*}
        X(z) = \frac{z-1}{z-2} = \frac{z(1-z^{-1})}{z(1-2z^{-1})} = \frac{1-z^{-1}}{1-2z^{-1}}
    \end{align*}
    From the denominator, let 1 be the leading term of the divisor. Then by long division,
    \begin{align*}
        X(z) = \frac{1-z^{-1}}{1-2z^{-1}} &= 1 + z^{-1} + 2z^{-2} + 4z^{-3} + 8z^{-4} + \cdots \\
        \Longrightarrow x[n] &= \left\{\underbar{1},1,2,4,8,...\right\}
    \end{align*}
    In fact, one could also use the table of $z$--transform pairs to verify this.
    \begin{align*}
        x[n] = \mathcal{Z}^{-1}\left[\frac{1-z^{-1}}{1-2z^{-1}}\right] &= \mathcal{Z}^{-1}\left[\frac{1}{1-2z^{-1}}\right] - \mathcal{Z}^{-1}\left[\frac{z^{-1}}{1-2z^{-1}}\right] \\
        &= 2^n u[n] - 2^{n-1}u[n-1] \\ 
        &= \left\{\underbar{1},1,2,4,8,...\right\}
    \end{align*}
\end{solution}

\begin{example}
    Find the first few terms of $x[n]$, given that its $z$--transform is  
    \begin{align*}
        X(z) = \frac{1}{1-1.5z^{-1}+0.5z^{-2}}, \text{ for } |z|<0.5.
    \end{align*}
\end{example}
\begin{solution}
    From the ROC alone, the signal $x[n]$ must be infinite and anticausal. Since $X(z)$ is already written in terms of polynomials in $z^{-1}$, 
    we choose $0.5z^{-2}$ as the leading term of the divisor such that 
    \begin{align*}
        X(z) = \frac{1}{0.5z^{-2}-1.5z^{-1}+1} &= 2z^2 + 6z^3 + 14z^4 + 30z^5 + \cdots \\
        \Longrightarrow x[n] &= \left\{\underbar{0},0,2,6,14,30,...\right\}
    \end{align*}
\end{solution}

\subsection{Inverse z--Transform by Partial Fraction Expansion}
When finding a closed form expression for infinite DT signals, inverse $z$--transform by partial fraction expansion is more ideal. 
Consider a $z$-domain rational function with positive powers in the numerator and denominator, of the form 
\begin{align}
    X(z) = \frac{N(z)}{D(z)},
\end{align}
written as a ratio of a polynomial numerator $N(z)$ to a polynomial denominator $D(z)$. 
The roots of the numerator are called the \emph{zeros} of $X(z)$ and are defined by 
\begin{align}
    N(z) = 0,
\end{align}
whereas the roots of the denominator are called the \emph{poles} of $X(z)$ and are defined by 
\begin{align}
    D(z) = 0.
\end{align}
Before even performing partial fraction expansion, first we examine the orders of the numerator and denominator. Let 
\begin{align}
    p &= \deg[N(z)], \\ 
    q &= \deg[D(z)].
\end{align}
\begin{tcolorbox}[width=\textwidth,colback={white}, sharp corners]
    There are three types of rational functions, each with a different task at hand.
    \begin{itemize}
        \item $p < q$: $X(z)$ is \emph{strictly proper}. It is ready for partial fraction expansion, if needed.
        \item $p = q$: $X(z)$ is \emph{proper}. There are two options. 
        \begin{itemize}
            \item If $N(z) = z^{M}$, first peel away a single $z$. Then perform partial fraction expansion on $X(z)/z$. Reintroduce $z$ to each residue term.
            \item Otherwise, perform long division and express $X(s)$ in terms of the quotient and remainder, then determine if the remainder term needs partial fraction expansion.
        \end{itemize}
        \item $p > q$: $X(z)$ is \emph{improper}. Perform long division and express $X(z)$ in terms of the quotient and remainder, then determine if the remainder term needs partial fraction expansion.
    \end{itemize}
\end{tcolorbox}
\noindent A term would need partial fraction expansion if there is not an easily matching $z$--transform pair provided by the table. Table \ref{z_pairs_short} lists the corresponding 
$z$--transform pairs for different partial fraction expansion cases. Note that unlike the CT case, the DT case requires ROC as well.

\begin{table}[hbt!]
    \centering
    \caption{Bilateral z--transform pairs based on poles.}
    \label{z_pairs_short}
    \resizebox{\textwidth}{!}{%
    \begin{tabular}{|c|c|c|}
        \hline
        Pole & $X(z)$ & $x[n]=\mathcal{Z}^{-1}[X(z)]$ \\[0.15cm]
        \hline
        & & \\
        \Centerstack{Distinct real\\(causal)} & $\dfrac{A}{1-az^{-1}}, \text{ for } |z|>|a|$ & $A\,a^n u[n]$ \\[0.5cm]
        \Centerstack{Distinct real\\(anticausal)} & $\dfrac{A}{1-az^{-1}}, \text{ for } |z|<|a|$ & $-A\,a^n u[-n-1]$ \\[0.5cm]
        \Centerstack{Repeated real\\(causal)} & $\dfrac{A}{(1-az^{-1})^m}, \text{ for } |z|>|a|$ & $\dfrac{A\,\prod_{k=1}^{m-1}(n+k)}{(m-1)!}\,a^n u[n]$ \\[0.5cm]
        \Centerstack{Repeated real\\(anticausal)} & $\dfrac{A}{(1-az^{-1})^m}, \text{ for } |z|<|a|$ & $\dfrac{-A\,\prod_{k=1}^{m-1}(n+k)}{(m-1)!}\,a^n u[-n-1]$ \\[0.5cm]
        \Centerstack{Distinct complex\\(causal)} & $\dfrac{Ae^{+j\alpha}}{1-Be^{+j\beta}z^{-1}}+\dfrac{Ae^{-j\alpha}}{1-Be^{-j\beta}z^{-1}}, \text{ for } |z|>|a|$ & $2A\,B^n\cos(\beta n+\alpha)u[n]$ \\[0.5cm]
        \Centerstack{Distinct complex\\(anticausal)} & $\dfrac{Ae^{+j\alpha}}{1-Be^{+j\beta}z^{-1}}+\dfrac{Ae^{-j\alpha}}{1-Be^{-j\beta}z^{-1}}, \text{ for } |z|<|a|$ & $-2A\,B^n\cos(\beta n+\alpha)u[-n-1]$ \\[0.5cm]
        \Centerstack{Repeated complex\\(causal)} & $\dfrac{Ae^{+j\alpha}}{(1-Be^{+j\beta}z^{-1})^m}+\dfrac{Ae^{-j\alpha}}{(1-Be^{-j\beta}z^{-1})^m}, \text{ for } |z|>|a|$ & $\dfrac{2A\,B^n\,\prod_{k=1}^{m-1}(n+k)}{(m-1)!}\,\cos(\beta n+\alpha)u[n]$ \\[0.5cm]
        \Centerstack{Repeated complex\\(anticausal)} & $\dfrac{Ae^{+j\alpha}}{(1-Be^{+j\beta}z^{-1})^m}+\dfrac{Ae^{-j\alpha}}{(1-Be^{-j\beta}z^{-1})^m}, \text{ for } |z|<|a|$ & $\dfrac{-2A\,B^n\,\prod_{k=1}^{m-1}(n+k)}{(m-1)!}\,\cos(\beta n+\alpha)u[-n-1]$ \\[0.5cm]
        \hline
    \end{tabular}
    }
\end{table}

\begin{example}
    Find the inverse $z$--transform of  
    \begin{align*}
        X(z) = \frac{1}{1-1.5z^{-1}+0.5z^{-2}}, \text{ for }
    \end{align*}
    (a) ROC: $|z|>1$ \\
    (b) ROC: $|z|<0.5$ \\
    (c) ROC: $0.5<|z|<1$
\end{example}
\begin{solution}
    Rewrite $X(z)$ in terms of positive powers such that 
    \begin{align*}
        X(z) = \frac{1}{1-1.5z^{-1}+0.5z^{-2}} = \frac{z^2}{z^2-1.5z+0.5}.
    \end{align*}
    Note that $X(z)$ is proper. Peel away a $z$ such that 
    \begin{align*}
        X(z) = z \left[\frac{z}{z^2-1.5z+0.5}\right] = z\left[\frac{z}{(z-1)(z-0.5)}\right].
    \end{align*}
    Then performing partial fraction expansion,
    \begin{align*}
        X(z) &= z\underbrace{\left[\frac{z}{(z-1)(z-0.5)}\right]}_{F(z)} = z\left[\frac{A_1}{z-1} + \frac{A_2}{z-0.5}\right], \text{ where } \\
        A_1 &= (z-1)F(z)\big|_{z=1} = \left[\frac{z}{z-0.5}\right|_{z=1} = 2 \\
        A_2 &= (z-0.5)F(z)\big|_{z=0.5} = \left[\frac{z}{z-1}\right|_{z=0.5} = -1
    \end{align*}
    Finally, converting the result back to negative powers,
    \begin{align*}
        X(z) = z\left[\frac{2}{z-1} - \frac{1}{z-0.5}\right] &= \frac{2z}{z-1} - \frac{z}{z-0.5} \\ 
        &= \frac{2}{1-z^{-1}} - \frac{1}{1-0.5z^{-1}}
    \end{align*}
    First, consider the ROC $|z|>1$. This indicates that the signal $x_a[n]$ is infinite and causal. Using the $z$--transform tables, 
    \begin{align*}
        x_a[n] = 2u[n] - (0.5)^nu[n] = [2-(0.5)^n]u[n].
    \end{align*}
    Next, consider the ROC $|z|<0.5$. This indicates that the signal $x_b[n]$ is infinite and anticausal. From the tables, it follows that 
    \begin{align*}
        x_b[n] = -2u[-n-1] + (0.5)^nu[-n-1] = [-2+(0.5)^n]u[-n-1]. 
    \end{align*}
    Last, consider the ROC $0.5<|z|<1$. This indicates that the signal $x_c[n]$ is infinite and two-sided. We can rewrite the ROC as the intersection of  
    two regions: 
    \begin{align*}
        \text{ROC}_1 \cap \text{ROC}_2 = (|z|<1) \cap (|z|>0.5)
    \end{align*}
    From above, we see that $\text{ROC}_1=(|z|<1)$ indicates that the pole $p_1=1$ contributes to the anticausal part, whereas 
    $\text{ROC}_2=(|z|>0.5)$ indicates that the pole $p_2=0.5$ contributes to the causal part. That is,
    \begin{align*}
        X(z) &= \underbrace{\frac{2}{1-z^{-1}}}_{|z|<1} - \underbrace{\frac{1}{1-0.5z^{-1}}}_{|z|>0.5} \\
        x_c[n] &= -2u[-n-1] - (0.5)^nu[n].
    \end{align*}
\end{solution}

\begin{example}
    Find the causal signal whose $z$--transform is given by 
    \begin{align*}
        X(z) = \frac{1}{(1+z^{-1})(1-z^{-1})^2}.
    \end{align*}
\end{example}
\begin{solution}
    Since the signal is said to be causal, ROC analysis is not needed. In fact, we can even use the unilateral $z$--transform notation such that 
    \begin{align*}
        X^+(z) = \frac{1}{(1+z^{-1})(1-z^{-1})^2} = \frac{z^3}{(z+1)(z-1)^2} = z\left[\frac{z^2}{(z+1)(z-1)^2}\right].
    \end{align*}
    Using partial fraction expansion,
    \begin{align*}
        X^+(z) &= z\left[\frac{z^2}{(z+1)(z-1)^2}\right] = z\left[\frac{A_1}{z+1} + \frac{B_1}{z-1} + \frac{B_2}{(z-1)^2}\right] \\
        A_1 &= \left[\frac{z^2}{(z-1)^2}\right|_{z=-1} = \frac{1}{4} \\ 
        A_2 &= \frac{d}{dz}\left[\frac{z^2}{z+1}\right]_{z=1} = \left[\frac{(z+1)(2z)-z^2(1)}{(z+1)^2}\right|_{z=1} = \frac{3}{4} \\
        A_3 &= \left[\frac{z^2}{z+1}\right|_{z=1} = \frac{1}{2}
    \end{align*}
    Since $x[n]$ is causal, the closed form expression is given by 
    \begin{align*}
        x[n] = \frac{1}{4}\,(-1)^n u[n] + \frac{3}{4\,}u[n] + \frac{1}{2}\,n\cdot u[n].
    \end{align*}
\end{solution}

Of course, even without a ROC provided, one can select the ROC based on the criteria of the DT signal $x[n]$. Suppose that 
\begin{align}
    X(z) = K\cdot\, \frac{\displaystyle\prod_{k=1}^{M}(1-r_kz^{-1})}{\displaystyle\prod_{k=1}^{N}(1-p_kz^{-1})} = \frac{A_1}{1-p_1z^{-1}} + \frac{A_2}{1-p_2z^{-1}} + \cdots + \frac{A_N}{1-p_Nz^{-1}},
\end{align}
with poles $p_k$ and no known ROC. Then depending on what type of infinite DT signal is desired, one can use the superposition property such that 
\begin{align}
    \text{ROC} = \text{ROC}_1 \cap \text{ROC}_2 \cap \cdots \cap \text{ROC}_N, \text{ at least,}
\end{align}
which leads to Table \ref{roc_poles}. Note that for the two-sided case, the ROC is only obvious when there are only two poles.

\begin{table}[hbt!]
    \centering
    \caption{ROC of infinite-duration DT signals based on poles.}
    \label{roc_poles}
    \begin{tabular}{|c|c|}
        \hline
        Infinite signal type & ROC \\[0.15cm]
        \hline
        & \\
        Causal & $|z|>\max|p_k|$, for $p_k=\{p_1,p_2,...,p_N\}$ \\[0.5cm]
        Anticausal & $|z|<\min|p_k|$, for $p_k=\{p_1,p_2,...,p_N\}$ \\[0.5cm]
        \Centerstack{Noncausal\\(two-sided)} & $|p_1|<|z|<|p_2|$, for $p_k=\{p_1,p_2\}$ \\[0.5cm]
        \hline
    \end{tabular}
\end{table}

\section{DT Transfer Function}

\begin{tcolorbox}[width=\textwidth,colback={white}, sharp corners]
    The \emph{discrete-time transfer function} is the $z$--transform of the impulse response and 
    is defined as 
    \begin{align}
        H(z) = \frac{Y(z)}{X(z)} = \mathcal{Z}[h[n]],
    \end{align}
    with some appropriate ROC such that $\text{ROC}_{Y(z)} = \text{ROC}_{X(z)} \cap \text{ROC}_{H(z)}$, at least. 
\end{tcolorbox}
\noindent In fact, by the convolution property, it follows that 
\begin{align}
    y[n] = x[n] * h[n] \iff Y(z) = X(z)H(z).
\end{align}
\newpage
\noindent Symbolically,
\begin{center}
    \begin{tikzpicture}
        \node [] (input) at (-3,0){$x[n]$};
        \node [] (output) at (+3,0){$y[n]$};
        \node [draw,
            fill=yellow!50, 
            minimum width=2cm, 
            minimum height=1.2cm
        ] (system) at (0,0){$h[n]$};
        \draw [-latex, line width=1.5pt] (input) -- (system);
        \draw [-latex, line width=1.5pt] (system) -- (output);
    \end{tikzpicture} \\[0.25cm]
    \begin{tikzpicture}
        \node at (0,0) [double arrow, draw=black, top color=red, bottom color=blue,
            minimum width = 15pt, double arrow head extend=5pt,
            minimum height=12mm,
            rotate=90] {};
    \end{tikzpicture} \\[0.25cm]
    \begin{tikzpicture}
        \node [] (input) at (-3,0){$X(z)$};
        \node [] (output) at (+3,0){$Y(z)$};
        \node [draw,
            fill=yellow!50, 
            minimum width=2cm, 
            minimum height=1.2cm
        ] (system) at (0,0){$H(z)$};
        \draw [-latex, line width=1.5pt] (input) -- (system);
        \draw [-latex, line width=1.5pt] (system) -- (output);
    \end{tikzpicture}
\end{center}
As the impulse response is the system response to input $x[n]=\delta[n]$, consider the $z$--transform of the impulse signal $\mathcal{Z}[\delta[n]]=1$. Then it also follows that 
\begin{center}
    \begin{tikzpicture}
        \node [] (input) at (-3,0){$x[n]=\delta[n]$};
        \node [] (output) at (+3,0){$y[n]=h[n]$};
        \node [draw,
            fill=yellow!50, 
            minimum width=2cm, 
            minimum height=1.2cm
        ] (system) at (0,0){$h[n]$};
        \draw [-latex, line width=1.5pt] (input) -- (system);
        \draw [-latex, line width=1.5pt] (system) -- (output);
    \end{tikzpicture} \\[0.25cm]
    \begin{tikzpicture}
        \node at (0,0) [double arrow, draw=black, top color=red, bottom color=blue,
            minimum width = 15pt, double arrow head extend=5pt,
            minimum height=12mm,
            rotate=90] {};
    \end{tikzpicture} \\[0.25cm]
    \begin{tikzpicture}
        \node [] (input) at (-3,0){$X(z)=1$};
        \node [] (output) at (+3,0){$Y(z)=H(z)$};
        \node [draw,
            fill=yellow!50, 
            minimum width=2cm, 
            minimum height=1.2cm
        ] (system) at (0,0){$H(z)$};
        \draw [-latex, line width=1.5pt] (input) -- (system);
        \draw [-latex, line width=1.5pt] (system) -- (output);
    \end{tikzpicture}
\end{center}

\begin{tcolorbox}[width=\textwidth,colback={white}, sharp corners]
    To find the transfer function of a DT LTI system, there are two approaches:
    \begin{enumerate}
        \item Find $X(z),Y(z)$. Then calculate $H(z)=Y(z)/X(z)$.
        \item Find $X(z),Y(z)$. Then substitute $X(z)=1$ and $Y(z)=H(z)$.
    \end{enumerate}
\end{tcolorbox}
While setting initial conditions to zero matters for the CT transfer function, the DT transfer function 
can handle nonzero initial conditions due to the two-sidedness of the $z$--transform as well as the 
DT application of being able to process offline. 

\subsection{System Duration}
If $H(z)$ is a non-everlasting polynomial in $z^{-1}$ or is a rational function that can be simplified to a non-everlasting polynomial in $z^{-1}$, 
then $H(z)$ is said to be \emph{finite}. Finite-duration systems have a finite number of nonzero terms in its impulse response. Typically, we are more 
interested in expressing finite-duration transfer functions as a length-$N$ polynomial in the form
\begin{align}
    H(z) = A_mz^{-m} + A_{m+1}z^{-(m+1)} + \cdots + A_{m+N-1}z^{-(m+N-1)}; \text{ for } m\in\mathbb{Z} \text{ and } N\in\mathbb{Z}^+.
\end{align}
Conversely, if $H(z)$ is an everlasting polynomial in $z^{-1}$ or is a rational function that can only be simplified to an everlasting polynomial in $z^{1}$, 
then it is said to be \emph{infinite}. Infinite-duration systems have infinite nonzero terms in its impulse response. Typically, we are more interested in expressing 
infinite-duration transfer functions in the partial fraction expansion form 
\begin{align}
    H(z) = \frac{A_1}{1-p_1z^{-1}} + \frac{A_2}{1-p_2z^{-1}} + \cdots + \frac{A_N}{1-p_Nz^{-1}},
\end{align}
where $p_k$ are poles of the system.

\subsection{System Causality}
Depending on the system duration, there are different ways to address the causality of the system. From Table \ref{causality}, we can see that while one can simply 
look at the $z$-terms of the transfer function for finite-duration systems, the ROC is more important for infinite-duration systems.

\begin{table}[hbt!]
    \centering
    \caption{Causality of DT systems based on transfer function.}
    \label{causality}
    \begin{tabular}{|c|c|c|}
        \hline
        System type & Finite-duration & Infinite-duration \\[0.1cm]
        \hline
        & & \\
        Causal & \Centerstack{Contains only \\negative powers of $z$} & \Centerstack{ROC: $|z|>\max|p_k|$, \\for poles $p_k=\{p_1,p_2,...,p_N\}$} \\[0.5cm]
        Anticausal & \Centerstack{Contains only \\positive powers of $z$} & \Centerstack{ROC: $|z|<\min|p_k|$, \\for poles $p_k=\{p_1,p_2,...,p_N\}$} \\[0.5cm]
        \Centerstack{Noncausal\\(two-sided)} & \Centerstack{Contains both \\negative and \\positive powers of $z$} & \Centerstack{ROC: $|p_1|<|z|<|p_2|$, \\for poles $p_k=\{p_1,p_2\}$} \\[0.5cm]
        \hline
    \end{tabular}
\end{table}

\subsection{System Stability}
A DT system is said to be \emph{stable} if the ROC of its transfer function contains the unit circle $|z|=1$. Since the ROC of finite-duration systems is 
nearly the entire $z$-plane (which obviously includes the unit circle), finite-duration DT systems will always be stable. 
\\ \\
While the ROC is necessary for infinite-duration DT systems, there is a quick method to analyzing causal and anticausal systems just by looking at the poles. 
If an infinite-duration DT system is causal, then the system can only be stable if the largest pole in magnitude is less than 1. That is, 
\begin{align}
    \text{Infinite and causal: } \max|p_k| < 1 \Longrightarrow \text{stable}.
\end{align}
Conversely, if the infinite-duration DT system is anticausal, then 
the system can only be stable if the smallest pole in magnitude is greater than 1. That is,
\begin{align}
    \text{Infinite and anticausal: } \min|p_k| > 1 \Longrightarrow \text{stable}.
\end{align}
If the infinite-duration DT system is noncausal and contains only two poles, then one pole must be greater than 1 in magnitude while the other pole is less than 1 in magnitude. That is, 
\begin{align}
    \text{Infinite and noncausal with only two poles: } |p_1| < 1 < |p_2| \Longrightarrow \text{stable}.
\end{align}
Unfortunately, if there are more than two poles for an infinite-duration noncausal DT system, then some more ROC analysis is needed.
\\ \\
Regardless, we typically want to choose an appropriate ROC such that the DT system is stable. 

\pagebreak
\begin{example}
    Given that a DT system generates an output $Y(z)$ from input $X(z)$, both given by, 
    \begin{align*}
        X(z) &= \frac{2z(z-\frac{10}{3})}{(z-\frac{1}{3})(z-3)}, \text{ for } \frac{1}{3}<|z|< 3, \\
        Y(z) &= \frac{-\frac{1}{4}z}{(z-\frac{1}{3})(z-\frac{1}{4})}, \text{ for } |z|>\frac{1}{3},
    \end{align*}
    find the impulse response of such a system. Then characterize the system.
\end{example}
\begin{solution}
    First, we find the transfer function.
    \begin{align*}
        H(z) = \frac{Y(z)}{X(z)} = \frac{-\frac{1}{4}z}{(z-\frac{1}{3})(z-\frac{1}{4})} \frac{(z-\frac{1}{3})(z-3)}{2z(z-\frac{10}{3})} = \frac{-\frac{1}{8}(z-3)}{(z-\frac{1}{4})(z-\frac{10}{3})}.
    \end{align*}
    Since $H(z)$ cannot be further simplified to a non-everlasting polynomial in $z^{-1}$, the system is of infinite duration. From the transfer function, we see $H(z)$ has two poles 
    at $z=\frac{1}{4}$ and $z=\frac{10}{3}$. Now considering causality, there are three possible ROCs to consider.
    \begin{align*}
        \text{Causal: } & |z| > \max|p_k| = \frac{10}{3} \\
        \text{Anticausal: } & |z| < \min|p_k| = \frac{1}{4} \\
        \text{Noncausal: } & \frac{1}{4} < |z| < \frac{10}{3}
    \end{align*}
    Only one of the three ROCs above actually overlaps with the ROC for $X(z)$, and that is the noncausal option. As a sanity check,
    \begin{align*}
        \text{ROC}_{Y(z)} = \text{ROC}_{X(z)} \cap \text{ROC}_{H(z)} &= \left(\frac{1}{3}<|z|< 3\right) \cap \left(\frac{1}{4}<|z|<\frac{10}{3}\right) \\
        &= \left(\frac{1}{3}<|z|< 3\right), \text{ at least.}
    \end{align*}
    We make an ``at least'' remark since there is a pole-zero cancellation at $z=3$ when multiplying $X(z)H(z)=Y(z)$, leaving $Y(z)$ with no pole at $z=3$ and hence a viable ROC of $|z|>\frac{1}{3}$.
    \\ \\
    Now we perform partial fraction expansion such that
    \begin{align*}
        H(z) &= \frac{-\frac{1}{8}(z-3)}{(z-\frac{1}{4})(z-\frac{10}{3})} = \left[\frac{A_1}{z-\frac{1}{4}} + \frac{A_2}{z-\frac{10}{3}}\right] \\
        A_1 &= \frac{-\frac{1}{8}(z-3)}{(z-\frac{10}{3})}\bigg|_{z=1/4} = -\frac{33}{296} \\
        A_2 &= \frac{-\frac{1}{8}(z-3)}{(z-\frac{1}{4})}\bigg|_{z=10/3} = -\frac{1}{74}.
    \end{align*}
    Finally, using ROC $\frac{1}{4}<|z|<\frac{10}{3}$, with $|z|>\frac{1}{4}$ contributing the causal part and $|z|<\frac{10}{3}$ contributing the anticausal part, we get the impulse response 
    \begin{align*}
        h[n] = -\frac{33}{296}\left(\frac{1}{4}\right)^n u[n] + \frac{1}{74}\left(\frac{10}{3}\right)^n u[-n-1].
    \end{align*}
    The DT system characterized by $h[n]$ and $H(z)$ can be described as 
    \begin{itemize}
        \item of infinite duration (rational function $H(z)$ cannot be further reduced), 
        \item noncausal (ROC is an annulus), and
        \item stable (ROC contains the unit circle).
    \end{itemize}
\end{solution}

\begin{example}
    Suppose a causal DT LTI system has the transfer function
    \begin{align*}
        H(z) = \frac{(z-1)(z-6)}{(z-2)(z-3)}.
    \end{align*}
    Determine if the system is stable. Then find the impulse response and difference equation which characterizes the system. 
\end{example}
\begin{solution}
    Since the system is causal, we can check if the largest pole in magnitude is less that 1. Since 
    \begin{align*}
        \max|p_k| = 3 \nless 1,
    \end{align*}
    the system is not stable. To get the difference equation with causality in mind, we rewrite the rational function such that
    \begin{align*}
        H(z) = H^+(z) = \frac{(z-1)(z-6)}{(z-2)(z-3)} = \frac{z^2-7z+6}{z^2-5z+6} &= \frac{1-7z^{-1}+6z^{-2}}{1-5z^{-1}+6z^{-2}} = \frac{Y^+(z)}{X^+(z)} \\
        \Longrightarrow (1-5z^{-1}+6z^{-2})Y^+(z) &= 1-7z^{-1}+6z^{-2}X^+(z) \\[0.25cm]
        \Longrightarrow y[n]-5y[n-1]+6y[n-2] &= x[n]-7x[n-1]+6x[n-2].
    \end{align*}
    Lastly, to get the impulse response, we perform long division and partial fraction expansion such that 
    \begin{align*}
        H^+(z) = \frac{z^2-7z+6}{z^2-5z+6} &= 1 + \frac{-2z}{z^2-5z+6} \\ 
        &= 1 + \frac{4}{z-2} - \frac{6}{z-3} \\
        &= 1 + \frac{4z^{-1}}{1-2z^{-1}} - \frac{6z^{-1}}{1-3z^{-1}} \\[0.25cm]
        \Longrightarrow h[n] &= \delta[n] + 4\cdot(2)^{n-1} u[n-1] - 6\cdot(3)^{n-1} u[n-1].
    \end{align*}
\end{solution}

\subsection{Invertible Systems}
As previously introduced, a system is invertible if there exists an inverse system that maps an output signal $y[n]$ back to its input $x[n]$. In the DT LTI case, 
an inverse system characterized by $g[n]$ exists such that 
\begin{center}
    \begin{tikzpicture}
        \node [] (input) at (-3,0){$x[n]$};
        \node [] (output) at (+6,0){$x[n]$};
        \node [draw,
            fill=yellow!50, 
            minimum width=2cm, 
            minimum height=1.2cm
        ] (fwd) at (0,0){$h[n]$};
        \node [draw,
            fill=blue!30, 
            minimum width=2cm, 
            minimum height=1.2cm
        ] (inv) at (3,0){$g[n]$};
        \draw [-latex, line width=1.5pt] (input) -- (fwd);
        \draw [-latex, line width=1.5pt] (fwd) -- (inv);
        \draw [-latex, line width=1.5pt] (inv) -- (output);
    \end{tikzpicture}
\end{center}
\begin{center}
    \begin{tikzpicture}
        \node [] (input) at (-3,0){$y[n]$};
        \node [] (output) at (+6,0){$y[n]$};
        \node [draw,
            fill=blue!30, 
            minimum width=2cm, 
            minimum height=1.2cm
        ] (inv) at (0,0){$g[n]$};
        \node [draw,
            fill=yellow!50, 
            minimum width=2cm, 
            minimum height=1.2cm
        ] (fwd) at (3,0){$h[n]$};
        \draw [-latex, line width=1.5pt] (input) -- (inv);
        \draw [-latex, line width=1.5pt] (inv) -- (fwd);
        \draw [-latex, line width=1.5pt] (fwd) -- (output);
    \end{tikzpicture}
\end{center}
While an inverse system $g[n]$ may exist, it does not necessarily mean it is stable. In fact, for 
\begin{align}
    G^+(z) = \frac{1}{H^+(z)},
\end{align}
in order for a causal and stable DT LTI system $H^+(z)$ to have a causal and stable inverse DT LTI system $G^+(z)$, the transfer function $H^+(z)$ 
must be a proper rational function (i.e., $\deg[N(z)]=\deg[D(z)]$) with both its poles and zeros all residing inside the unit circle on the $z$-plane. 
Such a system is called a \emph{minimum phase system}. 
\\ \\
In the event that it is not possible to have a stable inverse system that is causal, one can force its causal state by applying a delay to the system, just 
enough to create an \emph{auxiliary transfer function of the inverse system} which functions like a causal and stable inverse system, albeit delayed.

\begin{example}
    Find a causal and stable inverse of the system characterized by difference equation 
    \begin{align*}
        y[n] - 0.5y[n-1] = 0.5x[n-1].
    \end{align*}
\end{example}
\begin{solution}
    Converting to the $z$-domain, we get 
    \begin{align*}
        (1-0.5z^{-1})Y(z) &= 0.5z^{-1}X(z) \\
        \Longrightarrow H(z) = \frac{Y(z)}{X(z)} &= \frac{0.5z^{-1}}{1-0.5z^{-1}}
    \end{align*}
    Taking the inverse, we get
    \begin{align*}
        G(z) = \frac{1}{H(z)} = \frac{1-0.5z^{-1}}{0.5z^{-1}} = 2z - 1 &= \frac{Y_G(z)}{X_G(z)} \\
        \Longrightarrow Y_G(z) &= (2z - 1)X_G(z)
    \end{align*}
    Here we see that $G(z)$ is of finite duration and is stable. Now we convert back to the DT domain to get the difference equation 
    \begin{align*}
        y[n] = 2x[n+1] - x[n].
    \end{align*}
    Note that the inverse system above is stable but not causal. In order to force causality, we can delay the system by setting $m=n+1$ such that 
    \begin{align*}
        y[m-1] = 2x[m] - x[m-1],
    \end{align*}
    which now has an auxiliary transfer function 
    \begin{align*}
        \tilde{G}(z) = z^{-1}G(z) = z^{-1}(2z - 1) = 2 - z^{-1}.
    \end{align*}
\end{solution}

\section{System Response Partitions}
Given some system response $y[n]$, there are three different ways to partition the response, with each partition describing a particular 
characteristic of the system response:
\begin{enumerate}
    \item Zero-state / zero-input partition
    \item Natural / forced partition
    \item Transient / steady-state partition
\end{enumerate}
\begin{tcolorbox}[width=\textwidth,colback={white}, sharp corners]
    \textbf{\underline{Type 1: Zero-state / zero-input partition.}} \\[0.25cm]
    As the names imply,
    \begin{itemize}
        \item the \emph{zero-state response} (ZSR) is the system response when there are no initial conditions, and
        \item the \emph{zero-input response} (ZIR) is the system response when there is no input signal (but there are initial conditions).
    \end{itemize}
\end{tcolorbox}
It follows that 
\begin{align}
    y[n] = y_{ZSR}[n] + y_{ZIR}[n].
\end{align}
Any geometric terms $a^n$ in the zero-input response $y_{ZIR}[n]$ are referred to as the \emph{characteristic modes} (also called \emph{natural modes}, or simply modes) 
of the system. The zero-input response is comprised of only natural modes, whereas the zero-state response has a mix of natural modes and ``forced'' modes from the input excitation function $x[n]$.
\begin{tcolorbox}[width=\textwidth,colback={white}, sharp corners]
    \textbf{\underline{Type 2: Natural / forced partition.}} \\[0.25cm]
    On the discussion of natural modes and forced modes,
    \begin{itemize}
        \item the \emph{natural response} is the part of the system response with only natural modes which describe the natural identity of the system, and
        \item the \emph{forced response} is the part of the system response with only forced modes which mimic the character of the excitation function $x[n]$.
    \end{itemize}
\end{tcolorbox}
\noindent It follows that 
\begin{align}
    y[n] = y_{nat}[n] + y_{forc}[n].
\end{align}
The natural response is closely related to the zero-input response; since the zero-input response is composed exclusively of modes, we can use it to identify the modes, 
then collect all terms with those modes to find the natural response. Naturally, the characteristic modes can also be determined by finding the poles of the system.
\begin{tcolorbox}[width=\textwidth,colback={white}, sharp corners]
    \textbf{\underline{Type 3: Transient / steady-state partition.}} \\[0.25cm]
    The end behavior of the terms of the system response can be analyzed such that
    \begin{itemize}
        \item the \emph{transient response} is the part of the system response that decays to zero as $n\to\infty$, and
        \item the \emph{steady-state response} is the part of the system response that remains after the transient response goes to zero.
    \end{itemize}
\end{tcolorbox}
\noindent It follows that 
\begin{align}
    y[n] = y_{tr}[n] + y_{ss}[n].
\end{align}

\begin{example}
    Suppose an input signal $x[n]=10\cos(\pi n/4)u[n]$ is inputted into a system characterized by the difference equation 
    \begin{align*}
        y[n] = 0.5y[n-1] + x[n].
    \end{align*}
    The system is initially at rest. Identify all six partitioned responses of the system response.
\end{example}
\begin{solution}
    First, recognizing that we are dealing with causal signals and systems, it follows that 
    \begin{align*}
        H^+(z) &= \frac{Y^+(z)}{X^+(z)} = \frac{1}{1-0.5z^{-1}}, \\
        X^+(z) &= \mathcal{Z}^+[x[n]] = \frac{10(1-(1/\sqrt{2})z^{-1})}{1-\sqrt{2}z^{-1}+z^{-2}} = \frac{10(1-(1/\sqrt{2})z^{-1})}{(1-e^{+j\pi/4}z^{-1})(1-e^{-j\pi/4}z^{-1})}.
    \end{align*}
    Then it follows that 
    \begin{align*}
        Y^+(z) = X^+(z)H^+(z) &= \frac{10(1-(1/\sqrt{2})z^{-1})}{(1-0.5z^{-1})(1-e^{+j\pi/4}z^{-1})(1-e^{-j\pi/4}z^{-1})} \\
        &= \frac{6.3}{(1-0.5z^{-1})} + \frac{6.78e^{-j28.7^{\circ}}}{1-e^{+j\pi/4}z^{-1}} + \frac{6.78e^{+j28.7^{\circ}}}{1-e^{-j\pi/4}z^{-1}} \\
        \Longrightarrow y[n] &= 6.3(0.5)^n u[n] + 13.56\cos\left(\frac{\pi n}{4}-28.7^{\circ}\right)u[n].
    \end{align*}
    To find the modes, first solve for $y_{ZIR}[n]$. Setting $x[n]=0$, we get 
    \begin{align*}
        y_{ZIR}[n] &= 0 \\
        y_{ZSR}[n] &= y[n] - y_{ZIR}[n] = y[n] = 6.3(0.5)^n u[n] + 13.56\cos\left(\frac{\pi n}{4}-28.7^{\circ}\right)u[n].
    \end{align*}
    From $X^+(z)$, the forced modes are given by geometric term $(0.5)^n$. Therefore,
    \begin{align*}
        y_{forc}[n] &= 6.3(0.5)^n u[n] \\
        y_{nat}[n] &= y[n] - y_{forc}[n] = 13.56\cos\left(\frac{\pi n}{4}-28.7^{\circ}\right)u[n].
    \end{align*}
    Lastly, we can group decaying terms together and the remaining terms together such that
    \begin{align*}
        y_{tr}[n] &= 6.3(0.5)^n u[n] \\
        y_{ss}[n] &= y[n] - y_{tr}[n] = 13.56\cos\left(\frac{\pi n}{4}-28.7^{\circ}\right)u[n].
    \end{align*}
\end{solution}

\section{Sample-Rate Conversion}
Recall that a DT signal is simply a sampled signal with sampling frequency $f_s$ such that 
\begin{align}
    x[n] = x(nT_s) = x\left(\frac{n}{f_s}\right).
\end{align}
Suppose 
\begin{align}
    x_1[n] &= x\left(\frac{n}{f_1}\right) \\
    x_2[n] &= x\left(\frac{n}{f_2}\right),
\end{align}
where $f_1$ and $f_2$ are two different sampling frequencies. \emph{Sample-rate conversion} is 
the process of attaining $x_2[n]$ from $x_1[n]$ in the discrete-time domain. That is, it aims to 
change the sampling rate from $f_1$ to $f_2$ without need to recollect data.
\\ \\
There are two main types of rate conversions: \emph{upsampling} and \emph{downsampling}. All other rate 
conversions are a mix of the two. 

\subsection{Upsampling}
\emph{Upsampling} increases the sampling rate of a DT signal $x[n]$ by some integer factor $L$. It creates a rough estimate of 
what would have been obtained had the original signal $x(t)$ been sampled at rate $(L\cdot f_s)$. Analagous to 
a time expansion in the CT case, upsampling essentially stuffs zeros between data points in a DT sequence. This is given by 
\begin{align}
    (\uparrow L)x[n] = x_L[n] &=
    \begin{cases}
        x[n/L], & n=0,\pm L,\pm 2L,... \\
        0, & \text{else}
    \end{cases} \\
    &= \left\{...,0,\underbar{x[0]}, \underbrace{0,...,0}_{L-1 \text{ zeros}}, x[1], 0, ...\right\}
\end{align}
Let $m=n/L$. Then in the $z$-domain, it follows that
\begin{align}
    X_L(z) &= \sum_{n=-\infty}^{+\infty} x_L[n]z^{-n} = \sum_{m=-\infty}^{+\infty} x[mL]z^{-mL} = \sum_{m=-\infty}^{+\infty} x[m](z^L)^{-m} = X(z^L), \\
    & \text{with ROC } a<|z^L|<b \Longrightarrow a^{1/L}<|z|<b^{1/L} \Longrightarrow R^{1/L} \text{ for } a,b\in[0,\infty] \nonumber
\end{align}

\subsection{Downsampling}
\emph{Downsampling} decreases the sampling rate of a DT signal $x[n]$ by some integer factor $M$. It creates a rough estimate of 
what would have been obtained had the original signal $x(t)$ been sampled at rate $(f_s/M)$. Analagous to 
a time compression in the CT case, downsampling essentially only retains every $M^{th}$ sample in a DT sequence and discards the rest. 
This is denoted as 
\begin{align}
    (\downarrow M)x[n] = x_{1/M}[n] &= x[Mn] \\
    &= \left\{...,\underbar{x[0]}, x[M], x[2M], ...\right\}.
\end{align}
Let $m=Mn$. Then in the $z$-domain, it follows that
\begin{align}
    X_{1/M}(z) = \sum_{n=-\infty}^{+\infty} x_{1/M}[n]z^{-n} &= \sum_{m=-\infty}^{+\infty} x[m/M]z^{-m/M} \nonumber \\
    &= \sum_{m=-\infty}^{+\infty} x[m]z^{-m/M}\mathbbm{1}_{m=kM}[m] \nonumber \\
    &= \sum_{m=-\infty}^{+\infty} x[m]z^{-m/M}\left(\frac{1}{M}\sum_{r=0}^{M-1}e^{j(2\pi m/M)r}\right) \nonumber \\
    &= \frac{1}{M} \sum_{r=0}^{M-1}\sum_{m=-\infty}^{+\infty} x[m]z^{-m/M}e^{j(2\pi m/M)r} \nonumber \\
    &= \frac{1}{M} \sum_{r=0}^{M-1}\sum_{m=-\infty}^{+\infty} x[m]\left(z^{1/M}e^{-j2\pi r/M}\right)^{-m} \nonumber \\
    &= \frac{1}{M} \sum_{r=0}^{M-1} X\left(z^{1/M}e^{-j2\pi r/M}\right) \\
    \text{with ROC } a<\left|z^{1/M}e^{-j2\pi r/M}\right|<b &\Longrightarrow a<|z^{1/M}|<b \nonumber \\
    &\Longrightarrow a^M<|z|<b^M \Longrightarrow R^M \text{ for } a,b\in[0,\infty] \nonumber
\end{align}

\subsection{Resampling}
\emph{Resampling} uses both upsampling and downsampling to change the sampling rate by some rational factor $L/M$. As shown in the block diagram below, it 
requires some lowpass filter in-between upsampling and downsampling as a measure to interpolate the upsampled sequence as well as prevent aliasing when downsampling.

\begin{center}
    \begin{tikzpicture}
        \node [] (input) at (-3,0){$x[n]$};
        \node [] (output) at (+9,0){$x_{L/M}[n]$};
        \node [draw,
            fill=yellow!50, 
            minimum width=2cm, 
            minimum height=1.2cm
        ] (up) at (0,0){$\uparrow L$};
        \node [draw,
            fill=violet!30, 
            minimum width=2cm, 
            minimum height=1.2cm
        ] (lpf) at (3,0){\shortstack{Lowpass\\Filter}};
        \node [draw,
            fill=red!30, 
            minimum width=2cm, 
            minimum height=1.2cm
        ] (down) at (6,0){$\downarrow M$};
        \draw [-latex, line width=1.5pt] (input) -- (up);
        \draw [-latex, line width=1.5pt] (up) -- (lpf);
        \draw [-latex, line width=1.5pt] (lpf) -- (down);
        \draw [-latex, line width=1.5pt] (down) -- (output);
    \end{tikzpicture}
\end{center}
\pagebreak
Some common resampling factors $L/M$ include the following:
\begin{center}
    \begin{tikzpicture}
        \node [] (output) at (+9,0){\shortstack{Audio format\\conversion}};
        \node [draw,
            minimum width=2cm, 
            minimum height=1.2cm
        ] (b1) at (0,0){\shortstack{CD\\44.1 [kHz]}};
        \node [draw,
            minimum width=2cm, 
            minimum height=1.2cm
        ] (b2) at (3,0){$\dfrac{L}{M}=\dfrac{48}{44.1}$};
        \node [draw,
            minimum width=2cm, 
            minimum height=1.2cm
        ] (b3) at (6,0){\shortstack{DAT\\48 [kHz]}};
        \draw [-latex, line width=1.5pt] (b1) -- (b2);
        \draw [-latex, line width=1.5pt] (b2) -- (b3);
    \end{tikzpicture}
\end{center}
\begin{center}
    \begin{tikzpicture}
        \node [] (output) at (+9,0){\shortstack{Speech\\compression}};
        \node [draw,
            minimum width=2cm, 
            minimum height=1.2cm
        ] (b1) at (0,0){\shortstack{Speech\\on DAT\\48 [kHz]}};
        \node [draw,
            minimum width=2cm, 
            minimum height=1.2cm
        ] (b2) at (3,0){$\dfrac{L}{M}=\dfrac{1}{6}$};
        \node [draw,
            minimum width=2cm, 
            minimum height=1.2cm
        ] (b3) at (6,0){\shortstack{Speech\\for phone\\8 [kHz]}};
        \draw [-latex, line width=1.5pt] (b1) -- (b2);
        \draw [-latex, line width=1.5pt] (b2) -- (b3);
    \end{tikzpicture}
\end{center}
\begin{center}
    \begin{tikzpicture}
        \node [] (output) at (+9,0){\shortstack{Video format\\conversion}};
        \node [draw,
            minimum width=2cm, 
            minimum height=1.2cm
        ] (b1) at (0,0){\shortstack{Film\\24 [fps]}};
        \node [draw,
            minimum width=2cm, 
            minimum height=1.2cm
        ] (b2) at (3,0){$\dfrac{L}{M}=\dfrac{5}{4}$};
        \node [draw,
            minimum width=2cm, 
            minimum height=1.2cm
        ] (b3) at (6,0){\shortstack{TV\\30 [fps]}};
        \draw [-latex, line width=1.5pt] (b1) -- (b2);
        \draw [-latex, line width=1.5pt] (b2) -- (b3);
    \end{tikzpicture}
\end{center}
More on sample-rate conversion will be discussed in the next chapter.

\end{document}