\documentclass{report}
% PACKAGES
\usepackage{adjustbox}
\usepackage{amsmath}
\usepackage{amssymb}
\usepackage{bodegraph}
\usepackage{bbm}
\usepackage{circledsteps}
\usepackage{circuitikz}
\usepackage{enumerate}
\usepackage{mathtools}
\usepackage{nicematrix}
\usepackage{pgfplots}
\usepackage{polynom}
\usepackage{qtree}
\usepackage{rotating}
\usepackage[usestackEOL]{stackengine}
\usepackage[subpreambles=true]{standalone}
\usepackage{steinmetz}
\usepackage{subcaption}
\usepackage{tabularray}
\usepackage{tcolorbox}
\usepackage{tikz}
\usepackage{xcolor}

\usepackage[colorlinks=true,linkcolor=blue,urlcolor=black,bookmarksopen=true]{hyperref}
\usepackage{bookmark}

\usetikzlibrary{shapes.arrows}
\usetikzlibrary{shapes.misc}
\usetikzlibrary{backgrounds}
\tikzset{cross/.style={cross out, draw=black, minimum size=2*(#1-\pgflinewidth), inner sep=0pt, outer sep=0pt},
%default radius will be 1pt. 
cross/.default={1pt}}

\renewcommand{\Re}{\operatorname{Re}}
\renewcommand{\Im}{\operatorname{Im}}

\usepackage{pifont}
\newcommand{\cmark}{\text{\ding{51}}}
\newcommand{\xmark}{\text{\ding{55}}}

\newcommand{\circconv}[1]{\text{ \small\Circled{#1} }}

\newcommand{\tikzmark}[3][]{\tikz[remember picture,baseline] \node [anchor=base,#1](#2) {$#3$};}

\pgfplotsset{compat=1.18}
\pgfplotsset{
    dirac/.style={
        mark=triangle*,
        mark options={scale=1.5},
        ycomb,
        scatter,
        visualization depends on={y/abs(y)-1 \as \sign},
        scatter/@pre marker code/.code={\scope[rotate=90*\sign,yshift=-2pt]}
    }
}

\usepackage[letterpaper, portrait, margin=1.25in]{geometry}
\usepackage[font=bf]{caption}
\hbadness = 10000
\hfuzz=2pt

\newtheorem{theorem}{Theorem}[chapter]

\tikzset{every tree node/.style={anchor=north,align=center}}
\usetikzlibrary{decorations.markings}
\usetikzlibrary{arrows}
\definecolor{darkgreen}{rgb}{0.133,0.545,0.133}
\tikzstyle{n}= [circle, fill=blue, minimum size=4pt,inner sep=0pt, outer sep=0pt]

\patchcmd{\thebibliography}{\chapter*}{\section*}{}{}
\setcounter{secnumdepth}{5}

%%%%%%%%%%% EXAMPLE ENVIRONMENT %%%%%%%%%%
\usepackage{calc}
\usepackage{tabto}
\usepackage[framemethod=tikz]{mdframed} % for colored backgrounds
\newcommand{\halmos}{} % makes a box at the end

\newlength{\framedinnerleftmargin}
\newlength{\framedinnertopmargin}
\newlength{\framedreversedinnerleftmargin}
\setlength{\framedinnerleftmargin}{\widthof{Theoreme 10.10.10}+2em}
\setlength{\framedreversedinnerleftmargin}{\widthof{Theoreme 10.10.10}+1em}
\setlength{\framedinnertopmargin}{1em}
 
% first argument: label in upper left corner,
% second argument: background color
\newenvironment{boxedtext}[2]{\begin{mdframed}[%
hidealllines=true,%
backgroundcolor=#2,%
innertopmargin=\framedinnertopmargin,%
innerleftmargin=\framedinnerleftmargin,%
innerrightmargin=1em%
]%
\tabto{-\framedreversedinnerleftmargin}\textbf{#1}\tabto*{0em}%
}% begin code
{\hskip 0pt\\\hspace*{\fill}\halmos{}\end{mdframed}\vspace{1em}} % end code
 
\newenvironment{summary}[0]{\begin{center}\begin{minipage}[c]{\summarywidth}\begin{spacing}{0.9}\footnotesize} % begin code
{\end{spacing}\end{minipage}\end{center}} % end code

\newcounter{example}
 
% optional! if you want it to start at zero
% with every new chapter/section/etc.
\numberwithin{example}{section}
 
\newenvironment{example}[0]
{\refstepcounter{example}\vspace{1em plus 1em}\begin{boxedtext}{Example \theexample.}{blue!7}}%\setlength{\parskip}{0em}}
{\end{boxedtext}\vspace{-1em plus 1em}}
 
\newenvironment{example*}[0]
{\vspace{1em plus 1em}\begin{boxedtext}{Example.}{blue!7}}
{\end{boxedtext}\vspace{-1em plus 1em}}

\newmdenv[
  topline=false,
  bottomline=false,
  rightline=false,
  skipabove=\topsep,
  skipbelow=\topsep,
  linecolor=purple,
  frametitle={\noindent\textcolor{purple}{\textbf{SOLUTION }}},
  endinnercode={$\hfill\textcolor{purple}{\blacksquare}$}
]{solution}

%%%%%%%%%%%%%% COLOR BOXED %%%%%%%%%%%%%%%
% Syntax: \colorboxed[<color model>]{<color specification>}{<math formula>}
\newcommand*{\colorboxed}{}
\def\colorboxed#1#{%
  \colorboxedAux{#1}%
}
\newcommand*{\colorboxedAux}[3]{%
  % #1: optional argument for color model
  % #2: color specification
  % #3: formula
  \begingroup
    \colorlet{cb@saved}{.}%
    \color#1{#2}%
    \boxed{%
      \color{cb@saved}%
      #3%
    }%
  \endgroup
}
%%%%%%%%%%%%%%%%%%%%%%%%%%%%%%%%%%%%%%%%%

\begin{document}
\setcounter{chapter}{5}
\setcounter{page}{66}
\chapter{Discrete-Time Fourier Series}

As with the CT Fourier series, there exists a discrete-time Fourier series (DTFS) representation for periodic DT signals,  
as part of a collection of discrete-time \emph{Fourier analysis} techniques. Of course, while the DTFS is applicable for periodic DT signals, 
the DTFT is applicable for any DT signal regardless of periodicity or causality. 
\\ \\
However, understanding how the DTFS works is a precursor to understanding the third DT Fourier analysis technique: the discrete Fourier transform (DFT).
Before delving into the DFT, we briefly examine the DTFS.

\section{Discrete-Time Fourier Series}
Fourier's theorem can be applied to the discrete-time domain in the sense that any periodic DT signal $x[n]$ can be represented as a finite sum of DT sinusoids. 
Specifically, given that the period of $x[n]$ is $N_0$ and that sinusoids are closely related to complex exponentials, the signal can be expressed as a 
sum of $N_0$ complex exponentials.

\begin{tcolorbox}[width=\textwidth,colback={white}, sharp corners]
The \emph{discrete-time Fourier series} (DTFS) of a $N_0$-periodic DT signal $x[n]$ for $N_0=2\pi/\Omega_0$ is given by
    \begin{align}
        x[n] = \sum_{k=0}^{N_0-1} x_k e^{jk\Omega_0 n},
    \end{align}
    where the \emph{Fourier coefficients} are calculated from the summation: 
    \begin{align}
        x_k = \frac{1}{N_0} \sum_{n=0}^{N_0-1} x[n] e^{-jk\Omega_0 n}.
    \end{align}
\end{tcolorbox}
\noindent Additionally, from the equation for calculating Fourier coefficients, since $x[n]e^{-jk\Omega_0 n}$ is $N_0$-periodic, the coefficients $x_k$ themselves must also be $N_0$-periodic, 
despite only needing one period to compute such coefficients. That is,
\begin{align}
    x_k = x_{k+mN_0}, \text{ for } m\in\mathbb{Z}.
\end{align}
For $x_k=|x_k|\exp(j\phi_k)$, the magnitudes $|x_k|$ and phases $\phi_k$ can be collected and plotted against their corresponding harmonic frequencies 
on a \emph{magnitude spectrum} and a \emph{two-sided phase spectrum}, respectively. Often the two spectra together are collectively referred to as the 
\emph{two-sided Fourier spectrum}. 
\\ \\
As with the DTFT, since periodic DT signals are sequences, there is only one single necessary Dirichlet condition for the DTFS. It follows that if $N_0$-periodic signal $x[n]$ is absolutely summable 
over a period such that 
\begin{align}
    \sum_{n=0}^{N_0-1} |x[n]| < \infty,
\end{align}
then $x[n]$ has a DTFS. Unless the sequence contains the value $\pm\infty$ at a certain value of $n$, the sequence will almost always converge absolutely. Note here that the condition 
is necessary, not sufficient. The implication is that the DTFS applies to strictly DT power signals. 
\\ \\
Properties of the DTFS are outlined in Table \ref{dtfs_prop}.

\begin{table}[hbt!]
    \centering
    \caption{Properties of the DTFS.}
    \label{dtfs_prop}
    \begin{tabular}{|c|c|c|}
        \hline
        Property & Periodic signal, $x[n]$ & Fourier coefficients, $x_k$ \\[0.15cm]
        \hline
        & & \\
        Superposition & $A_1x_1[n]+A_2x_2[n]$ & $A_1(x_1)_k+A_2(x_2)_k$ \\[0.5cm]
        Time shift & $x[n-n_0]$ & $e^{-jk\Omega_0 n_0}x_k$ \\[0.5cm]
        Frequency shift & $e^{-jk_0\Omega_0 n}x[n]$ & $x_{k+k_0}$ \\[0.5cm]
        Time reversal & $x[-n]$ & $x_{-k}$ \\[0.5cm]
        Conjugate symmetry & $x[n]$ real & 
        $\begin{cases}
            x_{-k} = x_k^* \\
            \Re(x_k) = \Re(x_{-k}) \\
            \Im(x_k) = -\Im(x_{-k}) \\
            |x_k| = |x_{-k}| \\
            \phi_k = -\phi_{-k}
        \end{cases}$ \\[0.5cm]
         & & \\[0.25cm]
        Real and even signals & $x[n]$ real and even & $x_k$ purely real and even \\[0.5cm]
        Real and odd signals & $x[n]$ real and odd & $x_k$ purely imaginary and odd \\[0.5cm]
        \Centerstack{Even-odd decomposition \\ of real signals} & 
        $\begin{cases}
            x_e[n]=\frac{1}{2}[x[n]+x[-n]] \\
            x_o[n]=\frac{1}{2}[x[n]-x[-n]]
        \end{cases}$ & 
        $\begin{cases}
            \Re(x_k) \\
            j\Im(x_k)
        \end{cases}$ \\[0.5cm]
        \hline
    \end{tabular}
\end{table}

\pagebreak
\begin{example}
    Compute the DTFS of periodic sequence 
    \begin{align*}
        x[n] = \left\{...,\underbar{24},8,12,16,24,8,12,16,...\right\}.
    \end{align*}
\end{example}
\begin{solution}
    From visual inspection, we see that the period of $x[n]$ is $N_0=4$. Therefore, we need to find four Fourier coefficients. 
    \begin{align*}
        x_0 &= \frac{1}{4} \sum_{n=0}^{3} x[n] = \frac{1}{4}[24+8+12+16] = 15 \\
        x_1 &= \frac{1}{4} \sum_{n=0}^{3} x[n]e^{-j2\pi n/4} = \frac{1}{4}[24+8e^{-j\pi/2}+12e^{-j2\pi/2}+16e^{-j3\pi/2}] = 3.6e^{j33.7^\circ} \\
        x_2 &= \frac{1}{4} \sum_{n=0}^{3} x[n]e^{-j4\pi n/4} = \frac{1}{4}[24+8e^{-j\pi}+12e^{-j2\pi}+16e^{-j3\pi}] = 3 \\
        x_3 &= \frac{1}{4} \sum_{n=0}^{3} x[n]e^{-j6\pi n/4} = \frac{1}{4}[24+8e^{-j3\pi/2}+12e^{-j6\pi/2}+16e^{-j9\pi/2}] = 3.6e^{-j33.7^\circ}
    \end{align*}
    Then for $\Omega_0=2\pi/N_0=\pi/2$, the DTFS of $x[n]$ is 
    \begin{align*}
        x[n] = \sum_{k=0}^{N_0-1} x_k e^{jk\Omega_0 n} &= 15 + 3.6e^{j33.7^\circ}e^{j\pi n/2} + 3e^{j2\pi n/2} + 3.6e^{-j33.7^\circ}e^{j3\pi n/2} \\
        &= 15 + 3(-1)^n + 7.2\cos\left(\frac{\pi}{2}n+33.7^\circ\right).
    \end{align*}
\end{solution}

\section{Parseval's Theorem for DTFS}
Parseval's theorem for the DTFS is essentially a ``conservation of (average) power'' theorem. When periodic signals are mapped from the DTFS to the discrete-frequency Fourier spectrum, the 
total signal average power is conserved. It follows that the total average power of a DT periodic signal $x[n]$ can be evaluated as: 
\begin{align}
    P_x = \frac{1}{N_0}\sum_{n=0}^{N_0-1} |x[n]|^2 = \sum_{k=0}^{N_0-1} |x_k|^2.
\end{align}
Here, the DC power is given by $|x_0|^2$ whereas the rest of the summation is the average AC power.
\\ \\
Additionally, we can define the \emph{one-sided power spectral density} (1-sided PSD) to be 
\begin{align}
    PSD_1 = 
    \begin{cases}
        |x_0|^2, & k=0 \\
        2|x_k|^2, & k>0
    \end{cases}
\end{align}
and the \emph{two-sided power spectral density} (2-sided PSD) to be 
\begin{align}
    PSD_2 = |x_k|^2.
\end{align}
Note that the 2-sided PSD is defined for all frequencies, whereas the 1-sided PSD is defined for only nonnegative frequencies. 
Because of this, the AC component of the 2-sided PSD are half the values of the 1-sided PSD. In signal processing, we tend to be more interested in the 1-sided PSD.

\section{DT LTI Systems with DTFS}
As with the CT case, the superposition principle can be applied to find the output of a DT LTI system, especially if the input signal is a periodic DT signal which 
can be represented as a finite sum of DT complex exponentials. 

\begin{tcolorbox}[width=\textwidth,colback={white}, sharp corners]
\emph{DTFS ANALYSIS.}
    \begin{enumerate}
        \item Express the input signal $x[n]$ as a DTFS to obtain the Fourier coefficients $x_k$.
        \item Use the DTFT to find the generic frequency response function $H(e^{j\Omega})$ of the system.
        \item Compute the output Fourier coefficients $y_k = x_k H(e^{j\Omega})\big|_{\Omega=k\Omega_0}$.
        \item Determine the output signal $y[n] = \displaystyle\sum_{k=0}^{N_0-1} y_k e^{jk\Omega_0 n}$.
    \end{enumerate}
\end{tcolorbox}

\begin{example}
    Suppose a periodic sequence
    \begin{align*}
        x[n] = \left\{...,\underbar{4},-2,0,-2,4,-2,0,-2,...\right\}
    \end{align*}
    is inputted into a system characterized by impulse response
    \begin{align*}
        h[n] = \frac{\sin\left(\frac{\pi n}{3}\right)}{\pi n}.
    \end{align*}
    Find the system response $y[n]$.
\end{example}
\begin{solution}
    From visual inspection, we see that the period of $x[n]$ is $N_0=4$. Therefore, we need to find four Fourier coefficients. 
    \begin{align*}
        x_0 &= \frac{1}{4} \sum_{n=0}^{3} x[n] = \frac{1}{4}[4-2+0-2] = 0 \\
        x_1 &= \frac{1}{4} \sum_{n=0}^{3} x[n]e^{-j2\pi n/4} = \frac{1}{4}[4-2e^{-j\pi/2}+0e^{-j2\pi/2}-2e^{-j3\pi/2}] = 1 \\
        x_2 &= \frac{1}{4} \sum_{n=0}^{3} x[n]e^{-j4\pi n/4} = \frac{1}{4}[4-2e^{-j\pi}+0e^{-j2\pi}-2e^{-j3\pi}] = 2 \\
        x_3 &= \frac{1}{4} \sum_{n=0}^{3} x[n]e^{-j6\pi n/4} = \frac{1}{4}[4-2e^{-j3\pi/2}+0e^{-j6\pi/2}-2e^{-j9\pi/2}] = 1
    \end{align*}
    Using the DTFT table, we find that 
    \begin{align*}
        H(e^{j\Omega}) = \text{DTFT}[h[n]] = \sum_{k=-\infty}^{+\infty} \operatorname{rect}\left(\frac{\Omega-2\pi k}{2(\pi/3)}\right).
    \end{align*}
    Isolate one period such that the frequency response over one period is
    \begin{align*}
        H_{\langle 2\pi\rangle}(e^{j\Omega}) = \operatorname{rect}\left(\frac{\Omega}{2(\pi/3)}\right) = 
        \begin{cases}
            1, & 0\leq|\Omega|\leq\pi/3 \\
            0, & \pi/3<|\Omega|<\pi.
        \end{cases}
    \end{align*}
    Then for $\Omega_0=2\pi/N_0=\pi/2$, 
    \begin{align*}
        &H_{\langle 2\pi\rangle}(e^{j\Omega})\bigg|_{\Omega=0} = 1 \\
        &H_{\langle 2\pi\rangle}(e^{j\Omega})\bigg|_{(\Omega=\Omega_0=\pi/2)} = 0 \\
        &H_{\langle 2\pi\rangle}(e^{j\Omega})\bigg|_{(\Omega=2\Omega_0=\pi)} = 0 \\
        &H_{\langle 2\pi\rangle}(e^{j\Omega})\bigg|_{(\Omega=3\Omega_0=3\pi/2=-\pi/2)} = 0.
    \end{align*}
    The output Fourier coefficients are then 
    \begin{align*}
        y_0 &= x_0 H(e^{j\Omega})\big|_{\Omega=0} = 0 \\
        y_1 &= x_1 H(e^{j\Omega})\big|_{\Omega=\Omega_0} = 0 \\
        y_2 &= x_2 H(e^{j\Omega})\big|_{\Omega=2\Omega_0} = 0 \\
        y_3 &= x_3 H(e^{j\Omega})\big|_{\Omega=3\Omega_0} = 0,
    \end{align*}
    and the system response is thus 
    \begin{align*}
        y[n] = \sum_{k=0}^{N_0-1} y_k e^{jk\Omega_0 n} = \sum_{k=0}^{3} 0\,e^{jk (2\pi/4) n} = 0.
    \end{align*}
\end{solution}

\begin{example}
    Given the periodic sequences 
    \begin{align*}
        x[n] &= \left\{...,\underbar{4},2,1,0,4,2,1,0,...\right\} \\
        y[n] &= \left\{...,\underbar{10},4,10,4,10,4,...\right\} \\
    \end{align*}
    find $h[n]=\left\{\underbar{\emph{a}},b,c\right\}$ for $y[n]=x[n]*h[n]$.
\end{example}
\begin{solution}
    Because $h[n]$ is finite, we can avoid using the DTFS representation altogether by using the expanded tabular method for linear convolution. 
    Since $x[n]$ has a period of 4 (which is greater than the period of $h[n]$ which is 2), only 4 rows for $h[\cdot]$ are needed.
    \begin{center}
        \begin{tabular}{c|ccccccc|c}
            $n \rightarrow $ & -2 & -1 & 0 & 1 & 2 & 3 & 4 & $y[\cdot]$ \\
            \hline
            $x[n]$ & 1 & 0 & 4 & 2 & 1 & 0 & 4 & \\
            \hline
            $h[-n]$ & $c$ & $b$ & $a$ & & & & & $y[0] = c + 4a = 10$ \\
            $h[1-n]$ & & $c$ & $b$ & $a$ & & & & $y[1] = 4b + 2a = 4$ \\
            $h[2-n]$ & & & $c$ & $b$ & $a$ & & & $y[2] = 4c + 2b + a = 10$\\
            $h[3-n]$ & & & & $c$ & $b$ & $a$ & & $y[3] = 2c + b = 4$
        \end{tabular}
    \end{center}
    Then solving the system of equations from the right pane, we get 
    \begin{align*}
        a &= 2 \\
        b &= 0 \\
        c &= 2.
    \end{align*}
    Therefore, $h[n]=\left\{\underbar{2},0,2\right\}$.
\end{solution}

\end{document}