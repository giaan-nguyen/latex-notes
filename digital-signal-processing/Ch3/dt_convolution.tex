\documentclass{report}
% PACKAGES
\usepackage{adjustbox}
\usepackage{amsmath}
\usepackage{amssymb}
\usepackage{bodegraph}
\usepackage{bbm}
\usepackage{circledsteps}
\usepackage{circuitikz}
\usepackage{enumerate}
\usepackage{mathtools}
\usepackage{nicematrix}
\usepackage{pgfplots}
\usepackage{polynom}
\usepackage{qtree}
\usepackage{rotating}
\usepackage[usestackEOL]{stackengine}
\usepackage[subpreambles=true]{standalone}
\usepackage{steinmetz}
\usepackage{subcaption}
\usepackage{tabularray}
\usepackage{tcolorbox}
\usepackage{tikz}
\usepackage{xcolor}

\usepackage[colorlinks=true,linkcolor=blue,urlcolor=black,bookmarksopen=true]{hyperref}
\usepackage{bookmark}

\usetikzlibrary{shapes.arrows}
\usetikzlibrary{shapes.misc}
\usetikzlibrary{backgrounds}
\tikzset{cross/.style={cross out, draw=black, minimum size=2*(#1-\pgflinewidth), inner sep=0pt, outer sep=0pt},
%default radius will be 1pt. 
cross/.default={1pt}}

\renewcommand{\Re}{\operatorname{Re}}
\renewcommand{\Im}{\operatorname{Im}}

\usepackage{pifont}
\newcommand{\cmark}{\text{\ding{51}}}
\newcommand{\xmark}{\text{\ding{55}}}

\newcommand{\circconv}[1]{\text{ \small\Circled{#1} }}

\newcommand{\tikzmark}[3][]{\tikz[remember picture,baseline] \node [anchor=base,#1](#2) {$#3$};}

\pgfplotsset{compat=1.18}
\pgfplotsset{
    dirac/.style={
        mark=triangle*,
        mark options={scale=1.5},
        ycomb,
        scatter,
        visualization depends on={y/abs(y)-1 \as \sign},
        scatter/@pre marker code/.code={\scope[rotate=90*\sign,yshift=-2pt]}
    }
}

\usepackage[letterpaper, portrait, margin=1.25in]{geometry}
\usepackage[font=bf]{caption}
\hbadness = 10000
\hfuzz=2pt

\newtheorem{theorem}{Theorem}[chapter]

\tikzset{every tree node/.style={anchor=north,align=center}}
\usetikzlibrary{decorations.markings}
\usetikzlibrary{arrows}
\definecolor{darkgreen}{rgb}{0.133,0.545,0.133}
\tikzstyle{n}= [circle, fill=blue, minimum size=4pt,inner sep=0pt, outer sep=0pt]

\patchcmd{\thebibliography}{\chapter*}{\section*}{}{}
\setcounter{secnumdepth}{5}

%%%%%%%%%%% EXAMPLE ENVIRONMENT %%%%%%%%%%
\usepackage{calc}
\usepackage{tabto}
\usepackage[framemethod=tikz]{mdframed} % for colored backgrounds
\newcommand{\halmos}{} % makes a box at the end

\newlength{\framedinnerleftmargin}
\newlength{\framedinnertopmargin}
\newlength{\framedreversedinnerleftmargin}
\setlength{\framedinnerleftmargin}{\widthof{Theoreme 10.10.10}+2em}
\setlength{\framedreversedinnerleftmargin}{\widthof{Theoreme 10.10.10}+1em}
\setlength{\framedinnertopmargin}{1em}
 
% first argument: label in upper left corner,
% second argument: background color
\newenvironment{boxedtext}[2]{\begin{mdframed}[%
hidealllines=true,%
backgroundcolor=#2,%
innertopmargin=\framedinnertopmargin,%
innerleftmargin=\framedinnerleftmargin,%
innerrightmargin=1em%
]%
\tabto{-\framedreversedinnerleftmargin}\textbf{#1}\tabto*{0em}%
}% begin code
{\hskip 0pt\\\hspace*{\fill}\halmos{}\end{mdframed}\vspace{1em}} % end code
 
\newenvironment{summary}[0]{\begin{center}\begin{minipage}[c]{\summarywidth}\begin{spacing}{0.9}\footnotesize} % begin code
{\end{spacing}\end{minipage}\end{center}} % end code

\newcounter{example}
 
% optional! if you want it to start at zero
% with every new chapter/section/etc.
\numberwithin{example}{section}
 
\newenvironment{example}[0]
{\refstepcounter{example}\vspace{1em plus 1em}\begin{boxedtext}{Example \theexample.}{blue!7}}%\setlength{\parskip}{0em}}
{\end{boxedtext}\vspace{-1em plus 1em}}
 
\newenvironment{example*}[0]
{\vspace{1em plus 1em}\begin{boxedtext}{Example.}{blue!7}}
{\end{boxedtext}\vspace{-1em plus 1em}}

\newmdenv[
  topline=false,
  bottomline=false,
  rightline=false,
  skipabove=\topsep,
  skipbelow=\topsep,
  linecolor=purple,
  frametitle={\noindent\textcolor{purple}{\textbf{SOLUTION }}},
  endinnercode={$\hfill\textcolor{purple}{\blacksquare}$}
]{solution}

%%%%%%%%%%%%%% COLOR BOXED %%%%%%%%%%%%%%%
% Syntax: \colorboxed[<color model>]{<color specification>}{<math formula>}
\newcommand*{\colorboxed}{}
\def\colorboxed#1#{%
  \colorboxedAux{#1}%
}
\newcommand*{\colorboxedAux}[3]{%
  % #1: optional argument for color model
  % #2: color specification
  % #3: formula
  \begingroup
    \colorlet{cb@saved}{.}%
    \color#1{#2}%
    \boxed{%
      \color{cb@saved}%
      #3%
    }%
  \endgroup
}
%%%%%%%%%%%%%%%%%%%%%%%%%%%%%%%%%%%%%%%%%

\begin{document}
\setcounter{chapter}{2}
\setcounter{page}{19}
\chapter{Discrete-Time Convolution}

In the previous chapter, we were introduced to the concept of \emph{linear convolution}, an operation ($*$) defined by the \emph{convolution sum}
\begin{align*}
    x[n] * h[n] &= \sum_{k=-\infty}^{+\infty} x[k]h[n-k].
\end{align*}
We have seen that the operation is commutative since by change of variables, 
\begin{align*}
    x[n] * h[n] &= \sum_{k=-\infty}^{+\infty} x[k]h[n-k] \\
    &= \sum_{k=-\infty}^{+\infty} x[n-k]h[k] = h[n] * x[n].
\end{align*}
We have also already seen another linear convolution property in effect with the sampling property of the unit impulse sequence. 
\begin{align*}
    \sum_{k=-\infty}^{+\infty} x[k]\delta[k-N] &= x[N] \\
    \Longrightarrow \sum_{k=-\infty}^{+\infty} x[k]\delta[k-n] &= \sum_{k=-\infty}^{+\infty} x[k]\delta[n-k] = x[n] \\
    x[n] * \delta[n] &= x[n].
\end{align*}
We also introduced the concept that any LSI system characterized by impulse response $h[n]$ will always have predictable outputs since the output is simply the 
linear convolution of the input signal and the impulse response, provided that all initial conditions of the system are zero. That is, 
\begin{align}
    y[n] = x[n] * h[n] = \sum_{k=-\infty}^{+\infty} x[k]h[n-k].
\end{align}
In this chapter, different techniques to computing the linear convolution of two sequences are explored.

\section{Analytical Convolution in Discrete-Time}
One way to compute the linear convolution is to directly solve the convolution sum itself. Refer to Table \ref{conv_prop} for some 
linear convolution properties.
\\
\begin{table}[hbt!]
    \centering
    \caption{Linear convolution properties.}
    \label{conv_prop}
    \begin{tabular}{|c|c|}
        \hline
        Property & Description \\[0.15cm]
        \hline
        Commutativity & $x[n]*h[n]=h[n]*x[n]$ \\[0.5cm]
        Associativity & $x[n]*[h[n]*g[n]]=[x[n]*h[n]]*g[n]$ \\[0.5cm]
        Distributivity & $x[n]*[h_1[n]+\cdots+h_N[n]]=(x[n]*h_1[n])+\cdots+[x[n]*h_N[n]]$  \\[0.5cm]
        Time-shift & $x[n-N_1]*h[n-N_2]=y[n-N_1-N_2]$ \\[0.5cm]
        Convolution with impulse & $x[n]*\delta[n]=x[n]$ \\[0.5cm]
        Convolution with step & $x[n]*u[n]=\displaystyle\sum_{k=-\infty}^{n} x[k]$ (ideal accumulator) \\[0.5cm]
        Causal signals and systems & $y[n]=u[n]\, \displaystyle\sum_{k=0}^{n} h[k]x[n-k]$ \\[0.5cm]
        Width (duration) & $\text{Width}[y[n]]=\text{Width}[x[n]]+\text{Width}[h[n]]-1$ \\[0.5cm]
         & $\begin{cases} 
            (\text{domain } x[n])=[a,b], \\
            (\text{domain } h[n])=[c,d]
            \end{cases} \Longrightarrow (\text{domain } y[n])=[a+c,b+d]$ \\[0.5cm]
        ``Area'' (infinite sum) & $\displaystyle\sum_{k=-\infty}^{+\infty} y[n]=\left(\displaystyle\sum_{k=-\infty}^{+\infty} x[n]\right)\left(\displaystyle\sum_{k=-\infty}^{+\infty} h[n]\right)$ \\[0.5cm]
        \hline
    \end{tabular}
\end{table} 
\\
When convolving two causal sequences, one of the sequences will become time-reversed in the convolution sum. We can use this to our advantage in redefining the limits of summation. That is, for 
$f[n]$ representing the non-step factors, 
\begin{align}
    \sum_{n=-\infty}^{+\infty} f[n] \cdot \underbrace{u[n-a]}_\textrm{$n-a\geq 0$} \cdot \underbrace{u[b-n]}_\textrm{$b-n\geq 0$} = u[b-a]\sum_{n=a}^{b} f[n].
\end{align}
The inequalities above are defined for when the step sequences are ``switched on'', and combining the two inequalities gives us an interval of validity $a\leq n\leq b$, which is what is used to update the convolution sum.
\\ \\
When solving convolution sums, the need to compute the geometric series come up often. As a reminder, the (length-$N$) finite sum of a geometric series is defined as 
\begin{align}
    \sum_{n=0}^{N-1} r^n = \frac{1-r^N}{1-r}\, \text{ for } |r| < 1,
\end{align}
whereas the infinite sum of a geometric series is defined as 
\begin{align}
    \sum_{n=0}^{\infty} r^n = \frac{1}{1-r}\, \text{ for } |r| < 1.
\end{align}

While we are mostly interested in intervals of validity that follow causal signals and systems, there are other possible intervals of validity for different 
types of systems that may or may not be causal. Refer to Table \ref{sum_redef} for these summation redefinitions. 
\begin{table}[hbt!]
    \centering
    \caption{Summation redefinitions based on unit step sequences.}
    \label{sum_redef}
    \begin{mdframed}
    \begin{align}
        \sum_{n=-\infty}^{+\infty} f[n] u[n-a] &= \sum_{n=a}^{+\infty} f[n] \\[0.25cm]
        \sum_{n=-\infty}^{+\infty} f[n] u[b-n] &= \sum_{n=-\infty}^{b} f[n] \\[0.25cm]
        \sum_{n=-\infty}^{+\infty} f[n] u[n-a] u[b-n] &= u[b-a]\sum_{n=a}^{b} f[n] \\[0.25cm]
        \sum_{n=-\infty}^{+\infty} f[n] u[n-a] u[n-b] &= \sum_{n=\max(a,b)}^{+\infty} f[n] \nonumber \\ 
        &= u[a-b] \sum_{n=a}^{+\infty} f[n] + u[b-a] \sum_{n=b}^{+\infty} f[n] \\[0.25cm] 
        \sum_{n=-\infty}^{+\infty} f[n] u[a-n] u[b-n] &= \sum_{n=-\infty}^{\min(a,b)} f[n] \nonumber \\ 
        &= u[b-a] \sum_{n=-\infty}^{a} f[n] + u[a-b] \sum_{n=-\infty}^{b} f[n] \\[0.15cm] \nonumber
    \end{align}
    \end{mdframed}
\end{table} 

\begin{example}
    Find $y[n]=x[n]*h[n]$ for 
    \begin{align*}
        x[n] &= 2^{-n}u[n], \\
        h[n] &= u[n-2].
    \end{align*}
\end{example}
\begin{solution}
    Using the convolution sum, it follows that 
    \begin{align*}
        y[n] = [2^{-n} u[n]] * u[n-2] &= \sum_{k=-\infty}^{+\infty} 2^{-k} \underbrace{u[k]}_\textrm{$k\geq 0$} \underbrace{u[n-k-2]}_\textrm{$n-2\geq k$} \\
        &= u[n-2]\,\sum_{k=0}^{n-2} 2^{-k} = u[n-2]\,\sum_{k=0}^{n-2} \left(\frac{1}{2}\right)^k \\
        &= u[n-2]\,\sum_{k=0}^{n-2} 2^{-k} = u[n-2]\, \left(\frac{1 - \left(\frac{1}{2}\right)^{n-1}}{1 - \frac{1}{2}}\right) \\
        &= \left(2-\left(\frac{1}{2}\right)^{n-2}\right) u[n-2].
    \end{align*}
\end{solution}

\begin{example}
    Find $y[n]=x[n]*h[n]$ for 
    \begin{align*}
        x[n] &= 2^{-|n|}, \\
        h[n] &= u[n-2].
    \end{align*}
\end{example}
\begin{solution}
    First, rewrite $x[n]$ such that
    \begin{align*}
        x[n] &= 2^{-|n|} \\
        &= 2^{-n} u[n] + 2^n u[-n-1].
    \end{align*}
    Using the convolution sum, it follows that 
    \begin{align*}
        y[n] &= [2^{-n} u[n] + 2^n u[-n-1]] * u[n-2] \\
        &= \sum_{k=-\infty}^{+\infty} [2^{-k} u[k] + 2^k u[-k-1]] u[n-k-2] \\
        &= \sum_{k=-\infty}^{+\infty} 2^{-k} \underbrace{u[k]}_\textrm{$k\geq 0$} \underbrace{u[n-k-2]}_\textrm{$n-2\geq k$} + \sum_{k=-\infty}^{+\infty} 2^k \underbrace{u[-k-1]}_\textrm{$k\leq -1$}\underbrace{u[n-k-2]}_\textrm{$n-2\geq k$} \\
        &= u[n-2]\sum_{k=0}^{n-2} 2^{-k} + u[n-1]\sum_{k=-\infty}^{-1} 2^k + u[1-n]\sum_{k=-\infty}^{n-2} 2^k
    \end{align*}
    Via change of variables, the output is given by
    \begin{align*}
        y[n] &= \left(2-\left(\frac{1}{2}\right)^{n-2}\right) u[n-2] + u[n-1]\sum_{\ell=1}^{+\infty} \left(\frac{1}{2}\right)^{\ell} + u[1-n]\sum_{\ell=2-n}^{+\infty} \left(\frac{1}{2}\right)^{\ell} \\
        &= \left(2-\left(\frac{1}{2}\right)^{n-2}\right) u[n-2] + u[n-1]\sum_{m=0}^{+\infty} \left(\frac{1}{2}\right)^{m+1} + u[1-n]\sum_{m=0}^{+\infty} \left(\frac{1}{2}\right)^{m+2-n} \\
        &= \left(2-\left(\frac{1}{2}\right)^{n-2}\right) u[n-2] + \left(\frac{\frac{1}{2}}{1-\frac{1}{2}}\right)u[n-1] + \left(\frac{\left(\frac{1}{2}\right)^{2-n}}{1-\frac{1}{2}}\right)u[1-n] \\
        &= \left(2-\left(\frac{1}{2}\right)^{n-2}\right) u[n-2] + u[n-1] + \left(\frac{1}{2}\right)^{1-n}u[1-n] \\
        &= \begin{cases} 
            2^{n-1}, & n<1 \\
            2, & n=1 \\
            3-\left(\frac{1}{2}\right)^{n-2}, & n\geq 2
        \end{cases}.
    \end{align*}
\end{solution}

\section{Pointwise Graphical Convolution in Discrete-Time}
Another way to compute the convolution is to interpret the convolution graphically. We will first introduce the pointwise method, 
best used when only a few output samples are of interest. Since 
\begin{align*}
    y[n] = x[n] * h[n] = \sum_{k=-\infty}^{+\infty} x[k]h[n-k],
\end{align*}
the output at a certain single time $n$ is given by the sum of the points formed by multiplying $x[k]$ and $h[n-k]$, which is a time-reversed and translated 
version of $h[k]$. 

\begin{tcolorbox}[width=\textwidth,colback={white}, sharp corners]
    \noindent\emph{POINTWISE GRAPHICAL CONVOLUTION.}
    \begin{enumerate}[Step 1:]
        \item Determine the set of time values of interest $n_0, ..., n_N$.
        \item Plot $x[k]$ and $h[-k]$ along the $k$-axis.
        \item Let $n=n_0$ and shift $h[-k]$ to the right by $n$ units to represent $h[n-k]$.
        \item Using the region of overlap between $x[k]$ and $h[n-k]$, take the sum of the point-by-point product.
        \item Repeat Steps 2--4 for the remaining time values of interest.
    \end{enumerate}
\end{tcolorbox}

\begin{example}
    Find $y[n]=x[n]*h[n]$ for 
    \begin{align*}
        x[n] &= \left\{\underline{3},1,2\right\}, \\
        h[n] &= \left\{\underline{3},2,1\right\}.
    \end{align*}
\end{example}
\begin{solution}
    From the width property of linear convolution, we know that the duration of the output is five samples, which is the sum of the two widths less one. 
    Additionally, both $x[n]$ and $h[n]$ are finite signals that start at $n=0$ and end at $n=2$. Therefore, we are interested when $n=0,1,2,3,4$.
    \\ \\
    Suppose we are interested in only finding $y[2]$. First, we plot $x[k]$ and $h[-k]$ on the $k$-axis.
    \\ \\
    \begin{tikzpicture}
        \begin{axis}[
            axis x line=center, axis y line=center,
            ymin=0, ymax=3.5, ytick={0,...,3},
            xmin=-2.5, xmax=2.5, xtick={-3,...,3}, xlabel={$k$},
            width=7cm, height=4cm]
            \addplot+ [
                ycomb,
            ] coordinates {(0,3.05) (1,1) (2,2)};
            \addplot+ [
                ycomb,
            ] coordinates {(0,2.95) (-1,2) (-2,1)};
        \end{axis}
    \end{tikzpicture}
    \\ \\
    Since we want to observe at time $n=2$, we shift the red plot to the right by 2 units: \\ \\
    \begin{tikzpicture}
        \begin{axis}[
            axis x line=center, axis y line=center,
            ymin=0, ymax=3.5, ytick={0,...,3},
            xmin=-2.5, xmax=2.5, xtick={-3,...,3}, xlabel={$k$},
            width=7cm, height=4cm]
            \addplot+ [
                ycomb,
            ] coordinates {(0,3) (1,1) (2,2)};
            \addplot+ [
                ycomb,
            ] coordinates {(2,3) (1,2) (0,1)};
        \end{axis}
    \end{tikzpicture}
    \\ \\
    Performing point-by-point multiplication, we get: \\ \\
    \begin{tikzpicture}
        \begin{axis}[
            axis x line=center, axis y line=center,
            ymin=0, ymax=6.5, ytick={0,2,4,6},
            xmin=-2.5, xmax=2.5, xtick={-3,...,3}, xlabel={$k$},
            width=7cm, height=4cm]
            \addplot+ [ycomb] coordinates {(10,0)};
            \addplot+ [ycomb] coordinates {(10,0)};
            \addplot+ [
                ycomb,
            ] coordinates {(0,3) (1,2) (2,6)};
        \end{axis}
    \end{tikzpicture}
    \\ \\
    Taking the sum of the points, it follows that 
    \begin{align*}
        y[2] = 3 + 2 + 6 = 11.
    \end{align*}
    We can repeat the process for other values of $n\in[0,4]$ until we get $y[n]=\left\{\underline{9},9,11,5,2\right\}$.
\end{solution}

\section{Region-Based Graphical Convolution in Discrete-Time}
The pointwise method is impractical if we are interested in the output response over a larger, if not entire, time domain. Here, we introduce the region-based 
graphical convolution, which utilizes a signal's \emph{leading and trailing edges}, if any. 

\begin{tcolorbox}[width=\textwidth,colback={white}, sharp corners]
    \noindent\emph{REGION-BASED GRAPHICAL CONVOLUTION.}
    \begin{enumerate}[Step 1:]
        \item Plot $x[k]$ and $h[-k]$ along the $k$-axis, with $h[-k]$ being the simpler sequence to flip.
        \item Label the leading and trailing edges of $h[-k]$.
        \item Identify a region of overlap and use the edges to determine the interval of $n$-values. Then evaluate the convolution sum during that region of overlap. 
        \item Repeat Steps 2--3 for any other regions of overlap when shifting $h[-k]$ from left-to-right along the $k$-axis. Keep track of the width of $h[n-k]$. 
    \end{enumerate}
\end{tcolorbox}

\begin{example}
    Find $y[n]=x[n]*h[n]$ for 
    \begin{align*}
        x[n] &= (0.8)^n u[n], \\
        h[n] &= u[n]-u[n-3].
    \end{align*}
\end{example}
\begin{solution}
    First, plot $x[k]$ and $h[-k]$. Then identify the leading and trailing edges of $h[-k]$. \\ \\
    \begin{tikzpicture}
        \begin{axis}[
            axis x line=center, axis y line=center,
            ymin=0, ymax=1.2, ytick={0,0.5,1}, ylabel={$x[k]$},
            xmin=0, xmax=5.5, xtick={0,...,5}, xlabel={$k$},
            width=7cm, height=4cm]
            \addplot+ [
                ycomb,
            ] coordinates {(0,1) (1,0.8) (2,0.64) (3,0.512) (4,0.4096) (5,0.32768)};
        \end{axis}
    \end{tikzpicture}
    \begin{tikzpicture}
        \begin{axis}[
            axis x line=center, axis y line=center,
            ymin=0, ymax=1.2, ytick={0,0.5,1}, ylabel={$\big[h[n-k]\big|_{n=0} = h[-k]$},
            xmin=-3.5, xmax=2.5, xtick={-2,-1,0}, xlabel={$k$},
            width=7cm, height=4cm]
            \addplot+ [ycomb] coordinates {(-10,0)};
            \addplot+ [
                ycomb,
            ] coordinates {(0,1) (-1,1) (-2,1)};
            \node[red] at (-3,0.5) {\scriptsize \shortstack{Trailing \\ edge, \\ $n-2$}};
            \node[red] at (1,0.5) {\scriptsize \shortstack{Leading \\ edge, $n$}};
        \end{axis}
    \end{tikzpicture}
    \\
    From $h[-k]$ above, we see that the leading edge is $k=n$ and the trailing edge is $k=n-2$. Now we place $h[n-k]$ to the furthest left and gradually shift to the right. 
    \\ \\ \\
    \underline{Region 1: No overlap} \\ \\
    \begin{tabular}{ cc }
        \begin{tabular}{ c }
            \begin{tikzpicture}
                \begin{axis}[
                    axis x line=center, axis y line=center,
                    ymin=0, ymax=1.5, ytick={0,1}, ylabel={$x[k]$},
                    xmin=-4, xmax=3.5, xtick={0,...,4}, xlabel={$k$},
                    width=7cm, height=3cm]
                    \addplot+ [
                        ycomb,
                    ] coordinates {(0,1) (1,0.8) (2,0.64) (3,0.512)};
                \end{axis}
            \end{tikzpicture} \\
            \begin{tikzpicture}
                \begin{axis}[
                    axis x line=center, axis y line=center,
                    ymin=0, ymax=1.5, ytick={0,1}, ylabel={$h[n-k]$},
                    xmin=-4, xmax=3.5, xtick={0}, xlabel={$k$},
                    extra x ticks={-3,-1}, extra x tick labels={$n-2$,$n$},
                    width=7cm, height=3cm]
                    \addplot+ [ycomb] coordinates {(-10,0)};
                    \addplot+ [
                        ycomb,
                    ] coordinates {(-3,1) (-2,1) (-1,1)};
                \end{axis}
            \end{tikzpicture}
        \end{tabular} & 
        \begin{tabular}{ c }
            Leading edge: $n \leq -1$ \\ 
            \\
            No overlap, so $y[n]=0$ for $n\leq -1.$
        \end{tabular}
    \end{tabular} 
    \\ \\ \\
    \noindent \underline{Region 2: Partial overlap} \\ \\
    \begin{tabular}{ cc }
        \begin{tabular}{ c }
            \begin{tikzpicture}
                \begin{axis}[
                    axis x line=center, axis y line=center,
                    ymin=0, ymax=1.5, ytick={0,1}, ylabel={$x[k]$},
                    xmin=-4, xmax=3.5, xtick={0,...,4}, xlabel={$k$},
                    width=7cm, height=3cm]
                    \addplot+ [
                        ycomb,
                    ] coordinates {(0,1) (1,0.8) (2,0.64) (3,0.512)};
                \end{axis}
            \end{tikzpicture} \\
            \begin{tikzpicture}
                \begin{axis}[
                    axis x line=center, axis y line=center,
                    ymin=0, ymax=1.5, ytick={0,1}, ylabel={$h[n-k]$},
                    xmin=-4, xmax=3.5, xtick={0}, xlabel={$k$},
                    extra x ticks={-1,1}, extra x tick labels={$n-2$,$n$},
                    width=7cm, height=3cm]
                    \addplot+ [ycomb] coordinates {(-10,0)};
                    \addplot+ [
                        ycomb,
                    ] coordinates {(-1,1) (0,1) (1,1)};
                \end{axis}
            \end{tikzpicture}
        \end{tabular} & 
        \begin{tabular}{ c }
            Leading edge: $n \geq 0$ \\ 
            Trailing edge: $n-2 \leq -1 \Rightarrow n \leq 1$ \\ 
            \\
            $\Longrightarrow 0\leq n\leq 1$ \\
            \\
            $\Longrightarrow y[n]=\displaystyle\sum_{k=0}^{n} [(0.8)^k][1] = \frac{1-(0.8)^{n+1}}{1-0.8}$ 
        \end{tabular}
    \end{tabular} 
    \\ \\ \\
    \noindent \underline{Region 3: Complete overlap} \\ \\
    \begin{tabular}{ cc }
        \begin{tabular}{ c }
            \begin{tikzpicture}
                \begin{axis}[
                    axis x line=center, axis y line=center,
                    ymin=0, ymax=1.5, ytick={0,1}, ylabel={$x[k]$},
                    xmin=-4, xmax=3.5, xtick={0,...,4}, xlabel={$k$},
                    width=7cm, height=3cm]
                    \addplot+ [
                        ycomb,
                    ] coordinates {(0,1) (1,0.8) (2,0.64) (3,0.512)};
                \end{axis}
            \end{tikzpicture} \\
            \begin{tikzpicture}
                \begin{axis}[
                    axis x line=center, axis y line=center,
                    ymin=0, ymax=1.5, ytick={0,1}, ylabel={$h[n-k]$},
                    xmin=-4, xmax=3.5, xtick={0}, xlabel={$k$},
                    extra x ticks={1,3}, extra x tick labels={$n-2$,$n$},
                    width=7cm, height=3cm]
                    \addplot+ [ycomb] coordinates {(-10,0)};
                    \addplot+ [
                        ycomb,
                    ] coordinates {(1,1) (2,1) (3,1)};
                \end{axis}
            \end{tikzpicture}
        \end{tabular} & 
        \begin{tabular}{ c }
            Trailing edge: $n-2 \geq 0 \Rightarrow n \geq 2$ \\ 
            \\
            $\Longrightarrow y[n]=\displaystyle\sum_{k=n-2}^{n} [(0.8)^k][1] = \displaystyle\sum_{\ell=0}^{2} (0.8)^{\ell+n-2}$ \\
            $\Longrightarrow y[n] = (0.8)^{n-2}\left(\frac{1-(0.8)^3}{1-0.8}\right) = 2.44(0.8)^{n-2}$ 
        \end{tabular}
    \end{tabular} 
    \\ \\ \\
    Aggregating the outputs from the five regions, we can write a piecewise function for the overall output:
    \begin{align*}
        y[n] = 
        \begin{cases} 
            0, & n \leq -1 \\
            \frac{1-(0.8)^{n+1}}{0.2}, & 0\leq n\leq 1 \\
            2.44(0.8)^{n-2}, & n\geq 2
        \end{cases}.
    \end{align*}
\end{solution}

\section{Tabular Methods for Finite-Length Sequences}
Since offline processing (or even real-time processing with limited memory) involves analyzing data with finite duration (due to the sampled nature of existing collected data), linear convolutions in practice 
mostly involves finite-length sequences -- meaning the output sequence will also be of finite-length. There are various shortcut methods to solving 
these specific types of linear convolutions by hand.

\subsection{Expanded Tabular Method}
The expanded tabular method essentially is a tabular version of the pointwise graphical convolution, albeit saving time on sketching multiple plots.

\begin{tcolorbox}[width=\textwidth,colback={white}, sharp corners]
    \noindent\emph{EXPANDED TABULAR METHOD.} For $(\text{domain } x[n]) = [a,b]$ and $(\text{domain } h[n]) = [c,d]$:
    \begin{enumerate}[Step 1:]
        \item Create a header row for the values of $n$.
        \item Populate the first row with values of $x[n]$ under their respective $n$.
        \item Populate the second row with the time-reversed $h[-n]$ under the respective $n$.
        \item For every necessary time shift $N$, populate another row corresponding to $h[N-n]$.
        \item Vertically multiply for each $x[n]h[N-n]$ and take the sum of the pointwise products to get $y[N]$.
    \end{enumerate}
\end{tcolorbox}

\begin{example}
    Find $y[n]=x[n]*h[n]$ for 
    \begin{align*}
        x[n] &= \left\{\underline{3},4,5\right\}, \\
        h[n] &= \left\{\underline{6},7,8\right\}.
    \end{align*}
\end{example}
\begin{solution}
    From the width property, the output $y[n]$ should be of length 5 and defined for $n=0,1,2,3,4$. Setting up the table:
    \begin{center}
        \begin{tabular}{c|ccccccc|c}
            $n \rightarrow $ & -2 & -1 & 0 & 1 & 2 & 3 & 4 & $y[\cdot]$ \\
            \hline
            $x[n]$ & & & 3 & 4 & 5 & & & \\
            \hline
            $h[-n]$ & 8 & 7 & 6 & & & & & $y[0] = 3(6) = 18$ \\
            $h[1-n]$ & & 8 & 7 & 6 & & & & $y[1] = 3(7) + 4(6) = 45$ \\
            $h[2-n]$ & & & 8 & 7 & 6 & & & $y[2] = 3(8) + 4(7) + 5(6) = 82$\\
            $h[3-n]$ & & & & 8 & 7 & 6 & & $y[3] = 4(8) + 5(7) = 67$ \\
            $h[4-n]$ & & & & & 8 & 7 & 6 & $y[4] = 5(8) = 40$ 
        \end{tabular}
    \end{center}
    Therefore, the output is $y[n]=\left\{\underline{18},45,82,67,40\right\}$.
\end{solution}

\subsection{Condensed Tabular Method}
The condensed tabular method is visually more compact and the quickest to set up.

\begin{tcolorbox}[width=\textwidth,colback={white}, sharp corners]
    \noindent\emph{CONDENSED TABULAR METHOD.} For $(\text{domain } x[n]) = [a,b]$ and $(\text{domain } h[n]) = [c,d]$:
    \begin{enumerate}[Step 1:]
        \item Create a header column for the values of $x[n]$. Underline $x[0]$.
        \item Create a header row for the values of $h[n]$. Underline $h[0]$.
        \item Populate the table by taking the product of the corresponding $x[\cdot]$ and $h[\cdot]$ values.
        \item Add entries diagonally from right to left and list the sums in the order of their diagonals.
        \item Find where $x[0]$ and $h[0]$ intersect. The corresponding right-to-left diagonal represents $y[0]$.
    \end{enumerate}
\end{tcolorbox}

\begin{example}
    Find $y[n]=x[n]*h[n]$ for 
    \begin{align*}
        x[n] &= \left\{\underline{3},4,5\right\}, \\
        h[n] &= u[n]-u[n-4].
    \end{align*}
\end{example}
\begin{solution}
    First, note that $h[n] = u[n]-u[n-4] = \left\{\underline{1},1,1,1\right\}$ is of length 4. Setting up the table:
    \begin{center}
        \begin{NiceTabular}{p{1cm}|p{1cm} p{1cm} p{1cm} p{1cm}}[corners=NE,margin]
            \rule{0pt}{2em}\diagbox{$x[n]$}{$h[n]$} & \Block{1-1}{$\underline{1}$} & \Block{1-1}{1} & \Block{1-1}{1} & \Block{1-1}{1} \\
            \hline
            \Block{1-1}{$\underline{3}$} & \Block[fill=yellow!50]{1-1}{3} & \Block[fill=red!30]{1-1}{3} & \Block[fill=blue!30]{1-1}{3} & \Block[fill=green!30]{1-1}{3} \\
            \Block{1-1}{4} & \Block[fill=red!30]{1-1}{4} & \Block[fill=blue!30]{1-1}{4} & \Block[fill=green!30]{1-1}{4} & \Block[fill=violet!30]{1-1}{4} \\
            \Block{1-1}{5} & \Block[fill=blue!30]{1-1}{5} & \Block[fill=green!30]{1-1}{5} & \Block[fill=violet!30]{1-1}{5} & \Block[fill=orange!50]{1-1}{5}
        \end{NiceTabular}
    \end{center}
    Each right-to-left diagonal is highlighted a different color to show which entries should be summed. The output is then 
    \begin{align*}
        y[n] = \left\{\underline{3},\,3+4,\,3+4+5,\,3+4+5,\,4+5,\,5\right\} = \left\{\underline{3},7,12,12,9,5\right\}
    \end{align*}
\end{solution}

\begin{example}
    For $y[n]=x[n]*h[n]$, find $h[n]$ given that  
    \begin{align*}
        x[n] &= \left\{\underline{1},2,3\right\}, \\
        y[n] &= \left\{\underline{1},4,7,6\right\}.
    \end{align*}
\end{example}
\begin{solution}
    Since $(\text{domain }x[n]) = [0,2]$ and $(\text{domain }y[n]) = [0,3]$, it must be true by the width property for linear convolution that 
    \begin{align*}
        (\text{domain }h[n]) &= [0-0,3-2] = [0,1] \\
        \Longrightarrow h[n] &= \left\{\underline{h[0]},h[1]\right\}
    \end{align*}
    Using the condensed tabular method, we get:
    \begin{center}
        \begin{NiceTabular}{p{1cm}|p{1cm} p{1cm}}[corners=NE,margin]
            \rule{0pt}{2em}\diagbox{$x[n]$}{$h[n]$} & \Block{1-1}{$\underline{h[0]}$} & \Block{1-1}{$h[1]$} \\
            \hline
            \Block{1-1}{$\underline{1}$} & \Block[fill=yellow!50]{1-1}{$h[0]$} & \Block[fill=red!30]{1-1}{$h[1]$} \\
            \Block{1-1}{2} & \Block[fill=red!30]{1-1}{$2h[0]$} & \Block[fill=blue!30]{1-1}{$2h[1]$} \\
            \Block{1-1}{3} & \Block[fill=blue!30]{1-1}{$3h[0]$} & \Block[fill=green!30]{1-1}{$3h[1]$} 
        \end{NiceTabular}
    \end{center}
    Lining up each right-to-left diagonal with the corresponding $y[n]$ entry, we get the following system of equations:
    \begin{align*}
        h[0] &= 1 \\
        2h[0] + h[1] &= 4 \\
        3h[0] + 2h[1] &= 7 \\
        3h[1] &= 6
    \end{align*}
    Solving the system of equations, we get:
    \begin{align*}
        h[n] = \left\{\underline{h[0]},h[1]\right\} = \left\{\underline{1},2\right\}
    \end{align*}
\end{solution}

\end{document}