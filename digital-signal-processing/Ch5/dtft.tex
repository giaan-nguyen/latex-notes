\documentclass{report}
% PACKAGES
\usepackage{adjustbox}
\usepackage{amsmath}
\usepackage{amssymb}
\usepackage{bodegraph}
\usepackage{bbm}
\usepackage{circledsteps}
\usepackage{circuitikz}
\usepackage{enumerate}
\usepackage{mathtools}
\usepackage{nicematrix}
\usepackage{pgfplots}
\usepackage{polynom}
\usepackage{qtree}
\usepackage{rotating}
\usepackage[usestackEOL]{stackengine}
\usepackage[subpreambles=true]{standalone}
\usepackage{steinmetz}
\usepackage{subcaption}
\usepackage{tabularray}
\usepackage{tcolorbox}
\usepackage{tikz}
\usepackage{xcolor}

\usepackage[colorlinks=true,linkcolor=blue,urlcolor=black,bookmarksopen=true]{hyperref}
\usepackage{bookmark}

\usetikzlibrary{shapes.arrows}
\usetikzlibrary{shapes.misc}
\usetikzlibrary{backgrounds}
\tikzset{cross/.style={cross out, draw=black, minimum size=2*(#1-\pgflinewidth), inner sep=0pt, outer sep=0pt},
%default radius will be 1pt. 
cross/.default={1pt}}

\renewcommand{\Re}{\operatorname{Re}}
\renewcommand{\Im}{\operatorname{Im}}

\usepackage{pifont}
\newcommand{\cmark}{\text{\ding{51}}}
\newcommand{\xmark}{\text{\ding{55}}}

\newcommand{\circconv}[1]{\text{ \small\Circled{#1} }}

\newcommand{\tikzmark}[3][]{\tikz[remember picture,baseline] \node [anchor=base,#1](#2) {$#3$};}

\pgfplotsset{compat=1.18}
\pgfplotsset{
    dirac/.style={
        mark=triangle*,
        mark options={scale=1.5},
        ycomb,
        scatter,
        visualization depends on={y/abs(y)-1 \as \sign},
        scatter/@pre marker code/.code={\scope[rotate=90*\sign,yshift=-2pt]}
    }
}

\usepackage[letterpaper, portrait, margin=1.25in]{geometry}
\usepackage[font=bf]{caption}
\hbadness = 10000
\hfuzz=2pt

\newtheorem{theorem}{Theorem}[chapter]

\tikzset{every tree node/.style={anchor=north,align=center}}
\usetikzlibrary{decorations.markings}
\usetikzlibrary{arrows}
\definecolor{darkgreen}{rgb}{0.133,0.545,0.133}
\tikzstyle{n}= [circle, fill=blue, minimum size=4pt,inner sep=0pt, outer sep=0pt]

\patchcmd{\thebibliography}{\chapter*}{\section*}{}{}
\setcounter{secnumdepth}{5}

%%%%%%%%%%% EXAMPLE ENVIRONMENT %%%%%%%%%%
\usepackage{calc}
\usepackage{tabto}
\usepackage[framemethod=tikz]{mdframed} % for colored backgrounds
\newcommand{\halmos}{} % makes a box at the end

\newlength{\framedinnerleftmargin}
\newlength{\framedinnertopmargin}
\newlength{\framedreversedinnerleftmargin}
\setlength{\framedinnerleftmargin}{\widthof{Theoreme 10.10.10}+2em}
\setlength{\framedreversedinnerleftmargin}{\widthof{Theoreme 10.10.10}+1em}
\setlength{\framedinnertopmargin}{1em}
 
% first argument: label in upper left corner,
% second argument: background color
\newenvironment{boxedtext}[2]{\begin{mdframed}[%
hidealllines=true,%
backgroundcolor=#2,%
innertopmargin=\framedinnertopmargin,%
innerleftmargin=\framedinnerleftmargin,%
innerrightmargin=1em%
]%
\tabto{-\framedreversedinnerleftmargin}\textbf{#1}\tabto*{0em}%
}% begin code
{\hskip 0pt\\\hspace*{\fill}\halmos{}\end{mdframed}\vspace{1em}} % end code
 
\newenvironment{summary}[0]{\begin{center}\begin{minipage}[c]{\summarywidth}\begin{spacing}{0.9}\footnotesize} % begin code
{\end{spacing}\end{minipage}\end{center}} % end code

\newcounter{example}
 
% optional! if you want it to start at zero
% with every new chapter/section/etc.
\numberwithin{example}{section}
 
\newenvironment{example}[0]
{\refstepcounter{example}\vspace{1em plus 1em}\begin{boxedtext}{Example \theexample.}{blue!7}}%\setlength{\parskip}{0em}}
{\end{boxedtext}\vspace{-1em plus 1em}}
 
\newenvironment{example*}[0]
{\vspace{1em plus 1em}\begin{boxedtext}{Example.}{blue!7}}
{\end{boxedtext}\vspace{-1em plus 1em}}

\newmdenv[
  topline=false,
  bottomline=false,
  rightline=false,
  skipabove=\topsep,
  skipbelow=\topsep,
  linecolor=purple,
  frametitle={\noindent\textcolor{purple}{\textbf{SOLUTION }}},
  endinnercode={$\hfill\textcolor{purple}{\blacksquare}$}
]{solution}

%%%%%%%%%%%%%% COLOR BOXED %%%%%%%%%%%%%%%
% Syntax: \colorboxed[<color model>]{<color specification>}{<math formula>}
\newcommand*{\colorboxed}{}
\def\colorboxed#1#{%
  \colorboxedAux{#1}%
}
\newcommand*{\colorboxedAux}[3]{%
  % #1: optional argument for color model
  % #2: color specification
  % #3: formula
  \begingroup
    \colorlet{cb@saved}{.}%
    \color#1{#2}%
    \boxed{%
      \color{cb@saved}%
      #3%
    }%
  \endgroup
}
%%%%%%%%%%%%%%%%%%%%%%%%%%%%%%%%%%%%%%%%%

\begin{document}
\setcounter{chapter}{4}
\setcounter{page}{54}
\chapter{Discrete-Time Fourier Transform}

Here, we introduce a special case of the $z$--transform. If we let $z=e^{j\Omega}$, then we get the \emph{discrete-time Fourier transform} (DTFT).
\begin{align}
    X(e^{j\Omega}) &= X(z)\big|_{z=\exp(j\Omega)}
\end{align}
An advantage of using the DTFT is not having to worry about ROCs. In addition, we are able to analyze and exploit the frequency content of DT signals and systems 
using the DTFT, similar to the CT case. Interestingly, just as the $z$-transform is continuous in $z$, the DTFT is also continuous in $\Omega$.
\\ \\
Lastly, while 
\begin{align}
    e^{j\omega t} \neq e^{j(\omega+2\pi)t} \text{ for non-integer } t\in\mathbb{R},
\end{align}
it holds true that 
\begin{align}
    e^{j\Omega n} = e^{j(\Omega+2\pi)n} \text{ for } n\in\mathbb{Z}.
\end{align}
Therefore, the DTFT is $2\pi$-periodic in $\Omega$.

\section{Discrete-Time Fourier Transform}
\begin{tcolorbox}[width=\textwidth,colback={white}, sharp corners]
    The \emph{discrete-time Fourier transform} (DTFT) of a DT signal $x[n]$ is a $2\pi$-periodic continuous function of $\Omega$ defined as 
    \begin{align}
        X(e^{j\Omega}) = \text{DTFT}[x[n]] = \sum_{n=-\infty}^{+\infty} x[n] e^{-j\Omega n}.
    \end{align}
\end{tcolorbox}
Similar to the $z$--transform, the DTFT is essentially an operator that maps a signal defined in the \emph{time domain} to another signal defined 
in the \emph{frequency domain}. $x[n]$ and $X(e^{j\Omega})$ constitute a unique \emph{DTFT pair}. This relationship can be written as 
\begin{align}
    x[n] \iff X(e^{j\Omega}).
\end{align}
Because of this relationship, there exists an \emph{inverse DTFT} defined by
\begin{align}
    x[n] = \text{DTFT}^{-1}[X(e^{j\Omega})] = \frac{1}{2\pi}\int_{-\pi}^{+\pi} X(e^{j\Omega}) e^{+j\Omega n} \,d\Omega,
\end{align}
though referencing a table of DTFT pairs is a lot simpler than computing the integral. 
\\ \\
Since DT signals are evaluated as sequences, the CT Dirichlet conditions are now condensed to a single sufficient condition for the DT case. 
It follows that if $x[n]$ is absolutely summable such that 
\begin{align}
    \sum_{n=-\infty}^{+\infty}|x[n]| < \infty,
\end{align}
then $x[n]$ has a DTFT. However, it is a sufficient but not a necessary condition as both power signals and energy signals have DTFT pairs. 
\\ \\
Table \ref{dtft_prop} lists the properties of the Fourier transform.
\begin{table}
    \centering
    \caption{Properties of the DTFT.}
    \label{dtft_prop}
    \begin{tabular}{|c|c|c|}
        \hline
        Property & $x[n]$ & $X(e^{j\Omega})=\text{DTFT}[x[n]]$ \\[0.15cm]
        \hline
        & & \\
        Superposition & $K_1x_1[n]+K_2x_2[n]$ & $K_1X_1(e^{j\Omega})+K_2X_2(e^{j\Omega})$ \\[0.5cm]
        Time shift & $x[n-n_0]$ & $e^{-j\Omega n_0}X(e^{j\Omega})$ \\[0.5cm]
        Time reversal & $x[-n]$ & $X(e^{-j\Omega})$ \\[0.5cm]
        Conjugation & $x^*[n]$ & $X^*(e^{-j\Omega})$ \\[0.5cm]
        Frequency shift & $e^{j\Omega_0 n}x[n]$ & $X(e^{j(\Omega-\Omega_0)})$ \\[0.5cm]
        Frequency derivative & $n\cdot x[n]$ & $j\cdot\dfrac{dX(e^{j\Omega})}{d\Omega} = j\cdot X'(e^{j\Omega})$\\[0.5cm]
        Finite difference & $x[n]-x[n-1]$ & $(1-e^{-j\Omega})X(e^{j\Omega})$ \\[0.5cm]
        Accumulation & $\displaystyle\sum_{k=-\infty}^{n} x[k]$ & $\dfrac{1}{1-e^{-j\Omega}} X(e^{j\Omega})$ \\[0.5cm]
        Linear convolution & $x_1[n]*x_2[n]$ & $X_1(e^{j\Omega})X_2(e^{j\Omega})$ \\[0.5cm]
        Multiplication & $x_1[n]x_2[n]$ & $\dfrac{1}{2\pi}\left[X_1(e^{j\Omega})*X_2(e^{j\Omega})\right]$ \\[0.5cm]
        Modulation & $x(t)\cos(\omega_0 t)$ & $\dfrac{1}{2}[X(e^{j(\Omega-\Omega_0)}) + X(e^{j(\Omega+\Omega_0)})]$ \\[0.5cm]
        Conjugate symmetry & $x[n]$ real & 
        $\begin{cases}
            X(e^{-j\Omega}) = X^*(e^{j\Omega}) \\
            \Re(X(e^{j\Omega})) = \Re(X(e^{-j\Omega})) \\
            \Im(X(e^{j\Omega})) = -\Im(X(e^{j\Omega})) \\
            |X(e^{j\Omega})| = |X(e^{-j\Omega})| \\
            \arg[X(e^{j\Omega})] = -\arg[X(e^{-j\Omega})]
        \end{cases}$ \\[0.5cm]
         & & \\[0.25cm]
        \Centerstack{Even-odd decomposition \\ of real signals} & 
        $\begin{cases}
            x_e[n]=\frac{1}{2}[x[n]+x[-n]] \\
            x_o[n]=\frac{1}{2}[x[n]-x[-n]]
        \end{cases}$ & 
        $\begin{cases}
            \Re(X(e^{j\Omega})) \\
            j\Im(X(e^{j\Omega}))
        \end{cases}$ \\[0.5cm]
        \hline
    \end{tabular}
\end{table}
From the conjugate symmetry property, there are important implications about the properties of $x[n]$ and its DTFT $X(e^{j\Omega})$:
\begin{center}
    \begin{tabular}{|c|c|}
        \hline
        $x[n]$ & $X(e^{j\Omega})$ \\
        \hline
        Real and even & Real and even \\
        Real and odd & Imaginary and odd \\
        Imaginary and even & Imaginary and even \\
        Imaginary and odd & Real and odd \\
        \hline
    \end{tabular}
\end{center}
Table \ref{dtft_pairs} lists some common DTFT pairs.

\begin{example}
    Show that $\text{DTFT}[\operatorname{sgn}[n]]=-j\cot\left(\frac{\Omega}{2}\right)$. 
    Note that $\operatorname{sgn}[n] = u[n]-u[-n]$.
\end{example}
\begin{solution}
    From the hint, we see that $\operatorname{sgn}[n] = u[n]-u[-n]$ is really just twice the odd part of real signal $u[n]$. 
    That is, if we let $x[n]=u[n]$, then
    \begin{align*}
        \operatorname{sgn}[n] &= u[n]-u[-n] \\ 
        &= x[n]-x[-n] = 2x_o[n].
    \end{align*}
    Then using the even-odd decomposition property of real signals for the DTFT, we see that 
    \begin{align*}
        \text{DTFT}[\operatorname{sgn}[n]] = \text{DTFT}[2x_o[n]] = j2\Im(X(e^{j\Omega})).
    \end{align*} 
    Therefore, for $X(e^{j\Omega})=\text{DTFT}[u[n]]$, 
    \begin{align*}
        \Im(X(e^{j\Omega})) &= \Im\left(\frac{1}{1-e^{-j\Omega}} + \sum_{k=-\infty}^{+\infty}\pi\delta(\Omega-2\pi k)\right) = \Im\left(\frac{1}{1-e^{-j\Omega}}\right) \\
        &= \Im\left(\frac{e^{+j\Omega/2}}{e^{+j\Omega/2}-e^{-j\Omega/2}}\right) = \Im\left(\frac{\cos(\Omega/2)+j\sin(\Omega/2)}{j2\sin(\Omega/2)}\right) \\
        &= \Im\left(\frac{\cos(\Omega/2)}{j2\sin(\Omega/2)}\right) = \Im\left(-j\frac{\cot(\Omega/2)}{2}\right) = -\frac{\cot(\Omega/2)}{2}. \\
        \Longrightarrow\text{DTFT}[\operatorname{sgn}[n]] &= j2\Im(X(e^{j\Omega})) = j2\left(-\frac{\cot(\Omega/2)}{2}\right) = -j\cot\left(\frac{\Omega}{2}\right).
    \end{align*}
\end{solution}

\begin{table}
    \centering
    \caption{DTFT pairs.}
    \label{dtft_pairs}
    \resizebox{\textwidth}{!}{%
    \begin{tabular}{|c|c|}
        \hline
        $x[n]$ & $X(e^{j\Omega})=\text{DTFT}[x[n]]$ \\[0.15cm]
        \hline
        & \\[0.1cm]
        $\delta[n]$ & $1$ \\[0.5cm]
        $\delta[n-n_0],\, n_0\in\mathbb{Z}$ & $e^{-j\Omega n_0}$ \\[0.5cm]
        $1$ & $2\pi\displaystyle\sum_{k=-\infty}^{+\infty}\delta(\Omega-2\pi k)$ \\[0.5cm]
        $u[n]$ & $\dfrac{1}{1-e^{-j\Omega}} + \displaystyle\sum_{k=-\infty}^{+\infty}\pi\delta(\Omega-2\pi k)$ \\[0.5cm]
        $\operatorname{sgn}[n]=u[n]-u[-n]$ & $-j\cot\left(\dfrac{\Omega}{2}\right)$ \\[0.5cm]
        $u[n]-u[n-N]$ & $e^{-j\Omega(N-1)/2}\left[\dfrac{\sin\left(\frac{N\Omega}{2}\right)}{\sin\left(\frac{\Omega}{2}\right)}\right]$ (modulated Dirichlet kernel) \\[0.5cm]
        $u[n+N]-u[n-(N+1)]$ & $\dfrac{\sin\left[\Omega\left(N+\frac{1}{2}\right)\right]}{\sin\left(\frac{\Omega}{2}\right)}$ (Dirichlet kernel) \\[0.5cm]
        $a^n u[n],\, |a|<1$ & $\dfrac{1}{1-ae^{-j\Omega}}$ \\[0.5cm]
        $a^n u[-n-1],\, |a|>1$ & $\dfrac{1}{1-ae^{-j\Omega}}$ \\[0.5cm]
        $a^{-n} u[-n-1],\, |a|<1$ & $\dfrac{ae^{j\Omega}}{1-ae^{j\Omega}}$ \\[0.5cm]
        $n\cdot a^n u[n],\, |a|<1$ & $\dfrac{ae^{j\Omega}}{(e^{j\Omega}-a)^2}$ \\[0.5cm]
        $e^{j\Omega_0 n}$ & $2\pi\displaystyle\sum_{k=-\infty}^{+\infty}\delta(\Omega-\Omega_0-2\pi k)$ \\[0.5cm]
        $\cos(\Omega_0 n)$ & $\pi\displaystyle\sum_{k=-\infty}^{+\infty}[\delta(\Omega-\Omega_0-2\pi k)+\delta(\Omega+\Omega_0-2\pi k)]$ \\[0.5cm]
        $\sin(\Omega_0 n)$ & $\dfrac{\pi}{j}\displaystyle\sum_{k=-\infty}^{+\infty}[\delta(\Omega-\Omega_0-2\pi k)-\delta(\Omega+\Omega_0-2\pi k)]$ \\[0.5cm]
        $a^n\cos(\Omega_0 n+\theta)u[n],\,|a|<1$ & $\dfrac{e^{j2\Omega}\cos\theta-ae^{j\Omega}\cos(\Omega_0-\theta)}{e^{j2\Omega}-2ae^{j\Omega}\cos\Omega_0+a^2}$ \\[0.5cm]
        $\dfrac{\sin(\Omega_0 n)}{\pi n}$ & $\displaystyle\sum_{k=-\infty}^{+\infty}[u(\Omega+\Omega_0-2\pi k)-u(\Omega-\Omega_0-2\pi k)] = \displaystyle\sum_{k=-\infty}^{+\infty}\operatorname{rect}\left(\dfrac{\Omega-2\pi k}{2\Omega_0}\right)$ \\[0.5cm]
        $\dfrac{1}{\sqrt{2\pi}\sigma}\exp\left(-\dfrac{n^2}{2\sigma^2}\right)$ (Gaussian pulse) & $\displaystyle\sum_{k=-\infty}^{+\infty}\exp\left(-\dfrac{\sigma^2(\Omega-2\pi k)^2}{2}\right)$ \\[0.5cm]
        \hline
    \end{tabular}
    }
\end{table}

\begin{example}
    Find the DTFT of the discrete-time impulse train 
    \begin{align*}
        \Sha[n] = \sum_{m=-\infty}^{+\infty}\delta[n-mN], \text{ for } N\in\mathbb{Z}^+.
    \end{align*}
\end{example}
\begin{solution}
    Note that the DT impulse train is merely an upsampled version of $x[n]=1$ by a factor of $N$. That is, there are $N-1$ zeros stuffed 
    between each sample of $x[n]$:
    \begin{align*}
        x[n] = 1 \Longrightarrow \Sha[n] = (\uparrow N)x[n] = x_N[n].
    \end{align*}
    Therefore, for $z=e^{j\Omega}$,
    \begin{align*}
        X_N(z) = X(z^N) \Longrightarrow X_N(e^{j\Omega}) = X(e^{j\Omega N}).
    \end{align*}
    Then for $X(e^{j\Omega}) = \text{DTFT}[1] = 2\pi\sum_{k=-\infty}^{+\infty}\delta(\Omega-2\pi k)$,
    \begin{align*}
        \text{DTFT}[\Sha[n]] = X(e^{j\Omega N}) &= 2\pi\sum_{k=-\infty}^{+\infty}\delta(\Omega N-2\pi k) \\
        &= 2\pi\sum_{k=-\infty}^{+\infty}\delta\left(N\left(\Omega -\frac{2\pi k}{N}\right)\right) \\
        &= \frac{2\pi}{N}\sum_{k=-\infty}^{+\infty}\delta\left(\Omega -\frac{2\pi k}{N}\right).
    \end{align*}
\end{solution}

\section{DTFT of Periodic Signals}
Analagous to the CT form, Fourier's theorem states that any periodic DT signal $x[n]$ can be expressed as a finite sum of DT sinusoids. Here, a periodic DT signal can be 
written in exponential form: 
\begin{align}
    x[n] &= \sum_{k=0}^{N_0-1} x_k e^{jk\Omega_0 n}  \\
    x_k &= \frac{1}{N_0}\sum_{n=0}^{N_0-1} x[n]e^{-jk\Omega_0 n}, \text{ for } \Omega_0 = 2\pi/N_0.
\end{align}
Then the DTFT of a periodic DT signal $x[n]$ is given by 
\begin{align}
    \text{DTFT}[x[n]] &= \text{DTFT}\left[\sum_{k=0}^{N_0-1} x_k e^{jk\Omega_0 n}\right] = \sum_{k=0}^{N_0-1} \text{DTFT}[x_k e^{jk\Omega_0 n}] \\
    &= \sum_{k=0}^{N_0-1} x_k 2\pi \sum_{m=-\infty}^{+\infty}\delta(\Omega - k\Omega_0 - 2\pi m) = 2\pi \sum_{k=-\infty}^{+\infty}x_k\delta(\Omega - k\Omega_0).
\end{align}

\section{Parseval's Theorem for DTFT}
Similar to the CT case, Parseval's theorem for the DTFT is essentially a ``conservation of energy'' theorem for any DT signal (aperiodic or periodic). 
When signals are mapped from the discrete-time function to the continuous-frequency DTFT spectrum, the total signal energy is conserved. It follows that 
the total energy of a DT signal $x[n]$ can be evaluated as:
\begin{align}
    E_x = \sum_{n=-\infty}^{+\infty} \left|x[n]\right|^2  = \frac{1}{2\pi}\int_{-\pi}^{+\pi} \left|X(e^{j\Omega})\right|^2 \,d\Omega
\end{align}
Additionally, we can define the \emph{one-sided energy spectral density} (1-sided ESD) to be 
\begin{align}
    ESD_1 = \frac{1}{\pi}\left|X(e^{j\Omega})\right|^2,
\end{align}
and the \emph{two-sided energy spectral density} (2-sided ESD) to be 
\begin{align}
    ESD_2 = \frac{1}{2\pi}\left|X(e^{j\Omega})\right|^2.
\end{align}
Note that the 2-sided ESD is defined for all frequencies, whereas the 1-sided ESD is defined for only nonnegative frequencies. 
Because of this, the values of the 2-sided ESD are half the values of the 1-sided ESD. In signal processing, we tend to be more interested in the 1-sided ESD. 
Regardless, because $X(e^{j\Omega})$ is periodic, the ESD is also periodic.

\section{DT LTI Systems with DTFT}
Similar to the $z$--transform, DT LTI systems characterized by LCCDEs can be solved by applying the DTFT, computing the $\Omega$-domain solution, 
and mapping the solution back to the time domain. 
\\ \\
An LCCDE of the form 
\begin{align}
    \sum_{k=0}^{N} a_k y[n-k] = \sum_{k=0}^{M} b_k x[n-k].
\end{align}
can be transformed to the $\Omega$-domain to get 
\begin{align}
    \left[\sum_{k=0}^{N} a_k e^{-j\Omega k}\right]Y(e^{j\Omega}) = \left[\sum_{k=0}^{M} b_k e^{-j\Omega k}\right]X(e^{j\Omega}).
\end{align}

\begin{example}
    A DT LTI system is characterized by the LCCDE 
    \begin{align*}
        y[n]+y[n-1] = x[n]-x[n-1].
    \end{align*}
    Find the system response to input $x[n]=(0.5)^n u[n]$.
\end{example}
\begin{solution}
    Transforming the LCCDE to $\Omega$-domain, we get 
    \begin{align*}
        [1+e^{-j\Omega}]Y(e^{j\Omega}) &= [1-e^{-j\Omega}]X(e^{j\Omega}) \\
        \Longrightarrow Y(e^{j\Omega}) &= \left[\frac{1-e^{-j\Omega}}{1+e^{-j\Omega}}\right]X(e^{j\Omega}).
    \end{align*}
    From the table of DTFT pairs, we see that
    \begin{align*}
        \text{DTFT}\left[(0.5)^n u[n]\right] = \frac{1}{1-0.5e^{-j\Omega}}.
    \end{align*}
    Therefore, by applying partial fraction expansion with respect to $z=e^{j\Omega}$, we get
    \begin{align*}
        Y(e^{j\Omega}) &= \left[\frac{1-e^{-j\Omega}}{1+e^{-j\Omega}}\right]\frac{1}{1-0.5e^{-j\Omega}} \\
        \Longrightarrow Y(z) &= \frac{1-z^{-1}}{(1+z^{-1})(1-0.5z^{-1})} = z\left[\frac{z-1}{(z+1)(z-0.5)}\right] = \frac{4/3}{1+z^{-1}} - \frac{1/3}{1-0.5z^{-1}} \\
        \Longrightarrow Y(e^{j\Omega}) &= \frac{4/3}{1+e^{-j\Omega}}-\frac{1/3}{1-0.5e^{-j\Omega}},
    \end{align*}
    and the solution becomes 
    \begin{align*}
        y[n] = \frac{4}{3}(-1)^n u[n] - \frac{1}{3}(0.5)^n u[n].
    \end{align*}
\end{solution}

\section{DT Frequency Response}
\begin{tcolorbox}[width=\textwidth,colback={white}, sharp corners]
    The \emph{discrete-time frequency response function} is the DTFT of the DT impulse response:
    \begin{align}
        H(e^{j\Omega}) = \frac{Y(e^{j\Omega})}{X(e^{j\Omega})} = \text{DTFT}[h[n]].
    \end{align}
\end{tcolorbox}
\noindent In fact, by the linear convolution property, it follows that 
\begin{align}
    y[n] = x[n] * h[n] \iff Y(e^{j\Omega}) = X(e^{j\Omega})H(e^{j\Omega}).
\end{align}
Symbolically,
\begin{center}
    \begin{tikzpicture}
        \node [] (input) at (-3,0){$x[n]$};
        \node [] (output) at (+3,0){$y[n]$};
        \node [draw,
            fill=yellow!50, 
            minimum width=2cm, 
            minimum height=1.2cm
        ] (system) at (0,0){$h[n]$};
        \draw [-latex, line width=1.5pt] (input) -- (system);
        \draw [-latex, line width=1.5pt] (system) -- (output);
    \end{tikzpicture} \\[0.25cm]
    \begin{tikzpicture}
        \node at (0,0) [double arrow, draw=black, top color=red, bottom color=blue,
            minimum width = 15pt, double arrow head extend=5pt,
            minimum height=12mm,
            rotate=90] {};
    \end{tikzpicture} \\[0.25cm]
    \begin{tikzpicture}
        \node [] (input) at (-3,0){$X(e^{j\Omega})$};
        \node [] (output) at (+3,0){$Y(e^{j\Omega})$};
        \node [draw,
            fill=yellow!50, 
            minimum width=2cm, 
            minimum height=1.2cm
        ] (system) at (0,0){$H(e^{j\Omega})$};
        \draw [-latex, line width=1.5pt] (input) -- (system);
        \draw [-latex, line width=1.5pt] (system) -- (output);
    \end{tikzpicture}
\end{center}
As the impulse response is the system response to input $x[n]=\delta[n]$, consider the DTFT of the impulse signal $\text{DTFT}[\delta[n]]=1$. Then it also follows that 
\begin{center}
    \begin{tikzpicture}
        \node [] (input) at (-3,0){$x[n]=\delta[n]$};
        \node [] (output) at (+3,0){$y[n]=h[n]$};
        \node [draw,
            fill=yellow!50, 
            minimum width=2cm, 
            minimum height=1.2cm
        ] (system) at (0,0){$h[n]$};
        \draw [-latex, line width=1.5pt] (input) -- (system);
        \draw [-latex, line width=1.5pt] (system) -- (output);
    \end{tikzpicture} \\[0.25cm]
    \begin{tikzpicture}
        \node at (0,0) [double arrow, draw=black, top color=red, bottom color=blue,
            minimum width = 15pt, double arrow head extend=5pt,
            minimum height=12mm,
            rotate=90] {};
    \end{tikzpicture} \\[0.25cm]
    \begin{tikzpicture}
        \node [] (input) at (-3,0){$X(e^{j\Omega})=1$};
        \node [] (output) at (+3,0){$Y(e^{j\Omega})=H(e^{j\Omega})$};
        \node [draw,
            fill=yellow!50, 
            minimum width=2cm, 
            minimum height=1.2cm
        ] (system) at (0,0){$H(e^{j\Omega})$};
        \draw [-latex, line width=1.5pt] (input) -- (system);
        \draw [-latex, line width=1.5pt] (system) -- (output);
    \end{tikzpicture}
\end{center}
Interestingly, the DT frequency response is related to the DT transfer function:
\begin{align}
    H(e^{j\Omega}) = H(z)\big|_{z=\exp(j\Omega)}
\end{align}
\begin{tcolorbox}[width=\textwidth,colback={white}, sharp corners]
    As before, there are two approaches to finding the frequency response of a DT LTI system:
    \begin{enumerate}
        \item Find $X(e^{j\Omega}),Y(e^{j\Omega})$. Then calculate $H(e^{j\Omega})=Y(e^{j\Omega})/X(e^{j\Omega})$.
        \item Find $X(e^{j\Omega}),Y(e^{j\Omega})$. Then substitute $X(e^{j\Omega})=1$ and $Y(e^{j\Omega})=H(e^{j\Omega})$.
    \end{enumerate}
\end{tcolorbox}
While the frequency response can be used for determining system stability just like the transfer function, 
the frequency response has the additional tool of analyzing the gain and phase of an LTI system at particular 
frequencies of interest. 
\\ \\
The magnitude $|H(e^{j\Omega})|$ of the DT frequency response is called the \emph{magnitude response} and has an associated \emph{phase reponse} $\phase{H(e^{j\Omega})}$. 
The plots of the magnitude response and phase response are two-sided and continuous and are collectively referred to as the \emph{frequency spectrum}. 
\\ \\ 
When expressed as $H(e^{j\Omega})=\dfrac{N(e^{j\Omega})}{D(e^{j\Omega})}$, the magnitude and phase responses can be computed as 
\begin{align}
    |H(e^{j\Omega})| &= \frac{|N(e^{j\Omega})|}{|D(e^{j\Omega})|} = \frac{\sqrt{[\Re(N(e^{j\Omega}))]^2 + [\Im(N(e^{j\Omega})]^2}}{\sqrt{[\Re(D(e^{j\Omega}))]^2 + [\Im(D(e^{j\Omega}))]^2}} \\
    \phase{H(e^{j\Omega})} &=  \phase{N(e^{j\Omega})} -  \phase{D(e^{j\Omega})} = \arctan\left(\frac{\Im(N(e^{j\Omega}))}{\Re(N(e^{j\Omega}))}\right) - \arctan\left(\frac{\Im(D(e^{j\Omega}))}{\Re(D(e^{j\Omega}))}\right)
\end{align}
Note that DT LTI systems characterized by LCCDEs have the frequency response 
\begin{align}
    H(e^{j\Omega}) = \frac{\displaystyle\sum_{k=0}^{M} b_k e^{-j\Omega k}}{\displaystyle\sum_{k=0}^{N} a_k e^{-j\Omega k}}.
\end{align}

\pagebreak
\section{Sample-Rate Conversion}
As explored before in the $z$-domain, we now can examine the frequency content behind sample-rate conversion by substituting $z=e^{j\Omega}$ in previous analyses. 
As a result, the importance of using filters for sample-rate conversions will become more obvious.

\subsection{Interpolation}
As a reminder, upsampling stuffs $L-1$ zeros between consecutive points of a sequence $x[n]$ to give the appearance of having sampled at rate $(L\cdot f_s)$. Its $z$--transform is 
given by
\begin{align}
    X_L(z) = X(z^L).
\end{align}
Substituting $z=e^{j\Omega}$, we get 
\begin{align}
    X_L(e^{j\Omega}) = X(e^{j(\Omega L)}).
\end{align}
Because the DTFT is $2\pi$-periodic in $\Omega$, $X_L(e^{j\Omega})$ is thus periodic. Let 
\begin{align}
    X_{\langle2\pi\rangle}(e^{j\Omega}) = X(e^{j\Omega}), \text{ for } \Omega\in[-\pi,\pi].
\end{align}
Then the spectrum of $X_L(e^{j\Omega})$ can be described as the sum of shifted 
compressed (by factor $L$) copies of $X_{\langle2\pi\rangle}(e^{j\Omega})$, centered at $\Omega_k = 2\pi k/L$.

\begin{center}
    \begin{table}[!hbt]
    \small
    \centering
    \caption{Upsampling.}
    \resizebox{\textwidth}{!}{%
    \begin{tabular}{ c|c }
    DT Sequence & DTFT Spectrum \\
    \hline 
    \adjustbox{valign=m}{\resizebox{0.4\textwidth}{!}{
        \begin{tikzpicture}
            [declare function={
                func(\x)= ((3 * sin(deg(\x)) / \x) + (\x == 0) * (3))^2 ; }]
            \begin{axis}[
                axis x line=center, axis y line=center,
                ymin=0, ymax=11, ytick={0}, ylabel={$x[n]$},
                xmin=-7.5, xmax=7.5, xtick={0},
                domain=-7:7, samples=15,
                extra x ticks={0}, extra x tick labels={\scriptsize\Centerstack{ \\ }}, 
                width=7cm, height=4cm]
                \addplot+ [
                    ycomb,
                ] {func(x)};
            \end{axis}
        \end{tikzpicture}}} & 
    \adjustbox{valign=m}{\resizebox{0.4\textwidth}{!}{
        \begin{tikzpicture}
            [declare function={
                func(\x)= and(\x > -2, \x < 2) * (1-abs(\x/2)); }]
            \begin{axis}[
                axis x line=center, axis y line=center,
                ymin=0, ymax=2, ytick={0}, ylabel={$X(e^{j\Omega})$},
                xmin=-7.5, xmax=7.5, xtick={0},
                domain=-7:7, samples=200,
                extra x ticks={-2*pi,-pi,-2,2,pi,2*pi}, extra x tick labels={\scriptsize\Centerstack{$-2\pi$\\ }, \scriptsize\Centerstack{$-\pi$\\ }, \scriptsize\Centerstack{$-B$\\ }, \scriptsize\Centerstack{$B$\\ }, \scriptsize\Centerstack{$\pi$\\ }, \scriptsize\Centerstack{$2\pi$\\ }}, 
                width=7cm, height=4cm]
                \addplot [blue,thick]{func(x)+func(x+2*pi)+func(x-2*pi)};
                \draw[red,dashed] (-pi,0)--(-pi,1.1)--(pi,1.1)--(pi,0);
                \node[violet] at (0,1.25) {\scriptsize$X_{\langle 2\pi\rangle}(e^{j\Omega})$};
            \end{axis}
        \end{tikzpicture}}} \\[1cm]
    \adjustbox{valign=m}{\resizebox{0.4\textwidth}{!}{
        \begin{tikzpicture}
            \begin{axis}[
                axis x line=center, axis y line=center,
                ymin=0, ymax=11, ytick={0}, ylabel={$x_L[n]$},
                xmin=-7.5, xmax=7.5, xtick={0},
                extra x ticks={1.5}, extra x tick labels={\scriptsize\Centerstack{$L-1$\\zeros}}, 
                width=7cm, height=4cm]
                \addplot+ [
                    ycomb,
                ] coordinates {(-6,1.8604) (-5,0) (-4.5,0) (-4,0) (-3,6.3727) (-2,0) (-1.5,0) (-1,0) (0,9) (1,0) (1.5,0) (2,0) (3,6.3727) (4,0) (4.5,0) (5,0) (6,1.8604)};
            \end{axis}
        \end{tikzpicture}}} & 
    \adjustbox{valign=m}{\resizebox{0.4\textwidth}{!}{
        \begin{tikzpicture}
            [declare function={
                func(\x)= and(\x > -1, \x < 1) * (1-abs(\x)); }]
            \begin{axis}[
                axis x line=center, axis y line=center,
                ymin=0, ymax=2, ytick={0}, ylabel={$X_L(e^{j\Omega})$},
                xmin=-7.5, xmax=7.5, xtick={0},
                domain=-7:7, samples=200,
                extra x ticks={-2*pi,-pi,-1,1,pi,2*pi}, extra x tick labels={\scriptsize$-\dfrac{4\pi}{L}$, \scriptsize$-\dfrac{2\pi}{L}$, \scriptsize$-\dfrac{B}{L}$, \scriptsize$\dfrac{B}{L}$, \scriptsize$\dfrac{2\pi}{L}$, \scriptsize$\dfrac{4\pi}{L}$}, 
                width=7cm, height=4cm]
                \addplot [blue,thick]{func(x)+func(x+2*pi)+func(x+pi)+func(x-2*pi)+func(x-pi)};
            \end{axis}
        \end{tikzpicture}}} \\[1cm]
        \hline
    \end{tabular}
    }
    \end{table}
\end{center}

Of course, upsampling is not a perfect reconstruction of what a CT signal sampled at a higher rate would look like, due to the presence of stuffed zeros. In actuality, these zeros would be replaced with 
sampled values of the original CT signal being sampled. 
\\ \\
Here, we introduce the ideal reconstruction filter, which aims to remove image spectra such that the time-domain output is a perfectly reconstructed signal at a higher sample rate. Let 
\begin{align}
    H_{\langle 2\pi\rangle}(e^{j\Omega}) = 
    \begin{cases}
        1, & 0\leq|\Omega|\leq\pi/L \\
        0, & \pi/L<|\Omega|\leq\pi.
    \end{cases}
\end{align}
Note that the cutoff frequency of this ideal lowpass filter is at $|\Omega_c|=\pi/L$. Since the DTFT is $2\pi$-periodic in $\Omega$, the ideal reconstruction filter is then given by 
\begin{align}
    H(e^{j\Omega}) = \sum_{k=-\infty}^{+\infty} H_{\langle 2\pi\rangle}(e^{j(\Omega+2\pi k)}).
\end{align}
The process of upsampling followed by a reconstruction filter is called \emph{interpolation}.
\begin{center}
    \begin{tikzpicture}
        \node [] (input) at (-3,0){$x[n]$};
        \node [] (output) at (+6,0){\Centerstack{$x_L[n]$\\(interpolated)}};
        \node [draw,
            fill=yellow!50, 
            minimum width=2cm, 
            minimum height=1.2cm
        ] (up) at (0,0){$\uparrow L$};
        \node [draw,
            fill=green!30, 
            minimum width=2cm, 
            minimum height=1.2cm
        ] (lpf) at (3,0){\Centerstack{Lowpass\\Filter}};
        \draw [-latex, line width=1.5pt] (input) -- (up);
        \draw [-latex, line width=1.5pt] (up) -- (lpf);
        \draw [-latex, line width=1.5pt] (lpf) -- (output);
    \end{tikzpicture}
\end{center}

\subsection{Decimation}
As a reminder, downsampling only retains every $M^{th}$ point of a sequence $x[n]$ to give the appearance of having sampled at rate $(f_s/M)$. Its $z$--transform is 
given by
\begin{align}
    X_{1/M}(z) = \frac{1}{M} \sum_{r=0}^{M-1} X\left(z^{1/M}e^{-j2\pi r/M}\right).
\end{align}
Substituting $z=e^{j\Omega}$, we get 
\begin{align}
    X_{1/M}(e^{j\Omega}) = \frac{1}{M} \sum_{r=0}^{M-1} X\left(e^{j\Omega/M}e^{-j2\pi r/M}\right) = \frac{1}{M} \sum_{r=0}^{M-1} X\left(e^{j(\Omega-2\pi r)/M}\right).
\end{align}
Because the DTFT is $2\pi$-periodic in $\Omega$, $X_{1/M}(e^{j\Omega})$ is thus periodic. Again let 
\begin{align}
    X_{\langle2\pi\rangle}(e^{j\Omega}) = X(e^{j\Omega}), \text{ for } \Omega\in[-\pi,\pi].
\end{align}
Then the spectrum of $X_{1/M}(e^{j\Omega})$ can be described as the sum of shifted 
expanded (by factor $M$) copies of $X_{\langle2\pi\rangle}(e^{j\Omega})$, centered at $\Omega_k = 2\pi k$, albeit attenuated.
\begin{center}
    \begin{table}[!hbt]
    \small
    \centering
    \caption{Downsampling.}
    \resizebox{\textwidth}{!}{%
    \begin{tabular}{ c|c }
    DT Sequence & DTFT Spectrum \\
    \hline 
    \adjustbox{valign=m}{\resizebox{0.4\textwidth}{!}{
        \begin{tikzpicture}
            [declare function={
                func(\x)= ((3 * sin(deg(\x)) / \x) + (\x == 0) * (3))^2 ; }]
            \begin{axis}[
                axis x line=center, axis y line=center,
                ymin=0, ymax=11, ytick={0}, ylabel={$x[n]$},
                xmin=-7.5, xmax=7.5, xtick={0},
                domain=-7:7, samples=15,
                extra x ticks={0}, extra x tick labels={\scriptsize\Centerstack{ \\ }}, 
                width=7cm, height=4cm]
                \addplot+ [
                    ycomb,
                ] {func(x)};
            \end{axis}
        \end{tikzpicture}}} & 
    \adjustbox{valign=m}{\resizebox{0.4\textwidth}{!}{
        \begin{tikzpicture}
            [declare function={
                func(\x)= and(\x > -2, \x < 2) * (1-abs(\x/2)); }]
            \begin{axis}[
                axis x line=center, axis y line=center,
                ymin=0, ymax=2, ytick={0}, ylabel={$X(e^{j\Omega})$},
                xmin=-7.5, xmax=7.5, xtick={0},
                domain=-7:7, samples=200,
                extra x ticks={-2*pi,-pi,-2,2,pi,2*pi}, extra x tick labels={\scriptsize\Centerstack{$-2\pi$\\ }, \scriptsize\Centerstack{$-\pi$\\ }, \scriptsize\Centerstack{$-B$\\ }, \scriptsize\Centerstack{$B$\\ }, \scriptsize\Centerstack{$\pi$\\ }, \scriptsize\Centerstack{$2\pi$\\ }}, 
                width=7cm, height=4cm]
                \addplot [blue,thick]{func(x)+func(x+2*pi)+func(x-2*pi)};
                \draw[red,dashed] (-pi,0)--(-pi,1.1)--(pi,1.1)--(pi,0);
                \node[violet] at (0,1.25) {\scriptsize$X_{\langle 2\pi\rangle}(e^{j\Omega})$};
            \end{axis}
        \end{tikzpicture}}} \\[1cm]
    \adjustbox{valign=m}{\resizebox{0.4\textwidth}{!}{
        \begin{tikzpicture}
            [declare function={
                func(\x)= ((3 * sin(deg(\x)) / \x) + (\x == 0) * (3))^2 ; }]
            \begin{axis}[
                axis x line=center, axis y line=center,
                ymin=0, ymax=11, ytick={0}, ylabel={$x_{1/M}[n]$},
                xmin=-7.5, xmax=7.5, xtick={0},
                domain=-7:7, samples=15,
                extra x ticks={0}, extra x tick labels={\scriptsize\Centerstack{ \\ }}, 
                width=7cm, height=4cm]
                \addplot+ [
                    ycomb,
                ] {func(2*x)};
            \end{axis}
        \end{tikzpicture}}} & 
    \adjustbox{valign=m}{\resizebox{0.4\textwidth}{!}{
        \begin{tikzpicture}
            [declare function={
                func(\x)= and(\x > -4, \x < 4) * (1-abs(\x/4)); }]
            \begin{axis}[
                axis x line=center, axis y line=center,
                ymin=0, ymax=2, ytick={0}, ylabel={$X_{1/M}(e^{j\Omega})$},
                xmin=-7.5, xmax=7.5, xtick={0},
                domain=-7:7, samples=200,
                extra x ticks={-2*pi,-4,4,2*pi}, extra x tick labels={\scriptsize\Centerstack{$-2\pi$\\ }, \scriptsize\Centerstack{$-MB$\\ }, \scriptsize\Centerstack{$MB$\\ }, \scriptsize\Centerstack{$2\pi$\\ }}, 
                extra y ticks={0.5}, extra y tick labels={\scriptsize$1/M$}, 
                width=7cm, height=4cm]
                \addplot [blue,thick]{func(x)/2};
                \addplot [red,thick]{func(x+2*pi)/2};
                \addplot [red,thick]{func(x-2*pi)/2};
                \node[] at (3,0.5) {\scriptsize\Centerstack{Aliasing\\$\downarrow$}};
            \end{axis}
        \end{tikzpicture}}} \\[1cm]
        & \adjustbox{valign=m}{\resizebox{0.4\textwidth}{!}{
            \begin{tikzpicture}
                [declare function={
                    func(\x)= and(\x > -4, \x < 4) * (1-abs(\x/4)); }]
                \begin{axis}[
                    axis x line=center, axis y line=center,
                    ymin=0, ymax=2, ytick={0}, ylabel={$X_{1/M}(e^{j\Omega})$},
                    xmin=-7.5, xmax=7.5, xtick={0},
                    domain=-7:7, samples=200,
                    extra x ticks={-2*pi,-4,4,2*pi}, extra x tick labels={\scriptsize\Centerstack{$-2\pi$\\ }, \scriptsize\Centerstack{$-MB$\\ }, \scriptsize\Centerstack{$MB$\\ }, \scriptsize\Centerstack{$2\pi$\\ }}, 
                    extra y ticks={0.5}, extra y tick labels={\scriptsize$1/M$}, 
                    width=7cm, height=4cm]
                    \addplot [blue,thick]{func(x)/2 + func(x+2*pi)/2 + func(x-2*pi)/2};
                    \node[] at (3,0.5) {\scriptsize\Centerstack{Aliasing\\$\downarrow$}};
                \end{axis}
            \end{tikzpicture}}} \\[1cm]
        \hline
    \end{tabular}
    }
    \end{table}
\end{center}
However, unlike upsampling, there is a risk of aliasing with downsampling. Let $B\in[0,\pi]$ be the maximum frequency in $X_{\langle 2\pi\rangle}(e^{j\Omega})$ 
that contains nonzero frequency content. Then the only mathematical instance when aliasing is not an issue is when 
\begin{align}
    MB < 2\pi - MB \Longrightarrow B < \frac{\pi}{M} \text{ [rad/samp]},
\end{align}
which rarely happens as this would require the data to contain only very low frequency content. Therefore, there is a need to apply some sort
of filter before downsampling.

\pagebreak
\noindent\emph{Decimation} then is the process of applying an anti-aliasing filter followed by downsampling.
\begin{center}
    \begin{tikzpicture}
        \node [] (input) at (-3,0){$x[n]$};
        \node [] (output) at (+6,0){\Centerstack{$x_{1/M}[n]$\\(decimated)}};
        \node [draw,
            fill=blue!30, 
            minimum width=2cm, 
            minimum height=1.2cm
        ] (lpf) at (0,0){\Centerstack{Lowpass\\Filter}};
        \node [draw,
            fill=red!30, 
            minimum width=2cm, 
            minimum height=1.2cm
        ] (down) at (3,0){$\downarrow M$};
        \draw [-latex, line width=1.5pt] (input) -- (lpf);
        \draw [-latex, line width=1.5pt] (lpf) -- (down);
        \draw [-latex, line width=1.5pt] (down) -- (output);
    \end{tikzpicture}
\end{center}
The ideal anti-aliasing filter aims to avoid aliasing by removing higher frequency content in $X_{\langle 2\pi\rangle}(e^{j\Omega})$ such that the image spectra 
of $X_{DEC}(e^{j\Omega})$ will never overlap. As such, the filter does introduce some information loss, but the alternative is having corrupted data at such frequencies. 
Let 
\begin{align}
    H_{\langle 2\pi\rangle}(e^{j\Omega}) = 
    \begin{cases}
        1, & 0\leq|\Omega|\leq\pi/M \\
        0, & \pi/M<|\Omega|\leq\pi.
    \end{cases}
\end{align}
Note that the cutoff frequency of this ideal lowpass filter is at $|\Omega_c|=\pi/M$. Since the DTFT is $2\pi$-periodic in $\Omega$, the ideal anti-aliasing filter is then given by 
\begin{align}
    H(e^{j\Omega}) = \sum_{k=-\infty}^{+\infty} H_{\langle 2\pi\rangle}(e^{j(\Omega+2\pi k)}).
\end{align}

\subsection{Resampling}
As a reminder, resampling is the combination of upsampling and downsampling with a lowpass filter to attain a sampling rate change by a non-integer factor. Now we see that a filter 
following upsampling is required to obtain interpolated data points, while having a lowpass filter before downsampling is required to prevent aliasing. 
\\ \\
In fact, resampling is essentially just an interpolator and a decimator in cascade. 
\begin{center}
    \begin{tikzpicture}
        \node [] (input) at (-5.25,0){$x[n]$};
        \node [] (output) at (+8.75,0){\Centerstack{$x_{L/M}[n]$\\(resampled)}};
        \node [draw,
            fill=yellow!50, 
            minimum width=1.8cm, 
            minimum height=1.2cm
        ] (up) at (-3,0){$\uparrow L$};
        \node [draw,
            fill=green!30, 
            minimum width=1.8cm, 
            minimum height=1.2cm
        ] (lpf1) at (0,0){\Centerstack{Lowpass\\Filter,\\$|\Omega_c|=\pi/L$}};
        \node [draw,
            fill=blue!30, 
            minimum width=1.8cm, 
            minimum height=1.2cm
        ] (lpf2) at (3,0){\Centerstack{Lowpass\\Filter,\\$|\Omega_c|=\pi/M$}};
        \node [draw,
            fill=red!30, 
            minimum width=1.8cm, 
            minimum height=1.2cm
        ] (down) at (6,0){$\downarrow M$};
        \draw [-latex, line width=1.5pt] (input) -- (up);
        \draw [-latex, line width=1.5pt] (up) -- (lpf1);
        \draw [-latex, line width=1.5pt] (lpf1) -- (lpf2);
        \draw [-latex, line width=1.5pt] (lpf2) -- (down);
        \draw [-latex, line width=1.5pt] (down) -- (output);
        \draw[dashed] (-4.5,-1.5)--(-4.5,1.5)--(1.2,1.5)--(1.2,-1.5)--(-4.5,-1.5);
        \node[] at (-3,1.1) {Interpolator};
        \draw[dashed] (7.2,-1.5)--(7.2,1.5)--(1.5,1.5)--(1.5,-1.5)--(7.2,-1.5);
        \node[] at (3,1.1) {Decimator};
    \end{tikzpicture}
\end{center}
The two lowpass filters can be combined into a single lowpass filter such that 
\begin{center}
    \begin{tikzpicture}
        \node [] (input) at (-3,0){$x[n]$};
        \node [] (output) at (+9,0){$x_{L/M}[n]$};
        \node [draw,
            fill=yellow!50, 
            minimum width=2cm, 
            minimum height=1.2cm
        ] (up) at (0,0){$\uparrow L$};
        \node [draw,
            fill=violet!30, 
            minimum width=2cm, 
            minimum height=1.2cm
        ] (lpf) at (3,0){\shortstack{Lowpass\\Filter}};
        \node [draw,
            fill=red!30, 
            minimum width=2cm, 
            minimum height=1.2cm
        ] (down) at (6,0){$\downarrow M$};
        \draw [-latex, line width=1.5pt] (input) -- (up);
        \draw [-latex, line width=1.5pt] (up) -- (lpf);
        \draw [-latex, line width=1.5pt] (lpf) -- (down);
        \draw [-latex, line width=1.5pt] (down) -- (output);
    \end{tikzpicture}
\end{center}
Here, the ideal lowpass filter is given by
\begin{align}
    H(e^{j\Omega}) &= \sum_{k=-\infty}^{+\infty} H_{\langle 2\pi\rangle}(e^{j(\Omega+2\pi k)}), \text{ where } \\
    H_{\langle 2\pi\rangle}(e^{j\Omega}) &= 
    \begin{cases}
        1, & 0\leq|\Omega|\leq\min(\pi/L,\pi/M) \\
        0, & \min(\pi/L,\pi/M)<|\Omega|\leq\pi.
    \end{cases}
\end{align}
Note that the cutoff frequency of the lowpass filter is $|\Omega_c|=\min(\pi/L,\pi/M)$, which is the result of 
multiplying the frequency response functions of the ideal reconstruction filter and the ideal anti-aliasing filter.
\\ \\
Lastly, note that interpolation is done before decimation. Since decimation already suggests some information loss, it is 
better to interpolate the data first to minimize as much loss as possible.


\end{document}