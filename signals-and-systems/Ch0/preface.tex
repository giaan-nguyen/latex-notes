\documentclass{report}
% PACKAGES
\usepackage{adjustbox}
\usepackage{amsmath}
\usepackage{amssymb}
\usepackage{bodegraph}
\usepackage{bbm}
\usepackage{circledsteps}
\usepackage{circuitikz}
\usepackage{enumerate}
\usepackage{mathtools}
\usepackage{nicematrix}
\usepackage{pgfplots}
\usepackage{polynom}
\usepackage{qtree}
\usepackage{rotating}
\usepackage[usestackEOL]{stackengine}
\usepackage[subpreambles=true]{standalone}
\usepackage{steinmetz}
\usepackage{subcaption}
\usepackage{tabularray}
\usepackage{tcolorbox}
\usepackage{tikz}
\usepackage{xcolor}

\usepackage[colorlinks=true,linkcolor=blue,urlcolor=black,bookmarksopen=true]{hyperref}
\usepackage{bookmark}

\usetikzlibrary{shapes.arrows}
\usetikzlibrary{shapes.misc}
\usetikzlibrary{backgrounds}
\tikzset{cross/.style={cross out, draw=black, minimum size=2*(#1-\pgflinewidth), inner sep=0pt, outer sep=0pt},
%default radius will be 1pt. 
cross/.default={1pt}}

\renewcommand{\Re}{\operatorname{Re}}
\renewcommand{\Im}{\operatorname{Im}}

\usepackage{pifont}
\newcommand{\cmark}{\text{\ding{51}}}
\newcommand{\xmark}{\text{\ding{55}}}

\newcommand{\circconv}[1]{\text{ \small\Circled{#1} }}

\newcommand{\tikzmark}[3][]{\tikz[remember picture,baseline] \node [anchor=base,#1](#2) {$#3$};}

\pgfplotsset{compat=1.18}
\pgfplotsset{
    dirac/.style={
        mark=triangle*,
        mark options={scale=1.5},
        ycomb,
        scatter,
        visualization depends on={y/abs(y)-1 \as \sign},
        scatter/@pre marker code/.code={\scope[rotate=90*\sign,yshift=-2pt]}
    }
}

\usepackage[letterpaper, portrait, margin=1.25in]{geometry}
\usepackage[font=bf]{caption}
\hbadness = 10000
\hfuzz=2pt

\newtheorem{theorem}{Theorem}[chapter]

\tikzset{every tree node/.style={anchor=north,align=center}}
\usetikzlibrary{decorations.markings}
\usetikzlibrary{arrows}
\definecolor{darkgreen}{rgb}{0.133,0.545,0.133}
\tikzstyle{n}= [circle, fill=blue, minimum size=4pt,inner sep=0pt, outer sep=0pt]

\patchcmd{\thebibliography}{\chapter*}{\section*}{}{}
\setcounter{secnumdepth}{5}

%%%%%%%%%%% EXAMPLE ENVIRONMENT %%%%%%%%%%
\usepackage{calc}
\usepackage{tabto}
\usepackage[framemethod=tikz]{mdframed} % for colored backgrounds
\newcommand{\halmos}{} % makes a box at the end

\newlength{\framedinnerleftmargin}
\newlength{\framedinnertopmargin}
\newlength{\framedreversedinnerleftmargin}
\setlength{\framedinnerleftmargin}{\widthof{Theoreme 10.10.10}+2em}
\setlength{\framedreversedinnerleftmargin}{\widthof{Theoreme 10.10.10}+1em}
\setlength{\framedinnertopmargin}{1em}
 
% first argument: label in upper left corner,
% second argument: background color
\newenvironment{boxedtext}[2]{\begin{mdframed}[%
hidealllines=true,%
backgroundcolor=#2,%
innertopmargin=\framedinnertopmargin,%
innerleftmargin=\framedinnerleftmargin,%
innerrightmargin=1em%
]%
\tabto{-\framedreversedinnerleftmargin}\textbf{#1}\tabto*{0em}%
}% begin code
{\hskip 0pt\\\hspace*{\fill}\halmos{}\end{mdframed}\vspace{1em}} % end code
 
\newenvironment{summary}[0]{\begin{center}\begin{minipage}[c]{\summarywidth}\begin{spacing}{0.9}\footnotesize} % begin code
{\end{spacing}\end{minipage}\end{center}} % end code

\newcounter{example}
 
% optional! if you want it to start at zero
% with every new chapter/section/etc.
\numberwithin{example}{section}
 
\newenvironment{example}[0]
{\refstepcounter{example}\vspace{1em plus 1em}\begin{boxedtext}{Example \theexample.}{blue!7}}%\setlength{\parskip}{0em}}
{\end{boxedtext}\vspace{-1em plus 1em}}
 
\newenvironment{example*}[0]
{\vspace{1em plus 1em}\begin{boxedtext}{Example.}{blue!7}}
{\end{boxedtext}\vspace{-1em plus 1em}}

\newmdenv[
  topline=false,
  bottomline=false,
  rightline=false,
  skipabove=\topsep,
  skipbelow=\topsep,
  linecolor=purple,
  frametitle={\noindent\textcolor{purple}{\textbf{SOLUTION }}},
  endinnercode={$\hfill\textcolor{purple}{\blacksquare}$}
]{solution}

%%%%%%%%%%%%%% COLOR BOXED %%%%%%%%%%%%%%%
% Syntax: \colorboxed[<color model>]{<color specification>}{<math formula>}
\newcommand*{\colorboxed}{}
\def\colorboxed#1#{%
  \colorboxedAux{#1}%
}
\newcommand*{\colorboxedAux}[3]{%
  % #1: optional argument for color model
  % #2: color specification
  % #3: formula
  \begingroup
    \colorlet{cb@saved}{.}%
    \color#1{#2}%
    \boxed{%
      \color{cb@saved}%
      #3%
    }%
  \endgroup
}
%%%%%%%%%%%%%%%%%%%%%%%%%%%%%%%%%%%%%%%%%
\pagenumbering{roman}
\renewcommand\theequation{P-\arabic{equation}}

\begin{document}

\chapter*{Preface}

\section*{Complex Numbers}
\subsection*{Definition of Complex Numbers}
A complex number $z$ can be written in \emph{rectangular form} (also called \emph{Cartesian form})
\begin{align}
    z = x + jy,
\end{align}
where $x,y$ are real numbers and $j=\sqrt{-1}$ is the \emph{imaginary unit}. 
\\ \\
On its own, $x$ is called the \emph{real part} of $z$, and $y$ is called the \emph{imaginary part} of $z$. This can be written as 
\begin{align}
    \Re(z) &= x \\
    \Im(z) &= y.
\end{align}
Alternatively, $z$ can be written in \emph{polar form}
\begin{align}
    z = |z|e^{j\theta} = |z|\phase{\theta},
\end{align}
where 
\begin{align}
    |z|=\sqrt{x^2+y^2}
\end{align} 
is the magnitude of $z$, and 
\begin{align}
    \theta = \arg(z) = \arg(x+jy) = \operatorname{atan2}(y,x) = 
    \begin{cases}
        \arctan\left(\dfrac{y}{x}\right), & x > 0 \\[0.25cm]
        \arctan\left(\dfrac{y}{x}\right) \pm \pi, & x < 0 \\[0.25cm]
        \operatorname{sgn}(y)\cdot\dfrac{\pi}{2}, & x = 0 \text{ and } y \neq 0 \\[0.25cm]
        \text{undefined}, & x = 0 \text{ and } y = 0 
    \end{cases}
\end{align}
is its phase angle, calculated from the four-quadrant inverse tangent function $\operatorname{atan2}$.
\\ \\
From \emph{Euler's formula},
\begin{align}
    e^{j\theta} = \cos\theta + j\sin\theta.
\end{align}
Combining Eqs. (1), (4), and (7), it also then follows that 
\begin{align}
    x &= |z| \cos(\theta), \\
    y &= |z| \sin(\theta).
\end{align} 
\newpage
\noindent Graphically, a complex number $z_0$ can be drawn on the \emph{complex plane} as a vector from the origin to point $(x_0,y_0)$, with $x=\Re(z)$ as the 
horizontal axis and $y=\Im(z)$ as the vertical axis. 
\begin{center}
    \begin{tikzpicture}[scale=2.5]
        \draw (0,-0.25) -- (0,1.25) node[above] {\large $y=\Im(z)$} (-0.25,0)--(1.25,0) node[right] {\large $x=\Re(z)$};
        \draw [-latex, very thick, color=blue] (0,0) -- (1,1);
        \draw [dashed] (1,0) node[below] {\large $x_0$} -- (1,1);
        \draw[fill=black](1,1) circle (0.75 pt) node [above] {\large $z_0$};
        \draw [dashed] (0,1) node[left] {\large $y_0$} -- (1,1);
        \draw (0.35,0.6) node {\large $|z_0|$};
        \path [->] (0.5,0) edge[bend right] node [left] {} ({0.5/sqrt(2)},{0.5/sqrt(2)});
        \draw (0.65,0.25) node {\large $\theta_0$};
    \end{tikzpicture}
\end{center}

\subsection*{Operations with Complex Numbers}
The \emph{complex conjugate} of $z$ is given by 
\begin{align}
    z^* = (x+jy)^* = x-jy.
\end{align}
This is achieved by substituting $j\leftarrow (-j)$ and can be written as 
\begin{align}
    z^* = \big[z\big|_{j\leftarrow (-j)} = \big[x+jy\big|_{j\leftarrow (-j)} = x-jy.
\end{align}
It then follows that the norm of $z$ is
\begin{align}
    |z| = \sqrt{zz^*} = \sqrt{x^2+y^2}.
\end{align}
Some operations between two complex numbers 
\begin{align*}
    z_1=x_1+jy_1=|z_1|e^{j\theta_1} \\
    z_2=x_2+jy_2=|z_2|e^{j\theta_2}
\end{align*}
are defined in the following list:
\begin{itemize}
    \item Addition/subtraction: $z_1 \pm z_2=(x_1\pm x_2) + j(y_1\pm y_2)$
    \item Multiplication: $z_1z_2=|z_1||z_2|e^{j(\theta_1+\theta_2)}$
    \item Division: $\dfrac{z_1}{z_2}=\dfrac{|z_1|}{|z_2|}e^{j(\theta_1-\theta_2)}$
\end{itemize}
Recall from \emph{Euler's formula} that $e^{j\theta} = \cos\theta + j\sin\theta$. The sinusoids can be rewritten such that
\begin{align}
    \cos\theta = \frac{e^{j\theta}+e^{-j\theta}}{2} \\
    \sin\theta = \frac{e^{j\theta}-e^{-j\theta}}{j2}.
\end{align}
Generalizing Euler's formula, exponentiation of a complex number gives
\begin{align}
    e^z = e^{x+jy} = e^x e^{jy} = e^x(\cos\theta +j\sin\theta).
\end{align}
\emph{De Moivre's formula} defines the $n^{th}$ power of a complex number $z$, for $n$ is a positive integer:
\begin{align}
    z^n = [|z|(\cos\theta + j\sin\theta)]^n = |z|^n[\cos(n\theta)+j\sin(n\theta)].
\end{align}
In a similar vein, the $n^{th}$ root of a complex number $z$ (for $n$ is a positive integer) is
\begin{align}
    z^{1/n} &= [|z|(\cos\theta + j\sin\theta)]^{1/n} \nonumber \\ 
    &= |z|^{1/n}\left[\cos\left(\frac{\theta+2k\pi}{n}\right)+j\sin\left(\frac{\theta+2k\pi}{n}\right)\right], \text{ for } k=0,1,2,...,n-1.
\end{align}
Solutions to the equation $z^n=1$ are called \emph{$n^{th}$ roots of unity} and are defined as 
\begin{align}
    z=\cos\left(\frac{2k\pi}{n}\right)+j\sin\left(\frac{2k\pi}{n}\right)=e^{j2k\pi/n}, \text{ for } k=0,1,2,...,n-1.
\end{align}
The following are common equivalences for powers of imaginary unit $j$:
\begin{align}
    j &= \sqrt{-1} = e^{j\pi/2} \\ 
    j^2 &= -1 = e^{-j\pi} \\
    j^3 &= -j = e^{-j \pi/2} \\
    j^4 &= 1 \\
    \sqrt{j} &= \pm e^{j\pi/4} = \pm \frac{(1+j)}{\sqrt{2}} \\
    \sqrt{-j} &= \pm e^{-j\pi/4} = \pm \frac{(1-j)}{\sqrt{2}}
\end{align}

\subsection*{Regions in the Complex Plane}
Let $z$ be a complex variable and $z_0=x_0+jy_0$ be a complex number. Then 
\begin{align}
    |z-z_0| = r
\end{align} 
graphically represents a circle of radius 2 centered at $(x_0,y_0)$. In the inequality form,
\begin{align}
    |z-z_0| < r
\end{align} 
represents the interior of the circle, whereas 
\begin{align}
    |z-z_0| > r
\end{align} 
represents the entire region of the complex plane outside of the circle. 
\begin{center}
    \begin{tikzpicture} [scale=2.5]
        \draw (0,-1.25) -- (0,0.25) node[above] {\large $y=\Im(z)$} (-0.25,0)--(1.25,0) node[right] {\large $x=\Re(z)$};
        \draw[fill=none, color=darkgreen](0.5,-0.5) circle (0.5) node [yshift=-1.6cm] {\large $|z-z_0|=r$};
        \draw[fill=black](0.5,-0.5) circle (0.75 pt) node [above] {\large $z_0$};
        \draw[](0.5,-0.5) -- (1.0,-0.5) node [midway,above] {\large $r$};
    \end{tikzpicture}
\end{center}
This can be proven by plugging in $z=x+jy$ and $z_0=x_0+jy_0$ to get the rectangular equation of a circle
\begin{align}
    \sqrt{(x-x_0)^2+(y-y_0)^2} = r \Longrightarrow (x-x_0)^2+(y-y_0)^2 = r^2.
\end{align} 
$\Re(z_0)$ represents a vertical line at $x=x_0$, and $\Im(z_0)$ represents a horizontal line at $y=y_0$. The inequality forms 
are evident in the rectangular form.
\begin{center}
    \begin{tikzpicture} [scale=2.5]
        \draw (0,-1.25) -- (0,1.25) node[above] {\large $y=\Im(z)$} (-1.25,0)--(1.25,0) node[right] {\large $x=\Re(z)$};
        \draw[dashed, blue](0.5,-1) -- (0.5,1) node[above] {\normalsize $x=\Re(z_0)$};
        \draw[dashed, red](-1,-0.5) -- (1,-0.5) node[right] {\normalsize $y=\Im(z_0)$};
    \end{tikzpicture}
\end{center}
In general, any region described by a complex equation or inequality can be identified by solving in rectangular form and using candidate values of $z$ 
to test the regions of validity.

\subsection*{Complex Functions of Time}
Just as with complex numbers and complex variables, complex functions of time $z(t)$ also follow many of the same properties. In rectangular form,
\begin{align}
    z(t)&=x(t)+jy(t) \\
    \Re[z(t)] &= x(t) \\
    \Im[z(t)] &= y(t).
\end{align}
In polar form,
\begin{align}
    z(t)&=|z(t)|\exp[{j\theta(t)}] \\
    |z(t)| &= \sqrt{x^2(t)+y^2(t)} \\
    \theta(t) &= \arg[x(t)+jy(t)].
\end{align}
Here, while the Steinmetz phasor notation $\phase{\theta(t)}$ can be used as an equivalent notation for $exp[{j\theta(t)}]$, it can also be used as 
a phase operator, where
\begin{align}
    \phase{z(t)} = \theta(t).
\end{align}
The phase operator can also be extended to complex numbers and variables, with $\phase{z}=\theta$.
\\ \\
Just as before, $x(t),y(t),|z(t)|,$ and $\phase{z(t)}$ are all real-valued functions of time. 
\\ \\
The complex conjugate function $z^*(t)$ has the following properties:
\begin{align}
    z^*(t) &= [z(t)|_{j\leftarrow (-j)} \\
    &= [x(t)+jy(t)|_{j\leftarrow (-j)} = x(t) - jy(t) \\
    |z(t)|^2 &= z(t)z^*(t) = z^*(t)z(t).
\end{align}
While Euler's and De Moivre's formulas also apply to complex functions of time, we are not particularly interested in their applications. 
Often, the polar form of complex functions of time are the only complex equations of interest. 

\section*{Operations as Functions}
Here, we introduce some functions that do not necessarily represent signals, but rather they serve as mathematical operations for analysis. 
\\ \\
The \emph{indicator function}, also called the \emph{characteristic function}, is a Boolean function given by 
\begin{align}
    \mathbbm{1}_A(a) = 
    \begin{cases}
        1, & a\in A \\
        0, & otherwise
    \end{cases}.
\end{align}
The \emph{signum (or sign) function} captures the sign of the value $a$: 
\begin{align}
    \operatorname{sgn}(a) = 
    \begin{cases}
        -1, & a<1 \\
        0, & a=0 \\
        1, & a>1
    \end{cases}.
\end{align}
The \emph{floor function} lowers the value of $a$ to the nearest integer below $a$:
\begin{align}
    \operatorname{floor}(a) = 
    \begin{cases}
        a, & a\in \mathbb{Z} \\
        \operatorname{RoundDown}(a), & otherwise
    \end{cases}
\end{align}
The \emph{ceiling function} raises the value of $a$ to the nearest integer above $a$:
\begin{align}
    \operatorname{ceil}(a) = 
    \begin{cases}
        a, & a\in \mathbb{Z} \\
        \operatorname{RoundUp}(a), & otherwise
    \end{cases}
\end{align}
Note that 
\begin{align}
    \operatorname{floor}(a) \leq a \leq \operatorname{ceil}(a).
\end{align}
The \emph{remainder function} returns the remainder of division $a/b$ and is borrowed from the integer modulo operation:
\begin{align}
    \operatorname{mod}(a,b) = (a \mod b) = a - b\cdot\operatorname{floor}(a/b).
\end{align}

\end{document}
