\documentclass{report}
% PACKAGES
\usepackage{adjustbox}
\usepackage{amsmath}
\usepackage{amssymb}
\usepackage{bodegraph}
\usepackage{bbm}
\usepackage{circledsteps}
\usepackage{circuitikz}
\usepackage{enumerate}
\usepackage{mathtools}
\usepackage{nicematrix}
\usepackage{pgfplots}
\usepackage{polynom}
\usepackage{qtree}
\usepackage{rotating}
\usepackage[usestackEOL]{stackengine}
\usepackage[subpreambles=true]{standalone}
\usepackage{steinmetz}
\usepackage{subcaption}
\usepackage{tabularray}
\usepackage{tcolorbox}
\usepackage{tikz}
\usepackage{xcolor}

\usepackage[colorlinks=true,linkcolor=blue,urlcolor=black,bookmarksopen=true]{hyperref}
\usepackage{bookmark}

\usetikzlibrary{shapes.arrows}
\usetikzlibrary{shapes.misc}
\usetikzlibrary{backgrounds}
\tikzset{cross/.style={cross out, draw=black, minimum size=2*(#1-\pgflinewidth), inner sep=0pt, outer sep=0pt},
%default radius will be 1pt. 
cross/.default={1pt}}

\renewcommand{\Re}{\operatorname{Re}}
\renewcommand{\Im}{\operatorname{Im}}

\usepackage{pifont}
\newcommand{\cmark}{\text{\ding{51}}}
\newcommand{\xmark}{\text{\ding{55}}}

\newcommand{\circconv}[1]{\text{ \small\Circled{#1} }}

\newcommand{\tikzmark}[3][]{\tikz[remember picture,baseline] \node [anchor=base,#1](#2) {$#3$};}

\pgfplotsset{compat=1.18}
\pgfplotsset{
    dirac/.style={
        mark=triangle*,
        mark options={scale=1.5},
        ycomb,
        scatter,
        visualization depends on={y/abs(y)-1 \as \sign},
        scatter/@pre marker code/.code={\scope[rotate=90*\sign,yshift=-2pt]}
    }
}

\usepackage[letterpaper, portrait, margin=1.25in]{geometry}
\usepackage[font=bf]{caption}
\hbadness = 10000
\hfuzz=2pt

\newtheorem{theorem}{Theorem}[chapter]

\tikzset{every tree node/.style={anchor=north,align=center}}
\usetikzlibrary{decorations.markings}
\usetikzlibrary{arrows}
\definecolor{darkgreen}{rgb}{0.133,0.545,0.133}
\tikzstyle{n}= [circle, fill=blue, minimum size=4pt,inner sep=0pt, outer sep=0pt]

\patchcmd{\thebibliography}{\chapter*}{\section*}{}{}
\setcounter{secnumdepth}{5}

%%%%%%%%%%% EXAMPLE ENVIRONMENT %%%%%%%%%%
\usepackage{calc}
\usepackage{tabto}
\usepackage[framemethod=tikz]{mdframed} % for colored backgrounds
\newcommand{\halmos}{} % makes a box at the end

\newlength{\framedinnerleftmargin}
\newlength{\framedinnertopmargin}
\newlength{\framedreversedinnerleftmargin}
\setlength{\framedinnerleftmargin}{\widthof{Theoreme 10.10.10}+2em}
\setlength{\framedreversedinnerleftmargin}{\widthof{Theoreme 10.10.10}+1em}
\setlength{\framedinnertopmargin}{1em}
 
% first argument: label in upper left corner,
% second argument: background color
\newenvironment{boxedtext}[2]{\begin{mdframed}[%
hidealllines=true,%
backgroundcolor=#2,%
innertopmargin=\framedinnertopmargin,%
innerleftmargin=\framedinnerleftmargin,%
innerrightmargin=1em%
]%
\tabto{-\framedreversedinnerleftmargin}\textbf{#1}\tabto*{0em}%
}% begin code
{\hskip 0pt\\\hspace*{\fill}\halmos{}\end{mdframed}\vspace{1em}} % end code
 
\newenvironment{summary}[0]{\begin{center}\begin{minipage}[c]{\summarywidth}\begin{spacing}{0.9}\footnotesize} % begin code
{\end{spacing}\end{minipage}\end{center}} % end code

\newcounter{example}
 
% optional! if you want it to start at zero
% with every new chapter/section/etc.
\numberwithin{example}{section}
 
\newenvironment{example}[0]
{\refstepcounter{example}\vspace{1em plus 1em}\begin{boxedtext}{Example \theexample.}{blue!7}}%\setlength{\parskip}{0em}}
{\end{boxedtext}\vspace{-1em plus 1em}}
 
\newenvironment{example*}[0]
{\vspace{1em plus 1em}\begin{boxedtext}{Example.}{blue!7}}
{\end{boxedtext}\vspace{-1em plus 1em}}

\newmdenv[
  topline=false,
  bottomline=false,
  rightline=false,
  skipabove=\topsep,
  skipbelow=\topsep,
  linecolor=purple,
  frametitle={\noindent\textcolor{purple}{\textbf{SOLUTION }}},
  endinnercode={$\hfill\textcolor{purple}{\blacksquare}$}
]{solution}

%%%%%%%%%%%%%% COLOR BOXED %%%%%%%%%%%%%%%
% Syntax: \colorboxed[<color model>]{<color specification>}{<math formula>}
\newcommand*{\colorboxed}{}
\def\colorboxed#1#{%
  \colorboxedAux{#1}%
}
\newcommand*{\colorboxedAux}[3]{%
  % #1: optional argument for color model
  % #2: color specification
  % #3: formula
  \begingroup
    \colorlet{cb@saved}{.}%
    \color#1{#2}%
    \boxed{%
      \color{cb@saved}%
      #3%
    }%
  \endgroup
}
%%%%%%%%%%%%%%%%%%%%%%%%%%%%%%%%%%%%%%%%%

\begin{document}
\setcounter{chapter}{5}
\chapter{Fourier Transform}
In the previous chapter, we analyzed periodic signals using the Fourier series. However, not all everlasting signals are periodic. While the Fourier transform 
will not be derived here, the Fourier transform is closely related to the Fourier series, except now letting $T\rightarrow\infty$ to account for nonperiodic waveforms. 
\\ \\ 
Generally, the Fourier transform can be applied to any signal. However, for best practices, the following table describes what technique is best used for certain signals. \\ \\
\resizebox{\textwidth}{!}{
    \NiceMatrixOptions{cell-space-top-limit=2pt,cell-space-bottom-limit=2pt}
    \begin{NiceTabular}[c]{cccc}[hvlines]
    \Block{1-2}{Input $x(t)$} & & \Block{2-1}{Solution Method} & \Block{2-1}{Output $y(t)$} \\ 
    Duration & Waveform & & \\ 
    Everlasting & Sinusoid & Phasor Transform & Steady-State Component \\ 
    Everlasting & Periodic & Fourier Series and Phasor Transform & Steady-State Component \\ 
    Causal ($x(t)=0$ for $t<0$) & Any & Unilateral Laplace Transform & \shortstack{Complete Solution \\ (transient + steady-state)} \\
    Everlasting & Any & \shortstack{Fourier Transform \\ or Bilateral Laplace Transform} & \shortstack{Complete Solution \\ (transient + steady-state)} \\ 
\end{NiceTabular}}

\section{Fourier Transform}
The \emph{Fourier transform} of a signal $x(t)$ is defined as 
\begin{align}
    X(\omega) = \mathcal{F}[x(t)] = \int_{-\infty}^{+\infty} x(t) e^{-j\omega t} \,dt.
\end{align}
Similar to the Laplace transform, the Fourier transform is essentially an operator that maps a signal defined in the \emph{time domain} to another signal defined 
in the \emph{frequency domain}. $x(t)$ and $X(\omega)$ constitute a unique \emph{Fourier transform pair}. This relationship can be written as 
\begin{align}
    x(t) \iff X(\omega).
\end{align}
Because of this relationship, there exists an \emph{inverse Fourier transform} defined by
\begin{align}
    x(t) = \mathcal{F}^{-1}[X(\omega)] = \frac{1}{2\pi}\int_{-\infty}^{+\infty} X(\omega) e^{+j\omega t} \,d\omega,
\end{align}
though referencing a table of Fourier transform pairs is a lot simpler than computing the integral. 
Interestingly, the Fourier transform is related to the bilateral Laplace transform:
\begin{align}
    X(\omega) = X(s)\big|_{s=j\omega}
\end{align}
The Dirichlet conditions for the Fourier transform is as follows:
\begin{enumerate}
    \item $x(t)$ has a finite number of finite discontinuities in every finite interval of time.
    \item $x(t)$ has a finite number of maxima and minima in every finite interval of time.
    \item $x(t)$ is absolutely integrable such that $\displaystyle\int_{-\infty}^{+\infty}|x(t)| \,dt < \infty$.
\end{enumerate}
If the Dirichlet conditions are met, then $x(t)$ is physically realizable and has a Fourier transform. However, these are sufficient but not necessary conditions 
as some functions like the constant function or the unit step function are not absolutely integrable, yet they have Fourier transforms. 
(Instead, limit definitions of those functions are evaluated to obtain the Fourier transform.)
\\ \\
Table \ref{fourier_prop} lists the properties of the Fourier transform.
\begin{table}
    \centering
    \caption{Properties of the Fourier Transform}
    \label{fourier_prop}
    \begin{tabular}{|c|c|c|}
        \hline
        Property & $x(t)$ & $X(\omega)=\mathcal{F}[x(t)]$ \\[0.15cm]
        \hline
        & & \\
        Superposition & $K_1x_1(t)+K_2x_2(t)$ & $K_1X_1(\omega)+K_2X_2(\omega)$ \\[0.5cm]
        Time scaling & $x(at)$ & $\dfrac{1}{|a|}X\left(\dfrac{\omega}{a}\right)$ \\[0.5cm]
        Time shift & $x(t-t_0)$ & $e^{-j\omega t_0}X(\omega)$ \\[0.5cm]
        Frequency shift & $e^{j\omega_0 t}x(t)$ & $X(\omega-\omega_0)$ \\[0.5cm]
        Time derivative & $x'(t)=\dfrac{dx(t)}{dt}$ & $j\omega X(\omega)$ \\[0.5cm]
        Time $n$th derivative & $x^{(n)}(t)=\dfrac{d^nx(t)}{dt^n}$ & $(j\omega)^nX(\omega)$ \\[0.5cm]
        Time integral & $\displaystyle\int_{-\infty}^{t} x(\tau) \,d\tau$ & $\dfrac{X(\omega)}{j\omega} + \pi\delta(\omega)\displaystyle\int_{-\infty}^{+\infty}x(t) \,dt$ \\[0.5cm]
        Frequency derivative & $t^n x(t)$ & $(j)^n \dfrac{d^n X(\omega)}{d\omega^n} = (j)^n X^{(n)}(\omega)$\\[0.5cm]
        Convolution & $x_1(t)*x_2(t)$ & $X_1(\omega)X_2(\omega)$ \\[0.5cm]
        Multiplication & $x_1(t)x_2(t)$ & $X_1(\omega)*X_2(\omega)$ \\[0.5cm]
        Modulation & $x(t)\cos(\omega_0 t)$ & $\dfrac{1}{2}[X(\omega-\omega_0) + X(\omega+\omega_0)]$ \\[0.5cm]
        Duality & $X(t)$ & $2\pi x(-\omega)$ \\[0.5cm]
        Conjugate symmetry & $x(t)$ real & 
        $\begin{cases}
            X(-\omega) = X^*(\omega) \\
            \Re(X(\omega)) = \Re(X(-\omega)) \\
            \Im(X(\omega)) = -\Im(X(\omega)) \\
            |X(\omega)| = |X(-\omega)| \\
            \arg[X(\omega)] = -\arg[X(-\omega)]
        \end{cases}$ \\[0.5cm]
         & & \\[0.25cm]
        \shortstack{Even-odd decomposition \\ of real signals} & 
        $\begin{cases}
            x_e(t)=\frac{1}{2}[x(t)+x(-t)] \\
            x_o(t)=\frac{1}{2}[x(t)-x(-t)]
        \end{cases}$ & 
        $\begin{cases}
            \Re(X(\omega)) \\
            j\Im(X(\omega))
        \end{cases}$ \\[0.5cm]
        \hline
    \end{tabular}
\end{table}
From the conjugate symmetry property, there are important implications about the properties of $x(t)$ and its Fourier transform $X(\omega)$:
\begin{center}
    \begin{tabular}{|c|c|}
        \hline
        $x(t)$ & $X(\omega)$ \\
        \hline
        Real and even & Real and even \\
        Real and odd & Imaginary and odd \\
        Imaginary and even & Imaginary and even \\
        Imaginary and odd & Real and odd \\
        \hline
    \end{tabular}
\end{center}

\begin{example}
    Find the Fourier transform of the constant function $x(t)=1$.
\end{example}
\begin{solution}
    Consider the delta function $\delta(t)$. Its Fourier transform is given by 
    \begin{align*}
        \mathcal{F}[\delta(t)] = \int_{-\infty}^{+\infty} \delta(t) e^{-j\omega t} \,dt = 1.
    \end{align*}
    Then by the duality property, the Fourier transform of 1 is 
    \begin{align*}
        \mathcal{F}[1] = 2\pi\delta(-\omega) = 2\pi\delta(\omega).
    \end{align*}
\end{solution}

\begin{example}
    Find the Fourier transform of the unit step function $u(t)$.
\end{example}
\begin{solution}
    From the previous example, the Fourier transform of the delta function $\delta(t)$ is 1. 
    Recall that the unit step function can be defined as
    \begin{align*}
        u(t) = \int_{-\infty}^{t} \delta(\tau) \,d\tau.
    \end{align*}
    Then by the time integral property, the Fourier transform of $u(t)$ is 
    \begin{align*}
        \mathcal{F}[u(t)] = \mathcal{F}\left[\int_{-\infty}^{t} \delta(\tau) \,d\tau\right] = \frac{1}{j\omega} + \pi\delta(\omega)\int_{-\infty}^{\infty}\delta(t) \,dt = \frac{1}{j\omega} + \pi\delta(\omega).
    \end{align*}
\end{solution}
Table \ref{fourier_pairs} lists some common Fourier transform pairs. While one could find the inverse Fourier transform using the inverse Fourier integral, 
it is much simpler to reference a table of transform pairs.

\begin{table}
    \centering
    \caption{Fourier Transform Pairs}
    \label{fourier_pairs}
    \begin{tabular}{|c|c|}
        \hline
        $x(t)$ & $X(\omega)=\mathcal{F}[x(t)]$ \\[0.15cm]
        \hline
        & \\[0.1cm]
        $\delta(t)$ & $1$ \\[0.5cm]
        $\delta(t-t_0)$ & $e^{-j\omega t_0}$ \\[0.5cm]
        $1$ & $2\pi\delta(\omega)$ \\[0.5cm]
        $u(t)$ & $\pi\delta(\omega) + \dfrac{1}{j\omega}$ \\[0.5cm]
        $\operatorname{sgn}(t)$ & $\dfrac{2}{j\omega}$ \\[0.5cm]
        $\operatorname{rect}\left(\dfrac{t}{\tau}\right)$ & $\tau\operatorname{Sa}\left(\dfrac{\omega\tau}{2}\right)$ \\[0.5cm]
        $\dfrac{\exp[-t^2/(2\sigma^2)]}{\sqrt{2\pi\sigma^2}}$ & $\exp[-\omega^2\sigma^2/2]$ \\[0.5cm]
        $e^{-at}u(t)$ & $\dfrac{1}{a+j\omega}$ \\[0.5cm]
        $e^{at}u(-t)$ & $\dfrac{1}{a-j\omega}$ \\[0.5cm]
        $\cos(\omega_0 t)$ & $\pi[\delta(\omega-\omega_0) + \delta(\omega+\omega_0)]$ \\[0.5cm]
        $\sin(\omega_0 t)$ & $\dfrac{\pi}{j}[\delta(\omega-\omega_0) - \delta(\omega+\omega_0)]$ \\[0.5cm]
        $e^{j\omega_0 t}$ & $2\pi\delta(\omega-\omega_0)$ \\[0.5cm]
        $te^{-at}u(t)$ & $\dfrac{1}{(a+j\omega)^2}$ \\[0.5cm]
        $e^{-at}\sin(\omega_0 t)u(t)$ & $\dfrac{\omega_0}{(a+j\omega)^2+\omega_0^2}$ \\[0.5cm]
        $\sin(\omega_0 t)u(t)$ & $\dfrac{\pi}{2j}[\delta(\omega-\omega_0) - \delta(\omega+\omega_0)] + \dfrac{\omega_0}{\omega_0^2-\omega^2}$ \\[0.5cm]
        $e^{-at}\cos(\omega_0 t)u(t)$ & $\dfrac{a+j\omega}{(a+j\omega)^2+\omega_0^2}$ \\[0.5cm]
        $\cos(\omega_0 t)u(t)$ & $\dfrac{\pi}{2}[\delta(\omega-\omega_0) - \delta(\omega+\omega_0)] + \dfrac{j\omega}{\omega_0^2+\omega^2}$ \\[0.5cm]
        \hline
    \end{tabular}
\end{table}

\section{Fourier Transform of Periodic Signals}
Recall that the exponential form of the Fourier series for a physically realizable periodic signal $x(t)$ is given by 
\begin{align}
    x(t) = \sum_{n=-\infty}^{+\infty} x_n e^{jn\omega_0 t},
\end{align}
with exponential Fourier coefficients
\begin{align}
    x_n = \frac{1}{T_0} \int_{t_0}^{t_0+T_0} x(t) e^{-jn\omega_0 t} \,dt.
\end{align}
The Fourier transform of periodic signal $x(t)$ with exponential Fourier coefficients $x_n$ is then 
\begin{align}
    \mathcal{F}[x(t)] = \mathcal{F}\left[\sum_{n=-\infty}^{+\infty} x_n e^{jn\omega_0 t}\right] = \sum_{n=-\infty}^{+\infty} \mathcal{F}[x_n e^{jn\omega_0 t}] = \sum_{n=-\infty}^{+\infty} x_n 2\pi \delta(\omega - n\omega_0).
\end{align}

\section{Parseval's Theorem for Fourier Transforms}
Similar to how Parseval's theorem for Fourier series is essentially a ``conservation of (average) power'' theorem as periodic signals are power signals, 
Parseval's theorem for Fourier transforms is essentially a ``conservation of energy'' theorem for any physically realizable signal (aperiodic or periodic). 
When signals are mapped from the continuous-time function to the continuous-frequency Fourier spectrum, the total signal energy is conserved. It follows that 
the total energy of a physically realizable signal $x(t)$ can be evaluated as:
\begin{align}
    E_x = \int_{-\infty}^{+\infty} |x(t)|^2 \,dt = \frac{1}{2\pi}\int_{-\infty}^{+\infty} |X(\omega)|^2 \,d\omega
\end{align}
Additionally, we can define the \emph{one-sided energy spectral density} (1-sided ESD) to be 
\begin{align}
    ESD_1 = \frac{1}{\pi}|X(\omega)|^2,
\end{align}
and the \emph{two-sided energy spectral density} (2-sided ESD) to be 
\begin{align}
    ESD_2 = \frac{1}{2\pi}|X(\omega)|^2.
\end{align}
Note that the 2-sided ESD is defined for all frequencies, whereas the 1-sided ESD is defined for only nonnegative frequencies. 
Because of this, the values of the 2-sided ESD are half the values of the 1-sided ESD. In signal processing, we tend to be more interested in the 1-sided ESD.

\begin{example}
    The signal $x(t)$ is plotted below. Use the properties of the Fourier transform to find: 
    \begin{enumerate}[(a)]
        \item $\phase{X(\omega)}$
        \item $X(\omega)$ at $\omega=0$
        \item $\int_{-\infty}^{+\infty}X(\omega) \,d\omega$
        \item $\int_{-\infty}^{+\infty}|X(\omega)|^2 \,d\omega$
    \end{enumerate}
    \resizebox{0.7\textwidth}{!}{
        \begin{tikzpicture}
            [declare function={
                func(\x)= and(\x > -1, \x < 0) * (2) + and(\x > 0, \x < 2) * (abs(\x-1)+1) + and(\x > 2, \x < 3) * (2); }]
            \begin{axis}[
                axis x line=center, axis y line=center,
                ymin=0, ymax=2.5, ytick={0,1,2}, ylabel={$x(t)$},
                xmin=-3.5, xmax=4, xtick={-3,...,3}, xlabel={$t$ [s]},
                domain=-3.1:3.9,samples=200,
                width=9cm, height=3cm]
            \addplot [blue,thick]{func(x)};
            \end{axis}
        \end{tikzpicture}}
\end{example}
\begin{solution}
    For part (a), notice that $x(t)$ has hidden symmetry. If we let $y(t) = x(t+1)$, then $y(t)$ is real and even. Therefore, 
    \begin{align*}
        y(t) \text{ real and even} \iff Y(\omega) \text{ real and even}
    \end{align*}
    Since $Y(\omega)$ is purely real, it has zero phase. Then
    \begin{align*}
        y(t) = x(t+1) \iff &Y(\omega) = X(\omega)e^{j\omega} \\
        &\phase{Y(\omega)} = \phase{X(\omega)} + \omega = 0.
    \end{align*}
    Then the phase of $X(\omega)$ is 
    \begin{align*}
        \phase{X(\omega)} = -\omega.
    \end{align*}
    Substituting $\omega=0$ into the Fourier transform integral for part (b), we get 
    \begin{align*}
        X(\omega=0) = \int_{-\infty}^{+\infty} x(t) \,dt = [\text{Area under the curve}] = 7
    \end{align*}
    For part (c), we can manipulate the inverse Fourier transform integral such that 
    \begin{align*}
        x(t) &= \frac{1}{2\pi}\int_{-\infty}^{+\infty} X(\omega)e^{j\omega t} \,d\omega \\
        \Longrightarrow 2\pi x(t=0) &= \int_{-\infty}^{+\infty} X(\omega) \,d\omega = 2\pi \times 2 = 4\pi
    \end{align*}
    Lastly, for part (d), by Parseval's theorem, 
    \begin{align*}
        \int_{-\infty}^{+\infty} |X(\omega)|^2 \,d\omega &= 2\pi\int_{-\infty}^{+\infty} |x(t)|^2 \,dt = \frac{76\pi}{3}
    \end{align*}
\end{solution}

\section{LTI Systems with Fourier Transforms}
Similar to the Laplace transform, LTI systems characterized by LCCDEs can be solved by applying the Fourier transform, computing the $\omega$-domain solution, 
and mapping the solution back to the domain. While the unilateral Laplace transform can only be used on causal signals and systems, the Fourier transform 
can be used on all physically realizable signals and LTI systems, though everlasting signals are more adequate.
\\ \\
An LCCDE of the form 
\begin{align}
    \sum_{\ell=0}^{N} a_{N-\ell} \frac{d^{\ell} y(t)}{dt^{\ell}} = \sum_{\ell=0}^{M} b_{M-\ell} \frac{d^{\ell} x(t)}{dt^{\ell}}
\end{align}
can be transformed to the $\omega$-domain to get 
\begin{align}
    \left[\sum_{\ell=0}^{N} a_{N-\ell} (j\omega)^{\ell}\right]Y(\omega) = \left[\sum_{\ell=0}^{M} b_{M-\ell} (j\omega)^{\ell}\right]X(\omega).
\end{align}

\begin{example}
    An LTI system is characterized by the LCCDE 
    \begin{align*}
        \frac{d^2y(t)}{dt^2} + 7\frac{dy(t)}{dt} + 12y(t) = \frac{dx(t)}{dt} + 2x(t).
    \end{align*}
    Find the system response to input $x(t)=e^{-2t}u(t)$.
\end{example}
\begin{solution}
    Transforming the LCCDE to $\omega$-domain, we get
    \begin{align*}
        [(j\omega)^2 + 7j\omega + 12]Y(\omega) &= [j\omega + 2]X(\omega) \\
        \Longrightarrow Y(\omega) &= \left[\frac{2+j\omega}{(j\omega)^2 + 7j\omega + 12}\right]X(\omega)
    \end{align*}
    From the table of Fourier transform pairs, we see that 
    \begin{align*}
        \mathcal{F}[e^{-2t}u(t)] = \frac{1}{2+j\omega} 
    \end{align*}
    Therefore, by applying partial fraction expansion with respect to $j\omega$,
    \begin{align*}
        Y(\omega) &= \left[\frac{2+j\omega}{(j\omega)^2 + 7j\omega + 12}\right] \frac{1}{2+j\omega} = \frac{1}{(j\omega)^2 + 7j\omega + 12} = \frac{1}{3+j\omega} - \frac{1}{4+j\omega},
    \end{align*}
    we get the solution
    \begin{align*}
        y(t) &= e^{-3t}u(t) - e^{-4t}u(t).
    \end{align*}
\end{solution}

Just as differential equations can be solved in the $\omega$-domain, the systems represented by differential equations can be solved in the $\omega$-domain. For instance, 
an electric circuit can be transformed to an $\omega$-domain equivalent and solved using algebraic equations, with the $\omega$-domain electrical components referred to  
as impedances. The equivalent $\omega$-domain models can be seen in Table \ref{omega_circ}.

\begin{table}[!hbt]
    \caption{Circuit models for electrical components in the $\omega$-domain.}
    \label{omega_circ}
    \centering
    \begin{tabular}{ |c|c|c| }
        \hline
        \shortstack{Electrical \\ Component} & \shortstack{Time \\ Domain} & $\omega$-Domain \\
        \hline
        & & \\
        Resistor &
        \adjustbox{valign=m}{
        \begin{circuitikz}[american voltages,raised voltages]
            \draw (0,0) to [R=$R$, f=$i_R$, v=$v_R$, voltage shift = 1.2] (0,-3); 
        \end{circuitikz}} & 
        \adjustbox{valign=m}{
        \begin{circuitikz}[american voltages,raised voltages]
            \draw (0,0) to [R=$R$, f=$I_R$, v=$V_R$, voltage shift = 1.2] (0,-3); 
        \end{circuitikz}}\\[1.5cm]
        \hline
        & & \\
        Capacitor & 
        \adjustbox{valign=m}{
        \begin{circuitikz}[american voltages,raised voltages]
            \draw (0,0) to [C=$C$, f=$i_C(t)$, v=$v_C(t)$, voltage shift=1.2] (0,-3); 
        \end{circuitikz}} &
        \adjustbox{valign=m}{
        \begin{circuitikz}[american voltages,raised voltages]
            \draw (0,0) to [C=$\dfrac{1}{j\omega C}$, f=$I_C$, v=$V_C$, voltage shift=1.2] (0,-3); 
        \end{circuitikz}} \\[1.5cm]
        \hline
        & & \\
        Inductor & 
        \adjustbox{valign=m}{
        \begin{circuitikz}[american voltages,raised voltages]
            \draw (0,0) to [L=$L$, f=$i_L(t)$, v=$v_L(t)$, voltage shift=1.2] (0,-3); 
        \end{circuitikz}} &
        \adjustbox{valign=m}{
        \begin{circuitikz}[american voltages,raised voltages]
            \draw (0,0) to [L=$j\omega L$, f=$I_L$, v=$V_L$, voltage shift=1.2] (0,-3); 
        \end{circuitikz}} \\[1.5cm]
        \hline
    \end{tabular}
\end{table}

\begin{example}
    Find an equation for the current flowing through the capacitor $i_c(t)$ in the following circuit, given that the input excitation is 
    \begin{align*}
        v_s(t) = 10e^{-2t}u(t) \text{ [V]}.
    \end{align*}
    \resizebox{0.7\textwidth}{!}{\begin{circuitikz}[american voltages,raised voltages]
        \draw (0,0) to [V, l_={\large $v_s(t)$}] (0,-2); 
        \draw (0,0) to [R={$2\,[k\Omega]$}] (3,0);
        \draw (3,0) to [short, f={$i_c(t)$}] (6,0);
        \draw (0,-2) to [short] (6,-2);
        \draw (3,0) to [R={$4\,[k\Omega]$}] (3,-2);
        \draw (6,0) to [C={$0.25\,[mF]$}] (6,-2);
    \end{circuitikz}}
\end{example}
\begin{solution}
    First, we transform the circuit to the $\omega$-domain. \\ \\
    \resizebox{0.7\textwidth}{!}{\begin{circuitikz}[american voltages,raised voltages]
        \draw (0,0) to [V, l_={\large $V_s=\dfrac{10}{2+j\omega}$}] (0,-2); 
        \draw (0,0) to [R={$2000$}] (3,0);
        \draw (3,0) to [short, f={$I_c$}] (6,0);
        \draw (0,-2) to [short] (6,-2);
        \draw (3,0) to [R={$4000$}] (3,-2);
        \draw (6,0) to [C={$\dfrac{4000}{j\omega}$}] (6,-2);
        \draw[-latex, blue] (1.25,-0.5) arc(110:-110:6mm) node[blue, midway, left, font=\large] {$I_1$};
        \draw[-latex, blue] (4.25,-0.5) arc(110:-110:6mm) node[blue, midway, left, font=\large] {$I_2$};
    \end{circuitikz}}
    \\ \\
    Then by using the mesh current method, we get 
    \begin{align*}
        -V_s + 2000I_1 + 4000(I_1-I_2) = 0 &\Longrightarrow I_1 = \frac{V_s + 4000I_2}{6000} \\
        4000(I_2-I_1) + \frac{4000}{j\omega}I_2 = 0 &\Longrightarrow I_1 = \left(1+\frac{1}{j\omega}\right)I_2
    \end{align*}
    Setting the two equations together and using partial fraction expansion with respect to $j\omega$:
    \begin{align*}
        I_2 = \frac{V_s}{6000}\cdot\frac{3j\omega}{3+j\omega} = \frac{(5\times10^{-3})j\omega}{(2+j\omega)(3+j\omega)} &= \frac{-10\times10^{-3}}{2+j\omega} + \frac{15\times10^{-3}}{3+j\omega} \\ 
        &= \left(\frac{-10}{2+j\omega} + \frac{15}{3+j\omega}\right)\times 10^{-3}
    \end{align*}
    The current running through the capacitor is then given by 
    \begin{align*}
        i_c(t) = \mathcal{F}^{-1}[I_2] = (-10e^{-2t} + 15e^{-3t})u(t) \text{ [mA]}.
    \end{align*}
\end{solution}

\section{Frequency Response}
The \emph{frequency response function} is the Fourier transform of the impulse response and 
is defined as 
\begin{align}
    H(\omega) = \frac{Y(\omega)}{X(\omega)} = \mathcal{F}[h(t)].
\end{align}
In fact, by the convolution property, it follows that 
\begin{align}
    y(t) = x(t) * h(t) \iff Y(\omega) = X(\omega)H(\omega).
\end{align}
Symbolically,
\begin{center}
    \begin{tikzpicture}
        \node [] (input) at (-3,0){$x(t)$};
        \node [] (output) at (+3,0){$y(t)$};
        \node [draw,
            fill=yellow!50, 
            minimum width=2cm, 
            minimum height=1.2cm
        ] (system) at (0,0){$h(t)$};
        \draw [-latex, line width=1.5pt] (input) -- (system);
        \draw [-latex, line width=1.5pt] (system) -- (output);
    \end{tikzpicture} \\[0.25cm]
    \begin{tikzpicture}
        \node at (0,0) [double arrow, draw=black, top color=red, bottom color=blue,
            minimum width = 15pt, double arrow head extend=5pt,
            minimum height=12mm,
            rotate=90] {};
    \end{tikzpicture} \\[0.25cm]
    \begin{tikzpicture}
        \node [] (input) at (-3,0){$X(\omega)$};
        \node [] (output) at (+3,0){$Y(\omega)$};
        \node [draw,
            fill=yellow!50, 
            minimum width=2cm, 
            minimum height=1.2cm
        ] (system) at (0,0){$H(\omega)$};
        \draw [-latex, line width=1.5pt] (input) -- (system);
        \draw [-latex, line width=1.5pt] (system) -- (output);
    \end{tikzpicture}
\end{center}
As the impulse response is the system response to input $x(t)=\delta(t)$, consider the Fourier transform of the impulse signal $\mathcal{F}[\delta(t)]=1$. Then it also follows that 
\begin{center}
    \begin{tikzpicture}
        \node [] (input) at (-3,0){$x(t)=\delta(t)$};
        \node [] (output) at (+3,0){$y(t)=h(t)$};
        \node [draw,
            fill=yellow!50, 
            minimum width=2cm, 
            minimum height=1.2cm
        ] (system) at (0,0){$h(t)$};
        \draw [-latex, line width=1.5pt] (input) -- (system);
        \draw [-latex, line width=1.5pt] (system) -- (output);
    \end{tikzpicture} \\[0.25cm]
    \begin{tikzpicture}
        \node at (0,0) [double arrow, draw=black, top color=red, bottom color=blue,
            minimum width = 15pt, double arrow head extend=5pt,
            minimum height=12mm,
            rotate=90] {};
    \end{tikzpicture} \\[0.25cm]
    \begin{tikzpicture}
        \node [] (input) at (-3,0){$X(\omega)=1$};
        \node [] (output) at (+3,0){$Y(\omega)=H(\omega)$};
        \node [draw,
            fill=yellow!50, 
            minimum width=2cm, 
            minimum height=1.2cm
        ] (system) at (0,0){$H(s)$};
        \draw [-latex, line width=1.5pt] (input) -- (system);
        \draw [-latex, line width=1.5pt] (system) -- (output);
    \end{tikzpicture}
\end{center}
Interestingly, the frequency response is related to the transfer function:
\begin{align}
    H(\omega) = H(s)\big|_{s=j\omega}
\end{align}
As before, there are two approaches to finding the frequency response of an LTI system:
\begin{enumerate}
    \item Find $X(\omega),Y(\omega)$. Then calculate $H(\omega)=Y(\omega)/X(\omega)$.
    \item Find $X(\omega),Y(\omega)$. Then substitute $X(\omega)=1$ and $Y(\omega)=H(\omega)$.
\end{enumerate}
While the frequency response can be used for determining system stability just like the transfer function, 
the frequency response has the additional tool of analyzing the gain and phase of an LTI system at particular 
frequencies of interest, as borrowed from the Fourier series analysis. 
\\ \\
The magnitude $|H(\omega)|$ of the frequency response is called the \emph{magnitude response} and has an associated \emph{phase reponse} $\phase{H(\omega)}$. 
The plots of the magnitude response and phase response are two-sided and continuous and are collectively referred to as the \emph{frequency spectrum}. 
\\ \\ 
When expressed as $H(\omega)=\dfrac{N(\omega)}{D(\omega)}$, the magnitude and phase responses can be computed as 
\begin{align}
    |H(\omega)| &= \frac{|N(\omega)|}{|D(\omega)|} = \frac{\sqrt{[\Re(N(\omega))]^2 + [\Im(N(\omega))]^2}}{\sqrt{[\Re(D(\omega))]^2 + [\Im(D(\omega))]^2}} \\
    \phase{H(\omega)} &=  \phase{N(\omega)} -  \phase{D(\omega)} = \arctan\left(\frac{\Im(N(\omega))}{\Re(N(\omega))}\right) - \arctan\left(\frac{\Im(D(\omega))}{\Re(D(\omega))}\right)
\end{align}
\begin{example}
    Find the frequency response of the following RC circuit. \\ \\
    \resizebox{0.5\textwidth}{!}{
        \begin{circuitikz}[american voltages,raised voltages]
            \draw 
                (0,3) to [V, l_={\Large $v_s(t)$}] (0,0)
                to [short, -*] (6,0)
                (0,3) to [R, l^={\Large $R$}] (4,3)
                to [C, l_={\Large $C$}] (4,0)
                (4,3) to [short, -*] (6,3) 
                to [open, v^>={\Large $v_{out}(t)$}] (6,0);
        \end{circuitikz}}
\end{example}
\begin{solution}
    Mapping to the $\omega$-domain, we get
    \begin{center}
        \resizebox{0.45\textwidth}{!}{
        \begin{circuitikz}[american voltages,raised voltages]
            \draw 
                (0,3) to [V, l_={\Large $V_s(\omega)$}] (0,0)
                to [short, -*] (6,0)
                (0,3) to [R, l^={\Large $R$}] (4,3)
                to [C, l_={\Large $\dfrac{1}{j\omega C}$}] (4,0)
                (4,3) to [short, -*] (6,3) 
                to [open, v^>={\Large $V_{out}(\omega)$}] (6,0);
        \end{circuitikz}}
    \end{center}
    From here, we can use the voltage divider rule to get 
    \begin{align*}
        V_{out}(\omega) = \left[\frac{1/j\omega C}{R+1/j\omega C}\right] V_s(\omega)
    \end{align*}
    The frequency response is then given by 
    \begin{align*}
        H(\omega) = \frac{V_{out}(\omega)}{V_s(\omega)} = \frac{1/j\omega C}{R+1/j\omega C} = \frac{1}{1+j\omega RC},
    \end{align*}
    with magnitude and phase responses 
    \begin{align*}
        |H(\omega)| &= \frac{1}{\sqrt{1+(\omega RC)^2}} \\
        \phase{H(\omega)} &= -\arctan(\omega RC)
    \end{align*}
    Let $RC=1$. The frequency spectrum is plotted below.
    \begin{center}
        \resizebox{0.6\textwidth}{!}{
        \begin{tikzpicture}
            [declare function={
                func(\x)= 1 / sqrt(1 + (\x)^2); }]
            \begin{axis}[
                axis x line=center, axis y line=center,
                ymin=0, ymax=1.5, ytick={0,1}, ylabel={$|H(\omega)|$},
                xmin=-3.5, xmax=4, xtick={-3,...,3}, xlabel={$\omega$},
                domain=-3.1:3.9,samples=200,
                width=9cm, height=3.5cm]
            \addplot [blue,thick]{func(x)};
            \end{axis}
        \end{tikzpicture}}
    \end{center}
    \begin{center}
        \resizebox{0.6\textwidth}{!}{
        \begin{tikzpicture}
            [declare function={
                func(\x)= -rad(atan(\x)); }]
            \begin{axis}[
                axis x line=center, axis y line=center,
                ymin=-2, ymax=2, ytick={0}, ylabel={$\phase{H(\omega)}$},
                xmin=-3.5, xmax=4, xtick={-3,...,3}, xlabel={$\omega$},
                extra y ticks={-pi/2,pi/2}, extra y tick labels={$-\pi/2$, $\pi/2$},
                domain=-3.1:3.9,samples=200,
                width=9cm, height=4.5cm]
            \addplot [blue,thick]{func(x)};
            \end{axis}
        \end{tikzpicture}}
    \end{center}
\end{solution}

\pagebreak
\begin{example}
    Given the frequency spectrum of an LTI system plotted below, find the impulse response $h(t)$. \\ \\
    \resizebox{0.75\textwidth}{!}{
        \begin{tikzpicture}
            [declare function={
                func(\x)= and(\x > -1, \x < 1) * abs(\x); }]
            \begin{axis}[
                axis x line=center, axis y line=center,
                ymin=0, ymax=1.5, ytick={0,1}, ylabel={$|H(\omega)|$},
                xmin=-3.5, xmax=4, xtick={-3,...,3}, xlabel={$\omega$},
                domain=-3.1:3.9,samples=200,
                width=9cm, height=3.5cm]
            \addplot [blue,thick]{func(x)};
            \end{axis}
        \end{tikzpicture}} \\ \\
    \resizebox{0.75\textwidth}{!}{
            \begin{tikzpicture}
                [declare function={
                    func(\x)= -3 * \x; }]
                \begin{axis}[
                    axis x line=center, axis y line=center,
                    ymin=-3.5, ymax=3.5, ytick={-3,0,3}, ylabel={$\phase{H(\omega)}$},
                    xmin=-3.5, xmax=4, xtick={-3,...,3}, xlabel={$\omega$},
                    domain=-3.1:3.9,samples=200,
                    width=9cm, height=4.5cm]
                \addplot [blue,thick]{func(x)};
                \end{axis}
            \end{tikzpicture}}
\end{example}
\begin{solution}
    The phase is given by $\phase{H(\omega)}=-3\omega$. The frequency response is then 
    \begin{align*}
        H(\omega) = |H(\omega)|\exp(j\phase{H(\omega)}) = |H(\omega)|e^{-j3\omega}.
    \end{align*}
    Let $G(\omega)=|H(\omega)|$. Then by the time shift property, 
    \begin{align*}
        H(\omega) = G(\omega)e^{-j3\omega} \iff h(t) = g(t-3).
    \end{align*}
    Solving for $g(t)$ first,
    \begin{align*}
        g(t) = \mathcal{F}^{-1}[G(\omega)] &= \frac{1}{2\pi}\int_{-\infty}^{+\infty} G(\omega)e^{j\omega t} \,d\omega \\
        &= \frac{1}{2\pi}\left[\int_{-1}^{0} -\omega e^{j\omega t} \,d\omega + \int_{0}^{1} \omega e^{j\omega t} \,d\omega\right] \\
        &= \frac{1}{2\pi}\left[\frac{e^{-jt}+jte^{-jt}-1}{t^2} + \frac{e^{+jt}-jte^{+jt}-1}{t^2}\right] \\ 
        &= \frac{1}{\pi}\left[\frac{\cos(t)-1}{t^2} + \frac{\sin(t)}{t}\right]
    \end{align*}
    Then applying the time shift property,
    \begin{align*}
        h(t) = g(t-3) = \frac{1}{\pi}\left[\frac{\cos(t-3)-1}{(t-3)^2} + \frac{\sin(t-3)}{t-3}\right].
    \end{align*}
\end{solution}

\section{Bode Plots}
Rather than plotting the frequency spectrum with linear units, an alternative graph of the frequency response to consider is the \emph{Bode plot}, which is comprised 
of the \emph{Bode magnitude plot} and the \emph{Bode phase plot}. The Bode plot logarithmically scales the $\omega$-axis and additionally converts the magnitude 
values to decibels. As a result, Bode plots are plotted on semilog templates and are inherently one-sided.
\\ \\ 
Additionally, Bode plots are convenient for providing straight-line approximations to the magnitude and phase responses. While the Bode plot does not truely give the same response as the 
frequency response, LTI systems can be modeled such that the asymptotic approximations are close enough to where the errors are minimal. 
\\ \\
Figure \ref{semilog} depicts a semilog template, where Bode plots can be drawn on. The main feature to examine is the logarithmically scaled $\omega$-axis. Each prominent solid vertical line that 
takes on powers of 10 as its angular frequency $\omega$ is referred to as a \emph{cycle}. While never depicted on Bode plots, the unseen angular frequency $\omega=0$ is called the $origin$. 
\\ \\
Any solid light tick marks after particular cycle represent multiples of that cycle -- at least up until 
the next cycle. Generally, a \emph{decade} (or \emph{dec} for shorthand) is the horizontal spacing between one cycle and the next; this spacing however can also be shifted to start at any angular frequency.
\\ \\
Usually we choose an appropriate small power of 10 and appropriate large power of 10 for the minimum and maximum 
when labeling the $\omega$-axis such that any and all behaviors due to the poles and zeros of the system are within scope. 
\begin{figure}[!hbt]
    \centering
    \caption{Characteristics of a Semilog Template}
    \label{semilog}
    \resizebox{\textwidth}{!}{\begin{tikzpicture}
        \tikzset{
            semilog lines/.style={black},
            semilog lines 2/.style={gray!50},
            semilog half lines/.style={white, dotted},
            semilog label x/.style={below,font=\small},
            semilog label y/.style={above,font=\small} }
        \begin{scope}[xscale=10/5,yscale=5/100]
            \OrdBode{25}
            \semilog*{-2}{3}{0}{100}
            \draw[blue!80,line width=1.2,latex-latex] (-1,75) -- (0,75);
            \node[fill=white] at (-0.5,81){\small Decade};
            \draw[blue!80,line width=1.2,latex-latex] (-0.5229,50) -- (0.4771,50);
            \node[fill=white] at (0,56){\small Decade};
            \node[fill=white, rotate=90] at (0,12.5){\small Cycle};
            \node[fill=white, rotate=90] at (1,12.5){\small Cycle};
            \node[fill=white, rotate=90] at (1.3010,12.5){\tiny $\omega = 20$};
            \node[fill=white, rotate=90] at (1.4771,12.5){\tiny $\omega = 30$};
            \node[fill=white, rotate=90] at (1.6021,12.5){\tiny $\omega = 40$};
            \node[fill=white, rotate=90] at (2.3010,12.5){\tiny $\omega = 200$};
            \node[fill=white, rotate=90] at (2.4771,12.5){\tiny $\omega = 300$};
            \node[fill=white, rotate=90] at (2.6021,12.5){\tiny $\omega = 400$};
        \end{scope}
    \end{tikzpicture}}
\end{figure}
\\ \\
Before delving into producing Bode plots, first we must define the logarithmic unit decibels [dB]. 
Depending on how the frequency response was obtained, the magnitude response can be converted from unitless to decibels. 
\\ \\
Most of the time, we assume that the input and output signals are voltages such that the frequency response is unitless $[V/V]$. 
Therefore, its magnitude response in decibels is given by 
\begin{align}
    |H(\omega)| [dB] = 20\log\left(|H(\omega)| \left[\frac{V}{V}\right]\right) [dB].
\end{align}
Table \ref{decibels} lists some examples of decibel values corresponding to a select few magnitudes.

\begin{table}[!hbt]
    \centering
    \caption{Examples of Corresponding Decibel Values}
    \label{decibels}
    \begin{tabular}{|c|c|}
        $M$ [V/V] & $M$ [dB] \\
        \hline
        $10^N$ & $20N$ [dB] \\
        $10^3$ & $60$ [dB] \\
        $100$ & $40$ [dB] \\
        $10$ & $20$ [dB] \\
        $4$ & $\approx 12$ [dB] \\
        $2$ & $\approx 6$ [dB] \\
        $\sqrt{2}$ & $\approx 3$ [dB] \\
        $1$ & $0$ [dB] \\
        $1/\sqrt{2}$ & $\approx -3$ [dB] \\
        $0.5$ & $\approx -6$ [dB] \\
        $0.25$ & $\approx -12$ [dB] \\
        $0.1$ & $-20$ [dB] \\
        $10^{-N}$ & $-20N$ [dB] \\ 
    \end{tabular}
\end{table}

\begin{example}
    Find an expression for $|H(\omega)| = \dfrac{1}{\sqrt{1+(\omega RC)^2}}$ [V/V] in decibels.
\end{example}
\begin{solution}
    It follows that 
    \begin{align*}
        20\log(|H(\omega)|) &= 20\log\left(\frac{1}{\sqrt{1+(\omega RC)^2}}\right) \\
        &= 20\left[\log(1) - \log\left(\sqrt{1+(\omega RC)^2}\right)\right] \\
        &= 20\left[\log(1) - \frac{1}{2}\log\left[1+(\omega RC)^2\right]\right] \\
        &= -10\log\left[1+(\omega RC)^2\right] \text{ [dB]}
    \end{align*}
    We can let $\omega_c = 1/(RC)$ such that
    \begin{align*}
        |H(\omega)| = -10\log\left[1+\left(\frac{\omega}{\omega_c}\right)^2\right] \text{ [dB]}
    \end{align*}
    It follows that 
    \begin{align*}
        \text{Low-frequency asymptote: } \left(\frac{\omega}{\omega_c}\right)\rightarrow 0 &\Longrightarrow |H(\omega)|\rightarrow 0 \text{ [dB]} \\
        \text{High-frequency asymptote: } \left(\frac{\omega}{\omega_c}\right)\rightarrow \infty &\Longrightarrow |H(\omega)|\rightarrow -20\log\left(\frac{\omega}{\omega_c}\right) \text{ [dB]}
    \end{align*}
\end{solution}

\begin{tcolorbox}[width=\textwidth,colback={white}, sharp corners]
The standard form of a generic frequency response is given by 
\\ \\ \\
\begin{align}
    H(\omega) = \frac{\tikzmark{constant}{K}\cdot\tikzmark[violet]{z_orig}{(j\omega)}\tikzmark[blue]{zero}{\left(1+\frac{j\omega}{\omega_0}\right)}}{\tikzmark[darkgreen]{p_orig}{(j\omega)}\tikzmark[orange]{pole}{\left(1+\frac{j\omega}{\omega_0}\right)}^{\tikzmark[red]{mult}{N}}}.
\end{align}
\begin{tikzpicture}[overlay, remember picture,node distance =1.5cm]
    \node[,text width=2cm] (constantdescr) [above left of=constant ]{\small constant};
    \draw[,-latex,thick] (constantdescr) to (constant);
    \node[violet,text width=2cm] (z_origdescr) [above of=z_orig ]{\small zero at origin};
    \draw[violet,-latex,thick] (z_origdescr) to (z_orig);
    \node[blue,text width=2cm] (zerodescr) [above right of=zero ]{\small real zero};
    \draw[blue,-latex,thick] (zerodescr) to (zero);
    \node[darkgreen,text width=2cm] (p_origdescr) [below left of=p_orig ]{\small pole at origin};
    \draw[darkgreen,-latex,thick] (p_origdescr) to (p_orig);
    \node[orange,text width=2cm] (poledescr) [below of=pole ]{\small real pole};
    \draw[orange,-latex,thick] (poledescr) to (pole);
    \node[red,text width=2cm] (multdescr) [below right of=mult ]{\small multiplicity};
    \draw[red,-latex,thick] (multdescr) to (mult);
\end{tikzpicture}
\\ \\
\end{tcolorbox}
\noindent The corresponding effects of each term on the Bode plots are outlined in Table \ref{bode}.
\begin{table}
    \centering
    \caption{Straight-Line Approximations of Terms for Bode Plots}
    \label{bode}
    \resizebox{\textwidth}{!}{
    \begin{tabular}{|c|c|c|}
        \hline
        Term & Bode Magnitude Approximation [dB] & Bode Phase Approximation [deg] \\
        \hline
        \shortstack{Constant \\ $H(\omega)=K$} & 
        \adjustbox{valign=m,right=6.25cm}{\begin{tikzpicture}
            [declare function={
                func(\x)= 10; }]
            \begin{axis}[
                axis x line=center, axis y line=center,
                ymin=-30, ymax=30, ytick={0}, ylabel={$|H(\omega)|$ [dB]},
                xmin=0, xmax=4, xtick={0}, xlabel={$\omega$},
                extra x ticks={1,2,3}, extra x tick labels={0.1,1,10},
                extra y ticks={10}, extra y tick labels={$M$},
                domain=0:4,samples=50,
                width=7cm, height=4.5cm]
            \addplot [blue,thick]{func(x)};
            \node at (2,-20) {$M=20\log|K|$ [dB]};
            \end{axis}
        \end{tikzpicture}} & 
        \adjustbox{valign=m,right=6.25cm}{\begin{tikzpicture}
            [declare function={
                func(\x)= -10; }]
            \begin{axis}[
                axis x line=center, axis y line=center,
                ymin=-30, ymax=30, ytick={0}, ylabel={$\phase{H(\omega)}$ [deg]},
                xmin=0, xmax=4, xtick={0}, xlabel={$\omega$},
                extra x ticks={1,2,3}, extra x tick labels={0.1,1,10},
                extra y ticks={-10}, extra y tick labels={$\phase{K}$},
                domain=0:4,samples=50,
                width=7cm, height=4.5cm]
            \addplot [blue,thick]{func(x)};
            \end{axis}
        \end{tikzpicture}} \\
        \hline
        \shortstack{Zero at origin \\ $H(\omega)=j\omega$} & 
        \adjustbox{valign=m,right=6.25cm}{\begin{tikzpicture}
            [declare function={
                func(\x)= 20 * (\x - 2); }]
            \begin{axis}[
                axis x line=center, axis y line=center,
                ymin=-30, ymax=30, ytick={-20,0,20}, ylabel={$|H(\omega)|$ [dB]},
                xmin=0, xmax=4, xtick={0}, xlabel={$\omega$},
                extra x ticks={1,2,3}, extra x tick labels={0.1,1,10},
                domain=1:3.2,samples=50,
                width=7cm, height=4.5cm]
            \addplot [blue,thick]{func(x)};
            \node at (2,-20) {$+20$ [dB/dec]};
            \end{axis}
        \end{tikzpicture}} & 
        \adjustbox{valign=m,right=6.25cm}{\begin{tikzpicture}
            [declare function={
                func(\x)= 16; }]
            \begin{axis}[
                axis x line=center, axis y line=center,
                ymin=-30, ymax=30, ytick={0}, ylabel={$\phase{H(\omega)}$ [deg]},
                xmin=0, xmax=4, xtick={0}, xlabel={$\omega$},
                extra x ticks={1,2,3}, extra x tick labels={0.1,1,10},
                extra y ticks={16}, extra y tick labels={$90^{\circ}$},
                domain=0:4,samples=50,
                width=7cm, height=4.5cm]
            \addplot [blue,thick]{func(x)};
            \end{axis}
        \end{tikzpicture}} \\
        \hline
        \shortstack{Pole at origin \\ $H(\omega)=\dfrac{1}{j\omega}$} & 
        \adjustbox{valign=m,right=6.25cm}{\begin{tikzpicture}
            [declare function={
                func(\x)= -20 * (\x - 2); }]
            \begin{axis}[
                axis x line=center, axis y line=center,
                ymin=-30, ymax=30, ytick={-20,0,20}, ylabel={$|H(\omega)|$ [dB]},
                xmin=0, xmax=4, xtick={0}, xlabel={$\omega$},
                extra x ticks={1,2,3}, extra x tick labels={0.1,1,10},
                domain=1:3.2,samples=50,
                width=7cm, height=4.5cm]
            \addplot [blue,thick]{func(x)};
            \node at (2,-20) {$-20$ [dB/dec]};
            \end{axis}
        \end{tikzpicture}} & 
        \adjustbox{valign=m,right=6.25cm}{\begin{tikzpicture}
            [declare function={
                func(\x)= -16; }]
            \begin{axis}[
                axis x line=center, axis y line=center,
                ymin=-30, ymax=30, ytick={0}, ylabel={$\phase{H(\omega)}$ [deg]},
                xmin=0, xmax=4, xtick={0}, xlabel={$\omega$},
                extra x ticks={1,2,3}, extra x tick labels={0.1,1,10},
                extra y ticks={-16}, extra y tick labels={$-90^{\circ}$},
                domain=0:4,samples=50,
                width=7cm, height=4.5cm]
            \addplot [blue,thick]{func(x)};
            \end{axis}
        \end{tikzpicture}} \\
        \hline
        \shortstack{Real zero \\ $H(\omega)=1+\dfrac{j\omega}{\omega_0}$} & 
        \adjustbox{valign=m,right=6.25cm}{\begin{tikzpicture}
            [declare function={
                func(\x)= 20 * (\x - 2) * (\x > 2); }]
            \begin{axis}[
                axis x line=center, axis y line=center,
                ymin=-30, ymax=30, ytick={-20,0,20}, ylabel={$|H(\omega)|$ [dB]},
                xmin=0, xmax=4, xtick={0}, xlabel={$\omega$},
                extra x ticks={1,2,3}, extra x tick labels={$0.1\omega_0$,$\omega_0$,$10\omega_0$},
                domain=0:3.2,samples=50,
                width=7cm, height=4.5cm]
            \addplot [blue,thick]{func(x)};
            \node at (2,-20) {$+20$ [dB/dec]};
            \end{axis}
        \end{tikzpicture}} & 
        \adjustbox{valign=m,right=6.25cm}{\begin{tikzpicture}
            [declare function={
                func(\x)= 8 * (\x - 1) * and(\x > 1, \x < 3) + 16 * (\x > 3); }]
            \begin{axis}[
                axis x line=center, axis y line=center,
                ymin=-30, ymax=30, ytick={0}, ylabel={$\phase{H(\omega)}$ [deg]},
                xmin=0, xmax=4, xtick={0}, xlabel={$\omega$},
                extra x ticks={1,2,3}, extra x tick labels={$0.1\omega_0$,$\omega_0$,$10\omega_0$},
                extra y ticks={16}, extra y tick labels={$90^{\circ}$},
                domain=0:4,samples=50,
                width=7cm, height=4.5cm]
            \addplot [blue,thick]{func(x)};
            \node at (2,-20) {$+45$ [deg/dec]};
            \end{axis}
        \end{tikzpicture}} \\
        \hline
        \shortstack{Real pole \\ $H(\omega)=\dfrac{1}{1+\frac{j\omega}{\omega_0}}$} & 
        \adjustbox{valign=m,right=6.25cm}{\begin{tikzpicture}
            [declare function={
                func(\x)= -20 * (\x - 2) * (\x > 2); }]
            \begin{axis}[
                axis x line=center, axis y line=center,
                ymin=-30, ymax=30, ytick={-20,0,20}, ylabel={$|H(\omega)|$ [dB]},
                xmin=0, xmax=4, xtick={0}, xlabel={$\omega$},
                extra x ticks={1,2,3}, extra x tick labels={$0.1\omega_0$,$\omega_0$,$10\omega_0$},
                domain=0:3.2,samples=50,
                width=7cm, height=4.5cm]
            \addplot [blue,thick]{func(x)};
            \node at (2,-20) {$-20$ [dB/dec]};
            \end{axis}
        \end{tikzpicture}} & 
        \adjustbox{valign=m,right=6.25cm}{\begin{tikzpicture}
            [declare function={
                func(\x)= -8 * (\x - 1) * and(\x > 1, \x < 3) + -16 * (\x > 3); }]
            \begin{axis}[
                axis x line=center, axis y line=center,
                ymin=-30, ymax=30, ytick={0}, ylabel={$\phase{H(\omega)}$ [deg]},
                xmin=0, xmax=4, xtick={0}, xlabel={$\omega$},
                extra x ticks={1,2,3}, extra x tick labels={$0.1\omega_0$,$\omega_0$,$10\omega_0$},
                extra y ticks={-16}, extra y tick labels={$-90^{\circ}$},
                domain=0:4,samples=50,
                width=7cm, height=4.5cm]
            \addplot [blue,thick]{func(x)};
            \node at (2,-20) {$-45$ [deg/dec]};
            \end{axis}
        \end{tikzpicture}} \\
        \hline
        Multiplicity, $N$ & \shortstack{Constant: $M\times N$ \\ All zeros/poles: $N\times$(slope)} & \shortstack{Constant/origin: $\phase{H(\omega)}\times N$ \\ Real zeros/poles: $N\times$(slope)} \\
        \hline
    \end{tabular}}
\end{table}
\\ \\
\noindent Note that the Bode magnitude plot of a real zero or real pole has a slope of $\pm 20$ [dB/dec] for only $\omega > \omega_0$, and the Bode phase plot of a real zero or real pole has a slope of $\pm 45$ [deg/dec] 
for only $0.1\omega_0 < \omega < 10\omega_0$. All other types of terms (i.e., constants and origin terms) have everlasting effects on the Bode plots -- at least for the scope of $\omega > 0$.
\\ \\ 
Additionally, the multiplicity $N$ (i.e., exponent) of a term scales that term's behavior on the Bode plots by a factor of $N$. For instance, the term $(j\omega)^N$ means it contributes a Bode magnitude slope of $20N$ [dB/dec] and 
a Bode phase of $90N$ [deg]. On the other hand, the term $\left(1+\frac{j\omega}{\omega_0}\right)^N$ contributes a Bode magnitude slope of $20N$ [dB/dec] for only $\omega>\omega_c$ and a Bode phase slope of $45N$ [deg/dec] for only 
$0.1\omega_0<\omega<10\omega_0$.
\\ \\
Similar to the linear combination of signals, the effects of each individual term of the frequency response can be added. Each term has a corresponding slope change on the Bode plot; these graphical properties 
are outlined in Table \ref{add_terms}.
\begin{table}
    \small
    \centering
    \caption{Graphical Properties of Frequency Response Terms on the Bode Plot}
    \label{add_terms}
    \resizebox{\textwidth}{!}{\begin{tabular}{ |c|c|c| }
        \hline
        & \shortstack{Magnitude Slope Change \\ $\Delta m(\Omega) = m(\Omega^+) - m(\Omega^-)$} & \shortstack{Phase Slope Change \\ $\Delta \phi(\Omega) = \phi(\Omega^+) - \phi(\Omega^-)$} \\[0.1cm]
        \hline
        Constant & None ($20\log|K|$ everywhere) & None ( $\phase{K}$ everywhere) \\[0.1cm]
        Zero at origin & None ($+20$ [dB/dec] everywhere) & None ($+90^{\circ}$ everywhere) \\[0.1cm]
        Pole at origin & None ($-20$ [dB/dec] everywhere) & None ($-90^{\circ}$ everywhere) \\[0.1cm]
        Real zero & $\Delta m(\omega_0)=+20$ [dB/dec] & $\Delta \phi(0.1\omega_0) = +45$ [deg/dec], $\Delta \phi(10\omega_0) = -45$ [deg/dec] \\[0.1cm]
        Real pole & $\Delta m(\omega_0)=-20$ [dB/dec] & $\Delta \phi(0.1\omega_0) = -45$ [deg/dec], $\Delta \phi(10\omega_0) = +45$ [deg/dec] \\
        \hline
    \end{tabular}}
\end{table}
\\ \\
The same tips for waveform synthesis and deconstruction can be applied to Bode plots. 
\begin{tcolorbox}[width=\textwidth,colback={white}, sharp corners]
    Bode plot graphing tips:
    \begin{itemize}
        \item Rewrite the frequency response in standard form
        \item Mark all starting and ending angular frequencies on the $\omega$-axis
        \item Determine a starting value from any constant and origin terms (applying any multiplicity if necessary)
        \item From the starting point, make changes at each marked angular frequency, going from left to right along the $\omega$-axis
    \end{itemize}
\end{tcolorbox}

\begin{example}
    Sketch the Bode plots for the frequency response
    \begin{align*}
        H(\omega) = \frac{48j(20+4j\omega)(1000+2j\omega)}{(4+20j\omega)^2(400+2j\omega)}.
    \end{align*}
\end{example}
\begin{solution}
    Putting $H(\omega)$ in standard form, 
    \begin{align*}
        H(\omega) &= \frac{48j\left[20\left(1+\frac{j\omega}{5}\right)\right]\left[1000\left(1+\frac{j\omega}{500}\right)\right]}{\left[4^2\left(1+\frac{j\omega}{0.2}\right)^2\right]\left[400\left(1+\frac{j\omega}{200}\right)\right]} \\
        &= \frac{150j\left(1+\frac{j\omega}{5}\right)\left(1+\frac{j\omega}{500}\right)}{\left(1+\frac{j\omega}{0.2}\right)^2\left(1+\frac{j\omega}{200}\right)}
    \end{align*}
    Focusing on just the Bode magnitude plot first, we make a note of all slope changes due to each term at each frequency.
    \begin{align*}
        150j &\Longrightarrow 20\log|K| = 43.5 \text{ [dB] } \\
        \left(1+\frac{j\omega}{5}\right) &\Longrightarrow \Delta m = 20 \text{ [dB/dec] at } \omega_0 = 5 \\
        \left(1+\frac{j\omega}{500}\right) &\Longrightarrow \Delta m = 20 \text{ [dB/dec] at } \omega_0 = 500 \\
        \frac{1}{\left(1+\frac{j\omega}{0.2}\right)^2} &\Longrightarrow \Delta m = -40 \text{ [dB/dec] at } \omega_0 = 0.2 \\
        \frac{1}{1+\frac{j\omega}{200}} &\Longrightarrow \Delta m = -20 \text{ [dB/dec] at } \omega_0 = 200
    \end{align*}
    Collecting the magnitude changes with respect to angular frequency:
    \begin{align*}
        \omega = 0 &\Longrightarrow \text{start at 43.5 [dB]} \\ 
        \omega = 0.2 &\Longrightarrow \Delta m = -40 \\
        \omega = 5 &\Longrightarrow \Delta m = 20 \\
        \omega = 200 &\Longrightarrow \Delta m = -20 \\
        \omega = 500 &\Longrightarrow \Delta m = 20
    \end{align*}
    Using the collection of magnitude changes, plot the Bode magnitude plot.
    \\ \\ \\
    \resizebox{\textwidth}{!}{\begin{tikzpicture}
        \tikzset{
            semilog lines/.style={black},
            semilog lines 2/.style={gray!50},
            semilog half lines/.style={white, dotted},
            semilog label x/.style={below,font=\small},
            semilog label y/.style={above,font=\small} }
        \begin{scope}[xscale=10/6,yscale=5/200]
            \OrdBode{50}
            \semilog*{-2}{4}{-100}{100}
            \node at (1,-130){Angular Frequency, $\omega$ [rad/s]};
            \node[rotate=90] at (-2.7,0){Magnitude, $|H(\omega)|$ [dB]};
            \draw[blue, -] (-2,43.5) -- (-0.6990,43.5) -- (0.6990,-12.4176) -- (2.3010,-44.4588) -- (2.6990,-60.3764) -- (4,-86.3970);
            \foreach \Point in {(-0.6990,43.5), (0.6990,-12.4176), (2.3010,-44.4588), (2.6990,-60.3764)}{
                \node at \Point {\scriptsize \textbullet};
            }
        \end{scope}
    \end{tikzpicture}}
    \\ \\ \\
    Similarly for the Bode phase plot, we make a note of all slope changes.
    \begin{align*}
        150j &\Longrightarrow \phase{K} = 90^{\circ} \\
        \left(1+\frac{j\omega}{5}\right) &\Longrightarrow \Delta m = +45 \text{ [deg/dec] at } 0.1\omega_0 = 0.5, \Delta m = -45 \text{ [deg/dec] at } 10\omega_0 = 50 \\
        \left(1+\frac{j\omega}{500}\right) &\Longrightarrow \Delta m = +45 \text{ [deg/dec] at } 0.1\omega_0 = 50, \Delta m = -45 \text{ [deg/dec] at } 10\omega_0 = 5000 \\
        \frac{1}{\left(1+\frac{j\omega}{0.2}\right)^2} &\Longrightarrow \Delta m = -90 \text{ [deg/dec] at } 0.1\omega_0 = 0.02, \Delta m = +90 \text{ [deg/dec] at } 10\omega_0 = 2 \\
        \frac{1}{1+\frac{j\omega}{200}} &\Longrightarrow \Delta m = -45 \text{ [deg/dec] at } 0.1\omega_0 = 20, \Delta m = +45 \text{ [deg/dec] at } 10\omega_0 = 2000
    \end{align*}
    Collecting the phase changes with respect to angular frequency:
    \begin{align*}
        \omega = 0 &\Longrightarrow \text{start at } 90^{\circ} \\ 
        \omega = 0.02 &\Longrightarrow \Delta m = -90 \\
        \omega = 0.5 &\Longrightarrow \Delta m = +45 \\
        \omega = 2 &\Longrightarrow \Delta m = +90 \\
        \omega = 20 &\Longrightarrow \Delta = -45 \\
        \omega = 50 &\Longrightarrow \Delta m = -45 + 45 = 0 \\
        \omega = 2000 &\Longrightarrow \Delta m = +45 \\
        \omega = 5000 &\Longrightarrow \Delta m = -45
    \end{align*}
    Using the collection of phase changes, plot the Bode phase plot.
    \\ \\ \\
    \resizebox{\textwidth}{!}{\begin{tikzpicture}
        \tikzset{
            semilog lines/.style={black},
            semilog lines 2/.style={gray!50},
            semilog half lines/.style={white, dotted},
            semilog label x/.style={below,font=\small},
            semilog label y/.style={above,font=\small} }
        \begin{scope}[xscale=10/6,yscale=5/180]
            \OrdBode{45}
            \semilog*{-2}{4}{-90}{90}
            \node at (1,-120){Angular Frequency, $\omega$ [rad/s]};
            \node[rotate=90] at (-2.7,0){Phase, $\phase{H(\omega)}$ [deg]};
            \draw[blue, -] (-2,90) -- (-1.6990,90) -- (-0.3010,-35.8146) -- (0.3010,-62.9073) -- (1.3010,-17.9073) -- (1.6990,-17.9073) -- (3.3010,-17.9073) -- (3.6990,0) -- (4,0);
            \foreach \Point in {(-1.6990,90), (-0.3010,-35.8146), (0.3010,-62.9073), (1.3010,-17.9073), (1.6990,-17.9073), (3.3010,-17.9073), (3.6990,0)}{
                \node at \Point {\scriptsize \textbullet};
            }
        \end{scope}
    \end{tikzpicture}}
\end{solution}
\end{document}
