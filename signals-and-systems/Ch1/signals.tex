\documentclass{report}
% PACKAGES
\usepackage{adjustbox}
\usepackage{amsmath}
\usepackage{amssymb}
\usepackage{bodegraph}
\usepackage{bbm}
\usepackage{circledsteps}
\usepackage{circuitikz}
\usepackage{enumerate}
\usepackage{mathtools}
\usepackage{nicematrix}
\usepackage{pdfpages}
\usepackage{pgfplots}
\usepackage{polynom}
\usepackage{qtree}
\usepackage{rotating}
\usepackage[usestackEOL]{stackengine}
\usepackage[subpreambles=true]{standalone}
\usepackage{steinmetz}
\usepackage{subcaption}
\usepackage{tabularray}
\usepackage{tcolorbox}
\usepackage{tikz}
\usepackage{xcolor}

\usepackage[colorlinks=true,linkcolor=blue,urlcolor=black,bookmarksopen=true]{hyperref}
\usepackage{bookmark}

\usetikzlibrary{shapes.arrows}
\usetikzlibrary{shapes.misc}
\usetikzlibrary{backgrounds}
\tikzset{cross/.style={cross out, draw=black, minimum size=2*(#1-\pgflinewidth), inner sep=0pt, outer sep=0pt},
%default radius will be 1pt. 
cross/.default={1pt}}

\renewcommand{\Re}{\operatorname{Re}}
\renewcommand{\Im}{\operatorname{Im}}

\usepackage[OT2,T1]{fontenc}
\DeclareSymbolFont{cyrletters}{OT2}{wncyr}{m}{n}
\DeclareMathSymbol{\Sha}{\mathalpha}{cyrletters}{"58}

\usepackage{pifont}
\newcommand{\cmark}{\text{\ding{51}}}
\newcommand{\xmark}{\text{\ding{55}}}

\newcommand{\circconv}[1]{\text{ \small\Circled{#1} }}

\newcommand{\tikzmark}[3][]{\tikz[remember picture,baseline] \node [anchor=base,#1](#2) {$#3$};}

\pgfplotsset{compat=1.18}
\pgfplotsset{
    dirac/.style={
        mark=triangle*,
        mark options={scale=1.5},
        ycomb,
        scatter,
        visualization depends on={y/abs(y)-1 \as \sign},
        scatter/@pre marker code/.code={\scope[rotate=90*\sign,yshift=-2pt]}
    }
}

\usepackage[letterpaper, portrait, margin=1.25in]{geometry}
\usepackage[font=bf]{caption}
\hbadness = 10000
\hfuzz=2pt

\newtheorem{theorem}{Theorem}[chapter]

\tikzset{every tree node/.style={anchor=north,align=center}}
\usetikzlibrary{decorations.markings}
\usetikzlibrary{arrows}
\definecolor{darkgreen}{rgb}{0.133,0.545,0.133}
\tikzstyle{n}= [circle, fill=blue, minimum size=4pt,inner sep=0pt, outer sep=0pt]

\patchcmd{\thebibliography}{\chapter*}{\section*}{}{}
\setcounter{secnumdepth}{5}

%%%%%%%%%%% EXAMPLE ENVIRONMENT %%%%%%%%%%
\usepackage{calc}
\usepackage{tabto}
\usepackage[framemethod=tikz]{mdframed} % for colored backgrounds
\newcommand{\halmos}{} % makes a box at the end

\newlength{\framedinnerleftmargin}
\newlength{\framedinnertopmargin}
\newlength{\framedreversedinnerleftmargin}
\setlength{\framedinnerleftmargin}{\widthof{Theoreme 10.10.10}+2em}
\setlength{\framedreversedinnerleftmargin}{\widthof{Theoreme 10.10.10}+1em}
\setlength{\framedinnertopmargin}{1em}
 
% first argument: label in upper left corner,
% second argument: background color
\newenvironment{boxedtext}[2]{\begin{mdframed}[%
hidealllines=true,%
backgroundcolor=#2,%
innertopmargin=\framedinnertopmargin,%
innerleftmargin=\framedinnerleftmargin,%
innerrightmargin=1em%
]%
\tabto{-\framedreversedinnerleftmargin}\textbf{#1}\tabto*{0em}%
}% begin code
{\hskip 0pt\\\hspace*{\fill}\halmos{}\end{mdframed}\vspace{1em}} % end code
 
\newenvironment{summary}[0]{\begin{center}\begin{minipage}[c]{\summarywidth}\begin{spacing}{0.9}\footnotesize} % begin code
{\end{spacing}\end{minipage}\end{center}} % end code

\newcounter{example}
 
% optional! if you want it to start at zero
% with every new chapter/section/etc.
\numberwithin{example}{section}
 
\newenvironment{example}[0]
{\refstepcounter{example}\vspace{1em plus 1em}\begin{boxedtext}{Example \theexample.}{blue!7}}%\setlength{\parskip}{0em}}
{\end{boxedtext}\vspace{-1em plus 1em}}
 
\newenvironment{example*}[0]
{\vspace{1em plus 1em}\begin{boxedtext}{Example.}{blue!7}}
{\end{boxedtext}\vspace{-1em plus 1em}}

\newmdenv[
  topline=false,
  bottomline=false,
  rightline=false,
  skipabove=\topsep,
  skipbelow=\topsep,
  linecolor=purple,
  frametitle={\noindent\textcolor{purple}{\textbf{SOLUTION }}},
  endinnercode={$\hfill\textcolor{purple}{\blacksquare}$}
]{solution}

%%%%%%%%%%%%%% COLOR BOXED %%%%%%%%%%%%%%%
% Syntax: \colorboxed[<color model>]{<color specification>}{<math formula>}
\newcommand*{\colorboxed}{}
\def\colorboxed#1#{%
  \colorboxedAux{#1}%
}
\newcommand*{\colorboxedAux}[3]{%
  % #1: optional argument for color model
  % #2: color specification
  % #3: formula
  \begingroup
    \colorlet{cb@saved}{.}%
    \color#1{#2}%
    \boxed{%
      \color{cb@saved}%
      #3%
    }%
  \endgroup
}
%%%%%%%%%%%%%%%%%%%%%%%%%%%%%%%%%%%%%%%%%

\begin{document}

\chapter{Signals}
Signals are essentially functions that convey information about some (usually physical) phenomenon. 

\section{Classification of Signals Based on Mapping}
Since signals are functions, signals can be classified by their domain and codomain. 

\subsection{Multichannel vs Multidimensional}
Signals that can be expressed as vector-valued functions are called \emph{multichannel signals}. These signals typically are generated 
from multiple sources or multiple sensors. An \emph{N-channel} signal may be written as 
\begin{align}
    \mathbf{S}(t) = 
    \begin{bmatrix}
        S_1(t) \\
        S_2(t) \\
        \vdots \\
        S_N(t)
    \end{bmatrix} \text{, for } t \in \mathbb{R}.
\end{align}
An example of a multichannel signal is the electrocardiogram (ECG or EKG), which utilizes either 3 leads or 12 leads to measure 
the electrical activity of the heart. As a result, 3-channel or 12-channel signals are recorded.

Phased arrays, commonly used for radar and 5G MIMO communications, also transmit multichannel signals. Each element in the antenna array is applied different  
phase shifts such that the collective beam is steered to a specific direction. Here, each element represents a different channel.
\\ \\
Signals that can be expressed as multivariate functions are called \emph{multidimensional signals}. An \emph{M-dimensional} signal may 
be written as
\begin{align}
    S(x_1,x_2,...,x_M) \text{, for } 
    \begin{bmatrix}
        x_1 \\
        x_2 \\
        \vdots \\
        x_M
    \end{bmatrix} \in \mathbb{R}^M.
\end{align}

An example of a multidimensional signal is a black-and-white image. The intensity or brightness of each point in an image is a function of two 
variables, often notated as $I(x,y)$. This can be extended to 3-dimensional signals in the form of black-and-white videos, notated as $I(x,y,t)$. 
After all, a single frame of a video is merely just an image.
\\ \\
Some signals can be both multichannel and multidimensional. That is, 
\begin{align}
    \mathbf{S}(x_1,x_2,...x_M) = 
    \begin{bmatrix}
        S_1(x_1,x_2,...x_M) \\
        S_2(x_1,x_2,...x_M) \\
        \vdots \\
        S_N(x_1,x_2,...x_M)
    \end{bmatrix}  \text{, for } 
    \begin{bmatrix}
        x_1 \\
        x_2 \\
        \vdots \\
        x_M
    \end{bmatrix} \in \mathbb{R}^M.
\end{align}
With the introduction of color in digital media, the RGB (red, green, blue) model becomes prevalent in 
creating techological devices that can display color such as monitors and mobile phones. As a result, color images are both 3-channel and 2-dimensional, with intensity 
denoted as 
\begin{align}
    \mathbf{I}(x,y)= 
    \begin{bmatrix}
        I_R(x,y) \\
        I_G(x,y) \\
        I_B(x,y)
    \end{bmatrix},
\end{align}
and color videos are 3-channel and 3-dimensional, with intensity denoted as
\begin{align}
    \mathbf{I}(x,y,t)= 
    \begin{bmatrix}
        I_R(x,y,t) \\
        I_G(x,y,t) \\
        I_B(x,y,t)
    \end{bmatrix}.
\end{align}

For the rest of this text, we will only discuss single-channel, one-dimensional signals (specifically time-varying signals) -- these will simply be referred to as signals. There are 
countless examples of these signals, some of which include audio (mono, as stereo is 2-channel), speech, circuit voltages, and transmitted messages.

\subsection{Continuous-Time vs Discrete-Time}
\emph{Continuous-time signals} (or CT signals for short) are defined for every value of time $t\in\mathbb{R}$ -- whether or not many of these values end up being zero depends on the physical nature of the 
signal itself. Another way to rephrase is that continuous-time signals operate in the continuous time domain. These signals are often notated with parentheses, $x(t)$.
\\ \\
\emph{Discrete-time signals} (or DT signals for short) are defined only at specific values of time, usually integers $n\in\mathbb{Z}$. That is, discrete-time signals operate in the discrete time domain. 
Sampling a continuous-time signal at periodic intervals produces a discrete-time signal. Discrete-time signals are often notated with brackets, $x[n]$, and are represented as sequences.
\\ \\ 
Note that the terms continuous-time and discrete-time describe signals that vary with time. For multidimensional signals that vary with location such as images, the terms \emph{continuous-space} and \emph{discrete-space} 
are more appropriate.

\subsection{Continuous-Valued vs Discrete-Valued}
If the amplitude of a signal can take on any value within a finite or infinite range, the signal is \emph{continuous-valued}. Otherwise, if a signal can only take on a finite set of values 
within the range as its amplitude, the signal is said to be \emph{discrete-valued}. 
\\ \\
For instance, given a range of $[0,1]$, if the signal can capture values of 0.234567, $\pi/4$, 0.8999..., and every number between those values, then it is continuous-valued. A discrete-valued signal may only 
be able to take on $\{0,0.5\}$, or perhaps $\{0,0.25,0.5,0.75\}$, or $\{0, 1/8, ..., 7/8\}$ -- usually there is a logic to what intermediate values these countable sets are limited to. An $N$-bit quantizer would influence 
the precision of the values within $2^{-N}$.
\\ \\
Furthermore, a signal that is both continuous-valued and continuous-time is called an \emph{analog signal}. Similarly, a signal that is both discrete-valued and discrete-time is called 
a \emph{digital signal}.

\begin{center}
    \begin{tabular}{ |c|c|c| } 
    \hline
    & Continuous-Valued & Discrete-Valued \\ 
    \hline
    Continuous-Time (CT) & Analog & --- \\ 
    \hline
    Discrete-Time (DT) & --- & Digital \\ 
    \hline
    \end{tabular}
\end{center}

\begin{figure}[hbt!]
    \begin{center}
    \caption{Signals based on time and range}
    \label{diff_signals}
    \begin{tikzpicture}
        \begin{axis}[
            axis x line=center, axis y line=center,
            ymin=-1.75, ymax=1.75, ytick={-1,0,1}, ylabel={Analog signal, $x(t)$},
            xmin=0, xmax=6, xtick={0,...,5}, xlabel={$t$ [s]},
            domain=-1:5.5,samples=200,
            width=8cm, height=4.5cm]
            \addplot [blue,thick]{sin(deg(x))};
        \end{axis}
    \end{tikzpicture} \\
    \begin{tikzpicture}
        \begin{axis}[
            axis x line=center, axis y line=center,
            ymin=-1.75, ymax=1.75, ytick={-1,0,1}, ylabel={DT continuous-valued signal, $x[n]$},
            xmin=0, xmax=6, xtick={0,...,5}, xlabel={$n$},
            domain=-1:5.5,samples=200,
            width=8cm, height=4.5cm]
            \addplot+ [
                ycomb,
            ] coordinates {
                (0,0) (1,0.8415) (2,0.9093) (3,0.1411) (4,-0.7568) (5,-0.9589)
            };
        \end{axis}
    \end{tikzpicture} \\
    \begin{tikzpicture}
        \begin{axis}[
            axis x line=center, axis y line=center,
            ymin=-1.75, ymax=1.75, ytick={-1,0,1}, ylabel={Digital signal, $x_q[n]$},
            xmin=0, xmax=6, xtick={0,...,5}, xlabel={$n$},
            domain=-1:5.5,samples=200,
            width=8cm, height=4.5cm]
            \addplot+ [
                ycomb,
            ] coordinates {
                (0,0) (1,1) (2,1) (3,0) (4,0) (5,-1)
            };
        \end{axis}
    \end{tikzpicture} \\
    \begin{tikzpicture}
        [declare function={
            func(\x)= (\x < 1) * (0) + and(\x >= 1, \x < 3) * (1) + and(\x >= 3, \x < 5) * (0) + (\x >= 5) * (-1); }]
        \begin{axis}[
            axis x line=center, axis y line=center,
            ymin=-1.75, ymax=1.75, ytick={-1,0,1}, ylabel={CT discrete-valued signal, $x_q(t)$},
            xmin=0, xmax=6, xtick={0,...,5}, xlabel={$t$ [s]},
            domain=-1:5.5,samples=200,
            width=8cm, height=4.5cm]
            \addplot [blue,thick]{func(x)};
        \end{axis}
    \end{tikzpicture}
    \end{center}
\end{figure}

In Figure \ref{diff_signals}, an analog signal $x(t)$ is first plotted. When \emph{sampled}, a discrete-time continuous-valued signal $x[n]$ 
is formed. Because of the precision that modern processors offer, sometimes these signals are also confusingly referred to as digital signals; however, while $x[n]$ can have infinite precision, 
a digital signal $x_q[n]$ has limited precision. Digital signal $x_q[n]$ is plotted, where in this version, the signal has limited \emph{quantization levels} $\{-1,0,1\}$.  
This \emph{quantized} signal $x_q[n]$ can then be ``\emph{held}'' such that a continuous-time discrete-valued signal $x_q(t)$ is formed. \\ \\
For the rest of this text, only analog signals are of interest. 


\subsection{Deterministic vs Random}
Lastly, if a signal can be reproduced without randomness dictating the future of the signal, then it is said to be a \emph{deterministic signal}. Otherwise, it is 
classified as a \emph{random signal}. 
\\ \\
While one could argue a noisy sine wave is irreproducible, it is merely a sum of two signals: a deterministic sine wave and random noise. 
Each counterpart could be studied on its own. For the rest of this text, only deterministic signals are of interest. 

\begin{tcolorbox}[width=\textwidth,colback={white}, sharp corners]
Collectively, we will only be looking into signals with the following characteristics for the remainder of this text:
\begin{itemize}
    \item single-channel, one-dimensional
    \item continuous-time, continuous-valued (analog)
    \item deterministic
\end{itemize}
\end{tcolorbox}

\section{Signal Transformations}
One of the simplest ways a signal can be modified is by an \emph{affine transformation on the independent variable}. That is, $x(t) \mapsto x(at+b)$.

\subsection{Time Shifting}
A CT signal that is time-shifted by $T$ seconds can be expressed as 
\begin{align}
    y(t) = x(t-T).
\end{align}
If $T>0$, then $y(t)$ is said to be delayed by $T$ seconds relative to $x(t)$. If $T<0$, then $y(t)$ is advanced by $T$ seconds.

\subsection{Time Scaling}
A CT signal that is time-scaled by some factor $a$ can be expressed as 
\begin{align}
    y(t) = x(at).
\end{align}
If $|a|>1$, then $y(t)$ is a temporally compressed version of $x(t)$. If $|a|<1$, then $y(t)$ is a temporally expanded version of $x(t)$.

\subsection{Time Reversal}
A time reversal is a reflection of some signal over the vertical axis and can be expressed as 
\begin{align}
    y(t)=x(-t).
\end{align}

\subsection{Combined Transformation}
Let 
\begin{align}
    y(t) = x(at-b) = x\left(a\left(t-\frac{b}{a}\right)\right).
\end{align}
There are two approaches to transforming the signal $x(t)$ to $y(t)$.
\begin{table}[hbt!]
\centering
\begin{tabular}{ |p{7cm}|p{7cm}| }
    \hline
    Approach 1 & Approach 2 \\
    \hline
    \begin{enumerate}
        \itemsep0em
        \item Time scale by $|a|$. Then reflect if $a<0$.
        \item Shift by $b/a$ units.
    \end{enumerate} &
    \begin{enumerate}
        \itemsep0em
        \item Shift by $b$ units.
        \item Time scale by $|a|$. Then reflect if $a<0$.
    \end{enumerate}\\
    \hline
\end{tabular}
\end{table}

\begin{example}
    Let $x(t)$ be characterized by the following plot. \\ \\
    \begin{tikzpicture}
        [declare function={
            func(\x)= (\x < -1) * (0) + and(\x >= -1, \x < 2) * (abs(\x)) + (\x > 2) * (0); }]
        \begin{axis}[
            axis x line=center, axis y line=center,
            ymin=0, ymax=2.5, ytick={0,1,2}, ylabel={$x(t)$},
            xmin=-2, xmax=3, xtick={-2,...,2}, xlabel={$t$ [s]},
            domain=-2:3,samples=200,
            width=7cm, height=4cm]
        \addplot [blue,thick]{func(x)};
        \end{axis}
    \end{tikzpicture} \\ \\
    Plot $y(t)=x\left(\frac{1}{2}(t+4)\right)$.
\end{example}
\begin{solution}
    Using Approach 1: \\ \\
    \begin{tikzpicture}
        [declare function={
            func(\x)= (\x < -1) * (0) + and(\x >= -1, \x < 2) * (abs(\x)) + (\x > 2) * (0); }]
        \begin{axis}[
            axis x line=center, axis y line=center,
            ymin=0, ymax=2.5, ytick={0,1,2}, ylabel={$ $},
            xmin=-3, xmax=5, xtick={-2,...,4}, xlabel={$t$ [s]},
            domain=-3:5,samples=200,
            width=8cm, height=4cm]
        \addplot [blue,thick]{func(x)} node[above,pos=0.6] {$x(t)$};
        \addplot [dashed,red,thick]{func(x/2)} node[above,pos=0.75] {$x\left(\frac{1}{2}t\right)$};
        \end{axis}
    \end{tikzpicture} 
    \begin{tikzpicture}
        [declare function={
            func(\x)= (\x < -1) * (0) + and(\x >= -1, \x < 2) * (abs(\x)) + (\x > 2) * (0); }]
        \begin{axis}[
            axis x line=center, axis y line=center,
            ymin=0, ymax=2.5, ytick={0,1,2}, ylabel={$ $},
            xmin=-6.5, xmax=3, xtick={-6,...,2}, xlabel={$t$ [s]},
            domain=-7:3,samples=200,
            width=9.5cm, height=4cm]
        \addplot [blue,thick]{func(x)} node[above,pos=0.8] {$x(t)$};
        \addplot [red,thick]{func((x+4)/2)} node[above,left,pos=0.5] {$x\left(\frac{1}{2}(t+4)\right)$};
        \end{axis}
    \end{tikzpicture} \\ \\
    Here, the plot is first expanded to twice the duration. Then the expanded plot is translated 4 units to the left. \\ \\ 
    Using Approach 2, for $y(t)=x\left(\frac{1}{2}(t+4)\right)=x\left(\frac{1}{2}t+2\right)$: \\ \\ 
    \begin{tikzpicture}
        [declare function={
            func(\x)= (\x < -1) * (0) + and(\x >= -1, \x < 2) * (abs(\x)) + (\x > 2) * (0); }]
        \begin{axis}[
            axis x line=center, axis y line=center,
            ymin=0, ymax=2.5, ytick={0,1,2}, ylabel={$ $},
            xmin=-4, xmax=3, xtick={-3,...,2}, xlabel={$t$ [s]},
            domain=-4:3,samples=200,
            width=7.5cm, height=4cm]
        \addplot [blue,thick]{func(x)} node[above,pos=0.75] {$x(t)$};
        \addplot [dashed,red,thick]{func(x+2)} node[above,pos=0.2] {$x\left(t+2\right)$};
        \end{axis}
    \end{tikzpicture} 
    \begin{tikzpicture}
        [declare function={
            func(\x)= (\x < -1) * (0) + and(\x >= -1, \x < 2) * (abs(\x)) + (\x > 2) * (0); }]
        \begin{axis}[
            axis x line=center, axis y line=center,
            ymin=0, ymax=2.5, ytick={0,1,2}, ylabel={$ $},
            xmin=-6.5, xmax=3, xtick={-6,...,2}, xlabel={$t$ [s]},
            domain=-7:3,samples=200,
            width=9.5cm, height=4cm]
        \addplot [blue,thick]{func(x)} node[above,pos=0.8] {$x(t)$};
        \addplot [red,thick]{func(x/2+2)} node[above,left,pos=0.5] {$x\left(\frac{1}{2}t+2\right)$};
        \end{axis}
    \end{tikzpicture} \\ \\
    Here, the plot is first translated 2 units to the left. Then the shifted plot is expanded to twice the duration, relative 
    to time $t=0$.
\end{solution}

\section{Waveforms}
\subsection{Exponential and Sinusoidal Signals}
\emph{Complex exponential functions} are periodic functions that can be described as 
\begin{align}
    x(t) = Ae^{j\omega_0 t}=A\operatorname{exp}(j\omega_0 t),
\end{align}
with some complex scalar $A = |A|e^{j\theta}$ and \emph{fundamental angular frequency} $\omega_0$. As such, these functions can be rewritten as
\begin{align}
    x(t) = |A|e^{j(\omega_0 t + \theta)}.
\end{align}
The real and imaginary parts of the exponential function can be isolated to obtain \emph{sinusoids}:
\begin{align}
    \Re(Ae^{j\omega_0 t}) = |A| \cos(\omega_0 t + \theta) \\
    \Im(Ae^{j\omega_0 t}) = |A| \sin(\omega_0 t + \theta)
\end{align}
As complex exponential functions are periodic, it follows that both the real and imaginary parts are periodic as well. Therefore, both cosine and sine 
functions are periodic with fundamental angular frequency $\omega_0$.

\begin{figure}[hbt!]
    \caption{Complex exponential function and its components}
    \resizebox{\textwidth}{!}{%
        \includestandalone[mode=buildnew, width=\textwidth]{complex_3d}
    }
\end{figure}

\subsection{Singularity Signals}
\emph{Singularity functions} are a class of discontinuous functions that are continuous everywhere except at a singularity point. That is, a function which either is not continuous everywhere 
or has a some-ordered derivative that is not continuous everywhere is considered a singularity function. 
When not translated, the singularity point is at $t=0$, where the value is undefined. 
\\ \\
The \emph{unit impulse function}, also called the \emph{Dirac delta function}, is defined as 
\begin{align}
    \delta(t-T) = 
    \begin{cases} 
        +\infty, & t=T \\
        0, & t \neq T
    \end{cases}, \\
    \text{with } \int_{-\infty}^{+\infty} \delta(t-T) \,dt = 1.
\end{align}
Graphically, $\delta(t-T)$ is plotted with a vertical arrow of length 1 pointing up from the $t$-axis at time $t=T$, with length 1 representing the area under the curve of an impulse 
function is 1. When multiplied by a scalar $k$, the plot of $k\cdot\delta(t-T)$ has an arrow length $|k|$ and may point down if $k<0$. \\
Interestingly, when time-scaled by a factor $a$, 
\begin{align}
    \delta(at) = \frac{1}{|a|}\cdot \delta(t).
\end{align}
The impulse function also has a unique \emph{sampling property}, also called \emph{sifting property}, and is defined as
\begin{align}
    \int_{-\infty}^{+\infty} x(t)\delta(t-T) \,dt &= x(T), \\
    x(t)\delta(t-T) &= x(T)\delta(t-T).
\end{align} \\
The antiderivative of the unit impulse function is the \emph{unit step function}, also called the \emph{Heaviside step function}, and is defined as 
\begin{align}
    u(t-T) = 
    \begin{cases} 
        0, & t<T \\
        1, & t>T
    \end{cases}.
\end{align}
Essentially, the unit step function can be treated as an ``on switch'', where at time $t=T$, the signal being multiplied gets turned on. In this case, since $u(t-T)=1\cdot u(t-T)$, 
the constant function is inactive up until time $t=T$, at which the constant function is finally turned on. This concept becomes important in creating other signals. \\ \\
Similarly, the time-reversed step function $u(T-t)$ can be treated as an ``off switch'', where at time $t=T$, the signal being multiplied gets turned off.
\\ \\
The antiderivative of the unit step function is the \emph{unit ramp function}, which is defined as
\begin{align}
    r(t-T) = \operatorname{ramp}(t-T) &= (t-T)u(t-T) \\
    & =
    \begin{cases} 
        0, & t<T \\
        t-T, & t>T
    \end{cases}.
\end{align}
Notice that the ramp function can be written as a product of the linear function and the unit step function. Using the ``on switch'' concept, the linear function is suppressed 
for all times before $T$; however, once time $t=T$ has arrived, the linear function is activated.
\\ \\
The antiderivative of the unit ramp function is the \emph{unit parabolic function}, which is defined as 
\begin{align}
    \operatorname{quad}(t-T) &= \frac{1}{2} (t-T)^2 u(t-T) \\
    &=
    \begin{cases} 
        0, & t<T \\
        \frac{1}{2}(t-T)^2, & t>T
    \end{cases}.
\end{align}

\subsection{Modifying Exponential, Sinusoidal, and Singularity Signals}
The \emph{unit exponential decay} with time constant $\tau$ is defined as
\begin{align}
    \operatorname{exp}[-(t-T)/\tau]\ u(t-T), \text{ for } \tau>0.
\end{align} \\
The \emph{rectangular pulse} is defined as 
\begin{align}
    \operatorname{rect}(t) = \Pi(t) &= u\left(t+\frac{1}{2}\right) - u\left(t-\frac{1}{2}\right) \\
    &=
    \begin{cases} 
        1, & |t|<\frac{1}{2} \\
        0, & otherwise
    \end{cases}.
\end{align}
More commonly, when written as $\operatorname{rect}\left(\frac{t-T}{\tau}\right)$, the rectangular pulse has a width $\tau$ that is centered at $t=T$.
\\ \\
The \emph{unnormalized triangular pulse}, or simply the \emph{triangular pulse}, is defined as 
\begin{align}
    \operatorname{tri}(t) = \Lambda(t) =
    \begin{cases} 
        1-|t|, & |t|<1 \\
        0, & otherwise
    \end{cases}.
\end{align}
When written as $\operatorname{tri}\left(\frac{t-T}{\tau}\right)$, the unnormalized triangular pulse has a width $2\tau$ that is centered at $t=T$.
\\ \\
The \emph{normalized triangular pulse} is defined as 
\begin{align}
    \overline{\operatorname{tri}}(t) = \overline{\Lambda}(t) =
    \begin{cases} 
        1-2|t|, & |t|<\frac{1}{2} \\
        0, & otherwise
    \end{cases}.
\end{align}
When written as $\overline{\operatorname{tri}}\left(\frac{t-T}{\tau}\right)$, the normalized triangular pulse has a width $\tau$ that is centered at $t=T$.
\\ \\
The \emph{signum function}, also called the \emph{sign function}, is defined as 
\begin{align}
    \operatorname{sgn}(t) &= u(t)-u(-t) \\ 
    &= 2u(t)-1 \\
    &=
    \begin{cases} 
        -1, & t<0 \\
        0, & t=0 \\
        1, & t>0
    \end{cases}.
\end{align}
The \emph{unnormalized sinc function}, also called the \emph{sampling function}, is defined as 
\begin{align}
    \operatorname{sinc}_u(t) = \operatorname{Sa}(t)=\frac{\sin(t)}{t}, \text{ for } \operatorname{sinc}_u(0) = 1.
\end{align}
The \emph{normalized sinc function}, or simply the \emph{sinc function}, is defined as 
\begin{align}
    \operatorname{sinc}(t) = \frac{\sin(\pi t)}{\pi t}, \text{ for } \operatorname{sinc}(0) = 1.
\end{align} \\
The waveforms of all signals described in this section can be found in Table \ref{waveforms}. 
\begin{example}
    Given $x(t)=u(t-3)u(5-t)$, express $x(-2t-1)$ as a piecewise function.
\end{example}
\begin{solution}
    For $b>a$, the expression $u(t-a)u(b-t)$ can rewritten as $u(t-a)-u(t-b)$. Therefore,
    \begin{align*}
        x(t) = u(t-3)u(5-t) &= u(t-3) - u(t-5) \\
        &= \begin{cases}
            1, & 3 < t < 5 \\
            0, & otherwise
        \end{cases}
    \end{align*}
    Substituting $t\leftarrow (-2t-1)$,
    \begin{align*}
        x(-2t-1) &= \begin{cases}
            1, & 3 < -2t-1 < 5 \\
            0, & otherwise
        \end{cases} \\
        &= \begin{cases}
            1, & -3 < t < -2 \\
            0, & otherwise
        \end{cases} \\
        &= u(t+3)-u(t+2).
    \end{align*}
\end{solution}

\begin{center}
    \begin{table}
    \small
    \centering
    \caption{Common Waveforms}
    \label{waveforms}
    \resizebox{\textwidth}{!}{%
    \begin{tabular}{ c|c|c }
    Function & Expression & General Shape \\[0.1cm]
    \hline 
    Unit impulse & $\delta(t-T) = 
        \begin{cases} 
            +\infty, & t=T \\
            0, & t \neq T
        \end{cases}$ & 
    \adjustbox{valign=m}{\resizebox{0.25\textwidth}{!}{
        \begin{tikzpicture}
            \begin{axis}[
                axis x line=center, axis y line=center,
                ymin=0, ymax=1.75, ytick={0,1}, ylabel={$\delta(t)$},
                xmin=-2, xmax=2, xtick={-2,...,2}, xlabel={$t$ [s]},
                domain=-1.5:1.5,samples=200,
                width=6cm, height=3cm]
            \addplot +[dirac] coordinates {(0,1)};
            \end{axis}
        \end{tikzpicture}}} \\[1cm]
    Unit step & $u(t-T) = 
        \begin{cases} 
            0, & t<T \\
            1, & t>T
        \end{cases}$ & 
    \adjustbox{valign=m}{\resizebox{0.25\textwidth}{!}{
        \begin{tikzpicture}
            [declare function={
                func(\x)= (\x < 0) * (0) + (\x >= 0) * (1); }]
            \begin{axis}[
                axis x line=center, axis y line=center,
                ymin=0, ymax=1.75, ytick={0,1}, ylabel={$u(t)$},
                xmin=-2, xmax=2, xtick={-2,...,2}, xlabel={$t$ [s]},
                domain=-1.5:1.5,samples=200,
                width=6cm, height=3cm]
            \addplot [blue,thick]{func(x)};
            \end{axis}
        \end{tikzpicture}}} \\[1cm]
    Unit ramp & $r(t-T) = \operatorname{ramp}(t-T) = (t-T)u(t-T)$ & 
    \adjustbox{valign=m}{\resizebox{0.25\textwidth}{!}{
        \begin{tikzpicture}
            [declare function={
                func(\x)= (\x < 0) * (0) + (\x >= 0) * (\x); }]
            \begin{axis}[
                axis x line=center, axis y line=center,
                ymin=0, ymax=1.75, ytick={0,1}, ylabel={$r(t)$},
                xmin=-2, xmax=2, xtick={-2,...,2}, xlabel={$t$ [s]},
                domain=-1.5:1.5,samples=200,
                width=6cm, height=3cm]
            \addplot [blue,thick]{func(x)};
            \end{axis}
        \end{tikzpicture}}} \\[1cm]
    Unit parabolic & $\operatorname{quad}(t-T) = \frac{1}{2} (t-T)^2 u(t-T)$ & 
    \adjustbox{valign=m}{\resizebox{0.25\textwidth}{!}{
        \begin{tikzpicture}
            [declare function={
                func(\x)= (\x < 0) * (0) + (\x >= 0) * (0.5 * \x^2); }]
            \begin{axis}[
                axis x line=center, axis y line=center,
                ymin=0, ymax=1.75, ytick={0,1}, ylabel={$\text{quad}(t)$},
                xmin=-2, xmax=2, xtick={-2,...,2}, xlabel={$t$ [s]},
                domain=-1.5:1.5,samples=200,
                width=6cm, height=3cm]
            \addplot [blue,thick]{func(x)};
            \end{axis}
        \end{tikzpicture}}} \\[1cm]
    Unit exponential decay & $\operatorname{exp}[-(t-T)/\tau]\ u(t-T)$ & 
    \adjustbox{valign=m}{\resizebox{0.25\textwidth}{!}{
        \begin{tikzpicture}
            [declare function={
                func(\x)= (\x < 0) * (0) + (\x >= 0) * (exp(-x)); }]
            \begin{axis}[
                axis x line=center, axis y line=center,
                ymin=0, ymax=1.75, ytick={0,1}, ylabel={$e^{-t/\tau}u(t)$},
                xmin=-2, xmax=2, xtick={-2,...,2}, xlabel={$t$ [s]},
                domain=-1.5:1.5,samples=200,
                width=6cm, height=3cm]
            \addplot [blue,thick]{func(x)};
            \end{axis}
        \end{tikzpicture}}} \\[1cm]
    Rectangular pulse & $\begin{aligned}
        \operatorname{rect}\left(\frac{t-T}{\tau}\right) &= \Pi\left(\frac{t-T}{\tau}\right) = u(t-T_1) - u(t-T_2) \\
        T_1 &= T-\frac{\tau}{2},\text{\quad} T_2 = T+\frac{\tau}{2}
    \end{aligned}$ & 
    \adjustbox{valign=m}{\resizebox{0.25\textwidth}{!}{
        \begin{tikzpicture}
            [declare function={
                func(\x)= (\x < -0.5) * (0) + and(\x >= -0.5, \x <= 0.5) * (1) + (\x >= 0.5) * (0); }]
            \begin{axis}[
                axis x line=center, axis y line=center,
                ymin=0, ymax=1.75, ytick={0,1}, ylabel={$\text{rect}(t)$},
                xmin=-2, xmax=2, xtick={-2,...,2}, xlabel={$t$ [s]},
                domain=-1.5:1.5,samples=200,
                width=6cm, height=3cm]
            \addplot [blue,thick]{func(x)};
            \end{axis}
        \end{tikzpicture}}} \\[1cm]
    (Unnormalized) triangular pulse & $\begin{aligned}\operatorname{tri}\left(\frac{t-T}{\tau}\right) = &\Lambda\left(\frac{t-T}{\tau}\right) = r(t-T_1)-2r(t-T)-r(t-T_2) \\ 
        &T_1 = T-\tau,\text{\quad} T_2 = T+\tau
    \end{aligned}$ & 
    \adjustbox{valign=m}{\resizebox{0.25\textwidth}{!}{
        \begin{tikzpicture}
            [declare function={
                func(\x)= (\x < -1) * (0) + and(\x >= -1, \x <= 1) * (1-abs(\x)) + (\x >= 1) * (0); }]
            \begin{axis}[
                axis x line=center, axis y line=center,
                ymin=0, ymax=1.75, ytick={0,1}, ylabel={$\text{tri}(t)$},
                xmin=-2, xmax=2, xtick={-2,...,2}, xlabel={$t$ [s]},
                domain=-1.5:1.5,samples=200,
                width=6cm, height=3cm]
            \addplot [blue,thick]{func(x)};
            \end{axis}
        \end{tikzpicture}}} \\[1cm]
    Normalized triangular pulse & $\begin{aligned}\overline{\operatorname{tri}}\left(\frac{t-T}{\tau}\right) = &\overline{\Lambda}\left(\frac{t-T}{\tau}\right) = 2r(t-T_1)-4r(t-T)-2r(t-T_2) \\ 
        &T_1 = T-\frac{\tau}{2},\text{\quad} T_2 = T+\frac{\tau}{2}
    \end{aligned}$ & 
    \adjustbox{valign=m}{\resizebox{0.25\textwidth}{!}{
        \begin{tikzpicture}
            [declare function={
                func(\x)= (\x < -0.5) * (0) + and(\x >= -0.5, \x <= 0.5) * (1-2*abs(\x)) + (\x >= 0.5) * (0); }]
            \begin{axis}[
                axis x line=center, axis y line=center,
                ymin=0, ymax=1.75, ytick={0,1}, ylabel={$\overline{\text{tri}}(t)$},
                xmin=-2, xmax=2, xtick={-2,...,2}, xlabel={$t$ [s]},
                domain=-1.5:1.5,samples=200,
                width=6cm, height=3cm]
            \addplot [blue,thick]{func(x)};
            \end{axis}
        \end{tikzpicture}}} \\[1cm]
    Signum function & $\operatorname{sgn}(t) = u(t)-u(-t)$ & 
    \adjustbox{valign=m}{\resizebox{0.25\textwidth}{!}{
        \begin{tikzpicture}
            [declare function={
                func(\x)= (\x < 0) * (-1) + (\x == 0) * (0) + (\x > 0) * (1); }]
            \begin{axis}[
                axis x line=center, axis y line=center,
                ymin=-1.75, ymax=1.75, ytick={-1,0,1}, ylabel={$\text{sgn}(t)$},
                xmin=-2, xmax=2, xtick={-2,...,2}, xlabel={$t$ [s]},
                domain=-1.5:1.5,samples=200,
                width=6cm, height=3.75cm]
            \addplot [blue,thick]{func(x)};
            \end{axis}
        \end{tikzpicture}}} \\[1cm]
    Sampling function & $\operatorname{sinc}_u(t)=\operatorname{Sa}(t)=\dfrac{\sin(t)}{t}, \text{ for } \operatorname{sinc}_u(0)=1$ & 
    \adjustbox{valign=m}{\resizebox{0.25\textwidth}{!}{
        \begin{tikzpicture}
            [declare function={
                func(\x)= sin(deg(\x)) / (\x); }]
            \begin{axis}[
                axis x line=center, axis y line=center,
                ymin=-1.25, ymax=1.5, ytick={-1,0,1}, ylabel={$\text{Sa}(t)$},
                xmin=-5, xmax=5, xtick={-5,...,5}, xlabel={$t$ [s]},
                domain=-5:5,samples=200,
                width=6cm, height=3.75cm]
            \addplot [blue,thick]{func(x)};
            \end{axis}
        \end{tikzpicture}}} \\[1cm]
    Sinc function & $\operatorname{sinc}(t)=\dfrac{\sin(\pi t)}{\pi t}, \text{ for } \operatorname{sinc}(0)=1$ & 
    \adjustbox{valign=m}{\resizebox{0.25\textwidth}{!}{
        \begin{tikzpicture}
            [declare function={
                func(\x)= sin(deg(pi*\x)) / (pi*\x); }]
            \begin{axis}[
                axis x line=center, axis y line=center,
                ymin=-1.25, ymax=1.5, ytick={-1,0,1}, ylabel={$\text{sinc}(t)$},
                xmin=-5, xmax=5, xtick={-5,...,5}, xlabel={$t$ [s]},
                domain=-5:5,samples=200,
                width=6cm, height=3.75cm]
            \addplot [blue,thick]{func(x)};
            \end{axis}
        \end{tikzpicture}}} \\
    \hline
    \end{tabular}
    }
    \end{table}
\end{center}

\section{Linear Combination of Signals}
Just as mathematical expressions can be summed, signals (which essentially are functions) can be summed as well. \\ \\
While any and all real signals are valid candidates for summation, this section will only focus on the unit impulse, 
unit step, unit ramp, rectangular pulse, and unit exponential decay as addends of interest. 
In particular, only signals defined for $t\geq 0$ will be synthesized.
\\ \\
Ignoring points of discontinuity, these signals graphically represent either slope changes or vertical offsets, which happen 
at time of \emph{excitation} -- for rectangular pulses, they happen strictly between time of excitation and time of \emph{termination}. 
These graphical properties can be seen in Table \ref{waveform_fun}. 
\begin{table}[hbt!]
    \small
    \centering
    \caption{Graphical Properties of Select Waveforms}
    \label{waveform_fun}
    \begin{tabular}{ |c|c|c| }
        \hline
        & Slope Change, $\Delta m(T) = m(T^+) - m(T^-)$ & Offset, $\Delta y(T) = y(T^+) - y(T^-)$ \\[0.1cm]
        \hline
        $k\cdot\delta(t-T)$ & Vertical arrow at $t=T$ of length $k$ & None \\[0.1cm]
        $k\cdot u(t-T)$ & None & $\Delta y(T) = k$ \\[0.1cm]
        $k\cdot\operatorname{rect}\left(\frac{t-T}{\tau}\right)$ & None & $\Delta y(T-\tau/2)=k$, $\Delta y(T+\tau/2) = -k$ \\[0.1cm]
        $k\cdot r(t-T)$ & $\Delta m(T)=k$ & None \\[0.1cm]
        $k\cdot\exp\left(-\frac{t-T}{\tau}\right)$ & Exponentially decay rate $1/\tau$ for $t>T$ & $\Delta y(T)=k$ \\
        \hline
    \end{tabular}
\end{table}

\begin{tcolorbox}[width=\textwidth,colback={white}, sharp corners]
    Waveform synthesis tips:
    \begin{itemize}
        \item Sort the addends by their respective excitation times
        \item Mark all excitation and termination times on the time axis
        \item Starting at zero slope and zero offset, make changes at each marked time, going from left to right along the time axis
    \end{itemize}
\end{tcolorbox}

\begin{example}
    Given $x(t)=2\operatorname{rect}\left(\frac{t-2}{2}\right)+r(t-3)u(4-t)+e^{-(t-4)}u(t-4)+\delta(t-5)$, plot $x(-2t+2)$.
\end{example}
\begin{solution}
    We need to plot $x(t)$ before its transformed version. Since the addends are already ordered, we determine the excitation and termination times, labeled $t_0$ and $t_f$.
    \begin{align*}
        2\operatorname{rect}\left(\frac{t-2}{2}\right) &\Longrightarrow t_0 = 1, t_f = 3 \\
        r(t-3)u(4-t) &\Longrightarrow t_0=3, t_f=4 \\
        e^{-(t-4)}u(t-4) &\Longrightarrow t_0 = 4 \\
        \delta(t-5) &\Longrightarrow t_0 = 5
    \end{align*}
    Next, we make a note of all slope changes $\Delta m$ and offsets $\Delta y$ due to each addend at each time.
    \begin{align*}
        2\operatorname{rect}\left(\frac{t-2}{2}\right) &\Longrightarrow \Delta y = 2 \text{ at } t_0 = 1, \Delta y = -2 \text{ at } t_f = 3 \\
        r(t-3)u(4-t) &\Longrightarrow \Delta m = 1 \text{ at } t_0=3, \Delta m = -1 \text{ and } \Delta y = -1 \text{ at } t_f = 4 \\
        e^{-(t-4)}u(t-4) &\Longrightarrow \text{exponential decay and } \Delta y = 1 \text{ at } t_0 = 4 \\
        \delta(t-5) &\Longrightarrow \text{arrow length 1 at } t_0 = 5
    \end{align*}
    Collecting the changes with respect to time:
    \begin{align*}
        t=1 &\Longrightarrow \Delta y = 2 \\
        t=3 &\Longrightarrow \Delta m = 1, \Delta y = -2 \\
        t=4 &\Longrightarrow \text{exponential decay and } \Delta y = -1+1=0 \\
        t=5 &\Longrightarrow \text{arrow length 1 at } t_0 = 5
    \end{align*}
    Using the collection of changes, plot $x(t)$ from left-to-right. \\ \\
    \begin{tikzpicture}
        [declare function={
            func(\x)= (\x < 1) * (0) + and(\x >= 1, \x < 3) * (2) + and(\x > 3, \x < 4) * (\x - 3) + (\x > 4) * (exp(-(\x - 4))); }]
        \begin{axis}[
            axis x line=center, axis y line=center,
            ymin=0, ymax=2.5, ytick={0,...,2}, ylabel={$ $},
            xmin=-1, xmax=6, xtick={-1,...,5}, xlabel={$ $},
            domain=-1:6,samples=200,
            width=7.5cm, height=4cm]
            \addplot [blue,thick]{func(x)};
            \draw [blue,thick] (5,0.3678) -- (5,1.3678);
            \addplot [blue,only marks,mark=triangle*,mark options={scale=1.5}] (5,1.3678);
            \draw [decorate, decoration={brace,amplitude=5pt,raise=1pt,mirror}] (5.1,0.3678) -- (5.1,1.3678) node [midway, xshift=-1mm, auto, swap, outer sep=10pt,font=\scriptsize]{1};
        \end{axis}
    \end{tikzpicture} \\ \\
    When applying signal transformations, note that for the transformed impulse function:
    \begin{align*}
        \delta((-2t+2)-5) = \delta(-2t-3) &= \delta(-2(t+1.5)) \\
        &= \frac{1}{|-2|} \cdot \delta(t+1.5) = \frac{1}{2} \cdot \delta(t+1.5)
    \end{align*}
    The new transformed signal is plotted below: \\ \\
    \begin{tikzpicture}
        [declare function={
            func(\x)= (\x < 1) * (0) + and(\x >= 1, \x < 3) * (2) + and(\x > 3, \x < 4) * (\x - 3) + (\x > 4) * (exp(-(\x - 4))); }]
        \begin{axis}[
            axis x line=center, axis y line=center,
            ymin=0, ymax=2.5, ytick={0,...,2}, ylabel={$ $},
            xmin=-3, xmax=2, xtick={-2,...,2}, xlabel={$ $},
            domain=-2:2,samples=200,
            width=7.5cm, height=4cm]
            \addplot [red,thick]{func(-2*x+2)};
            \draw [red,thick] (-1.5,0.3678) -- (-1.5,0.8678);
            \addplot [red,only marks,mark=triangle*,mark options={scale=1.5}] (-1.5,0.8678);
            \draw [decorate, decoration={brace,amplitude=2.5pt,raise=1pt}] (-1.6,0.3678) -- (-1.6,0.8678) node [midway, xshift=-1mm, auto, outer sep=5pt,font=\scriptsize]{0.5};
        \end{axis}
    \end{tikzpicture}
\end{solution}
\begin{tcolorbox}[width=\textwidth,colback={white}, sharp corners]
    Waveform deconstruction tips:
    \begin{itemize}
        \item Mark all times when slope changes or offsets happen, and also note when there are impulse or exponential behaviors
        \item If there is a slope change $\Delta m=m_2-m_1$ at $t=T$, then use addend $\Delta m \cdot r(t-T)$.
        \item If there is an instantaneous jump $\Delta y=y_2-y_1$ at $t=T$, then use addend $\Delta y\cdot u(t-T)$. If the jump is followed by  
            exponential decay, then use $\Delta y\cdot \exp(-(t-T)/\tau)$ instead.
        \item If there is a vertical arrowhead of length $k$ at $t=T$, then use $k\cdot\delta(t-T)$.
    \end{itemize}
\end{tcolorbox}

\pagebreak
\begin{example}
    Given the plot of $x(t)$ below, find an expression for $x(t)$. \\ \\
    \begin{tikzpicture}
        [declare function={
            func(\x)= (\x < -0) * (0) + and(\x >= 0, \x < 2) * (2*\x) + and(\x > 2, \x < 4) * (2*(\x-4)) + (\x > 4) * (0); }]
        \begin{axis}[
            axis x line=center, axis y line=center,
            ymin=-5, ymax=5, ytick={-4,4}, ylabel={$x(t)$},
            xmin=-1, xmax=5, xtick={0,...,4}, xlabel={$t$ [s]},
            domain=-1:5,samples=200,
            width=5cm, height=5cm]
        \addplot [blue,thick]{func(x)};
        \end{axis}
    \end{tikzpicture}
\end{example}
\begin{solution}
    First, make note of all times that introduce some behavioral change.
    \begin{align*}
        t=0 &\Longrightarrow \Delta m = 2-0 = 2 \\
        t=2 &\Longrightarrow \Delta y = -4-4 = -8 \\
        t=4 &\Longrightarrow \Delta m = 0-2 = -2
    \end{align*}
    From above, we are likely dealing with a ramp function at $t=0$, a step function at $t=2$, and another ramp function at $t=4$. That is,
    \begin{align*}
        x(t) = 2r(t)-8u(t-2)-2r(t-4).
    \end{align*}
\end{solution}

\section{Classification of Signals Based on Properties}
Signals can also be classified by the content of the signals themselves. 

\subsection{Causality}
In the real world, only causal signals  can ever occur, though theoretical types can be defined.
\begin{tcolorbox}[width=\textwidth,colback={white}, sharp corners]
    \begin{itemize}
        \item Causal signals: $x(t)=0$ for $t<0$
        \item Noncausal signals: $x(t)\neq 0$ for $t<0$
        \item Anticausal signals: $x(t)=0$ for $t>0$
    \end{itemize}
\end{tcolorbox}

\subsection{Symmetry}
A signal could have even symmetry, odd symmetry, or no symmetry.
\begin{tcolorbox}[width=\textwidth,colback={white}, sharp corners]
    \begin{itemize}
        \item Even signals: $x(t)=x(-t)$
        \item Odd signals: $x(t)=-x(-t)$, or alternatively $x(-t)=-x(t)$
    \end{itemize}
\end{tcolorbox}

One way to visualize symmetry is to imagine ``folding'' the graph of the signal. If the graph aligns with itself when 
folded along the vertical axis, then it has \emph{even symmetry} (sometimes called \emph{mirror symmetry}). If not, then 
if the graph aligns with itself when folded along the vertical and then the horizontal (time) axis, then it has 
\emph{odd symmetry} (sometimes called \emph{rotation symmetry}). If neither the 1-fold nor 2-fold tests pass, 
then it has no symmetry.

\begin{tcolorbox}[width=\textwidth,colback={white}, sharp corners]
    When multiplying symmetric functions, take note of the following properties:
    \begin{itemize}
        \item $(\text{even}) \times (\text{even}) = \text{even}$
        \item $(\text{even}) \times (\text{odd}) = \text{odd}$
        \item $(\text{odd}) \times (\text{odd}) = \text{even}$
    \end{itemize}
    This is analagous to using $+1$ for even and $-1$ for odd.
\end{tcolorbox}

Lastly, while a signal might not have any symmetry, any signal can be expressed as a sum of two component signals: one with even symmetry 
and the other with odd symmetry. Sometimes it is easier to analyze each individual component rather than analyze the whole signal on its own. For 
\begin{align}
    x(t) = x_e(t) + x_o(t),
\end{align}
the even and odd components are given by
\begin{align}
    x_e(t) &= \frac{1}{2}\left[x(t)+x(-t)\right] \\
    x_o(t) &= \frac{1}{2}\left[x(t)-x(-t)\right]
\end{align}

\begin{example}
    Given $x(t)=e^{2t}$, find its even and odd components.
\end{example}
\begin{solution}
    The even component is given by 
    \begin{align*}
        x_e(t) &= \frac{1}{2}\left[x(t)+x(-t)\right] \\
        &= \frac{1}{2}\left[e^{2t}+e^{-2t}\right] = \cosh(2t),
    \end{align*}
    and the odd component is given by 
    \begin{align*}
        x_e(t) &= \frac{1}{2}\left[x(t)-x(-t)\right] \\
        &= \frac{1}{2}\left[e^{2t}-e^{-2t}\right] = \sinh(2t).
    \end{align*}
\end{solution}
\subsection{Periodicity}
A signal $x(t)$ is said to be $periodic$ if $x(t)=x(t+nT_0)$ for all integers $n$ and time $t$ with fundamental period $T_0$. 
Note that $\omega_0=2\pi/T_0$ is the fundamental angular frequency, and $f_0=1/T_0=\omega_0/2\pi$ is the fundamental (linear) frequency.

\begin{tcolorbox}[width=\textwidth,colback={white}, sharp corners]
    The sum of $N$ periodic signals are itself periodic if:    \begin{itemize}
        \item $\omega_0=\mathbf{GCD}(\omega_1,\omega_2,...,\omega_N)$ exists, where $\omega_k$ is the fundamental angular frequency of the $k^{th}$ signal
        \item $\forall n\in\left\{\frac{\omega_1}{\omega_0}, \frac{\omega_2}{\omega_0}, ..., \frac{\omega_N}{\omega_0}\right\}$ are integers, 
        with each $n$ called the $n^{th}$ harmonic (or mode) of fundamental angular frequency $\omega_0$ (or of fundamental frequency $f_0$)
    \end{itemize}
\end{tcolorbox}

\begin{example}
    Determine if $x(t)=\sin(\frac{5\pi}{6}t)+\cos(\frac{3\pi}{4}t)-\exp\left(j\frac{\pi}{3}t\right)$ is periodic or not.
\end{example}
\begin{solution}
Notice that each addend in $x(t)$ is periodic itself. Let $\omega_1=5\pi/6, \omega_2=3\pi/4, \omega_3=\pi/3$. Then
\begin{align*}
    \omega_0 &= \mathbf{GCD}(\omega_1,\omega_2,\omega_3) \\
    &= \mathbf{GCD}\left(\frac{5\pi}{6}, \frac{3\pi}{4}, \frac{\pi}{3}\right) \\
    &= \mathbf{GCD}\left(\frac{10\pi}{12}, \frac{9\pi}{12}, \frac{4\pi}{12}\right) = \frac{\pi}{12}.
\end{align*}
The corresponding harmonics can be calculated
\begin{align*}
    \omega_1/\omega_0 &= 10 \\
    \omega_2/\omega_0 &= 9 \\
    \omega_3/\omega_0 &= 4
\end{align*}
and are shown to be all integers. \\ \\
Therefore, $x(t)=\sin(10\omega_0 t)+\cos(9\omega_0 t)+e^{j4\omega_0 t}$ is periodic.
\end{solution}

\subsection{Signal Power and Energy}
The total \emph{energy} of a signal $x(t)$ is given by 
\begin{align}
    E = \int_{-\infty}^{+\infty} |x(t)|^2 \,dt,
\end{align}
whereas the \emph{time-average power} of a signal is given by
\begin{align}
    P_{av} = \lim_{T\rightarrow \infty} \frac{1}{T}\int_{-T/2}^{+T/2} |x(t)|^2 \,dt.
\end{align}
Furthermore, if a signal $x(t)$ is periodic with fundamental period $T_0$, then 
\begin{align}
    P_{av} = \frac{1}{T_0}\int_{-T_0/2}^{+T_0/2} |x(t)|^2 \,dt.
\end{align}
In fact, if $x(t)=A\sin(\omega_0 t+\theta)$ for $A$ is real, then 
\begin{align}
    P_{av} = \frac{A^2}{2}.
\end{align}
It can then be said that all periodic signals are power signals.
\begin{tcolorbox}[width=\textwidth,colback={white}, sharp corners]
    A signal can be classified based on the values of $P_{av}$ and $E$.
    \begin{itemize}
        \item Power signals: $P_{av}$ is finite and $E\rightarrow \infty$ 
        \item Energy signals: $P_{av}=0$ and $E$ is finite
        \item Non-physical signals: $P_{av}\rightarrow \infty$ and $E\rightarrow \infty$ 
    \end{itemize}
\end{tcolorbox}

\begin{example}
    Find the total energy of the signal 
    \begin{align*}
        x(t) = \begin{cases}
            0, & t\geq 0 \\
            3t, & 0\leq t\leq 2 \\
            6e^{-(t-2)}, & t\geq 2
        \end{cases}
    \end{align*}
\end{example}
\begin{solution}
    \begin{align*}
        E = \int_{-\infty}^{+\infty} |x(t)|^2 \,dt &= \int_{0}^{2} |3t|^2 \,dt + \int_{2}^{+\infty} |6e^{-(t-2)}|^2 \,dt \\
        &= \left[\frac{9t^3}{3}\right|_0^2 + 36e^4 \left[\frac{-e^{-2t}}{2}\right|_2^{+\infty} \\
        &= (24-0) + 36e^4\left(0+\frac{e^{-4}}{2}\right) \\
        &= 24 + 18 =  42.
    \end{align*}
\end{solution}

\begin{example}
    Fully describe what type of signal $x(t)=(3-j4)e^{j\pi t/3}$ is.
\end{example}
\begin{solution}
    Since $x(t)$ is a complex exponential function, it is periodic with 
    \begin{align*}
        \omega_0 = \frac{\pi}{3} = \frac{2\pi}{T_0} \Longrightarrow T_0 = 6.
    \end{align*}
    Therefore, $x(t)$ is also a power signal with average power 
    \begin{align*}
        P_{av} = \frac{1}{T_0}\int_{-T_0/2}^{+T_0/2} |x(t)|^2 \,dt &= \frac{1}{6}\int_{-3}^{3} x(t)x^*(t) \,dt \\
        &= \frac{1}{6}\int_{-3}^{3} [(3-j4)e^{j\pi t/3}][(3+j4)e^{-j\pi t/3}] \,dt \\
        &= \frac{1}{6}\int_{-3}^{3} 25 \,dt = 25.
    \end{align*}
    Checking for symmetry, we find there is none since 
    \begin{align*}
        x(-t) &= (3-j4)e^{-j\pi t/3} \neq x(t), \\
        -x(-t) &= -(3-j4)e^{-j\pi t/3} \neq x(t).
    \end{align*}
    Therefore, $x(t)$ is:
    \begin{itemize}
        \item an analog signal,
        \item deterministic, 
        \item noncausal, 
        \item asymmetric,
        \item periodic,
        \item and a power signal.
    \end{itemize}
\end{solution}
The signal processing equations for signal average power and energy are borrowed from physics itself. The physics equation for instantaneous power is given by 
\begin{align}
    p(t) = i^2(t)R = \frac{v^2(t)}{R}
\end{align}
It then follows that the physics equation for energy is 
\begin{align}
    E = \int_{-\infty}^{+\infty} p(t) \,dt
\end{align}
and the physics equation for average power for periodic signals is 
\begin{align}
    P_{av} = \frac{1}{T}\int_{-T/2}^{+T/2} p(t) \,dt
\end{align}
In this case, we take $|x(t)|^2$ to be analogous to $p(t)$. With signal processing, 
the equations for signal average power and energy are actually normalized. If the signal of interest $x(t)$ 
ends up being a voltage or current signal, the units must be accounted for when unnormalizing the values.
\\ \\
For cases where normalized signal power and energy need to be converted to unnormalized power and energy describing physical systems such as circuits, we redefine 
\begin{align}
    E_{norm} &= \int_{-\infty}^{+\infty} |x(t)|^2 \,dt \\
    P_{av,norm} &= \lim_{T\rightarrow \infty} \frac{1}{T}\int_{-T/2}^{+T/2} |x(t)|^2 \,dt
\end{align}
If $x(t)=v(t)$ is the voltage across a resistor, then it follows that 
\begin{align}
    E &= \frac{E_{norm}}{R} \\ 
    P_{av} &= \frac{P_{av,norm}}{R}
\end{align}
Otherwise if $x(t)=i(t)$ is the current through a resistor, then it follows that 
\begin{align}
    E &= E_{norm}R \\
    P_{av} &= P_{av,norm}R
\end{align}

\end{document}
