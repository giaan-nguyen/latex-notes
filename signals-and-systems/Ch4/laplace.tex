\documentclass{report}
% PACKAGES
\usepackage{adjustbox}
\usepackage{amsmath}
\usepackage{amssymb}
\usepackage{bodegraph}
\usepackage{bbm}
\usepackage{circledsteps}
\usepackage{circuitikz}
\usepackage{enumerate}
\usepackage{mathtools}
\usepackage{nicematrix}
\usepackage{pgfplots}
\usepackage{polynom}
\usepackage{qtree}
\usepackage{rotating}
\usepackage[usestackEOL]{stackengine}
\usepackage[subpreambles=true]{standalone}
\usepackage{steinmetz}
\usepackage{subcaption}
\usepackage{tabularray}
\usepackage{tcolorbox}
\usepackage{tikz}
\usepackage{xcolor}

\usepackage[colorlinks=true,linkcolor=blue,urlcolor=black,bookmarksopen=true]{hyperref}
\usepackage{bookmark}

\usetikzlibrary{shapes.arrows}
\usetikzlibrary{shapes.misc}
\usetikzlibrary{backgrounds}
\tikzset{cross/.style={cross out, draw=black, minimum size=2*(#1-\pgflinewidth), inner sep=0pt, outer sep=0pt},
%default radius will be 1pt. 
cross/.default={1pt}}

\renewcommand{\Re}{\operatorname{Re}}
\renewcommand{\Im}{\operatorname{Im}}

\usepackage{pifont}
\newcommand{\cmark}{\text{\ding{51}}}
\newcommand{\xmark}{\text{\ding{55}}}

\newcommand{\circconv}[1]{\text{ \small\Circled{#1} }}

\newcommand{\tikzmark}[3][]{\tikz[remember picture,baseline] \node [anchor=base,#1](#2) {$#3$};}

\pgfplotsset{compat=1.18}
\pgfplotsset{
    dirac/.style={
        mark=triangle*,
        mark options={scale=1.5},
        ycomb,
        scatter,
        visualization depends on={y/abs(y)-1 \as \sign},
        scatter/@pre marker code/.code={\scope[rotate=90*\sign,yshift=-2pt]}
    }
}

\usepackage[letterpaper, portrait, margin=1.25in]{geometry}
\usepackage[font=bf]{caption}
\hbadness = 10000
\hfuzz=2pt

\newtheorem{theorem}{Theorem}[chapter]

\tikzset{every tree node/.style={anchor=north,align=center}}
\usetikzlibrary{decorations.markings}
\usetikzlibrary{arrows}
\definecolor{darkgreen}{rgb}{0.133,0.545,0.133}
\tikzstyle{n}= [circle, fill=blue, minimum size=4pt,inner sep=0pt, outer sep=0pt]

\patchcmd{\thebibliography}{\chapter*}{\section*}{}{}
\setcounter{secnumdepth}{5}

%%%%%%%%%%% EXAMPLE ENVIRONMENT %%%%%%%%%%
\usepackage{calc}
\usepackage{tabto}
\usepackage[framemethod=tikz]{mdframed} % for colored backgrounds
\newcommand{\halmos}{} % makes a box at the end

\newlength{\framedinnerleftmargin}
\newlength{\framedinnertopmargin}
\newlength{\framedreversedinnerleftmargin}
\setlength{\framedinnerleftmargin}{\widthof{Theoreme 10.10.10}+2em}
\setlength{\framedreversedinnerleftmargin}{\widthof{Theoreme 10.10.10}+1em}
\setlength{\framedinnertopmargin}{1em}
 
% first argument: label in upper left corner,
% second argument: background color
\newenvironment{boxedtext}[2]{\begin{mdframed}[%
hidealllines=true,%
backgroundcolor=#2,%
innertopmargin=\framedinnertopmargin,%
innerleftmargin=\framedinnerleftmargin,%
innerrightmargin=1em%
]%
\tabto{-\framedreversedinnerleftmargin}\textbf{#1}\tabto*{0em}%
}% begin code
{\hskip 0pt\\\hspace*{\fill}\halmos{}\end{mdframed}\vspace{1em}} % end code
 
\newenvironment{summary}[0]{\begin{center}\begin{minipage}[c]{\summarywidth}\begin{spacing}{0.9}\footnotesize} % begin code
{\end{spacing}\end{minipage}\end{center}} % end code

\newcounter{example}
 
% optional! if you want it to start at zero
% with every new chapter/section/etc.
\numberwithin{example}{section}
 
\newenvironment{example}[0]
{\refstepcounter{example}\vspace{1em plus 1em}\begin{boxedtext}{Example \theexample.}{blue!7}}%\setlength{\parskip}{0em}}
{\end{boxedtext}\vspace{-1em plus 1em}}
 
\newenvironment{example*}[0]
{\vspace{1em plus 1em}\begin{boxedtext}{Example.}{blue!7}}
{\end{boxedtext}\vspace{-1em plus 1em}}

\newmdenv[
  topline=false,
  bottomline=false,
  rightline=false,
  skipabove=\topsep,
  skipbelow=\topsep,
  linecolor=purple,
  frametitle={\noindent\textcolor{purple}{\textbf{SOLUTION }}},
  endinnercode={$\hfill\textcolor{purple}{\blacksquare}$}
]{solution}

%%%%%%%%%%%%%% COLOR BOXED %%%%%%%%%%%%%%%
% Syntax: \colorboxed[<color model>]{<color specification>}{<math formula>}
\newcommand*{\colorboxed}{}
\def\colorboxed#1#{%
  \colorboxedAux{#1}%
}
\newcommand*{\colorboxedAux}[3]{%
  % #1: optional argument for color model
  % #2: color specification
  % #3: formula
  \begingroup
    \colorlet{cb@saved}{.}%
    \color#1{#2}%
    \boxed{%
      \color{cb@saved}%
      #3%
    }%
  \endgroup
}
%%%%%%%%%%%%%%%%%%%%%%%%%%%%%%%%%%%%%%%%%

\begin{document}
\setcounter{chapter}{3}
\setcounter{page}{44}
\chapter{Laplace Transform}
Previously, we have seen that LTI systems can be characterized by their impulse responses $h(t)$, and the outputs of those systems can be computed 
by taking the convolution of the input signal and the impulse response. 
\\ \\ 
In fact, all LTI systems can be characterized by some impulse response. Going back 
to the affine transformation on the independent variable, its impulse response is given by 
\begin{align}
    h(t)=\delta(at-b)=\frac{1}{|a|}\delta\left(t-\frac{b}{a}\right).
\end{align}
As seen in the first two chapters, some LTI systems can be represented by LCCDEs. While one could find the system response by solving the differential equation, 
it is more difficult to solve it that way. Here, we introduce a concept called the Laplace transform which offers a simpler way to solve differential equations 
by mapping them to a domain that uses algebraic equations.
\\ \\
As signals and systems in the physical world are causal, we are particularly interested in the unilateral (one-sided) Laplace transform. 
\begin{center}
    \begin{tikzpicture}
        \node [] (in) at (-3,+4){$x(t)$};
        \node [] (out) at (+3,+4){$y(t)$};
        \node [draw,
                fill=green!30, 
                minimum width=2.8cm, 
                minimum height=0.8cm
            ] (sys) at (-3,+2.25){System};
        \node [draw,
                fill=blue!30, 
                minimum width=2.8cm, 
                minimum height=1.2cm
            ] (diff) at (-3,0){\shortstack{Differential Eq. \\ (Time Domain)}};
        \node [draw,
                fill=blue!30, 
                minimum width=2.8cm, 
                minimum height=1.2cm
            ] (sol_t) at (+3,0){\shortstack{Solution \\ (Time Domain)}};
        \node [draw,
                fill=red!30, 
                minimum width=2.8cm, 
                minimum height=1.2cm
            ] (alg) at (-3,-3){\shortstack{Algebraic Eq. \\ ($s$-Domain)}};
        \node [draw,
                fill=red!30, 
                minimum width=2.8cm, 
                minimum height=1.2cm
            ] (sol_s) at (+3,-3){\shortstack{Solution \\ ($s$-Domain)}};
        \draw [-latex, line width=1.5pt] (in) -- (sys) node[midway,left]{Input};
        \draw [-latex, line width=1.5pt] (sys) -- (diff);
        \draw [-latex, line width=1.5pt] (diff) -- (alg) node[midway,right]{\shortstack{Laplace \\ Transform}};
        \draw [-latex, line width=1.5pt] (alg) -- (sol_s);
        \draw [-latex, line width=1.5pt] (sol_s) -- (sol_t) node[midway,right]{\shortstack{Inverse Laplace \\ Transform}};
        \draw [-latex, line width=1.5pt] (sol_t) -- (out) node[midway,right]{Output};
    \end{tikzpicture}
\end{center}

\section{Unilateral Laplace Transform}
\begin{tcolorbox}[width=\textwidth,colback={white}, sharp corners]
    The \emph{unilateral Laplace transform} of a signal $x(t)$ is defined as  
    \begin{align}
        X(s) = \mathcal{L}[x(t)] = \int_{0^-}^{\infty} x(t) e^{-st} \,dt,
    \end{align}
    where $s=\sigma + j\omega$ is the \emph{complex frequency}. 
\end{tcolorbox}
\noindent The Laplace transform essentially is an operator that maps a signal defined in the \emph{time domain} to another signal 
defined in the \emph{$s$-domain}. $x(t)$ and $X(s)$ constitute a \emph{Laplace transform pair}. This relationship can be written as 
\begin{align}
    x(t) \iff X(s).
\end{align}
Because of this relationship, there exists an \emph{inverse Laplace transform} defined by 
\begin{align}
    x(t) = \mathcal{L}^{-1}[X(s)] = \frac{1}{j2\pi}\int_{\sigma-j\infty}^{\sigma+j\infty} X(s) e^{+st} \,ds,
\end{align}
though direct computation of the inverse integral is often more cumbersome than referencing a table of Laplace transform pairs.
\\ \\
Additionally, for signal $x(t)$ to be transformable, the unilateral Laplace integral must absolutely converge such that 
\begin{align}
    \int_{0^-}^{\infty} |x(t) e^{-st}| \,dt < \infty. 
\end{align}
The \emph{region of convergence} (ROC) is then defined as the values for $s$ for which the unilateral Laplace integral converges to some finite value  
for a given signal $x(t)$ for all $t$. Here, the ROC is defined by a set of values for $\sigma = \Re(s)$.
\\ \\
For the unilateral Laplace transform, the ROC informs if there exists a unilateral Laplace transform for some signal $x(t)$ that can be mapped back to the time domain. 
Since we are more interested in causal signals and systems, signals that are physically realizable always have a unilateral Laplace transform. 
\\ \\
Since all physically realizable signals and systems have a Laplace transform, the ROC is not necessary when modeling those entities. In fact, all causal signals have a unique 
unilateral Laplace transform.
\\ \\
Table \ref{laplace_prop} lists the properties of the unilateral Laplace transform.
\newpage
\begin{table}[hbt!]
    \centering
    \caption{Properties of the unilateral Laplace transform.}
    \label{laplace_prop}
    \begin{tabular}{|c|c|c|}
        \hline
        Property & $x(t)$ & $X(s)=\mathcal{L}[x(t)]$ \\[0.15cm]
        \hline
        & & \\
        Superposition & $K_1x_1(t)+K_2x_2(t)$ & $K_1X_1(s)+K_2X_2(s)$ \\[0.5cm]
        Time scaling & $x(at),\text{ } a>0$ & $\dfrac{1}{a}X\left(\dfrac{s}{a}\right)$ \\[0.5cm]
        Time shift & $x(t-T)u(t-T),\text{ } T>0$ & $e^{-Ts}X(s)$ \\[0.5cm]
        Frequency shift & $e^{-at}x(t)$ & $X(s+a)$ \\[0.5cm]
        Time 1st derivative & $x'(t)=\dfrac{dx(t)}{dt}$ & $sX(s) - x(0^-)$ \\[0.5cm]
        Time 2nd derivative & $x''(t)=\dfrac{d^2x(t)}{dt^2}$ & $s^2X(s) - sx(0^-) - x'(0^-)$ \\[0.5cm]
        Time $n$th derivative & $x^{(n)}(t)=\dfrac{d^nx(t)}{dt^n}$ & $s^nX(s) - \displaystyle\sum_{k=1}^{n}s^{n-k}x^{(k-1)}(0^-)$ \\[0.5cm]
        Time integral & $\displaystyle\int_{0^-}^{t} x(\tau) \,d\tau$ & $\dfrac{1}{s} X(s)$ \\[0.5cm]
        Frequency derivative & $t\cdot x(t)$ & $-\dfrac{dX(s)}{ds} = -X'(s)$\\[0.5cm]
        Frequency integral & $\dfrac{x(t)}{t}$ & $\displaystyle\int_{s}^{\infty} X(\gamma) \,d\gamma$ \\[0.5cm]
        Convolution & $x_1(t)*x_2(t)$ & $X_1(s)X_2(s)$ \\[0.5cm]
        Multiplication & $x_1(t)x_2(t)$ & $X_1(s)*X_2(s)$ \\[0.5cm]
        Initial value theorem & $x(0^+)$ & $\displaystyle\lim_{s\to\infty} X(s)$ \\[0.5cm]
        Final value theorem & $x(\infty)$ & $\displaystyle\lim_{s\to 0} X(s)$ \\[0.5cm]
        \hline
    \end{tabular}
\end{table}

\begin{example}
    Find the unilateral Laplace transform of $x(t)=e^{-at}u(t)$, for $a>0$. 
\end{example}
\begin{solution}
    Using the unilateral Laplace integral, 
    \begin{align*}
        X(s) = \mathcal{L}[x(t)] = \int_{0^-}^{\infty} x(t) e^{-st} \,dt, 
    \end{align*}
    we get
    \begin{align*}
        X(s) = \int_{0^-}^{\infty} [e^{-at}u(t)]e^{-st} \,dt = \int_{0^-}^{\infty} e^{-(s+a)t} \,dt = \left[\frac{e^{-(s+a)t}}{-(s+a)}\right|_{0^-}^{\infty} = \frac{1}{s+a}.
    \end{align*}
\end{solution}

Table \ref{laplace_pairs} lists some common unilateral Laplace transform pairs. While one could find the inverse Laplace transform using the inverse Laplace integral, 
it is much simpler to reference a table of transform pairs. 

\begin{example}
    Suppose a signal $x(t)= u(t-2)-u(t-4)$ is inputted into an LTI system characterized by $h(t)=t\cdot u(t)$. Find the unilateral Laplace transform of 
    the output signal $y(t)$.
\end{example}
\begin{solution}
    First, let $x_0(t) = u(t)$. From the convolution property, we can find the unilateral Laplace transforms of both $x_0(t)$ and $h(t)$, then take the product $X_0(s)H(s)$. 
    \begin{align*}
        X_0(s) &= \int_{0^-}^{\infty} u(t)e^{-st} \,dt = \left[\frac{e^{-st}}{-s}\right|_{0^-}^{\infty} = \frac{1}{s} \\
        H(s) &=\int_{0^-}^{\infty} t\cdot u(t)e^{-st} \,dt = \left[\frac{e^{-st}}{s^2}(-st-1)\right|_{0^-}^{\infty} = \frac{1}{s^2} \\
        \Longrightarrow Y_0(s) &= X_0(s)H(s) = \frac{1}{s^3}
    \end{align*}
    Using the table of unilateral transform pairs, we inverse-transform the product back to the time domain to get 
    \begin{align*}
        y_0(t) = \mathcal{L}^{-1}[Y_0(s)] &= \mathcal{L}^{-1}\left[\frac{1}{s^3}\right] \\
        &= \mathcal{L}^{-1}\left[\frac{1}{2}\cdot\frac{2}{s^3}\right] \\
        &= \frac{1}{2}\cdot\mathcal{L}^{-1}\left[\frac{2}{s^3}\right] = \frac{1}{2}\cdot t^2u(t).
    \end{align*}
    By the superposition principle for LTI systems, it follows that 
    \begin{center}
        \begin{tikzpicture}
            \node [] (input) at (-4,0){$x(t)=x_0(t-2)-x_0(t-4)$};
            \node [] (output) at (+4,0){$y(t)=y_0(t-2)-y_0(t-4)$};
            \node [draw,
                fill=yellow!50, 
                minimum width=2cm, 
                minimum height=1.2cm
            ] (system) at (0,0){$h(t)$};
            \draw [-latex, line width=1.5pt] (input) -- (system);
            \draw [-latex, line width=1.5pt] (system) -- (output);
        \end{tikzpicture}
    \end{center}
    Therefore, 
    \begin{align*}
        y(t) &= y_0(t-2)-y_0(t-4) \\
        &= \frac{1}{2}\cdot (t-2)^2u(t-2) - \frac{1}{2}\cdot (t-4)^2u(t-4).
    \end{align*}
\end{solution}

\begin{table}[hbt!]
    \centering
    \caption{Unilateral Laplace transform pairs.}
    \label{laplace_pairs}
    \begin{tabular}{|c|c|}
        \hline
        $x(t)$ & $X(s)=\mathcal{L}[x(t)]$ \\[0.15cm]
        \hline
        & \\[0.1cm]
        $\delta(t)$ & $1$ \\[0.5cm]
        $u(t)$ & $\dfrac{1}{s}$ \\[0.5cm]
        $e^{-at}u(t)$ & $\dfrac{1}{s+a}$ \\[0.5cm]
        $t\cdot u(t)$ & $\dfrac{1}{s^2}$ \\[0.5cm]
        $t^2\cdot u(t)$ & $\dfrac{2}{s^3}$ \\[0.5cm]
        $te^{-at}u(t)$ & $\dfrac{1}{(s+a)^2}$ \\[0.5cm]
        $t^2e^{-at}u(t)$ & $\dfrac{2}{(s+a)^3}$ \\[0.5cm]
        $t^{n-1}e^{-at}u(t)$ & $\dfrac{(n-1)!}{(s+a)^n}$ \\[0.5cm]
        $\sin(\omega_0 t)u(t)$ & $\dfrac{\omega_0}{s^2+\omega_0^2}$ \\[0.5cm]
        $\sin(\omega_0 t+\theta)u(t)$ & $\dfrac{s\sin\theta + \omega_0\cos\theta}{s^2+\omega_0^2}$ \\[0.5cm]
        $\cos(\omega_0 t)u(t)$ & $\dfrac{s}{s^2+\omega_0^2}$ \\[0.5cm]
        $\cos(\omega_0 t+\theta)u(t)$ & $\dfrac{s\cos\theta - \omega_0\sin\theta}{s^2+\omega_0^2}$ \\[0.5cm]
        $e^{-at}\sin(\omega_0 t)u(t)$ & $\dfrac{\omega_0}{(s+a)^2+\omega_0^2}$ \\[0.5cm]
        $e^{-at}\cos(\omega_0 t)u(t)$ & $\dfrac{s+a}{(s+a)^2+\omega_0^2}$ \\[0.5cm]
        $2e^{-at}\cos(bt-\theta)u(t)$ & $\dfrac{e^{+j\theta}}{s+a+jb}+\dfrac{e^{-j\theta}}{s+a-jb}$ \\[0.5cm]
        $e^{-at}\cos(bt-\theta)u(t)$ & $\dfrac{(s+a)\cos\theta+b\sin\theta}{(s+a)^2+b^2}$ \\[0.5cm]
        $\dfrac{2t^{n-1}}{(n-1)!}e^{-at}\cos(bt-\theta)u(t)$ & $\dfrac{e^{+j\theta}}{(s+a+jb)^n}+\dfrac{e^{-j\theta}}{(s+a-jb)^n}$ \\[0.5cm]
        \hline
    \end{tabular}
\end{table}

\pagebreak [4]
\section{Bilateral Laplace Transform}
\begin{tcolorbox}[width=\textwidth,colback={white}, sharp corners]
    The \emph{bilateral Laplace transform} of a signal $x(t)$ is defined as  
    \begin{align}
        X(s) = \mathcal{B}[x(t)] = \int_{-\infty}^{+\infty} x(t) e^{-st} \,dt.
    \end{align}
\end{tcolorbox}
\noindent Similar to the unilateral Laplace transform, for a signal $x(t)$ to be transformable, the bilateral Laplace integral must absolutely converge such that  
\begin{align}
    \int_{-\infty}^{+\infty} |x(t) e^{-st}| \,dt < \infty
\end{align}
holds true. However, the region of convergence (ROC) plays a bigger role for the bilateral Laplace transform. 
\\ \\ 
While the unilateral Laplace transform refers to the ROC 
to validate the existence of a unilateral transform pair, the bilateral Laplace transform needs the ROC to correctly map the $s$-domain solution to the time domain. 
That is, the ROC informs the mapping for the inverse bilateral transform. Two different signals may have the same Laplace transform and only differ by their ROCs. 
\\ \\
For instance, consider the causal signal $x(t)=e^{-at}u(t)$ with the integrable condition
\begin{align*}
    \int_{-\infty}^{+\infty} |e^{-at}u(t)| e^{-\sigma t} \,dt = \int_{0^-}^{\infty} e^{-(a+\sigma) t} \,dt < \infty.
\end{align*}
If $\Re(s) > -a$, then it follows that 
\begin{align*}
    \int_{0^-}^{\infty} e^{-(a+\sigma) t} \,dt = \left[\frac{e^{-(a+\sigma) t}}{-(a+\sigma)}\right|_{0^-}^{\infty} < \infty.
\end{align*}
If $\Re(s) \leq -a$, then for $\epsilon=-(a+\sigma)\geq 0$, it follows that
\begin{align*}
    \int_{0^-}^{\infty} e^{-(a+\sigma) t} \,dt = \int_{0^-}^{\infty} e^{+\epsilon t} \,dt = \left[\frac{e^{\epsilon t}}{\epsilon}\right|_{0^-}^{\infty} = \infty.
\end{align*}
However, the anticausal signal $x_{A}(t)= -e^{-at}u(-t)$ has an ROC of $\Re(s)\leq -a$ such that 
\begin{align*}
    \int_{-\infty}^{+\infty} |-e^{-at}u(-t)|e^{-\sigma t} \,dt = \int_{-\infty}^{0^+} e^{-(a+\sigma)t} \,dt = \left[\frac{e^{-(a+\sigma) t}}{-(a+\sigma)}\right|_{-\infty}^{0^+} < \infty.
\end{align*}
Here, we see two signals -- one of which is causal and the other anticausal -- that would have the same bilateral Laplace transform $X(s)=\frac{1}{s+a}$ but with 
different ROCs.
\\ \\
Generally, while the unilateral Laplace transform is convenient for solving differential equations with initial conditions, the bilateral Laplace transform provides insight into system characteristics 
such as stability, causality, and frequency response. 
\\ \\ 
However, we can avoid using the bilateral Laplace transform (and avoid ROC calculations) when analyzing system characteristics by setting zero initial conditions and/or using other 
tools such as the Fourier transform, which will be covered in a later chapter. 
\\ \\
For the rest of the text, when referring to the Laplace transform, we will be using the unilateral form.

\section{Partial Fraction Expansion}
When converting a solution from $s$-domain back to time domain, we will often have a $s$-domain solution in the form of a rational function that may need to be rewritten as a sum of parts, with 
each part having a Laplace transform pair.
\\ \\
Consider an $s$-domain function with the form
\begin{align}
    X(s) = \frac{N(s)}{D(s)},
\end{align}
written as a ratio of a polynomial numerator $N(s)$ to a polynomial denominator $D(s)$. 
The roots of the numerator are called the \emph{zeros} of $X(s)$ and are defined by 
\begin{align}
    N(s) = 0,
\end{align}
whereas the roots of the denominator are called the \emph{poles} of $X(s)$ and are defined by 
\begin{align}
    D(s) = 0.
\end{align}
Before even performing partial fraction expansion, first we examine the orders of the numerator and denominator. Let 
\begin{align}
    m &= \deg[N(s)], \\ 
    n &= \deg[D(s)].
\end{align}
\begin{tcolorbox}[width=\textwidth,colback={white}, sharp corners]
    There are three types of rational functions, each with a different task at hand.
    \begin{itemize}
        \item $m < n$: $X(s)$ is \emph{strictly proper}. It is ready for partial fraction expansion, if needed.
        \item $m = n$: $X(s)$ is \emph{proper}. Perform long division and express $X(s)$ in terms of the quotient and remainder, then determine if the remainder term needs partial fraction expansion.
        \item $m > n$: $X(s)$ is \emph{improper}. Perform long division and express $X(s)$ in terms of the quotient and remainder, then determine if the remainder term needs partial fraction expansion.
    \end{itemize}
\end{tcolorbox}
\noindent A term would need partial fraction expansion if there is not an easily matching Laplace transform pair provided by the table. Usually this occurs when $\deg[D(s)]>2$.

\begin{example}
    Determine if 
    \begin{align*}
        X(s) = \frac{2s^2+8s+6}{s^2+2s+1}
    \end{align*}
    needs to undergo partial fraction expansion.
\end{example}
\begin{solution}
    Since both the numerator and denominator are of the same polynomic order, first we need to long divide. \\
    \[ \polylongdiv[vars=s]{2s^2+8s+6}{s^2+2s+1} \]
    From here, we can rewrite $X(s)$ as 
    \begin{align*}
        X(s) = 2 + \frac{4s+4}{s^2+2s+1} = 2 + \frac{4(s+1)}{(s+1)^2} = 2 + \frac{4}{s+1}.
    \end{align*}
    Since both addends are similar to the ones in the Laplace transform pairs table, partial fraction expansion is not needed, and 
    \begin{align*}
        x(t) = \mathcal{L}^{-1}\left[2 + \frac{4}{s+1}\right] = 2\delta(t) + 4e^{-t}u(t).
    \end{align*}
\end{solution}

\begin{example}
    Determine if 
    \begin{align*}
        X(s) = \frac{s^4+4}{s^3-3s-1}
    \end{align*}
    needs to undergo partial fraction expansion.
\end{example}
\begin{solution}
    Since both the numerator and denominator are of the same polynomic order, first we need to long divide. \\
    \[ \polylongdiv[vars=s]{s^4+4}{s^3-3s-1} \]
    From here, we can rewrite $X(s)$ as 
    \begin{align*}
        X(s) = s + \frac{3s^2+s+4}{s^3-3s-1}.
    \end{align*}
    Since the remainder term does not have a matching Laplace transform pair, partial fraction expansion is needed.
\end{solution}

Given that a remainder term requires partial fraction expansion, all poles $p_n$ of $X(s)$ must be solved first so that $D(s)$ is expressed in factored form. That is, 
\begin{align}
    X(s) = \frac{N(s)}{D(s)} = \frac{N(s)}{\prod_{\ell=1}^{n}(s-p_{\ell})}.
\end{align}
\newpage
\noindent There are four different cases for partial fraction expansion. Depending on the poles, one or more cases may apply. Each case requires using a technique called the \emph{residue method}, which evaluates a modified expression 
at a pole $s=p$ so that factors cancel and expansion coefficients can be calculated. 
\begin{tcolorbox}[width=\textwidth,colback={white}, sharp corners]
    \textbf{\underline{Case 1: Distinct real poles.}} \\[0.25cm]
    Suppose there are $n$ distinct real poles in $X(s)$. Then $X(s)$ can be expanded as 
    \begin{align}
        X(s) &= \frac{N(s)}{(s-p_1)(s-p_2)\cdots(s-p_n)}=\frac{A_1}{s-p_1}+\frac{A_2}{s-p_2}+\cdots+\frac{A_n}{s-p_n}.
    \end{align}
    The expansion coefficients for Case 1 is then given by
    \begin{align}
        A_n = (s-p_n)X(s)\big|_{s=p_n}.
    \end{align}
\end{tcolorbox}
\begin{example}
    Use the residue method to expand $X(s) = \dfrac{s^2-4s+3}{s(s+1)(s+3)}$. \\
\end{example}
\begin{solution}
    First, set up the expanded form 
    \begin{align*}
        X(s) = \frac{A_1}{s} + \frac{A_2}{s+1} + \frac{A_3}{s+3}.
    \end{align*}
    The expansion coefficients are given by 
    \begin{align*}
        A_1 &= sX(s)\big|_{s=0} = s\frac{s^2-4s+3}{s(s+1)(s+3)}\bigg|_{s=0} = \frac{s^2-4s+3}{(s+1)(s+3)}\bigg|_{s=0} = 1 \\[0.5cm]
        A_2 &= (s+1)X(s)\big|_{s=-1} = (s+1)\frac{s^2-4s+3}{s(s+1)(s+3)}\bigg|_{s=-1} = \frac{s^2-4s+3}{s(s+3)}\bigg|_{s=0} = -4 \\[0.5cm]
        A_3 &= (s+3)X(s)\big|_{s=-3} = (s+3)\frac{s^2-4s+3}{s(s+1)(s+3)}\bigg|_{s=-1} = \frac{s^2-4s+3}{s(s+1)}\bigg|_{s=0} = 4
    \end{align*}
    The full expansion is then 
    \begin{align*}
        X(s) = \frac{1}{s} - \frac{4}{s+1} + \frac{4}{s+3}.
    \end{align*}
    Here, we see that inverse Laplace transform is 
    \begin{align*}
        x(t) &= \mathcal{L}^{-1}\left[\frac{1}{s} - \frac{4}{s+1} + \frac{4}{s+3}\right] \\
        &= u(t) - 4e^{-t}u(t) + 4e^{-3t}u(t).
    \end{align*}
\end{solution}
\begin{tcolorbox}[width=\textwidth,colback={white}, sharp corners]
    \textbf{\underline{Case 2: Repeated real poles.}} \\[0.25cm]
    Suppose there is a repeated pole at $s=p$. Then $X(s)$ can be expanded as 
    \begin{align}
        X(s) = \frac{N(s)}{(s-p)^m} = \frac{B_1}{s-p} + \frac{B_2}{(s-p)^2} + \cdots + \frac{B_m}{(s-p)^m},
    \end{align}
    with expansion coefficients given by 
    \begin{align}
        B_m &= (s-p)^mX(s)\big|_{s=p} \\ 
        B_{m-1} &= \left\{\frac{d}{ds}\left[(s-p)^mX(s)\right]\right\}\bigg|_{s=p} \\ 
        B_{m-2} &= \left\{\frac{1}{2!}\cdot\frac{d^2}{ds^2}\left[(s-p)^mX(s)\right]\right\}\bigg|_{s=p}, 
    \end{align}
    or more generally, 
    \begin{align}
        B_n = \left\{\frac{1}{(m-n)!}\cdot\frac{d^{m-n}}{ds^{m-n}}\left[(s-p)^mX(s)\right]\right\}\bigg|_{s=p}, \text{ for } n=1,2,...,m.
    \end{align}
\end{tcolorbox}
\begin{example}
    Use the residue method to expand $X(s) = \dfrac{s^2+3s+3}{(s+2)(s+3)^3}$.\\
\end{example}
\begin{solution}
    Setting up the expanded form, 
    \begin{align*}
        X(s) = \frac{A_1}{s+2} + \frac{B_1}{s+3} + \frac{B_2}{(s+3)^2} + \frac{B_3}{(s+3)^3}.
    \end{align*}
    The expansion coefficients are given by 
    \begin{align*}
        A_1 &= sX(s)\big|_{s=-2} = s\frac{s^2+3s+3}{(s+2)(s+3)^3}\bigg|_{s=-2} = \frac{s^2+3s+3}{(s+3)^3}\bigg|_{s=-2} = 1 \\[0.5cm]
        B_3 &= (s+3)^3X(s)\big|_{s=-3} = (s+3)^3\frac{s^2+3s+3}{(s+2)(s+3)^3}\bigg|_{s=-3} = \frac{s^2+3s+3}{s+2}\bigg|_{s=-3} = -3 \\[0.5cm]
        B_2 &= \left\{\frac{d}{ds}[(s+3)^3X(s)]\right\}\bigg|_{s=-3} = \frac{d}{ds}\left[\frac{s^2+3s+3}{s+2}\right]\bigg|_{s=-3} = \frac{s^2+4s+3}{(s+2)^2}\bigg|_{s=-3} = 0 \\[0.5cm]
        B_1 &= \left\{\frac{1}{2!}\cdot\frac{d^2}{ds^2}[(s+3)^3X(s)]\right\}\bigg|_{s=-3} = \frac{1}{2}\cdot\frac{d^2}{ds^2}\left[\frac{s^2+3s+3}{s+2}\right]\bigg|_{s=-3} = \frac{1}{2}\cdot\frac{2}{(s+2)^3}\bigg|_{s=-3} = -1
    \end{align*}
    The full expanded form is then 
    \begin{align*}
        X(s) = \frac{1}{s+2} - \frac{1}{s+3} - \frac{3}{(s+3)^3},
    \end{align*}
    with inverse Laplace transform 
    \begin{align*}
        x(t) &= e^{-2t}u(t) - e^{-3t}u(t) - \frac{3}{2}\,t^2e^{-3t}u(t) \\ 
        &= \left[e^{-2t} - e^{-3t} - \frac{3}{2}\,t^2e^{-3t}\right] u(t).
    \end{align*}
\end{solution}
\begin{tcolorbox}[width=\textwidth,colback={white}, sharp corners]
    \textbf{\underline{Case 3: Distinct complex poles.}} \\[0.25cm]
    If $X(s)$ has a complex pole at $s=p$, it also follows that its complex conjugate $s=p^*$ is also a pole. Interestingly, the expansion coefficients associated with 
    both complex poles are also complex conjugates of each other. That is, 
    \begin{align}
        X(s) = \frac{N(s)}{s^2+as+b} = \frac{C}{s-p} + \frac{C^*}{s-p^*},
    \end{align}
    with expansion coefficient 
    \begin{align}
        C = (s-p)X(s)\big|_{s=p}.
    \end{align}
\end{tcolorbox}
\begin{example}
    Use the residue method to expand $X(s)=\dfrac{s(s-8)}{(s+2)(s^2+16)}$. \\
\end{example}
\begin{solution}
    It follows that for $s^2+16=0$, there are poles at $s=\pm j4$, along with $s=-2$. The expanded form then is given by 
    \begin{align*}
        X(s) = \frac{A_1}{s+2} + \frac{C_1}{s+j4} + \frac{C_1^*}{s-j4},
    \end{align*}
    with expansion coefficients
    \begin{align*}
        A_1 &= (s+2)X(s)\big|_{s=-2} = (s+2)\frac{s(s-8)}{(s+2)(s^2+16)}\bigg|_{s=-2} = \frac{s(s-8)}{(s^2+16)}\bigg|_{s=-2} = 1 \\[0.5cm]
        C_1 &= (s+j4)X(s)\big|_{s=-j4} = \frac{s(s-8)}{(s+2)(s-j4)}\bigg|_{s=-j4} = -j = e^{-j\pi/2} \\[0.5cm]
        C_1^* &= C_1\big|_{j\leftarrow(-j)} = e^{+j\pi/2}
    \end{align*}
    The full expanded form is then 
    \begin{align*}
        X(s) = \frac{1}{s+2} + \frac{e^{-j\pi/2}}{s+j4} + \frac{e^{+j\pi/2}}{s-j4},
    \end{align*}
    with inverse Laplace transform 
    \begin{align*}
        x(t) = e^{-2t}u(t) + 2\cos(4t+\pi/2)u(t).
    \end{align*}
\end{solution}
\begin{tcolorbox}[width=\textwidth,colback={white}, sharp corners]
    \textbf{\underline{Case 4: Repeated complex poles.}} \\[0.25cm]
    Lastly, for repeated complex poles, the expansion is given by 
    \begin{align}
        X(s) &= \frac{N(s)}{(s^2+as+b)^m} = \frac{N(s)}{(s-p)^m(s-p^*)^m} \nonumber \\
        &= \left(\frac{D_1}{s-p} + \frac{D_2}{(s-p)^2} + \cdots + \frac{D_m}{(s-p)^m}\right) + \left(\frac{D_1^*}{s-p^*} + \frac{D_2^*}{(s-p^*)^2} + \cdots + \frac{D_m^*}{(s-p^*)^m}\right),
    \end{align}
    with expansion coefficients
    \begin{align}
        D_n = \left\{\frac{1}{(m-n)!}\cdot \frac{d^{m-n}}{ds^{m-n}}[(s-p)^mX(s)]\right\}\bigg|_{s=p}, \text{ for } n=1,2,...,m.
    \end{align}
\end{tcolorbox}
Table \ref{laplace_pairs_short} lists the corresponding Laplace transform pairs for the four partial fraction expansion cases. 
\begin{table}[hbt!]
    \centering
    \caption{Unilateral Laplace transform pairs based on poles.}
    \label{laplace_pairs_short}
    \begin{tabular}{|c|c|c|}
        \hline
        Pole & $X(s)$ & $x(t)=\mathcal{L}^{-1}[X(s)]$ \\[0.15cm]
        \hline
        & & \\
        Distinct real & $\dfrac{A}{s+a}$ & $Ae^{-at}u(t)$ \\[0.5cm]
        Repeated real & $\dfrac{A}{(s+a)^n}$ & $\dfrac{At^{n-1}}{(n-1)!}e^{-at}u(t)$ \\[0.5cm]
        Distinct complex & $\dfrac{Ae^{+j\theta}}{s+a+jb}+\dfrac{Ae^{-j\theta}}{s+a-jb}$ & $2Ae^{-at}\cos(bt-\theta)u(t)$ \\[0.5cm]
        Repeated complex & $\dfrac{Ae^{+j\theta}}{(s+a+jb)^n}+\dfrac{Ae^{-j\theta}}{(s+a-jb)^n}$ & $\dfrac{2At^{n-1}}{(n-1)!}e^{-at}\cos(bt-\theta)u(t)$ \\[0.5cm]
        \hline
    \end{tabular}
\end{table}

\section{Transfer Function}
\begin{tcolorbox}[width=\textwidth,colback={white}, sharp corners]
    Given that all initial conditions of the input and output signals of an LTI system are zero, the \emph{transfer function} is the Laplace transform of the impulse response and 
    is defined as 
    \begin{align}
        H(s) = \frac{Y(s)}{X(s)} = \mathcal{L}[h(t)].
    \end{align}
\end{tcolorbox}
\noindent In fact, by the convolution property, it follows that 
\begin{align}
    y(t) = x(t) * h(t) \iff Y(s) = X(s)H(s).
\end{align}
Symbolically,
\begin{center}
    \begin{tikzpicture}
        \node [] (input) at (-3,0){$x(t)$};
        \node [] (output) at (+3,0){$y(t)$};
        \node [draw,
            fill=yellow!50, 
            minimum width=2cm, 
            minimum height=1.2cm
        ] (system) at (0,0){$h(t)$};
        \draw [-latex, line width=1.5pt] (input) -- (system);
        \draw [-latex, line width=1.5pt] (system) -- (output);
    \end{tikzpicture} \\[0.25cm]
    \begin{tikzpicture}
        \node at (0,0) [double arrow, draw=black, top color=red, bottom color=blue,
            minimum width = 15pt, double arrow head extend=5pt,
            minimum height=12mm,
            rotate=90] {};
    \end{tikzpicture} \\[0.25cm]
    \begin{tikzpicture}
        \node [] (input) at (-3,0){$X(s)$};
        \node [] (output) at (+3,0){$Y(s)$};
        \node [draw,
            fill=yellow!50, 
            minimum width=2cm, 
            minimum height=1.2cm
        ] (system) at (0,0){$H(s)$};
        \draw [-latex, line width=1.5pt] (input) -- (system);
        \draw [-latex, line width=1.5pt] (system) -- (output);
    \end{tikzpicture}
\end{center}
\pagebreak
As the impulse response is the system response to input $x(t)=\delta(t)$, consider the Laplace transform of the impulse signal $\mathcal{L}[\delta(t)]=1$. Then it also follows that 
\begin{center}
    \begin{tikzpicture}
        \node [] (input) at (-3,0){$x(t)=\delta(t)$};
        \node [] (output) at (+3,0){$y(t)=h(t)$};
        \node [draw,
            fill=yellow!50, 
            minimum width=2cm, 
            minimum height=1.2cm
        ] (system) at (0,0){$h(t)$};
        \draw [-latex, line width=1.5pt] (input) -- (system);
        \draw [-latex, line width=1.5pt] (system) -- (output);
    \end{tikzpicture} \\[0.25cm]
    \begin{tikzpicture}
        \node at (0,0) [double arrow, draw=black, top color=red, bottom color=blue,
            minimum width = 15pt, double arrow head extend=5pt,
            minimum height=12mm,
            rotate=90] {};
    \end{tikzpicture} \\[0.25cm]
    \begin{tikzpicture}
        \node [] (input) at (-3,0){$X(s)=1$};
        \node [] (output) at (+3,0){$Y(s)=H(s)$};
        \node [draw,
            fill=yellow!50, 
            minimum width=2cm, 
            minimum height=1.2cm
        ] (system) at (0,0){$H(s)$};
        \draw [-latex, line width=1.5pt] (input) -- (system);
        \draw [-latex, line width=1.5pt] (system) -- (output);
    \end{tikzpicture}
\end{center}
\begin{tcolorbox}[width=\textwidth,colback={white}, sharp corners]
    To find the transfer function of an LTI system, there are two approaches:
    \begin{enumerate}
        \item Find $X(s),Y(s)$. Then calculate $H(s)=Y(s)/X(s)$.
        \item Find $X(s),Y(s)$. Then substitute $X(s)=1$ and $Y(s)=H(s)$.
    \end{enumerate}
\end{tcolorbox}
Typically, the transfer function $H(s)$ is used to analyze the characteristics of the LTI system. Sometimes, when one is interested in  
representing systems solely for the purpose of finding an output response, the \emph{conditional transfer function} $H_C(s)$ 
is calculated, where initial conditions are not necessarily zero; this is when the substitution approach works best. However, when including nonzero 
initial conditions, the system can no longer be called LTI.

\begin{example}
    Find the transfer function for an LTI system characterized by LCCDE
    \begin{align*}
        \frac{d^2y(t)}{dt^2} + 3\frac{dy(t)}{dt} + 2y(t) &= \frac{dx(t)}{dt} + 3x(t), \\
        x(0^-) &= 0, \\
        y(0^-)=y'(0^-) &= 1.
    \end{align*}
    Then find the conditional transfer function of the system.
\end{example}
\begin{solution}
    To find the transfer function $H(s)$, set all initial conditions to zero and take the Laplace transform of the LCCDE.
    \begin{align*}
        \mathcal{L}\left[\frac{d^2y(t)}{dt^2} + 3\frac{dy(t)}{dt} + 2y(t)\right] &= \mathcal{L}\left[\frac{dx(t)}{dt} + 3x(t)\right] \\
        s^2Y(s)+3sY(s)+2Y(s) &= sX(s) + 3X(s) \\
        Y(s)[s^2+3s+2] &= X(s)[s+3]
    \end{align*}
    Rearranging the equation, we get the transfer function
    \begin{align*}
        H(s) = \frac{Y(s)}{X(s)} = \frac{s+3}{s^2+3s+2}.
    \end{align*}
    Now we consider the initial conditions to solve the conditional transfer function.
    \begin{align*}
        [s^2Y(s)-s-1]+3[sY(s)-1]+2Y(s) &= [sX(s)-0] + 3X(s) \\ 
        Y(s)[s^2+3s+2] - (s+4) &= X(s)[s+3].
    \end{align*}
    Now let $X(s)=1$ and $Y(s)=H_C(s)$. The conditional transfer function is given by 
    \begin{align*}
        H_C(s)[s^2+3s+2] - (s+4) &= [1][s+3] \\
        \Longrightarrow H_C(s) &= \frac{2s+7}{s^2+3s+2}
    \end{align*}
\end{solution}

For the rest of this text, unless explicitly specified, only transfer functions with zero initial conditions will be considered.

\subsection{System Stability}
As previously introduced, zeros are the roots of the numerator of a transfer function ($N(s)=0$), and poles are the roots of the denominator of a transfer function ($D(s)=0$). 
We can visualize both zeros and poles on the \emph{$s$-plane}, which plots $\Im(s)$ versus $\Re(s)$. As seen in Figure \ref{s_plane}, zeros are plotted with circles, and poles are plotted with crosses.
\begin{tcolorbox}[width=\textwidth,colback={white}, sharp corners]
    The \emph{open left half-plane} (OLHP) refers to the region of the $s$-plane which describes $\Re(s)<0$. 
    \begin{itemize}
        \item If an LTI system with transfer function $H(s)$ is strictly proper or proper, then the system is BIBO stable if and only if its poles reside in the OLHP.
        \item If $H(s)$ is improper, then the system is unstable.
    \end{itemize}
\end{tcolorbox}
Note that the imaginary axis is not included in the OLHP; a system with poles on the imaginary axis is not stable.

\begin{figure}[!hbt]
    \caption{$s$-plane plot of a system with zeros and poles.}
    \label{s_plane}
    \centering
    \begin{tikzpicture}[scale=1.2]
        \begin{axis}[
            axis x line=center, axis y line=center,
            ymin=-3, ymax=3, ymajorticks=false, ylabel={$\Im(s)$},
            xmin=-3, xmax=3, xtick={-2,...,2}, xlabel={$\Re(s)$},
            domain=-3:3,samples=200,
            width=8cm, height=6cm]
            \begin{scope}[on background layer={color=orange!10}]
                \draw[fill=orange!10] (-3,3) -- (0,3) -- (0,-3) -- (-3,-3);
            \end{scope}
            \draw (-1.5,2) node{\small left half-plane};
            \draw (1.5,-2) node{\small $s$-plane};
            \draw (-2,0) node[cross=4pt,red, very thick] {};
            \draw (-1,0) node[cross=4pt,red, very thick] {};
            \draw (0,0) node[cross=4pt,red, very thick] {};
            \draw[red, very thick] (1,0) circle (3pt);
            \draw[red, very thick] (2,0) circle (3pt);
            % labels
            \draw (-1.5,1) node{\footnotesize poles};
            \draw[-latex] (-1.75,0.75) -- (-2,0.25);
            \draw[-latex] (-1.5,0.75) -- (-1,0.25);
            \draw[-latex] (-1.1,0.75) -- (-0.25,0.25);
            \draw (1.5,1) node{\footnotesize zeros};
            \draw[-latex] (1.75,0.75) -- (2,0.25);
            \draw[-latex] (1.25,0.75) -- (1,0.25);
        \end{axis}
    \end{tikzpicture}
\end{figure}

\begin{example}
    Determine if an LTI system with the following impulse response is BIBO stable:
    \begin{align*}
        h(t)=(4+j5)e^{(2+j3)t}u(t)+(4-j5)e^{(2-j3)t}u(t)
    \end{align*}
\end{example}
\begin{solution}
    First, find the transfer function by taking the Laplace transform of the impulse response.
    \begin{align*}
        H(s) = \mathcal{L}[h(t)] &= (4+j5)\mathcal{L}[e^{(2+j3)t}u(t)]+(4-j5)\mathcal{L}[e^{(2-j3)t}u(t)] \\
        &= \frac{4+j5}{s-2-j3} + \frac{4-j5}{s-2+j3} \\
        &= \frac{8s-46}{(s-2-j3)(s-2+j3)}
    \end{align*}
    We can see that the poles are given by $p_1=2+j3$ and $p_2=2-j3$, and the zero is given by $z_1=46/8$. 
    Since $\Re(p_k)>0$, the poles are not in the OHLP, and thus the system is unstable. \\
\end{solution}

\subsection{Invertible Systems}
As previously introduced, a system is invertible if there exists an inverse system that maps an output signal $y(t)$ back to its input $x(t)$. In the LTI case, 
an inverse system characterized by $g(t)$ exists such that 
\begin{center}
    \begin{tikzpicture}
        \node [] (input) at (-3,0){$x(t)$};
        \node [] (output) at (+6,0){$x(t)$};
        \node [draw,
            fill=yellow!50, 
            minimum width=2cm, 
            minimum height=1.2cm
        ] (fwd) at (0,0){$h(t)$};
        \node [draw,
            fill=blue!30, 
            minimum width=2cm, 
            minimum height=1.2cm
        ] (inv) at (3,0){$g(t)$};
        \draw [-latex, line width=1.5pt] (input) -- (fwd);
        \draw [-latex, line width=1.5pt] (fwd) -- (inv);
        \draw [-latex, line width=1.5pt] (inv) -- (output);
    \end{tikzpicture}
\end{center}
\begin{center}
    \begin{tikzpicture}
        \node [] (input) at (-3,0){$y(t)$};
        \node [] (output) at (+6,0){$y(t)$};
        \node [draw,
            fill=blue!30, 
            minimum width=2cm, 
            minimum height=1.2cm
        ] (inv) at (0,0){$g(t)$};
        \node [draw,
            fill=yellow!50, 
            minimum width=2cm, 
            minimum height=1.2cm
        ] (fwd) at (3,0){$h(t)$};
        \draw [-latex, line width=1.5pt] (input) -- (inv);
        \draw [-latex, line width=1.5pt] (inv) -- (fwd);
        \draw [-latex, line width=1.5pt] (fwd) -- (output);
    \end{tikzpicture}
\end{center}
While an inverse system $g(t)$ may exist, it does not necessarily mean it is stable. In fact, for 
\begin{align}
    G(s) = \frac{1}{H(s)},
\end{align}
in order for a BIBO stable and causal LTI system $H(s)$ to have a stable and causal inverse LTI system $G(s)$, the transfer function $H(s)$ 
must be a proper rational function (i.e., $\deg[N(s)]=\deg[D(s)]$) with both its poles and zeros all residing in the OLHP of the $s$-plane. 
Such a system is called a \emph{minimum phase system}. 

\begin{example}
    An LTI system is characterized by its impulse response 
    \begin{align*}
        h(t) = \delta(t) - 4e^{-3t}u(t).
    \end{align*}
    Determine if it is a minimum phase system.
\end{example}
\begin{solution}
    \begin{align*}
        H(s) = \mathcal{L}[h(t)] &= \mathcal{L}[\delta(t)] - 4 \mathcal{L}[e^{-3t}u(t)] \\ 
        &= 1 - \frac{4}{s+3} = \frac{s-1}{s+3}.
    \end{align*}
    From above, there is a pole $p=-3$ and a zero $z=1$. While the forward system may be stable for $\Re(p)<0$, the inverse is not since 
    $\Re(z)>0$. Therefore, the LTI system is not a minimum phase system.
\end{solution}

\section{System Response of Second-Order LCCDEs}
Consider second-order LCCDEs of the form 
\begin{align}
    \frac{d^2y(t)}{dt^2} + a_1\frac{dy(t)}{dt} + a_2y(t) = b_1\frac{dx(t)}{dt} + b_2x(t).
\end{align}
Setting initial conditions to zero such that the system is LTI, the transfer function is given by 
\begin{align}
    s^2Y(s) + a_1sY(s) + a_2Y(s) &= b_1sX(s) + b_2X(s) \nonumber \\
    \Longrightarrow H(s) = \frac{Y(s)}{X(s)} &= \frac{b_1s+b_2}{s^2+a_1s+a_2s}.
\end{align}
\begin{tcolorbox}[width=\textwidth,colback={white}, sharp corners]
    Define the following system attributes
    \begin{itemize}
        \item Attenuation coefficient [Np/s]: $\alpha = \dfrac{a_1}{2}$
        \item Undamped natural frequency [rad/s]: $\omega_0 = \sqrt{a_2}$
        \item Damping coefficient [unitless]: $\xi = \dfrac{\alpha}{\omega_0} = \dfrac{a_1}{2\sqrt{a_2}}$
    \end{itemize}
\end{tcolorbox}
\noindent such that the solutions to the \emph{characteristic equation}
\begin{align}
    s^2 + a_1s + a_2 &= 0
\end{align}
can be written as 
\begin{align}
    s = \frac{-a_1\pm\sqrt{a_1^2-4a_2}}{2} = -\alpha \pm \sqrt{\alpha^2-\omega_0^2} = \omega_0\left[-\xi\pm\sqrt{\xi^2-1}\right].
\end{align}
In fact, the solutions to the characteristic equation are the poles $p_1, p_2$ of the system. Since $\Re(p_k) < 0$ must be true for a system to be BIBO stable, 
it also holds from above that $a_1>0$ and $a_2>0$ must be true for the system to be BIBO stable.
\begin{tcolorbox}[width=\textwidth,colback={white}, sharp corners]
    Depending on the damping coefficient $\xi$, there are five possible system responses for an input impulse or step signal:
    \begin{itemize}
        \item $\xi > 1$: overdamped response
        \item $\xi = 1$: critically damped response
        \item $0 < \xi < 1$: underdamped response
        \item $\xi = 0$: undamped response
        \item $\xi < 0$: unstable response
    \end{itemize}
\end{tcolorbox}
\begin{figure}[!hbt]
    \caption{Second-order system response on the $s$-plane.}
    \label{s_plane_second}
    \centering
    \begin{tikzpicture}[scale=1.2]
        \begin{axis}[
            axis x line=center, axis y line=center,
            ymin=-3, ymax=3, ymajorticks=false, ylabel={$\Im(s)$},
            xmin=-3, xmax=3, xmajorticks=false, xlabel={$\Re(s)$},
            domain=-3:3,samples=200,
            width=8cm, height=6cm]
            \begin{scope}[on background layer={color=orange!10}]
                \draw[fill=orange!30] (-3,3) -- (-0.1,3) -- (-0.1,0.1) -- (-3,0.1);
                \draw[fill=orange!30] (-3,-0.1) -- (-0.1,-0.1) -- (-0.1,-3) -- (-3,-3);
            \end{scope}
            \draw (-1.5,2) node[text=red]{\scriptsize underdamped};
            \draw (-1.5,-2) node[text=red]{\scriptsize underdamped};
            \begin{scope}[on background layer={color=green!50}]
                \draw[fill=green!50] (-3,0.1) -- (-0.1,0.1) -- (-0.1,-0.1) -- (-3,-0.1);
            \end{scope}
            \draw (-1.5,0.2) node[text=darkgreen]{\scriptsize overdamped or};
            \draw (-1.5,-0.2) node[text=darkgreen]{\scriptsize critically damped};
            \begin{scope}[on background layer={color=blue!30}]
                \draw[fill=blue!30] (-0.1,3) -- (0.1,3) -- (0.1,-3) -- (-0.1,-3);
            \end{scope}
            \draw (0.2,0) node[text=blue, rotate=90]{\scriptsize undamped};
            \begin{scope}[on background layer={color=gray!20}]
                \draw[fill=gray!20] (0.1,3) -- (3,3) -- (3,-3) -- (0.1,-3);
            \end{scope}
            \draw (1.5,2) node[text=black]{\scriptsize unstable};
            \draw (1.5,-2) node[text=black]{\scriptsize unstable};
        \end{axis}
    \end{tikzpicture}
\end{figure}
We are particularly interested in the first three damping responses as they describe how a system response reaches \emph{steady-state}, or \emph{equilibrium}. That is, it describes 
how fast and in what manner a system approaches the final value of the system response. 
\\ \\
As seen in Figure \ref{damping}, overdamped responses slowly approach steady-state, whereas underdamped responses quickly but oscillatorily approach steady-state at a \emph{damped natural frequency} of 
\begin{align}
    \omega_d = \sqrt{\omega_0^2-\alpha^2} = \omega\sqrt{1-\xi^2}.
\end{align}
Critically damped responses represent the quickest non-oscillatory path to steady-state. 
\begin{figure}[!hbt]
    \caption{Damping responses.}
    \label{damping}
    \centering
    \resizebox{0.65\textwidth}{!}{%
        \includestandalone[mode=buildnew, width=\textwidth]{damping}
    }
\end{figure}
\pagebreak
\begin{example}
    An LTI system is described by the LCCDE 
    \begin{align*}
        \frac{d^2y(t)}{dt^2} + B\frac{dy(t)}{dt} + 25y(t) = \frac{dx(t)}{dt} + 23x(t).
    \end{align*}
    Compute the range of values for constant $B$ such that the system impulse response is:
    \begin{enumerate}[(a)]
        \item overdamped
        \item critically damped
        \item underdamped
    \end{enumerate}
    Then let $B=26$ and compute the impulse and step responses of the system.
\end{example}
\begin{solution}
    The characteristic equation of the LCCDE is 
    \begin{align*}
        s^2 + Bs + 25 = 0.
    \end{align*}
    The damping coefficient is then given by 
    \begin{align*}
        \xi = \frac{a_1}{2\sqrt{a_2}} = \frac{B}{2\sqrt{25}} = \frac{B}{10}.
    \end{align*}
    It then follows that
    \begin{itemize}
        \item overdamped: $\xi > 1 \Longrightarrow B > 10$
        \item critically damped: $\xi = 1 \Longrightarrow B = 10$
        \item underdamped: $0<\xi<1 \Longrightarrow 0<B<10$
    \end{itemize}
    Letting $B=26$ and setting initial conditions to zero, the Laplace transform of the LCCDE is
    \begin{align*}
        Y(s)[s^2+26s+25] &= X(s)[s+23].
    \end{align*}
    The transfer function is given by
    \begin{align*}
        H(s) &= \frac{Y(s)}{X(s)} = \frac{s+23}{s^2+26s+25} = \frac{s+23}{(s+1)(s+25)} = \frac{A_1}{s+1} + \frac{A_2}{s+25}, \\
        A_1 &= (s+1)H(s)\big|_{s=-1} = 22/24 = 11/12, \\
        A_2 &= (s+25)H(s)\big|_{s=-25} = (-2)/(-24) = 1/12.
    \end{align*}
    Therefore the impulse response is 
    \begin{align*}
        h(t) = \mathcal{L}^{-1}[H(s)] = \frac{1}{12}\left[11e^{-t}u(t) + e^{-25t}u(t)\right].
    \end{align*}
    The step response can be calculated from 
    \begin{align*}
        y_{step}(t) &= \int_{-\infty}^{t} h(\tau) \,d\tau \iff Y_{step}(s) = \frac{1}{s}H(s). \\[0.5cm]
        Y_{step}(s) &= \frac{s+23}{s(s+1)(s+25)} = \frac{A_0}{s} + \frac{A_1}{s+1} + \frac{A_2}{s+25}, \\
        A_0 &= sY_{step}(s)\big|_{s=0} = 23/25 \\
        A_1 &= (s+1)Y_{step}(s)\big|_{s=-1} = 22/(-24) = -11/12, \\
        A_2 &= (s+25)Y_{step}(s)\big|_{s=-25} = (-2)/600 = -1/300. \\[0.5cm]
        y_{step}(t) &= \mathcal{L}^{-1}[Y_{step}(s)] = \frac{23}{25}u(t) - \frac{11}{12}e^{-t}u(t) - \frac{1}{300}e^{-25t}u(t).
    \end{align*}
\end{solution}

\section{System Response Partitions}
Given some system response $y(t)$, there are three different ways to partition the response, with each partition describing a particular 
characteristic of the system response:
\begin{enumerate}
    \item Zero-state / zero-input partition
    \item Natural / forced partition
    \item Transient / steady-state partition
\end{enumerate}
\begin{tcolorbox}[width=\textwidth,colback={white}, sharp corners]
    \textbf{\underline{Type 1: Zero-state / zero-input partition.}} \\[0.25cm]
    As the names imply,
    \begin{itemize}
        \item the \emph{zero-state response} (ZSR) is the system response when there are no initial conditions, and
        \item the \emph{zero-input response} (ZIR) is the system response when there is no input signal (but there are initial conditions).
    \end{itemize}
\end{tcolorbox}
It follows that 
\begin{align}
    y(t) = y_{ZSR}(t) + y_{ZIR}(t).
\end{align}
Any exponentials $e^{-\lambda_k t}$ in the zero-input response $y_{ZIR}(t)$ are referred to as the \emph{characteristic modes} (also called \emph{natural modes}, or simply modes) 
of the system. The zero-input response is comprised of only natural modes, whereas the zero-state response has a mix of natural modes and ``forced'' modes from the input excitation function $x(t)$.
\begin{tcolorbox}[width=\textwidth,colback={white}, sharp corners]
    \textbf{\underline{Type 2: Natural / forced partition.}} \\[0.25cm]
    On the discussion of natural modes and forced modes,
    \begin{itemize}
        \item the \emph{natural response} is the part of the system response with only natural modes which describe the natural identity of the system, and
        \item the \emph{forced response} is the part of the system response with only forced modes which mimic the character of the excitation function $x(t)$.
    \end{itemize}
\end{tcolorbox}
\noindent It follows that 
\begin{align}
    y(t) = y_{nat}(t) + y_{forc}(t).
\end{align}
The natural response is closely related to the zero-input response; since the zero-input response is composed exclusively of modes, we can use it to identify the modes, 
then collect all terms with those modes to find the natural response. Naturally, the characteristic modes can also be determined by finding the solutions $s$ of the characteristic equation 
and calculating $e^{-st}$.
\begin{tcolorbox}[width=\textwidth,colback={white}, sharp corners]
    \textbf{\underline{Type 3: Transient / steady-state partition.}} \\[0.25cm]
    The end behavior of the terms of the system response can be analyzed such that
    \begin{itemize}
        \item the \emph{transient response} is the part of the system response that decays to zero as $t\to\infty$, and
        \item the \emph{steady-state response} is the part of the system response that remains after the transient response goes to zero.
    \end{itemize}
\end{tcolorbox}
\noindent It follows that 
\begin{align}
    y(t) = y_{tr}(t) + y_{ss}(t).
\end{align}
We have seen this behavior before with the damping response of a stable LTI system described by a second-order LCCDE. The curves eventually approach a final value $K$, with 
the approach rate and shape dependent on the damping coefficient $\xi$. The damping response actually can be written as a sum of transient terms and the steady-state response $K\cdot u(t)$.
\\ \\
Again, notice here that we reserve the term ``LTI system'' for systems that have zero initial conditions, in addition to the superposition principle condition. Partitions can be applied to any LTI or ``LTI-like'' system response, 
whether or not the initial conditions are zero.
\begin{example}
    Suppose a system can be characterized by the LCCDE 
    \begin{align*}
        \frac{d^2y(t)}{dt^2} + 5\frac{dy(t)}{dt} + 4y(t) &= \frac{dx(t)}{dt} + 3x(t), \\ 
        y(0) = y'(0) &= 1, \\ 
        x(t) &= u(t).
    \end{align*}
    Identify all six partitioned responses of the system response.
\end{example}
\begin{solution}
    Taking the Laplace transform of the LCCDE, it follows that 
    \begin{align*}
        \mathcal{L}\left[\frac{d^2y(t)}{dt^2} + 3\frac{dy(t)}{dt} + 2y(t)\right] &= \mathcal{L}\left[\frac{dx(t)}{dt} + 3x(t)\right] \\
        [s^2Y(s)-s-1] + 3[sY(s)-1] + 2Y(s) &= [sX(s)-0] + 3X(s) \\
        Y(s)[s^2+3s+2] - (s+4) &= X(s)[s+3].
    \end{align*}
    Since $X(s) = \mathcal{L}[u(t)] = \dfrac{1}{s}$, it follows that
    \begin{align*}
        Y(s)[s^2+3s+2] - (s+4) &= \frac{s+3}{s} \\
        Y(s)[(s+1)(s+2)] &= \frac{s^2+5s+3}{s} \\
        \Longrightarrow Y(s) &= \frac{s^2+5s+3}{s(s+1)(s+2)} = \frac{A_1}{s} + \frac{A_2}{s+1} + \frac{A_3}{s+2}.
    \end{align*}
    After partial fraction decomposition, the inverse Laplace transform gives 
    \begin{align*}
        y(t) = 1.5u(t) + e^{-t}u(t) -1.5e^{-2t}u(t).
    \end{align*}
    To find the modes, first solve for $y_{ZIR}(t)$. We can do this by taking the Laplace transform of the LCCDE and setting $X(s)=0$.
    \begin{align*}
        Y(s)[s^2+3s+2] - (s+4) &= X(s)[s+3] = 0 \\
        \Longrightarrow Y_{ZIR}(s) &= \frac{s+4}{s^2+3s+2} = \frac{s+4}{(s+1)(s+2)} = \frac{3}{s+1} - \frac{2}{s+2}.
    \end{align*}
    Then it follows that 
    \begin{align*}
        y_{ZIR}(t) &= 3e^{-t}u(t) - 2e^{-2t}u(t), \\
        y_{ZSR}(t) &= y(t) - y_{ZIR}(t) = 1.5u(t) - 2e^{-t}u(t) + 0.5e^{-2t}u(t).
    \end{align*}
    From $y_{ZIR}(t)$, the characteristic modes are given by exponentials $\{e^{-t}, e^{-2t}\}$. Therefore, by grouping natural modes together and forced modes together, 
    \begin{align*}
        y_{nat}(t) &= e^{-t}u(t) - 1.5e^{-2t}u(t), \\
        y_{forc}(t) &= 1.5u(t).
    \end{align*}
    Lastly, we can group decaying terms together and the remaining terms together such that 
    \begin{align*}
        y_{tr}(t) &= e^{-t}u(t) - 1.5e^{-2t}u(t), \\
        y_{ss}(t) &= 1.5u(t).
    \end{align*}
\end{solution}

\section{$s$--Domain Circuit Analysis}
Just as differential equations can be solved in the $s$-domain, the systems represented by differential equations can be solved in the $s$-domain. For instance, 
an electric circuit can be transformed to an $s$-domain equivalent and solved using algebraic equations, with the $s$-domain electrical components referred to  
as impedances. The equivalent $s$-domain models can be seen in Table \ref{s_circ}.
\\ \\
While only Thévenin equivalent models with voltage sources are shown in Table \ref{s_circ}, Norton equivalent models with current sources can be drawn instead, though the relation between the Thévenin equivalents 
and the steady-state equivalents are easier to visualize.
\begin{table}[!hbt]
    \caption{Circuit models for electrical components in the $s$-domain.}
    \label{s_circ}
    \centering
    \begin{tabular}{ |c|c|c|c| }
        \hline
        \shortstack{Electrical \\ Component} & \shortstack{Time \\ Domain} & \shortstack{$s$-Domain \\ with Initial Conditions} & \shortstack{$s$-Domain \\ at Steady State} \\
        \hline
        & & & \\
        Resistor &
        \adjustbox{valign=m}{
        \begin{circuitikz}[american voltages,raised voltages]
            \draw (0,0) to [R=$R$, f=$i_R$, v=$v_R$, voltage shift = 1.2] (0,-3); 
        \end{circuitikz}} & 
        \adjustbox{valign=m}{
        \begin{circuitikz}[american voltages,raised voltages]
            \draw (0,0) to [R=$R$, f=$I_R$, v=$V_R$, voltage shift = 1.2] (0,-3); 
        \end{circuitikz}} & 
        \adjustbox{valign=m}{
        \begin{circuitikz}[american voltages,raised voltages]
            \draw (0,0) to [R=$R$, f=$I_R$, v=$V_R$, voltage shift = 1.2] (0,-3); 
        \end{circuitikz}}\\[1.5cm]
        \hline
        & & & \\
        Capacitor & 
        \adjustbox{valign=m}{
        \begin{circuitikz}[american voltages,raised voltages]
            \draw (0,0) to [C=$C$, f=$i_C(t)$, v=$v_C(t)$, voltage shift=1.2] (0,-3); 
        \end{circuitikz}} &
        \adjustbox{valign=m}{
        \begin{circuitikz}[american voltages,raised voltages]
            \draw (0,0) to [C=$\dfrac{1}{sC}$] (0,-1); 
            \draw (0,-1) to [short, f=$I_C$] (0,-1.5);
            \draw (0,-1.5) to [V, l^=$\dfrac{v_c(0^-)}{s}$] (0,-3); 
            \draw (-1,0) to [open,v=$V_C$] (-1,-3);
        \end{circuitikz}} &
        \adjustbox{valign=m}{
        \begin{circuitikz}[american voltages,raised voltages]
            \draw (0,0) to [C=$\dfrac{1}{sC}$, f=$I_C$, v=$V_C$, voltage shift=1.2] (0,-3); 
        \end{circuitikz}} \\[1.5cm]
        \hline
        & & & \\
        Inductor & 
        \adjustbox{valign=m}{
        \begin{circuitikz}[american voltages,raised voltages]
            \draw (0,0) to [L=$L$, f=$i_L(t)$, v=$v_L(t)$, voltage shift=1.2] (0,-3); 
        \end{circuitikz}} &
        \adjustbox{valign=m}{
        \begin{circuitikz}[american voltages]
            \draw (0,0) to [L=$sL$, f=$I_L$] (0,-1.5); 
            \draw (0,-3) to [V, l_=$Li_L(0^-)$] (0,-1.5); 
            \draw (-1,0) to [open,v=$V_L$] (-1,-3);
        \end{circuitikz}} &
        \adjustbox{valign=m}{
        \begin{circuitikz}[american voltages,raised voltages]
            \draw (0,0) to [L=$sL$, f=$I_L$, v=$V_L$, voltage shift=1.2] (0,-3); 
        \end{circuitikz}} \\[1.5cm]
        \hline
    \end{tabular}
\end{table}
\newpage
\noindent Recall that the unit step function $u(t)$ can be treated as an ``on switch'', where the signal being multiplied gets turned on at time $t=0$. 
Similarly here, circuit switches can be represented using signals multiplied by $u(t)$, as seen in Figure \ref{switch}. Because of how switches operate, 
here we see that the initial condition is zero.
\begin{figure}[!hbt]
    \caption{Equivalent signal representation of switches.}
    \label{switch}
    \centering
    \begin{circuitikz}[american voltages,raised voltages]
        \draw (0,0) to [V, l_=$v_{in}(t)$] (0,-2); 
        \draw (0,-2) to [short] (3,-2);
        \draw (3,0) to [R=$R$] (3,-2);
        \draw (0,0) to [closing switch] (3,0);
        \draw (1.5,0.5) node {$t=0$};
    \end{circuitikz}
    \begin{tikzpicture}
        \node at (0,0) [double arrow, draw=white, left color=white, right color=white,
      minimum width = 15pt, double arrow head extend=5pt,
      minimum height=12mm] {};
    \end{tikzpicture}
    \begin{circuitikz}[american voltages,raised voltages]
        \draw (0,0) to [V, l_=$v_{in}(t)u(t)$] (0,-2); 
        \draw (0,-2) to [short] (3,-2);
        \draw (3,0) to [R=$R$] (3,-2);
        \draw (0,0) to [short] (3,0);
    \end{circuitikz}
\end{figure}

\begin{example}
    Find the output response $v_{out}(t)$ of the following circuit, given that the switch closes at time $t=0$ and $v_1(0^-)= 20\,[V]$. \\
    \resizebox{\textwidth}{!}{\begin{circuitikz}[american voltages,raised voltages]
        \draw (0,0) to [V, l_={$35\,[V]$}] (0,-2); 
        \draw (0,0) to [closing switch] (3,0);
        \draw (1.5,0.5) node {$t=0$};
        \draw (3,0) to [R={$1\,[\Omega]$}] (4.5,0);
        \draw (4.5,0) to [C={$1\,[F]$},v=$v_1(t)$] (8,0);
        \draw (8,0) to [C={$2\,[F]$}] (8,-2);
        \draw (8,0) to [short, -*] (12.5,0);
        \draw (10,0) to [R={$0.5\,[\Omega]$}] (10,-2);
        \draw (0,-2) to [short, -*] (12.5,-2);
        \draw (12.5,0) to [open, v^>={\large $v_{out}(t)$}] (12.5,-2);
    \end{circuitikz}}
\end{example}
\begin{solution}
    Using the equivalent model for switches, first we redraw the circuit as
    \begin{center}
        \resizebox{0.8\textwidth}{!}{\begin{circuitikz}[american voltages,raised voltages]
            \draw (2,0) to [V, l_={$35u(t)\,[V]$}] (2,-2); 
            \draw (2,0) to [R={$1\,[\Omega]$}] (4.5,0);
            \draw (4.5,0) to [C={$1\,[F]$},v=$v_1(t)$] (8,0);
            \draw (8,0) to [C={$2\,[F]$}] (8,-2);
            \draw (8,0) to [short, -*] (12.5,0);
            \draw (10,0) to [R={$0.5\,[\Omega]$}] (10,-2);
            \draw (2,-2) to [short, -*] (12.5,-2);
            \draw (12.5,0) to [open, v^>={\large $v_{out}(t)$}] (12.5,-2);
        \end{circuitikz}}
    \end{center}
    Now we can transform the circuit into an $s$-domain equivalent model:
    \begin{center}
        \resizebox{0.8\textwidth}{!}{\begin{circuitikz}[american voltages,raised voltages]
            \draw (2,0) to [V, l_={$\dfrac{35}{s}$}] (2,-2); 
            \draw (2,0) to [R={$1$}] (4.5,0);
            \draw (4.5,0) to [C={$1/s$}] (6,0);
            \draw (6,0) to [V, l^={$20/s$}] (8,0);
            \draw (8,0) to [C={$1/(2s)$}] (8,-2);
            \draw (8,0) to [short, -*] (12.5,0);
            \draw (10,0) to [R={$0.5$}] (10,-2);
            \draw (2,-2) to [short, -*] (12.5,-2);
            \draw (12.5,0) to [open, v^>={\large $V_{out}(s)$}] (12.5,-2);
        \end{circuitikz}}
    \end{center}
    We can algebraically combine the two voltage sources together, the two series impedances together, and the two parallel impedances together:
    \begin{center}
        \begin{circuitikz}[american voltages,raised voltages]
            \draw (0,0) to [V, l_={$\dfrac{15}{s}$}] (0,-2); 
            \draw (0,0) to [european resistor, l={$Z_1=1+1/s$}] (3,0);
            \draw (3,0) to [european resistor, l={$Z_2=\dfrac{0.5}{s+1}$}] (3,-2);
            \draw (3,0) to [short, -*] (6.5,0);
            \draw (0,-2) to [short, -*] (6.5,-2);
            \draw (6.5,0) to [open, v^>={\large $V_{out}(s)$}] (6.5,-2);
        \end{circuitikz}
    \end{center}
    From here, the $s$-domain output response can be calculated using the voltage divider:
    \begin{align*}
        V_{out}(s) = \frac{15}{s}\cdot\left[\frac{Z_2}{Z_1+Z_2}\right] = \frac{7.5}{s^2+2.5s+1} = \frac{5}{s+0.5} - \frac{5}{s+2}.
    \end{align*}
    Lastly, taking the solution to the time domain, we get
    \begin{align*}
        v_{out}(t) = 5e^{-0.5t}u(t) - 5e^{-2t}u(t) \,[V].
    \end{align*}
\end{solution}

\begin{example}
    Given the plot of input excitation $v_s(t)$, find the conditional transfer function $H_C(s)=V_{out}(s)/V_s(s)$ of the circuit below for time $t\geq 0$. Then find $v_{out}(t)$ for $t\geq 0$. 
    Note that an ellipsis indicates a continuing pattern on the corresponding end. \\ \\
    \begin{tikzpicture}
        [declare function={
            func(\x)= (\x < 0) * (1) + and(\x >= 0, \x < 1) * (0) + and(\x >= 1, \x <= 2) * (1) + (\x > 2) * (0); }]
        \begin{axis}[
            axis x line=center, axis y line=center,
            ymin=0, ymax=2, ytick={0,1}, ylabel={$v_s(t)$},
            xmin=-2, xmax=3, xtick={0,...,3}, xlabel={$t$ [s]},
            domain=-1.3:3,samples=200,
            width=7cm, height=4cm]
        \addplot [blue,thick]{func(x)};
        \addplot [blue!50,only marks,mark options={scale=0.5}] (-1.5,1);
        \addplot [blue!50,only marks,mark options={scale=0.5}] (-1.65,1);
        \addplot [blue!50,only marks,mark options={scale=0.5}] (-1.8,1);
        \end{axis}
    \end{tikzpicture} \\ \\
    \resizebox{0.85\textwidth}{!}{\begin{circuitikz}[american voltages,raised voltages]
        \draw (0,0) to [V, l_={\large $v_s(t)$}] (0,-2); 
        \draw (0,0) to [L={$1\,[H]$}] (3,0);
        \draw (3,0) to [R={$1\,[\Omega]$}] (6,0);
        \draw (6,0) to [short, -*] (9,0);
        \draw (0,-2) to [short, -*] (9,-2);
        \draw (3,0) to [R={$1\,[\Omega]$}] (3,-2);
        \draw (6,0) to [C={$1\,[F]$}] (6,-2);
        \draw (9,0) to [open, v^>={\large $v_{out}(t)$}] (9,-2);
    \end{circuitikz}}
\end{example}
\begin{solution}
    First, find the initial conditions at time $t=0^-$. Since $v_s(t)=1$ for $t<0$, we can assume the circuit is at steady-state at $t=0^-$. 
    The steady-state circuit for $t < 0$ is given by 
    \begin{center}
        \resizebox{0.5\textwidth}{!}{\begin{circuitikz}[american voltages,raised voltages]
            \draw (0,0) to [V, l_={$1\,[V]$}] (0,-2); 
            \draw (0,0) to [short, f={$i_L(0^-)$}] (3,0);
            \draw (3,0) to [R={$1\,[\Omega]$}, -*] (6,0);
            \draw (0,-2) to [short, -*] (6,-2);
            \draw (3,0) to [R={$1\,[\Omega]$}] (3,-2);
            \draw (6,0) to [open, v^>={\large $v_c(0^-)$}] (6,-2);
        \end{circuitikz}}
    \end{center}
    Here we see that $i_L(0^-) = 1\,[V]/1\,[\Omega] = 1\,[A]$ and $v_c(0^-) = 1\,[V]$. 
    Using these initial conditions, we can set up an $s$-domain equivalent of the circuit for $t\geq 0$.
    \begin{center}
        \resizebox{0.7\textwidth}{!}{\begin{circuitikz}[american voltages,raised voltages]
            \draw (0,0) to [V, l_={\large $V_s(s)$}] (0,-3); 
            \draw (0,0) to [L={$s$}] (1.5,0);
            \draw (3,0) to [V, l_={$1$}] (1.5,0);
            \draw (3,0) to [R={$1$}] (6,0);
            \draw (6,0) to [short, -*] (9,0);
            \draw (0,-3) to [short, -*] (9,-3);
            \draw (3,0) to [R={$1$}] (3,-3);
            \draw (6,0) to [C={$1/s$}] (6,-1.5);
            \draw (6,-1.5) to [V, l^={$1/s$}] (6,-3);
            \draw (9,0) to [open, v^>={\large $V_{out}(s)$}] (9,-3);
            \draw[-latex, blue] (1.25,-1) arc(110:-110:6mm) node[blue, midway, left, font=\large] {$I_1$};
            \draw[-latex, blue] (4.25,-1) arc(110:-110:6mm) node[blue, midway, left, font=\large] {$I_2$};
        \end{circuitikz}}
    \end{center}
    We can use the mesh current method to solve the circuit. 
    \begin{align*}
        -V_s(s) + sI_1 - 1 + 1\cdot(I_1 - I_2) &= 0 \Longrightarrow (s+1)I_1 - I_2 = V_s(s) + 1 \\
        1\cdot(I_2 - I_1) + I_2 (1 + 1/s) + 1/s &= 0 \Longrightarrow -sI_1 + (2s + 1)I_2 = -1
    \end{align*}
    In matrix form, the mesh current equations can be written as 
    \begin{align*}
        \begin{bmatrix}
            s+1 & -1 \\
            -s & 2s+1
        \end{bmatrix}
        \begin{bmatrix}
            I_1 \\
            I_2
        \end{bmatrix} = 
        \begin{bmatrix}
            V_s(s)+1 \\
            -1
        \end{bmatrix}.
    \end{align*}
    From here, we can use Cramer's rule to solve for $I_2$ which can then be used to solve for $V_{out}(s)$.
    \begin{align*}
        \Delta &= 
        \begin{vmatrix}
            s+1 & -1 \\
            -s & 2s+1
        \end{vmatrix} 
        = 2s^2 + 2s + 1 \\[1pt]
        \Delta_2 &= 
        \begin{vmatrix}
            s+1 & V_s(s)+1 \\
            -s & -1
        \end{vmatrix} 
        = sV_s(s) - 1 \\[1pt]
        I_2 &= \frac{\Delta_2}{\Delta} = \frac{sV_s(s)-1}{2s^2+2s+1}
    \end{align*}
    The conditional transfer function can now be solved.
    \begin{align*}
        V_{out}(s) &= I_2\cdot \frac{1}{s} + \frac{1}{s} = \frac{sV_s(s)-1}{s(2s^2+2s+1)} + \frac{1}{s} = \frac{sV_s(s)+2s^2+2s}{s(2s^2+2s+1)} \\
        \Longrightarrow H_C(s) &= [V_{out}(s)\big|_{V_s(s)=1} = \frac{2s+3}{2s^2+2s+1} = \frac{s+1.5}{s^2+s+0.5}
    \end{align*}
    With the Laplace transform of input $v_s(t)$ for $t\geq 0$ given by 
    \begin{align*}
        V_s(s)=\mathcal{L}[v_s(t)]=\mathcal{L}[u(t-1)-u(t-2)]=\frac{1}{s}(e^{-s}-e^{-2s}),
    \end{align*}
    the equation for $V_{out}(s)$ is given by
    \begin{align*}
        V_{out}(s) = H_C(s)V_s(s) &= (e^{-s}-e^{-2s})\left[\frac{s+1.5}{s(s^2+s+0.5)}\right] \\ 
        &= (e^{-s}-e^{-2s})\left[\frac{3}{s} + \frac{-1.5-j0.5}{s+0.5+j0.5} + \frac{-1.5+j0.5}{s+0.5-j0.5}\right] \\
        &= (e^{-s}-e^{-2s})\underbrace{\left[\frac{3}{s} + \frac{1.581e^{-j2.82}}{s+0.5+j0.5} + \frac{1.581e^{+j2.82}}{s+0.5-j0.5}\right]}_{G(s)}.
    \end{align*}
    The inverse Laplace transform of $G(s)$ is 
    \begin{align*}
        g(t) = 3u(t) + 3.162e^{-0.5t}\cos(0.5t+2.82\text{ [rad]})u(t)
    \end{align*}
    The output voltage $v_{out}(t)$ for $t\geq 0$ (with phase expressed in radians) is then 
    \begin{align*}
        v_{out}(t) &= \mathcal{L}^{-1}[(e^{-s}-e^{-2s})G(s)] = g(t-1) - g(t-2) \\ 
        &= 3u(t-1) + 3.162e^{-0.5(t-1)}\cos(0.5(t-1)+2.82)u(t-1) \\ 
        &\qquad - 3u(t-2) - 3.162e^{-0.5(t-2)}\cos(0.5(t-2)+2.82)u(t-2).
    \end{align*}
\end{solution}

\pagebreak
\section{$s$--Domain Block Diagrams}
Similar to the interconnection of LTI systems in the time domain, LTI systems can be combined in the $s$-domain. When LTI systems are connected in series, the interconnection has 
the following equivalence:
\begin{center}
    \begin{tikzpicture}
        \node [] (input) at (-3,0){$X(s)$};
        \node [] (output) at (+6,0){$Y(s)$};
        \node [draw,
            fill=yellow!50, 
            minimum width=2cm, 
            minimum height=1.2cm
        ] (sys1) at (0,0){$H_1(s)$};
        \node [draw,
            fill=red!30, 
            minimum width=2cm, 
            minimum height=1.2cm
        ] (sys2) at (3,0){$H_2(s)$};
        \draw [-latex, line width=1.5pt] (input) -- (sys1);
        \draw [-latex, line width=1.5pt] (sys1) -- (sys2);
        \draw [-latex, line width=1.5pt] (sys2) -- (output);
    \end{tikzpicture} \\[0.25cm] 
    \begin{tikzpicture}
        \node at (0,0) [double arrow, draw=black, top color=red, bottom color=blue,
      minimum width = 15pt, double arrow head extend=5pt,
      minimum height=12mm,
      rotate=90] {};
    \end{tikzpicture} \\[0.25cm]
    \begin{tikzpicture}
        \node [] (input) at (-3,0){$X(s)$};
        \node [] (output) at (+6,0){$Y(s)$};
        \node [draw,
            fill=blue!30, 
            minimum width=2cm, 
            minimum height=1.2cm
        ] (system) at (1.5,0){$H_1(s)H_2(s)$};
        \draw [-latex, line width=1.5pt] (input) -- (system);
        \draw [-latex, line width=1.5pt] (system) -- (output);
    \end{tikzpicture} \\
\end{center}
When LTI systems are connected in parallel, the interconnection has the following equivalence:
\begin{center}
    \begin{tikzpicture}
        \node [] (input) at (-3,0){$X(s)$};
        \node [] (output) at (+6,0){$Y(s)$};
        \node [draw,
            fill=yellow!50, 
            minimum width=2cm, 
            minimum height=1.2cm
        ] (sys1) at (1.5,1){$H_1(s)$};
        \node [draw,
            fill=orange!50, 
            minimum width=2cm, 
            minimum height=1.2cm
        ] (sys2) at (1.5,-1){$H_2(s)$};
        \node[draw,
            circle,
            minimum size=0.6cm,
            fill=red!30
        ] (sum) at (4,0){\large +};
        \draw [-, line width=1.5pt] (input) -- (-1,0);
        \draw [-, line width=1.5pt] (-1,-1) -- (-1,1);
        \draw [-latex, line width=1.5pt] (-1,1) -- (sys1);
        \draw [-latex, line width=1.5pt] (-1,-1) -- (sys2);
        \draw [-, line width=1.5pt] (sys1) -- (4,1);
        \draw [-, line width=1.5pt] (sys2) -- (4,-1);
        \draw [-latex, line width=1.5pt] (4,1) -- (sum);
        \draw [-latex, line width=1.5pt] (4,-1) -- (sum);
        \draw [-latex, line width=1.5pt] (sum) -- (output);
    \end{tikzpicture} \\[0.25cm]
    \begin{tikzpicture}
        \node at (0,0) [double arrow, draw=black, top color=red, bottom color=blue,
      minimum width = 15pt, double arrow head extend=5pt,
      minimum height=12mm,
      rotate=90] {};
    \end{tikzpicture} \\[0.25cm]
    \begin{tikzpicture}
        \node [] (input) at (-3,0){$X(s)$};
        \node [] (output) at (+6,0){$Y(s)$};
        \node [draw,
            fill=blue!30, 
            minimum width=2cm, 
            minimum height=1.2cm
        ] (system) at (1.5,0){$H_1(s)+H_2(s)$};
        \draw [-latex, line width=1.5pt] (input) -- (system);
        \draw [-latex, line width=1.5pt] (system) -- (output);
    \end{tikzpicture}
\end{center}
A minus sign adjacent to an arrowhead at a summation node indicates that the corresponding addend signal is multiplied by $-1$ before being summed. 
When an interconnection of LTI systems implement \emph{negative feedback}, it has the following equivalence:
\begin{center}
    \begin{tikzpicture}
        \node [] (input) at (-3,1){$X(s)$};
        \node [] (output) at (+6,1){$Y(s)$};
        \node [draw,
            fill=green!50, 
            minimum width=2cm, 
            minimum height=1.2cm
        ] (sys1) at (1.5,1){$G(s)$};
        \node [draw,
            fill=yellow!50, 
            minimum width=2cm, 
            minimum height=1.2cm
        ] (sys2) at (1.5,-1){$H(s)$};
        \node[draw,
            circle,
            minimum size=0.6cm,
            fill=red!30
        ] (sum) at (-1,1){\large +};
        \node [] at (-0.5,0.35){\Huge -};
        \node [] at (+5,-0.75){\shortstack{Feedback \\ loop}};
        \draw [-latex, line width=1.5pt] (input) -- (sum);
        \draw [-latex, line width=1.5pt] (-1,-1) -- (sum);
        \draw [-latex, line width=1.5pt] (sum) -- (sys1);
        \draw [-, line width=1.5pt] (4,1) -- (4,-1);
        \draw [-, line width=1.5pt] (sys2) -- (-1,-1);
        \draw [-, line width=1.5pt] (sys1) -- (4,1);
        \draw [-latex, line width=1.5pt] (4,-1) -- (sys2);
        \draw [-latex, line width=1.5pt] (sys1) -- (output);
    \end{tikzpicture} \\[0.25cm]
    \begin{tikzpicture}
        \node at (0,0) [double arrow, draw=black, top color=red, bottom color=blue,
      minimum width = 15pt, double arrow head extend=5pt,
      minimum height=12mm,
      rotate=90] {};
    \end{tikzpicture} \\[0.25cm]
    \begin{tikzpicture}
        \node [] (input) at (-3,0){$X(s)$};
        \node [] (output) at (+6,0){$Y(s)$};
        \node [draw,
            fill=blue!30, 
            minimum width=2cm, 
            minimum height=1.2cm
        ] (system) at (1.5,0){$\dfrac{G(s)}{1+G(s)H(s)}$};
        \draw [-latex, line width=1.5pt] (input) -- (system);
        \draw [-latex, line width=1.5pt] (system) -- (output);
    \end{tikzpicture}
\end{center}
\emph{Feedback} takes a sample of the output signal and feeds it back to the input signal. A system without feedback is called an \emph{open-loop system}, whereas a system that utilizes 
feedback to form a \emph{feedback loop} is called a \emph{closed-loop system}.
\\ \\
A closed-loop system has \emph{positive feedback} if the strength of the input signal is increased as a result; the closed-loop system has \emph{negative feedback} if the strength 
of the input signal is decreased. 
\begin{tcolorbox}[width=\textwidth,colback={white}, sharp corners]
    An interconnection of LTI systems with feedback loops can be simplified to an equivalent single-transfer-function model.
    \begin{enumerate}
        \item Condense any systems in parallel or series first.
        \item Count the number of summation nodes left. The number of equations needed is the number of summation nodes plus one, with the one representing the output.
        \item Define an intermediary signal $E_n(s)$ after each summation node $n$.
        \item Starting from the output signal $Y(s)$, write an equation for the output. Then working backwards from $Y(s)$, write equations for each summation node.
        \item Begin substituting and rearranging equations until a single equation in terms of only $X(s)$ and $Y(s)$ is left.
    \end{enumerate}
\end{tcolorbox}

\begin{example}
    Find an equivalent transfer function for the following interconnection of LTI systems. \\ \\
    \resizebox{0.95\textwidth}{!}{
    \begin{tikzpicture}
        \node [] (input) at (-3,0){\LARGE $X(s)$};
        \node [] (output) at (+12,0){\LARGE $Y(s)$};
        \node [draw,
            fill=yellow!50, 
            minimum width=2cm, 
            minimum height=1.2cm
        ] (sys1) at (3,0){\LARGE $\dfrac{10}{s+1}$};
        \node [draw,
            fill=orange!50, 
            minimum width=2cm, 
            minimum height=1.2cm
        ] (sys2) at (8,0){\LARGE $\dfrac{1}{s}$};
        \node [draw,
            fill=green!30, 
            minimum width=2cm, 
            minimum height=1.2cm
        ] (sys3) at (6,-2){\LARGE $2$};
        \node[draw,
            circle,
            minimum size=0.6cm,
            fill=red!30
        ] (sum1) at (0,0){\Huge +};
        \node [] at (0.65,-0.75){\Huge -};
        \node[draw,
            circle,
            minimum size=0.6cm,
            fill=red!30
        ] (sum2) at (6,-4){\Huge +};
        \node [] at (6.75,-4.65){\Huge -};
        \draw [-latex, line width=1.5pt] (input) -- (sum1);
        \draw [-latex, line width=1.5pt] (sum1) -- (sys1);
        \draw [-latex, line width=1.5pt] (sys1) -- (sys2);
        \draw [-latex, line width=1.5pt] (sys2) -- (output);
        \draw [-latex, line width=1.5pt] (6,0) -- (sys3);
        \draw [-latex, line width=1.5pt] (sys3) -- (sum2);
        \draw [-, line width=1.5pt] (10,0) -- (10,-4);
        \draw [-latex, line width=1.5pt] (10,-4) -- (sum2);
        \draw [-, line width=1.5pt] (sum2) -- (0,-4);
        \draw [-latex, line width=1.5pt] (0,-4) -- (sum1);
    \end{tikzpicture}}
\end{example}
\begin{solution}
    There are two summation nodes. We will need to define $E_1(s), E_2(s)$ and write three equations. \\ \\
    \resizebox{0.65\textwidth}{!}{
    \begin{tikzpicture}
        \node [] (input) at (-3,0){\LARGE $X(s)$};
        \node [] (output) at (+12,0){\LARGE $Y(s)$};
        \node [] at (2,0.5){\LARGE $E_1(s)$};
        \node [] at (4,-3.5){\LARGE $E_2(s)$};
        \node [draw,
            fill=yellow!50, 
            minimum width=2cm, 
            minimum height=1.2cm
        ] (sys1) at (4,0){\LARGE $\dfrac{10}{s+1}$};
        \node [draw,
            fill=orange!50, 
            minimum width=2cm, 
            minimum height=1.2cm
        ] (sys2) at (8,0){\LARGE $\dfrac{1}{s}$};
        \node [draw,
            fill=green!30, 
            minimum width=2cm, 
            minimum height=1.2cm
        ] (sys3) at (6,-2){\LARGE $2$};
        \node[draw,
            circle,
            minimum size=0.6cm,
            fill=red!30
        ] (sum1) at (0,0){\Huge +};
        \node [] at (0.65,-0.75){\Huge -};
        \node[draw,
            circle,
            minimum size=0.6cm,
            fill=red!30
        ] (sum2) at (6,-4){\Huge +};
        \node [] at (6.75,-4.65){\Huge -};
        \draw [-latex, line width=1.5pt] (input) -- (sum1);
        \draw [-latex, line width=1.5pt] (sum1) -- (sys1);
        \draw [-latex, line width=1.5pt] (sys1) -- (sys2);
        \draw [-latex, line width=1.5pt] (sys2) -- (output);
        \draw [-latex, line width=1.5pt] (6,0) -- (sys3);
        \draw [-latex, line width=1.5pt] (sys3) -- (sum2);
        \draw [-, line width=1.5pt] (10,0) -- (10,-4);
        \draw [-latex, line width=1.5pt] (10,-4) -- (sum2);
        \draw [-, line width=1.5pt] (sum2) -- (0,-4);
        \draw [-latex, line width=1.5pt] (0,-4) -- (sum1);
    \end{tikzpicture}} \\ \\
    We can first write an equation for $Y(s)$, following the feedforward path at the top.
    \begin{align*}
        Y(s) &= E_1(s) \cdot \frac{10}{s+1} \cdot \frac{1}{s} \\ 
        &= E_1(s) \left[\frac{10}{s(s+1)}\right]
    \end{align*}
    Now we write an equation for $E_2(s)$ using the rightmost branch.
    \begin{align*}
        E_2(s) &= E_1(s) \cdot \frac{10}{s+1} \cdot 2 - Y(s) \\
        &= E_1(s) \cdot \frac{10}{s+1} \cdot 2 - E_1(s) \left[\frac{10}{s(s+1)}\right] \\ 
        &= E_1(s) \cdot \frac{10}{s+1} \left[2-\frac{1}{s}\right] \\
        &= E_1(s) \left[\frac{10(2s-1)}{s(s+1)}\right]
    \end{align*}
    Lastly, we write an equation for $E_1(s)$ using the leftmost branch.
    \begin{align*}
        E_1(s) &= X(s) - E_2(s) = X(s) - E_1(s) \left[\frac{10(2s-1)}{s(s+1)}\right] \\
        \Longrightarrow X(s) &= E_1(s)\left[1 + \frac{10(2s-1)}{s(s+1)}\right]
    \end{align*}
    We can rearrange the equation for $Y(s)$ and substitute into $X(s)$.
    \begin{align*}
        E_1(s) &= Y(s)\left[\frac{s(s+1)}{10}\right] \\
        \Longrightarrow X(s) &= E_1(s)\left[1 + \frac{10(2s-1)}{s(s+1)}\right] \\
        &= Y(s)\left[\frac{s(s+1)}{10}\right]\left[1 + \frac{10(2s-1)}{s(s+1)}\right] \\
        &= Y(s)\left[\frac{s(s+1)}{10} + (2s-1)\right] \\
        &= Y(s)\left[\frac{s^2+21s-10}{10}\right]
    \end{align*}
    Then the equivalent transfer function $H(s)$ is  
    \begin{align*}
        H(s) = \frac{Y(s)}{X(s)} = \frac{10}{s^2+21s-10}.
    \end{align*}
\end{solution}
\end{document}
