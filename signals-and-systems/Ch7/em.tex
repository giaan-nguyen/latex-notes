\documentclass[border=5pt]{standalone}
\usepackage{contour}
\usepackage{tikz}
\usepackage{siunitx}
    \usetikzlibrary{
        arrows,
        backgrounds,
        calc,
        fit,
        positioning,
        shapes,
    }
\usepackage{pgf-spectra}
\begin{document}
\begin{tikzpicture}[
    raylabel/.style={font=\scriptsize}
]

%    % make the horizontal shading and set the UV and IR colors -->
%    \pgfspectraStyle[gamma=.6]     % uncomment to change the gamma
    \wlcolor{380}\colorlet{UV}{wlcolor}
    \wlcolor{780}\colorlet{IR}{wlcolor}
    \pgfspectraplotshade[logarithmic,shade opacity=0.3]{visiblelight}
%    \pgfspectraStyleReset          % uncomment to reset the style

    \def\minexponent{-6}
    \def\maxexponent{6}
    \def\spectrumheight{9em}

    \pgfmathtruncatemacro{\nextminexponent}{\minexponent + 1}

    % Main foreach loop, drawing the wavelengths as powers of 10 in an alternating
    % fashion: even on top, odd at bottom. Then connects them with help lines
    \foreach [
        count=\i,
        remember=\exponent as \previousexponent,
        evaluate=\i as \currentposition using int(\i/2),
    ] \exponent in {\minexponent,\nextminexponent,...,\maxexponent}{
        \ifodd\exponent
            \def\height{0}
        \else
            \def\height{\spectrumheight}
        \fi

        \node [anchor=base] (WAVELENGTH_\exponent) at (\exponent, \height)
            {\contour{white}{\num[print-unity-mantissa = false]{e\exponent}}};

        \ifnum\i > 1
            \ifodd\i
                \node (LABEL_\currentposition)
                    at ($(WAVELENGTH_\exponent)!0.5!(WAVELENGTH_\previousexponent)$)
                    % This is left as a node as opposed to coordinate:
                    % fill it out with '\currentposition' for debugging
                    {}
                ;
            \else
                % Do not draw connection at exponent 1:
                % (`\pgfmathparse` stores result (0 or 1) in macro `\pgfmathresult`)
                \pgfmathparse{\exponent != 1}
                \ifnum\pgfmathresult = 1
                    \draw[help lines]
                        (WAVELENGTH_\previousexponent) --(WAVELENGTH_\exponent)
                        % This is left as a node as opposed to coordinate:
                        % fill it out with '\currentposition' for debugging
                        node [midway] (CONNECTION_\currentposition) {}
                        coordinate [at start] (CONNECTION_\currentposition_START)
                        coordinate [at end] (CONNECTION_\currentposition_END);
                \fi
            \fi
        \fi
    }

    % Create an arrow shape that fits around all relevant nodes, but do not draw it.
    % Draw it manually later to leave out the 'bottom' of the arrow.
    % We still need this invisible arrow for lining up of coordinates
    \node [
        single arrow,
        single arrow head extend=0pt,
        % Inner angle of arrow tip
        single arrow tip angle=150,
        fit={
            ([xshift=-3em]CONNECTION_1_START)
            (CONNECTION_1_END)
            (CONNECTION_\maxexponent_START)
            ([xshift=5em]CONNECTION_\maxexponent_END)
        },
        inner sep=0pt
    ]
    (ARROW) {};

    % Only works because exponent 1 is between -1 and 3
    \node [align=center] (THERM) at ([yshift=3em]WAVELENGTH_1|-ARROW.after tail)
        {thermal\\radiation};
    \draw (THERM) -| ([yshift=-1.5em]WAVELENGTH_-1|-THERM);
    \draw (THERM) -| ([yshift=-1.5em]WAVELENGTH_3|-THERM);

    % On background layer so already drawn arrow and scale lines cover it up nicely
    \begin{scope}[on background layer]
        \node [
            inner sep=0pt,
            outer sep=0pt,
            fit={
                ([xshift=-1.9em]WAVELENGTH_0|-ARROW.after tail)
                ([xshift=-3em]WAVELENGTH_1|-ARROW.before tail)
            },
            shading=visiblelight,
        ] (SMALL_VISIBLE_LIGHT) {};
        \shade [
            left color=white,
            right color=UV!25,
            middle color=UV!5,
            outer sep=0pt
        ] (CONNECTION_3_START)
            -- (CONNECTION_3_END)
            -- ([xshift=\pgflinewidth]SMALL_VISIBLE_LIGHT.south west)
            -- ([xshift=\pgflinewidth]SMALL_VISIBLE_LIGHT.north west)
            -- cycle
        ;
        \shade [
            left color=IR!25,
            right color=white,
            middle color=IR!5,
            outer sep=0pt,
        ] (CONNECTION_5_START)
            -- (CONNECTION_5_END)
            -- ([xshift=-\pgflinewidth]SMALL_VISIBLE_LIGHT.south east)
            -- ([xshift=-\pgflinewidth]SMALL_VISIBLE_LIGHT.north east)
            -- cycle
        ;
    \end{scope}

    % Some labels can be drawn automatically at the designated label coordinates:
    \foreach [count=\i] \label in {
        {Gamma\\rays},
        {X-rays},
        {},%Skip this one
        {infrared}
    }{
        \node[raylabel, align=center] at (LABEL_\i) {\label};
    }

    % These do not fit the loop and are drawn manually:
    \node [raylabel, align=right, anchor=north] at
        ([yshift=-1em,xshift=-2.5pt]$(WAVELENGTH_-2)!0.45!(WAVELENGTH_0)$)
            {Ultra-\\violet};
    \node [raylabel, fill=white] at (CONNECTION_6) {radio waves};
    \node [raylabel, left=0.1em of CONNECTION_1, align=right] {cosmic\\rays};
    \node [
        draw,
        fill=black!20,
        below=4em of SMALL_VISIBLE_LIGHT,
        align=center,
        label=above:{\textbf{Visible Spectrum}}
    ] (FULL_VISIBLE_LIGHT) {%
        \pgfspectra[
            width=13em,height=3em,axis,axis unit=micron,axis step=100,
            axis ticks=4,axis unit precision=2,axis color=black!20,
            axis font=\normalsize,axis font color=black,
        ]\\%
            \footnotesize
        blue \hspace{0.5em}
        green
        yellow \hspace{1.5em}
        red
    };

    % Draw 'magnifying' trapeze, on background so it is covered by scale labels
    \begin{scope}[on background layer]
        \filldraw [help lines, fill=black!10] (FULL_VISIBLE_LIGHT.north east)
            -- (SMALL_VISIBLE_LIGHT.south east)
            -- (SMALL_VISIBLE_LIGHT.south west)
            -- (FULL_VISIBLE_LIGHT.north west)
            -- cycle
        ;
    \end{scope}

    % Draw around arrow manually, leaving its tail open
    \draw [draw, thick] (ARROW.after tail)
        -- (ARROW.before head)
        -- (ARROW.before tip)
        -- (ARROW.tip)
        -- (ARROW.after tip)
        -- (ARROW.after head)
        -- (ARROW.before tail)
    ;
\end{tikzpicture}
\end{document}