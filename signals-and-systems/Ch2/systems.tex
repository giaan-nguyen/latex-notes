\documentclass{report}
% PACKAGES
\usepackage{adjustbox}
\usepackage{amsmath}
\usepackage{amssymb}
\usepackage{bodegraph}
\usepackage{bbm}
\usepackage{circledsteps}
\usepackage{circuitikz}
\usepackage{enumerate}
\usepackage{mathtools}
\usepackage{nicematrix}
\usepackage{pgfplots}
\usepackage{polynom}
\usepackage{qtree}
\usepackage{rotating}
\usepackage[usestackEOL]{stackengine}
\usepackage[subpreambles=true]{standalone}
\usepackage{steinmetz}
\usepackage{subcaption}
\usepackage{tabularray}
\usepackage{tcolorbox}
\usepackage{tikz}
\usepackage{xcolor}

\usepackage[colorlinks=true,linkcolor=blue,urlcolor=black,bookmarksopen=true]{hyperref}
\usepackage{bookmark}

\usetikzlibrary{shapes.arrows}
\usetikzlibrary{shapes.misc}
\usetikzlibrary{backgrounds}
\tikzset{cross/.style={cross out, draw=black, minimum size=2*(#1-\pgflinewidth), inner sep=0pt, outer sep=0pt},
%default radius will be 1pt. 
cross/.default={1pt}}

\renewcommand{\Re}{\operatorname{Re}}
\renewcommand{\Im}{\operatorname{Im}}

\usepackage{pifont}
\newcommand{\cmark}{\text{\ding{51}}}
\newcommand{\xmark}{\text{\ding{55}}}

\newcommand{\circconv}[1]{\text{ \small\Circled{#1} }}

\newcommand{\tikzmark}[3][]{\tikz[remember picture,baseline] \node [anchor=base,#1](#2) {$#3$};}

\pgfplotsset{compat=1.18}
\pgfplotsset{
    dirac/.style={
        mark=triangle*,
        mark options={scale=1.5},
        ycomb,
        scatter,
        visualization depends on={y/abs(y)-1 \as \sign},
        scatter/@pre marker code/.code={\scope[rotate=90*\sign,yshift=-2pt]}
    }
}

\usepackage[letterpaper, portrait, margin=1.25in]{geometry}
\usepackage[font=bf]{caption}
\hbadness = 10000
\hfuzz=2pt

\newtheorem{theorem}{Theorem}[chapter]

\tikzset{every tree node/.style={anchor=north,align=center}}
\usetikzlibrary{decorations.markings}
\usetikzlibrary{arrows}
\definecolor{darkgreen}{rgb}{0.133,0.545,0.133}
\tikzstyle{n}= [circle, fill=blue, minimum size=4pt,inner sep=0pt, outer sep=0pt]

\patchcmd{\thebibliography}{\chapter*}{\section*}{}{}
\setcounter{secnumdepth}{5}

%%%%%%%%%%% EXAMPLE ENVIRONMENT %%%%%%%%%%
\usepackage{calc}
\usepackage{tabto}
\usepackage[framemethod=tikz]{mdframed} % for colored backgrounds
\newcommand{\halmos}{} % makes a box at the end

\newlength{\framedinnerleftmargin}
\newlength{\framedinnertopmargin}
\newlength{\framedreversedinnerleftmargin}
\setlength{\framedinnerleftmargin}{\widthof{Theoreme 10.10.10}+2em}
\setlength{\framedreversedinnerleftmargin}{\widthof{Theoreme 10.10.10}+1em}
\setlength{\framedinnertopmargin}{1em}
 
% first argument: label in upper left corner,
% second argument: background color
\newenvironment{boxedtext}[2]{\begin{mdframed}[%
hidealllines=true,%
backgroundcolor=#2,%
innertopmargin=\framedinnertopmargin,%
innerleftmargin=\framedinnerleftmargin,%
innerrightmargin=1em%
]%
\tabto{-\framedreversedinnerleftmargin}\textbf{#1}\tabto*{0em}%
}% begin code
{\hskip 0pt\\\hspace*{\fill}\halmos{}\end{mdframed}\vspace{1em}} % end code
 
\newenvironment{summary}[0]{\begin{center}\begin{minipage}[c]{\summarywidth}\begin{spacing}{0.9}\footnotesize} % begin code
{\end{spacing}\end{minipage}\end{center}} % end code

\newcounter{example}
 
% optional! if you want it to start at zero
% with every new chapter/section/etc.
\numberwithin{example}{section}
 
\newenvironment{example}[0]
{\refstepcounter{example}\vspace{1em plus 1em}\begin{boxedtext}{Example \theexample.}{blue!7}}%\setlength{\parskip}{0em}}
{\end{boxedtext}\vspace{-1em plus 1em}}
 
\newenvironment{example*}[0]
{\vspace{1em plus 1em}\begin{boxedtext}{Example.}{blue!7}}
{\end{boxedtext}\vspace{-1em plus 1em}}

\newmdenv[
  topline=false,
  bottomline=false,
  rightline=false,
  skipabove=\topsep,
  skipbelow=\topsep,
  linecolor=purple,
  frametitle={\noindent\textcolor{purple}{\textbf{SOLUTION }}},
  endinnercode={$\hfill\textcolor{purple}{\blacksquare}$}
]{solution}

%%%%%%%%%%%%%% COLOR BOXED %%%%%%%%%%%%%%%
% Syntax: \colorboxed[<color model>]{<color specification>}{<math formula>}
\newcommand*{\colorboxed}{}
\def\colorboxed#1#{%
  \colorboxedAux{#1}%
}
\newcommand*{\colorboxedAux}[3]{%
  % #1: optional argument for color model
  % #2: color specification
  % #3: formula
  \begingroup
    \colorlet{cb@saved}{.}%
    \color#1{#2}%
    \boxed{%
      \color{cb@saved}%
      #3%
    }%
  \endgroup
}
%%%%%%%%%%%%%%%%%%%%%%%%%%%%%%%%%%%%%%%%%

\begin{document}
\setcounter{chapter}{1}
\chapter{Systems}
From the previous chapter, we introduce the idea of modifying signals by an affine transformation on the independent variable. 
In fact, this was the very first introduction to the concept of \emph{systems}. A system essentially transforms an \emph{input signal} 
into an \emph{output signal}. Furthermore, a system can be modeled as an operator that maps an input function to an output function.
\\ \\
Given an input signal $x(t)$ and an output signal $y(t)$, a system can be depicted using a block diagram:

\begin{center}
\begin{tikzpicture}
    \node [] (input) at (-3,0){$x(t)$};
    \node [] (output) at (+3,0){$y(t)$};
    \node [draw,
        fill=yellow!50, 
        minimum width=2cm, 
        minimum height=1.2cm
    ] (system) at (0,0){$system$};
    \draw [-latex, line width=1.5pt] (input) -- (system);
    \draw [-latex, line width=1.5pt] (system) -- (output);
\end{tikzpicture}
\end{center}

There are many practical examples of systems in the real world, including circuits, signal processing systems, 
communications systems, and feedback control systems, those of which are depicted in Figures \ref{RC_circuit} -- \ref{feedback}.

\begin{figure}[!hbt]
    \caption{A simple RC circuit.}
    \label{RC_circuit}
    \centering
    \resizebox{0.4\textwidth}{!}{
        \begin{circuitikz}[american voltages,raised voltages]
            \draw 
                (0,3) to [V, l_={\Large $v_s(t)$}] (0,0)
                to [short, -*] (6,0)
                (0,3) to [R, v_={\Large $i(t)$}, voltage=straight, l^={\Large $R$}] (4,3)
                to [C, l_={\Large $C$}] (4,0)
                (4,3) to [short, -*] (6,3) 
                to [open, v^>={\Large $v_c(t)$}] (6,0);
        \end{circuitikz}}
\end{figure}

\begin{figure}[!hbt]
    \caption{Signal processing systems.}
    \label{sig_proc_sys}
    \centering
    \resizebox{\textwidth}{!}{
        \begin{tikzpicture}
            \node [] (in) at (-8,0){\shortstack{Input \\ CT Signal}};
            \node [] (out) at (+8,0){\shortstack{Output \\ CT Signal}};
            \node [] at (-2.25,0.7){\shortstack{Pre-processed \\ DT Signal}};
            \node [] at (+2.25,0.7){\shortstack{Post-processed \\ DT Signal}};
            \node [draw,
                fill=orange!50, 
                minimum width=2cm, 
                minimum height=1.2cm
            ] (cd) at (-4.5,0){\shortstack{C/D \\ Converter}};
            \node [draw,
                fill=yellow!50, 
                minimum width=2cm, 
                minimum height=1.2cm
            ] (sys) at (0,0){DT System};
            \node [draw,
                fill=green!50, 
                minimum width=2cm, 
                minimum height=1.2cm
            ] (dc) at (4.5,0){\shortstack{D/C \\ Converter}};
            \draw [-latex, line width=1.5pt] (in) -- (cd);
            \draw [-latex, line width=1.5pt] (cd) -- (sys);
            \draw [-latex, line width=1.5pt] (sys) -- (dc);
            \draw [-latex, line width=1.5pt] (dc) -- (out);
        \end{tikzpicture}}
        \resizebox{\textwidth}{!}{
        \begin{tikzpicture}
            \node [] (in) at (-8,0){\shortstack{Input \\ DT Signal}};
            \node [] (out) at (+8,0){\shortstack{Output \\ DT Signal}};
            \node [] at (-2.25,0.7){\shortstack{Pre-processed \\ CT Signal}};
            \node [] at (+2.25,0.7){\shortstack{Post-processed \\ CT Signal}};
            \node [draw,
                fill=green!50, 
                minimum width=2cm, 
                minimum height=1.2cm
            ] (dc) at (-4.5,0){\shortstack{D/C \\ Converter}};
            \node [draw,
                fill=yellow!50, 
                minimum width=2cm, 
                minimum height=1.2cm
            ] (sys) at (0,0){CT System};
            \node [draw,
                fill=orange!50, 
                minimum width=2cm, 
                minimum height=1.2cm
            ] (cd) at (4.5,0){\shortstack{C/D \\ Converter}};
            \draw [-latex, line width=1.5pt] (in) -- (dc);
            \draw [-latex, line width=1.5pt] (dc) -- (sys);
            \draw [-latex, line width=1.5pt] (sys) -- (cd);
            \draw [-latex, line width=1.5pt] (cd) -- (out);
        \end{tikzpicture}}
\end{figure}

\begin{figure}[!hbt]
    \caption{Communications system.}
    \label{comms}
    \centering
    \resizebox{\textwidth}{!}{
        \begin{tikzpicture}
            \node [] (msg) at (-8,0){\shortstack{Message \\ Signal}};
            \node [] (est) at (+8,0){\shortstack{Estimate of \\ Message Signal}};
            \node [] at (-2.25,0.7){\shortstack{Transmitted \\ Signal}};
            \node [] at (+2.25,0.7){\shortstack{Received \\ Signal}};
            \node [draw,
                fill=blue!30, 
                minimum width=2cm, 
                minimum height=1.2cm
            ] (tx) at (-4.5,0){\shortstack{Transmitter \\ (Tx)}};
            \node [draw,
                fill=yellow!50, 
                minimum width=2cm, 
                minimum height=1.2cm
            ] (ch) at (0,0){Channel};
            \node [draw,
                fill=red!30, 
                minimum width=2cm, 
                minimum height=1.2cm
            ] (rx) at (4.5,0){\shortstack{Receiver \\ (Rx)}};
            \draw [-latex, line width=1.5pt] (msg) -- (tx);
            \draw [-latex, line width=1.5pt] (tx) -- (ch);
            \draw [-latex, line width=1.5pt] (ch) -- (rx);
            \draw [-latex, line width=1.5pt] (rx) -- (est);
        \end{tikzpicture}}
\end{figure}

\begin{figure}[!hbt]
    \caption{Feedback control system.}
    \label{feedback}
    \centering
    \resizebox{\textwidth}{!}{
        \begin{tikzpicture}
            \node [] (in) at (-8.5,0){\shortstack{Reference \\ Input}};
            \node [] (out) at (+7,0){\shortstack{Actual \\ Output}};
            \node [] at (-5.5,-0.65){\Huge -};
            \node [] at (-5,0.5){\shortstack{Error \\ Signal }};
            \node [] at (+3,-1.5){\shortstack{Feedback \\ Signal}};
            \node [] at (-3,-1.5){\shortstack{Measurement \\ Output}};
            \node[draw,
                circle,
                minimum size=0.6cm,
                fill=red!30
            ] (sum) at (-6,0){\LARGE +};
            \node [draw,
                fill=green!50, 
                minimum width=2cm, 
                minimum height=1.2cm
            ] (ctrl) at (-3,0){Controller};
            \node [draw,
                fill=orange!50, 
                minimum width=2cm, 
                minimum height=1.2cm
            ] (actr) at (0,0){Actuator};
            \node [draw,
                fill=yellow!50, 
                minimum width=2cm, 
                minimum height=1.2cm
            ] (proc) at (3,0){\shortstack{Process \\ (Plant)}};
            \node [draw,
                fill=blue!30, 
                minimum width=2cm, 
                minimum height=1.2cm
            ] (sens) at (0,-2){Sensor};
            \draw [-latex, line width=1.5pt] (in) -- (sum);
            \draw [-latex, line width=1.5pt] (sum) -- (ctrl);
            \draw [-latex, line width=1.5pt] (ctrl) -- (actr);
            \draw [-latex, line width=1.5pt] (actr) -- (proc);
            \draw [-latex, line width=1.5pt] (proc) -- (out);
            \draw [-, line width=1.5pt] (5.25,0) -- (5.25,-2);
            \draw [-latex, line width=1.5pt] (5.25,-2) -- (sens);
            \draw [-, line width=1.5pt] (sens) -- (-6,-2);
            \draw [-latex, line width=1.5pt] (-6,-2) -- (sum);
        \end{tikzpicture}}
\end{figure}

This chapter will go over different types of systems and why we are particularly interested in a class of systems called 
\emph{linear time-invariant (LTI) systems}.

\section{Classification of Systems Based on Mapping}
Systems can be classified by the type of signals that are being inputted and outputted.

\subsection{Multichannel vs Multidimensional}
A system that deals with multichannel input and output systems is called a \emph{multichannel system}. There are four types of multichannel systems, described by the number 
of channels in the input and output signals and aptly named so:
\begin{itemize}
    \item Single-input single-output (SISO) systems
    \item Single-input multiple-output (SIMO) systems
    \item Multiple-input single-output (MISO) systems
    \item Multiple-input multiple-output (MIMO) systems
\end{itemize}
A system that deals with multidimensional input and output signals is called a \emph{multidimensional system}. 
\\ \\
The rest of this text will cover one-dimensional SISO systems.

\subsection{Continuous-Time vs Discrete-Time}
A system that handles CT input and output signals is called a \emph{continuous-time system}, whereas a system that deals with DT input and output signals 
is called a \emph{discrete-time system}. Furthermore, a system that handles analog input and output signals is called a \emph{analog system}, and a system 
that deals with digital input and output signals is called a \emph{digital system}. 
\\ \\
Interestingly, there are cases where a system maps a CT input signal to a DT output signal, or a DT input signal to a CT output signal. The most common example is 
the analog-to-digital and digital-to-analog converters. Such systems are called \emph{hybrid systems}.
\\ \\
The rest of this text will cover only analog systems.

\subsection{Deterministic vs Stochastic}
A system that maps a deterministic input signal to a \emph{predictable} deterministic output signal is called a \emph{deterministic system}. A system that maps a random input signal 
to an \emph{unpredictable} random signal is called a \emph{stochastic system}; stochastic systems may be modeled by some probability distribution but cannot be precisely predicted. \\ \\ 
A system that deals with deterministic input signals and unpredictable output signals is called a \emph{chaotic system}.
\\ \\
The rest of this text will cover only deterministic systems.

\section{Classification of Systems Based on Properties}
Systems can also be classified by $how$ it maps inputs to outputs. Here, the content of the output signal with respect to the input signal is analyzed.
\\ \\
For this chapter, let $\Psi[\cdot]$ represent some operator that the system resembles, with the new block diagram shown.
\begin{center}
    \begin{tikzpicture}
        \node [] (input) at (-3,0){$x(t)$};
        \node [] (output) at (+3,0){$y(t)=\Psi[x(t)]$};
        \node [draw,
            fill=yellow!50, 
            minimum width=2cm, 
            minimum height=1.2cm
        ] (system) at (0,0){$\Psi[\cdot]$};
        \draw [-latex, line width=1.5pt] (input) -- (system);
        \draw [-latex, line width=1.5pt] (system) -- (output);
    \end{tikzpicture}
\end{center}
From the previous chapter, an affine transformation on the independent variable can be represented as a system with $y(t)=\Psi[x(t)]=x(at-b)$.

\subsection{Linear vs Nonlinear}
A system is said to be \emph{linear} if it follows the \emph{superposition principle}, as depicted in the block diagram below with constants $c_1,c_2$, input signal 
addends $x_1(t), x_2(t)$, and output signal addends $y_1(t),y_2(t)$:
\begin{center}
    \begin{tikzpicture}
        \node [] (input) at (-3,0){$c_1x_1(t)+c_2x_2(t)$};
        \node [] (output) at (+3,0){$c_1y_1(t)+c_2y_2(t)$};
        \node [draw,
            fill=yellow!50, 
            minimum width=2cm, 
            minimum height=1.2cm
        ] (system) at (0,0){$\Psi[\cdot]$};
        \draw [-latex, line width=1.5pt] (input) -- (system);
        \draw [-latex, line width=1.5pt] (system) -- (output);
    \end{tikzpicture}
\end{center}
In practice, if the superposition principle is met, then a sum of $N$ input addends will generate a sum of $N$ output addends. \\ \\
The superposition principle can be broken down into its two properties: additivity and scalability. 
\\ \\ 
A system has \emph{additive property} if for $y_1(t)=\Psi[x_1(t)]$ and $y_2(t)=\Psi[x_2(t)]$, the system-generated output from 
the summed inputs is the sum of the outputs $\Psi[x_1(t)+x_2(t)]=y_1(t)+y_2(t)$. That is:
\begin{center}
    \begin{tikzpicture}
        \node [] (input) at (-3,0){$x_1(t)+x_2(t)$};
        \node [] (output) at (+3,0){$y_1(t)+y_2(t)$};
        \node [draw,
            fill=yellow!50, 
            minimum width=2cm, 
            minimum height=1.2cm
        ] (system) at (0,0){$\Psi[\cdot]$};
        \draw [-latex, line width=1.5pt] (input) -- (system);
        \draw [-latex, line width=1.5pt] (system) -- (output);
    \end{tikzpicture}
\end{center}
A system has \emph{scaling property} if for some constant $c$, the system-generated output for a scaled input is 
an appropriately scaled output $\Psi[c\cdot x(t)]=c\cdot y(t)$. That is:
\begin{center}
    \begin{tikzpicture}
        \node [] (input) at (-3,0){$c\cdot x(t)$};
        \node [] (output) at (+3,0){$c\cdot y(t)$};
        \node [draw,
            fill=yellow!50, 
            minimum width=2cm, 
            minimum height=1.2cm
        ] (system) at (0,0){$\Psi[\cdot]$};
        \draw [-latex, line width=1.5pt] (input) -- (system);
        \draw [-latex, line width=1.5pt] (system) -- (output);
    \end{tikzpicture}
\end{center}
If a system has both scaling and additive properties, then the superposition principle is met, and the system is linear. If the superposition principle is not met, 
then the system is said to be \emph{nonlinear}.

\begin{example}
    Determine if the system characterized by the following equation is linear:
    \begin{align*}
        \frac{d^2y(t)}{dt}+2\frac{dy(t)}{dt}+3y(t)=4\frac{dx(t)}{dt}+5x(t)
    \end{align*}
\end{example}
\begin{solution}
    First, determine if the system is additive. Consider the following sum: \\
    \begin{center}
        \begin{tabular}{ cc }
            & $\left[\dfrac{d^2y_1(t)}{dt}+2\dfrac{dy_1(t)}{dt}+3y_1(t)=4\dfrac{dx_1(t)}{dt}+5x_1(t)\right]$ \\[0.3cm]
            + & $\left[\dfrac{d^2y_2(t)}{dt}+2\dfrac{dy_2(t)}{dt}+3y_2(t)=4\dfrac{dx_2(t)}{dt}+5x_2(t)\right]$ \\[0.3cm]
        \end{tabular}
        \begin{tabular}{ c }
            \hline\\
            $\dfrac{d^2[y_1(t)+y_2(t)]}{dt}+2\dfrac{d[y_1(t)+y_2(t)]}{dt}+3[y_1(t)+y_2(t)]=4\dfrac{d[x_1(t)+x_2(t)]}{dt}+5[x_1(t)+x_2(t)]$ \\[0.75cm]
        \end{tabular}
    \end{center} 
    We are able to combine input components and output components due to the additive nature of the derivative. Here we can see that 
    $\Psi[x_1(t)+x_2(t)]=\Psi[x_1(t)]+\Psi[x_2(t)]=y_1(t)+y_2(t)$. Therefore, the system is additive. 
    \\ \\
    Lastly, determine if the system is scalable. Consider the following product:
    \begin{align*}
        \dfrac{d^2[c\cdot y(t)]}{dt}+2\dfrac{d[c\cdot y(t)]}{dt}+3[c\cdot y(t)] &= 4\dfrac{d[c\cdot x(t)]}{dt}+5[c\cdot x(t)] \\
        \Longrightarrow c\cdot \left[\dfrac{d^2y(t)}{dt}+2\dfrac{dy(t)}{dt}+3y(t)\right] &= c\cdot \left[4\dfrac{dx(t)}{dt}+5x(t)\right]
    \end{align*}
    We are able to factor out $c$ because of the scaling property of the derivative. Here we can see that $\Psi[c\cdot x(t)]=c\cdot\Psi[x(t)]=c\cdot y(t)$. 
    Therefore, the system is scalable.
    \\ \\
    Since the system is both additive and scalable, the superposition principle is met, and the system is linear.
\end{solution}

\subsection{Time-Invariant vs Time-Variant}
A system is said to be \emph{time-invariant} if a delay in the input signal results in a corresponding delay in the output signal such that $\Psi[x(t-T)]=y(t-T)$. That is:
\begin{center}
    \begin{tikzpicture}
        \node [] (input) at (-3,0){$x(t-T)$};
        \node [] (output) at (+3,0){$y(t-T)$};
        \node [draw,
            fill=yellow!50, 
            minimum width=2cm, 
            minimum height=1.2cm
        ] (system) at (0,0){$\Psi[\cdot]$};
        \draw [-latex, line width=1.5pt] (input) -- (system);
        \draw [-latex, line width=1.5pt] (system) -- (output);
    \end{tikzpicture}
\end{center}
One way to check is to consider the following two block diagrams, with the ``D''-block representing a delay of $T$ seconds:
\begin{center}
    \begin{tikzpicture}
        \node [] (input) at (-3,0){$x_1(t)$};
        \node [] (output) at (+3,0){$y_1(t)$};
        \node [] (delayed) at (+9,0){$y_1(t-T)$};
        \node [draw,
            fill=yellow!50, 
            minimum width=2cm, 
            minimum height=1.2cm
        ] (system) at (0,0){$\Psi[\cdot]$};
        \node [draw,
            fill=green!50,
            minimum width=1.2cm,
            minimum height=1.2cm
        ] (delay) at (5.8,0){$\mathbf{D}$};
        \draw [-latex, line width=1.5pt] (input) -- (system);
        \draw [-latex, line width=1.5pt] (system) -- (output);
        \draw [-latex, line width=1.5pt] (output) -- (delay);
        \draw [-latex, line width=1.5pt] (delay) -- (delayed);
    \end{tikzpicture}
    \\[0.5cm]
    \begin{tikzpicture}
        \node [] (input) at (-3,0){$x_1(t)$};
        \node [] (delayed) at (+2,0){$x_2(t)=x_1(t-T)$};
        \node [] (output) at (+8,0){$y_2(t)$};
        \node [draw,
            fill=green!50,
            minimum width=1.2cm,
            minimum height=1.2cm
        ] (delay) at (-1,0){$\mathbf{D}$};
        \node [draw,
            fill=yellow!50, 
            minimum width=2cm, 
            minimum height=1.2cm
        ] (system) at (+5.5,0){$\Psi[\cdot]$};
        \draw [-latex, line width=1.5pt] (input) -- (delay);
        \draw [-latex, line width=1.5pt] (delay) -- (delayed);
        \draw [-latex, line width=1.5pt] (delayed) -- (system);
        \draw [-latex, line width=1.5pt] (system) -- (output);
    \end{tikzpicture}
\end{center}
From the outputs above, if $y_2(t)=y_1(t-T)$, then the system is time-invariant. Otherwise, if the outputs are not equal, 
then the system is \emph{time-variant}, in which $\Psi[x(t-T),T]=y(t-T)$ results in an additional dependence on $T$.

\begin{example}
    Determine if the system characterized by the following equation is time-invariant.
    \begin{align*}
        y(t) = \frac{t\cdot x(t+2)}{x(t-1)}
    \end{align*}
\end{example}
\begin{solution}
    If we feed the input into the system first then delay the output, then we get:
    \begin{align*}
        y_1(t) &= \frac{t\cdot x_1(t+2)}{x_1(t-1)} \\
        y_1(t-T) &= \frac{(t-T)\cdot x_1((t-T)+2)}{x_1((t-T)-1)}
    \end{align*}
    If we delay the input first then feed the delayed input into the system, then we get:
    \begin{align*}
        x_2(t) &= x_1(t-T) \\
        y_2(t) &= \frac{t\cdot x_2(t+2)}{x_2(t-1)} = \frac{t\cdot x_1((t-T)+2)}{x_1((t-T)-1)}
    \end{align*}
    Since $y_2(t) \neq y_1(t-T)$, the system is time-variant.
\end{solution}

Systems that are both linear and time-invariant are called linear time-invariant (LTI) systems. We will explore LTI systems more later. 

\subsection{Dynamic vs Memoryless}
A system is said to be \emph{memoryless} or \emph{static} if the output $y(t)$ at time $t$ depends only on the input $x(t)$ at time $t$. The only LTI system 
that is also memoryless has the form $y(t)=ax(t)$ for $a$ is a constant. \\ \\
Otherwise, a system whose output additionally depends on past and/or future values of the input is a \emph{dynamic} system; that is, the output at $t$ 
depends on the input at $t-T$ for any $T\neq 0$. Most systems in the physical world are dynamic.

\subsection{Causal vs Noncausal}
\emph{Causal} systems, also called \emph{physically realizable systems}, are systems whose output at time $t$ only depend on the past and present values of the input at time $t-T$ for $T>0$ and time $t$, respectively. 
Otherwise, a system that anticipates a future value of the input at time $t+T$ before generating an output at time $t$ is said to be \emph{noncausal}. \\ \\
Essentially, noncausal systems see the future, which is impossible in the physical world. 
Generally, from the affine transformation on the independent variable, any system with a time advance or a temporal expansion of the input signal is classified as noncausal.
\\ \\
One way to determine causality is by using an impulse signal $\delta(t)$ as input and observing its output $h(t)$, also called the \emph{impulse response}. If $h(t)=0$ for all $t<0$, 
then the system is causal; otherwise it is noncausal.

\begin{center}
    \begin{tikzpicture}
        \node [] (input) at (-3,0){$\delta(t)$};
        \node [] (output) at (+3,0){$h(t)$};
        \node [draw,
            fill=yellow!50, 
            minimum width=2cm, 
            minimum height=1.2cm
        ] (system) at (0,0){$\Psi[\cdot]$};
        \draw [-latex, line width=1.5pt] (input) -- (system);
        \draw [-latex, line width=1.5pt] (system) -- (output);
    \end{tikzpicture}
\end{center}

\begin{example}
    Determine if the following circuit is a causal system. \\ \\
    \resizebox{0.5\textwidth}{!}{
        \begin{circuitikz}[american voltages,raised voltages]
            \draw 
                (0,3) to [V, l_={\Large $x(t)$}] (0,0)
                to [short, -*] (6,0)
                (0,3) to [R, v_={\Large $i(t)$}, voltage=straight, l^={\Large $R$}] (4,3)
                to [C, l_={\Large $C$}] (4,0)
                (4,3) to [short, -*] (6,3) 
                to [open, v^>={\Large $y(t)$}] (6,0);
        \end{circuitikz}}
\end{example}
\begin{solution}
    From Kirchhoff's voltage law (KVL):
    \begin{align*}
        R\cdot i(t) + y(t) &= x(t) \\
        \Longrightarrow RC \cdot \frac{dy(t)}{dt} + y(t) &= x(t).
    \end{align*}
    Let $\tau=RC$. Then
    \begin{align*}
        \frac{dy(t)}{dt} + \frac{1}{\tau} \cdot y(t) = \frac{1}{\tau} \cdot x(t).
    \end{align*}
    Since the integrating factor is $\exp\left(\int 1/\tau \,dt\right) = \exp(t/\tau)$, multiply the whole equation by $\exp(t/\tau)$ such that 
    \begin{align*}
        e^{t/\tau} \cdot \frac{dy(t)}{dt} + \frac{e^{t/\tau}}{\tau} \cdot y(t) &= \frac{e^{t/\tau}}{\tau} \cdot x(t) \\
        \frac{d}{dt} \big[e^{t/\tau} y(t)\big] &= \frac{e^{t/\tau}}{\tau} \cdot x(t).
    \end{align*}
    Inputting in an impulse function for the input, we get
    \begin{align*}
        \frac{d}{dt} \big[e^{t/\tau} h(t)\big] &= \frac{e^{t/\tau}}{\tau} \cdot \delta(t).
    \end{align*}
    Integrating both sides:
    \begin{align*}
        \int_{0^-}^{T} \left(\frac{d}{dt} \big[e^{t/\tau} h(t)\big]\right) \,dt &= \int_{0^-}^{T} \left(\frac{e^{t/\tau}}{\tau} \cdot \delta(t)\right) \,dt \\
        e^{T/\tau} h(T) &= \int_{0^-}^{T} \left(\frac{e^{0}}{\tau} \cdot \delta(t)\right) \,dt \\
        e^{T/\tau} h(T) &= \frac{1}{\tau} u(T).
    \end{align*}
    Rearranging the equation and substituting $T\leftarrow t$, we get
    \begin{align*}
        h(t) = \frac{e^{-t/\tau}}{\tau} u(t).
    \end{align*}
    Since $h(t)$ is causal, the circuit is a causal system.
\end{solution}

\subsection{BIBO Stable vs Unstable}
A signal $x(t)$ is said to be \emph{bounded} if there exists some constant $C$ such that 
\begin{align}
    |x(t)| \leq C, \text{ for } \forall t.
\end{align}
A system with a \emph{bounded input, bounded output} (BIBO, for short) is said to be \emph{BIBO stable}; that is, every bounded input signal results in a bounded output signal. Otherwise, if a system 
produces an unbounded output for a bounded input, it is said to be \emph{unstable}.
\\ \\
For LTI systems, we can particularly look at the impulse response $h(t)$. An LTI system is BIBO stable if and only if $h(t)$ is \emph{absolutely integrable} such that
\begin{align}
    \int_{-\infty}^{+\infty} |h(t)| \,dt = C, \text{ for $C$ is finite.}
\end{align}

\begin{example}
    Determine if an LTI system with impulse response $h(t)=e^{-|t|}$ is stable.
\end{example}
\begin{solution}
    \begin{align*}
        \int_{-\infty}^{+\infty} |h(t)| \,dt = \int_{-\infty}^{+\infty} |e^{-|t|}| \,dt &= 2\int_{0}^{+\infty} e^{-t} \,dt \\
        &= 2 \left[-e^{-t}\right|_0^{+\infty} = 2 [0 - (-1)] = 2 < \infty
    \end{align*}
    Since $h(t)$ is absolutely integrable, the system is BIBO stable.
\end{solution}

\begin{example}
    Let $C=A+jB$ and $\gamma = \alpha + j\beta$ be complex values. 
    Show that all LTI systems with exponential decay $h(t)=Ce^{\gamma t}u(t)$ as impulse responses are BIBO stable if and only if $\Re(\gamma) < 0$. 
\end{example}
\begin{solution}
    First, let $\alpha < 0$. Then 
    \begin{align*}
        \int_{-\infty}^{+\infty} |h(t)| \,dt &= \int_{-\infty}^{+\infty} |Ce^{\gamma t}u(t)| \,dt = \int_{0}^{+\infty} |C|\cdot |e^{\alpha t}|\cdot |e^{j\beta t}| \,dt \\
        &= |C|\int_{0}^{+\infty} e^{\alpha t}\,dt = |C|\int_{0}^{+\infty} e^{-|\alpha| t}\,dt = |C| \left[-\frac{e^{-|\alpha|t}}{|\alpha|}\right|_0^{+\infty} = \frac{|C|}{|\alpha|} < \infty.
    \end{align*}
    Now, let $\alpha = 0$. Then 
    \begin{align*}
        \int_{-\infty}^{+\infty} |h(t)| \,dt = \left[|C|\int_{0}^{+\infty} e^{\alpha t}\,dt\right|_{\alpha=0}  = |C|\int_{0}^{+\infty} 1\,dt = |C| \big[t\big|_0^{+\infty} = \infty.
    \end{align*}
    Lastly, let $\alpha>0$. Then
    \begin{align*}
        \int_{-\infty}^{+\infty} |h(t)| \,dt = |C|\int_{0}^{+\infty} e^{\alpha t}\,dt = |C| \left[\frac{e^{|\alpha|t}}{|\alpha|}\right|_0^{+\infty} = \infty.
    \end{align*}
    Since $h(t)$ is only absolutely integrable when $\alpha<0$, the system is only stable when $\Re(\gamma)<0$. \\
\end{solution}

\subsection{Invertible vs Non-Invertible}
A system is said to be \emph{invertible} if it generates unique output signals for every unique input signal. That is, an invertible system has a one-to-one mapping between 
inputs and outputs. If a system has a many-to-one mapping between inputs and outputs, then it is said to be \emph{non-invertible}. 
\\ \\
Alternatively, an invertible system has an inverse system $\Psi^{-1}[\cdot]$ that maps outputs back to the inputs of the forward system $\Psi[\cdot]$ -- it may be delayed but the shape of the waveform is maintained. 
\begin{center}
    \begin{tikzpicture}
        \node [] (input) at (-3,0){$x(t)$};
        \node [] (output) at (+6,0){$x(t)$};
        \node [draw,
            fill=yellow!50, 
            minimum width=2cm, 
            minimum height=1.2cm
        ] (fwd) at (0,0){$\Psi[\cdot]$};
        \node [draw,
            fill=blue!30, 
            minimum width=2cm, 
            minimum height=1.2cm
        ] (inv) at (3,0){$\Psi^{-1}[\cdot]$};
        \draw [-latex, line width=1.5pt] (input) -- (fwd);
        \draw [-latex, line width=1.5pt] (fwd) -- (inv);
        \draw [-latex, line width=1.5pt] (inv) -- (output);
    \end{tikzpicture}
\end{center}
\begin{center}
    \begin{tikzpicture}
        \node [] (input) at (-3,0){$y(t)$};
        \node [] (output) at (+6,0){$y(t)$};
        \node [draw,
            fill=blue!30, 
            minimum width=2cm, 
            minimum height=1.2cm
        ] (inv) at (0,0){$\Psi^{-1}[\cdot]$};
        \node [draw,
            fill=yellow!50, 
            minimum width=2cm, 
            minimum height=1.2cm
        ] (fwd) at (3,0){$\Psi[\cdot]$};
        \draw [-latex, line width=1.5pt] (input) -- (inv);
        \draw [-latex, line width=1.5pt] (inv) -- (fwd);
        \draw [-latex, line width=1.5pt] (fwd) -- (output);
    \end{tikzpicture}
\end{center}
While a system can be invertible, it does not mean it is implementable. For this very reason, when inverting LTI systems, generally we are also interested in the properties of the 
inverse system; that is, if the original system is BIBO stable and causal, we would want the inverse system to be BIBO stable and causal as well. 

\begin{example}
    Determine if the system characterized by $y(t)=x(2t-3)$ is invertible.
\end{example}
\begin{solution}
    Here we can solve for $x(t)$. First, define $y(\tau)=x(2\tau-3)$. Then 
    \begin{align*}
        t = 2\tau - 3 \Longrightarrow \tau = \frac{1}{2}\left(t+3\right).
    \end{align*}
    Substituting this relationship in, we get 
    \begin{align*}
        x(t) = y\left(\frac{1}{2}(t+3)\right).
    \end{align*}
    Since both forward and inverse systems are affine transformations on the independent variable, the system is invertible.
\end{solution}

\begin{example}
    Determine if the system characterized by $y(t)=\dfrac{d}{dt}\big[x(t)\big]$ is invertible.
\end{example}
\begin{solution}
    Let $x(t)=x_0(t)+c$, where $x_0(t)$ is the non-constant portion of the function and $c$ is the constant in the function.
    \\ \\
    Regardless of the value of $c$, the function $x(t)$  will always map to the derivative of $x_0(t)$ since the derivative of a constant is zero. 
    \begin{center}
        \begin{tikzpicture}
            \node [] (input) at (-3.6,0){$x(t)=x_0(t)+c$};
            \node [] (output) at (+4.4,0){$y(t)=\Psi[x(t)]=\dfrac{d}{dt}\big[x_0(t)\big]$};
            \node [draw,
                fill=yellow!50, 
                minimum width=2cm, 
                minimum height=1.2cm
            ] (system) at (0,0){$\Psi[\cdot]$};
            \draw [-latex, line width=1.5pt] (input) -- (system);
            \draw [-latex, line width=1.5pt] (system) -- (output);
        \end{tikzpicture}
    \end{center}
    This is a many-to-one mapping. Therefore the system is non-invertible.
\end{solution}

\section{LTI Systems}
By now, we know that an \emph{LTI system} is a system that is both linear and time-invariant. But why are we particularly study in 
LTI systems? We study LTI systems because of their predictable nature. As long as we know the system response to a few select input signals, we can 
accurately predict the output for all input signals.
\\ \\
Although many systems in the real world are not LTI systems, they might be ``locally nearly LTI'', where the behavior of the system could possibly be approximated with 
an LTI system within a small analytic region. Because we are modeling real systems using LTI systems, we would need to choose an LTI model that is also dynamic, causal, and stable.

\subsection{LTI System Response to Singularity Signals}
As prefaced before, the \emph{impulse response} $h(t)$ of a system is the system response to an inputted impulse signal $\delta(t)$, given zero initial conditions. 
The impulse response of an LTI system is depicted in the following block diagram:
\begin{center}
    \begin{tikzpicture}
        \node [] (input) at (-3,0){$\delta(t)$};
        \node [] (output) at (+3,0){$h(t)$};
        \node [draw,
            fill=yellow!50, 
            minimum width=2cm, 
            minimum height=1.2cm
        ] (system) at (0,0){LTI};
        \draw [-latex, line width=1.5pt] (input) -- (system);
        \draw [-latex, line width=1.5pt] (system) -- (output);
    \end{tikzpicture}
\end{center}
For LTI systems, the impulse response plays an important role. An LTI system can be characterized by its impulse response such that any output signal can be predicted by 
performing an operation called \emph{convolution} between the input signal and the impulse response. The block diagram of an LTI system can be depicted as:
\begin{center}
    \begin{tikzpicture}
        \node [] (input) at (-3,0){$x(t)$};
        \node [] (output) at (+3,0){$y(t)$};
        \node [draw,
            fill=yellow!50, 
            minimum width=2cm, 
            minimum height=1.2cm
        ] (system) at (0,0){$h(t)$};
        \draw [-latex, line width=1.5pt] (input) -- (system);
        \draw [-latex, line width=1.5pt] (system) -- (output);
    \end{tikzpicture}
\end{center}
Convolution is an operation defined by the \emph{convolution integral}
\begin{align}
    x(t) * h(t) &= \int_{-\infty}^{+\infty} x(\tau)h(t-\tau) \,d\tau \\
    &= \int_{-\infty}^{+\infty} x(t-\tau)h(\tau) \,d\tau.
\end{align}
All LTI systems characterized by $h(t)$ abide by convolution. Convolution will be explored further in the next chapter.
\\ \\
The \emph{step response} $y_{step}(t)$ of a system is the system response to an inputted unit step signal $u(t)$, given zero initial conditions. 
The step response of an LTI system is depicted in the following block diagram:
\begin{center}
    \begin{tikzpicture}
        \node [] (input) at (-3,0){$u(t)$};
        \node [] (output) at (+3,0){$y_{step}(t)$};
        \node [draw,
            fill=yellow!50, 
            minimum width=2cm, 
            minimum height=1.2cm
        ] (system) at (0,0){LTI};
        \draw [-latex, line width=1.5pt] (input) -- (system);
        \draw [-latex, line width=1.5pt] (system) -- (output);
    \end{tikzpicture}
\end{center}
While not as prevalent in signal processing applications, the step response appears in primarily control-based applications, where 
we are interested in how well a system ``tracks'' a step input.
\\ \\
Interestingly, just as $u(t)$ is the antiderivative of $\delta(t)$, the step response $y_{step}(t)$ is the antiderivative of the impulse response $h(t)$. That is,
\begin{align}
    h(t) &= \frac{dy_{step}(t)}{dt}, \\
    y_{step}(t) &= \int_{-\infty}^{t} h(\tau) \,d\tau.
\end{align}
For both the impulse response and step response, all initial conditions must be zero in order to generalize the performance of any system modeled 
by that particular LTI system. If initial conditions were left as nonzero, we must deal with case-specific systems, where an LTI model with nonzero initial 
conditions can only be applicable for one system with those specific initial conditions. 

\subsection{LTI System Response to Exponential and Sinusoidal Signals}
Using the convolution integral, we can generalize the LTI system response to a complex exponential signal $e^{j\omega t}$.
\begin{align*}
    x(t) * h(t) &= \int_{-\infty}^{+\infty} x(t-\tau)h(\tau) \,d\tau \\
    \Longrightarrow e^{j\omega t} * h(t) &= \int_{-\infty}^{+\infty} e^{j\omega (t-\tau)}h(\tau) \,d\tau \\
    &= e^{j\omega t} \underbrace{\int_{-\infty}^{+\infty} h(\tau)e^{-j\omega\tau} \,d\tau}_\textrm{$H(\omega)$}.
\end{align*}
Therefore, the LTI system response to a complex exponential can be depicted as: 
\begin{center}
    \begin{tikzpicture}
        \node [] (input) at (-3,0){$e^{j\omega t}$};
        \node [] (output) at (+3,0){$H(\omega)e^{j\omega t}$};
        \node [draw,
            fill=yellow!50, 
            minimum width=2cm, 
            minimum height=1.2cm
        ] (system) at (0,0){$h(t)$};
        \draw [-latex, line width=1.5pt] (input) -- (system);
        \draw [-latex, line width=1.5pt] (system) -- (output);
    \end{tikzpicture}
\end{center}
Here, $H(\omega)$ is called the \emph{frequency response} of the system. Note that $H(\omega)$ is independent of time $t$, 
is a complex function of $j\omega$, and is defined for an everlasting complex exponential. $H(\omega)$ also has conjugate symmetry, in which 
\begin{align}
    H^*(\omega) = H(-\omega)
\end{align}
It then also follows that 
\begin{align}
    |H(\omega)| = |H(-\omega)| \\
    \phase{H(-\omega)} = -\phase{H(\omega)}
\end{align}
In a later chapter, the relationship between the impulse reponse and the frequency response will be explored. There exists some operator in which the impulse response can be 
mapped to the frequency response and vice versa.
\\ \\
Since sinusoids form the real and imaginary parts of a complex exponential, the LTI sinusoidal response can also be depicted: 
\begin{center}
    \begin{tikzpicture}
        \node [] (input) at (-3,0){$A\cos(\omega t+\phi)$};
        \node [] (output) at (+4.5,0){$A|H(\omega)|\cos\big(\omega t+\phase{H(\omega)}+\phi\big)$};
        \node [draw,
            fill=yellow!50, 
            minimum width=2cm, 
            minimum height=1.2cm
        ] (system) at (0,0){$h(t)$};
        \draw [-latex, line width=1.5pt] (input) -- (system);
        \draw [-latex, line width=1.5pt] (system) -- (output);
    \end{tikzpicture}
\end{center}

\begin{example}
    The input and output to an LTI system is given by 
    \begin{align*}
        x(t) &= u(t)+2\cos(2t), \\ 
        y(t) &= u(t) - e^{-2t}u(t) +\sqrt{2}\cos(2t-45^{\circ}).
    \end{align*}
    Suppose a different signal is now inputted to the very same LTI system and is defined as 
    \begin{align*}
        x(t) = 5u(t-3) + 3\sqrt{2}\cos(2t-60^{\circ}).
    \end{align*}
    Determine the new output $y(t)$.
\end{example}
\begin{solution}
    From the superposition principle, we know that as long as we know the response to each addend of the input signal, 
    we can combine each of the responses to get the overall output.
    \\ \\
    By observation, we first analyze just the unit step addends.
    \begin{center}
        \begin{tikzpicture}
            \node [] (input) at (-3,0){$u(t)$};
            \node [] (output) at (+4,0){$(1-e^{-2t})u(t)$};
            \node [draw,
                fill=yellow!50, 
                minimum width=2cm, 
                minimum height=1.2cm
            ] (system) at (0,0){LTI};
            \draw [-latex, line width=1.5pt] (input) -- (system);
            \draw [-latex, line width=1.5pt] (system) -- (output);
        \end{tikzpicture}
    \end{center}
    Since this is an LTI system, any delay in the input will have the same corresponding delay in the output. Additionally, an input signal that is scaled by 
    some factor will be scaled by the same factor in the output. Therefore,
    \begin{center}
        \begin{tikzpicture}
            \node [] (input) at (-3,0){$5u(t-3)$};
            \node [] (output) at (+4,0){$5(1-e^{-2(t-3)})u(t-3)$};
            \node [draw,
                fill=yellow!50, 
                minimum width=2cm, 
                minimum height=1.2cm
            ] (system) at (0,0){LTI};
            \draw [-latex, line width=1.5pt] (input) -- (system);
            \draw [-latex, line width=1.5pt] (system) -- (output);
        \end{tikzpicture}
    \end{center}
    Next, we analyze just the sinusoidal addends.
    \begin{center}
        \begin{tikzpicture}
            \node [] (input) at (-3,0){$2\cos(2t)$};
            \node [] (output) at (+4,0){$\sqrt{2}\cos(2t-45^{\circ})$};
            \node [draw,
                fill=yellow!50, 
                minimum width=2cm, 
                minimum height=1.2cm
            ] (system) at (0,0){LTI};
            \draw [-latex, line width=1.5pt] (input) -- (system);
            \draw [-latex, line width=1.5pt] (system) -- (output);
        \end{tikzpicture}
    \end{center}
    From the LTI sinusoidal response, it follows that 
    \begin{align*}
        |H(\omega)| &= \frac{\sqrt{2}}{2}, \\ 
        \phase{H(\omega)} &= -45^{\circ}.
    \end{align*}
    Therefore, the LTI sinusoidal response of the new addend is given by 
    \begin{center}
        \begin{tikzpicture}
            \node [] (input) at (-3,0){$3\sqrt{2}\cos(2t-60^{\circ})$};
            \node [] (output) at (+4,0){$3\sqrt{2}\left(\dfrac{\sqrt{2}}{2}\right)\cos(2t-60^{\circ}-45^{\circ})$};
            \node [draw,
                fill=yellow!50, 
                minimum width=2cm, 
                minimum height=1.2cm
            ] (system) at (0,0){LTI};
            \draw [-latex, line width=1.5pt] (input) -- (system);
            \draw [-latex, line width=1.5pt] (system) -- (output);
        \end{tikzpicture}
    \end{center}
    Summing the two LTI system responses, the overall output is given by 
    \begin{align*}
        y(t) = 5[1-e^{-2(t-3)}]u(t-3) + 3\cos(2t-105^{\circ}).
    \end{align*}
\end{solution}

\subsection{Eigenfunctions of LTI Systems}
An \emph{eigenfunction} of an LTI system is some function $x(t)$ such that the system response is $\lambda x(t)$ for some scalar $\lambda$ (independent of $t$). That is,  
\begin{center}
    \begin{tikzpicture}
        \node [] (input) at (-3,0){$x(t)$};
        \node [] (output) at (+3,0){$\lambda x(t)$};
        \node [draw,
            fill=yellow!50, 
            minimum width=2cm, 
            minimum height=1.2cm
        ] (system) at (0,0){LTI};
        \draw [-latex, line width=1.5pt] (input) -- (system);
        \draw [-latex, line width=1.5pt] (system) -- (output);
    \end{tikzpicture}
\end{center}
For LTI systems characterized by $h(t)$, we can use the convolution integral to determine if a function 
is an eigenfunction of the system. 
\\ \\
Previously, we saw that for a complex exponential input $e^{j\omega t}$, 
we get a scaled output $H(\omega)e^{j\omega t}$. Therefore, the complex exponential is an eigenfunction of the system. 
In fact, this particular eigenfunction yields the frequency response function $H(\omega)$, which is crucial for Fourier transforms 
in a later chapter.
\begin{center}
    \begin{tikzpicture}
        \node [] (input) at (-3,0){$e^{j\omega t}$};
        \node [] (output) at (+3,0){$H(\omega)e^{j\omega t}$};
        \node [draw,
            fill=yellow!50, 
            minimum width=2cm, 
            minimum height=1.2cm
        ] (system) at (0,0){LTI};
        \draw [-latex, line width=1.5pt] (input) -- (system);
        \draw [-latex, line width=1.5pt] (system) -- (output);
    \end{tikzpicture}
\end{center}

\begin{example}
    Determine if $e^{st}$ is an eigenfunction of LTI systems, for $s=\sigma + j\omega$.
\end{example}
\begin{solution}
    \begin{align*}
        e^{st} * h(t) &= \int_{-\infty}^{+\infty} e^{s(t-\tau)}h(\tau) \,d\tau = e^{st} \underbrace{\int_{-\infty}^{+\infty} h(\tau)e^{-s\tau} \,d\tau}_\textrm{$H(s)$}
    \end{align*}
    Since $H(s)$ is a constant for a given $s$ and is scaling $e^{st}$, the function $e^{st}$ is an eigenfunction of LTI systems. 
    In fact, it yields the transfer function $H(s)$, which is crucial for Laplace transforms in a later chapter.
    \\ \\ 
    \begin{center}
    \begin{tikzpicture}
        \node [] (input) at (-3,0){$e^{st}$};
        \node [] (output) at (+3,0){$H(s)e^{st}$};
        \node [draw,
            fill=yellow!50, 
            minimum width=2cm, 
            minimum height=1.2cm
        ] (system) at (0,0){LTI};
        \draw [-latex, line width=1.5pt] (input) -- (system);
        \draw [-latex, line width=1.5pt] (system) -- (output);
    \end{tikzpicture}
    \end{center}
\end{solution}

\begin{example}
    Determine if $3^t u(t)$ is an eigenfunction of LTI systems.
\end{example}
\begin{solution}
    \begin{align*}
        3^t u(t) * h(t) &= \int_{-\infty}^{+\infty} 3^{t-\tau}u(t-\tau)h(\tau) \,d\tau = 3^t \int_{-\infty}^{t} 3^{-\tau}h(\tau) \,d\tau
    \end{align*}
    Since the integral is dependent on time $t$, it is not a constant. Therefore $3^t u(t)$ is not an eigenfunction of LTI systems.
\end{solution}

\subsection{Linear Constant-Coefficient Differential Equations (LCCDE)}
LTI systems can be characterized by a class of differential equations called 
\emph{linear constant-coefficient differential equations} (LCCDE):
\begin{align}
    \sum_{k=0}^{N} a_{N-k} \frac{d^k y(t)}{dt^k} = \sum_{k=0}^{M} b_{M-k} \frac{d^k x(t)}{dt^k}.
\end{align}
However, higher-order nonhomogeneous LCCDEs are generally difficult to solve, which is why it is generally not covered until after Laplace and/or Fourier transforms are studied. 
Because of that, the LTI system response of LCCDEs will not be addressed until a later chapter.

\section{Interconnection of LTI Systems}
We have briefly seen that the convolution of input $x(t)$ and impulse response $h(t)$ of an LTI system gives us the output $y(t)$ of the system.
\begin{center}
    \begin{tikzpicture}
        \node [] (input) at (-3,0){$x(t)$};
        \node [] (output) at (+3,0){$y(t)=x(t)*h(t)$};
        \node [draw,
            fill=yellow!50, 
            minimum width=2cm, 
            minimum height=1.2cm
        ] (system) at (0,0){$h(t)$};
        \draw [-latex, line width=1.5pt] (input) -- (system);
        \draw [-latex, line width=1.5pt] (system) -- (output);
    \end{tikzpicture}
\end{center}
Suppose we begin introducing multiple LTI systems that are somehow connected to each other. When we are only interested in the corresponding output signal to an 
input signal, we do not particularly care about an individual LTI system -- rather, we are more interested in the interconnection of LTI systems as a whole. In these cases, 
we want to find an equivalent impulse response that characterizes the interconnection and not just a single LTI system. 
\\ \\
When LTI systems are connected in series, the interconnection has the following equivalence:
\begin{center}
    \begin{tikzpicture}
        \node [] (input) at (-3,0){$x(t)$};
        \node [] (output) at (+6,0){$y(t)$};
        \node [draw,
            fill=yellow!50, 
            minimum width=2cm, 
            minimum height=1.2cm
        ] (sys1) at (0,0){$h_1(t)$};
        \node [draw,
            fill=red!30, 
            minimum width=2cm, 
            minimum height=1.2cm
        ] (sys2) at (3,0){$h_2(t)$};
        \draw [-latex, line width=1.5pt] (input) -- (sys1);
        \draw [-latex, line width=1.5pt] (sys1) -- (sys2);
        \draw [-latex, line width=1.5pt] (sys2) -- (output);
    \end{tikzpicture} \\[0.25cm] 
    \begin{tikzpicture}
        \node at (0,0) [double arrow, draw=black, top color=red, bottom color=blue,
      minimum width = 15pt, double arrow head extend=5pt,
      minimum height=12mm,
      rotate=90] {};
    \end{tikzpicture} \\[0.25cm]
    \begin{tikzpicture}
        \node [] (input) at (-3,0){$x(t)$};
        \node [] (output) at (+6,0){$y(t)$};
        \node [draw,
            fill=blue!30, 
            minimum width=2cm, 
            minimum height=1.2cm
        ] (system) at (1.5,0){$h_1(t)*h_2(t)$};
        \draw [-latex, line width=1.5pt] (input) -- (system);
        \draw [-latex, line width=1.5pt] (system) -- (output);
    \end{tikzpicture} \\
\end{center}
When LTI systems are connected in parallel, the interconnection has the following equivalence:
\begin{center}
    \begin{tikzpicture}
        \node [] (input) at (-3,0){$x(t)$};
        \node [] (output) at (+6,0){$y(t)$};
        \node [draw,
            fill=yellow!50, 
            minimum width=2cm, 
            minimum height=1.2cm
        ] (sys1) at (1.5,1){$h_1(t)$};
        \node [draw,
            fill=orange!50, 
            minimum width=2cm, 
            minimum height=1.2cm
        ] (sys2) at (1.5,-1){$h_2(t)$};
        \node[draw,
            circle,
            minimum size=0.6cm,
            fill=red!30
        ] (sum) at (4,0){\large +};
        \draw [-, line width=1.5pt] (input) -- (-1,0);
        \draw [-, line width=1.5pt] (-1,-1) -- (-1,1);
        \draw [-latex, line width=1.5pt] (-1,1) -- (sys1);
        \draw [-latex, line width=1.5pt] (-1,-1) -- (sys2);
        \draw [-, line width=1.5pt] (sys1) -- (4,1);
        \draw [-, line width=1.5pt] (sys2) -- (4,-1);
        \draw [-latex, line width=1.5pt] (4,1) -- (sum);
        \draw [-latex, line width=1.5pt] (4,-1) -- (sum);
        \draw [-latex, line width=1.5pt] (sum) -- (output);
    \end{tikzpicture} \\[0.25cm]
    \begin{tikzpicture}
        \node at (0,0) [double arrow, draw=black, top color=red, bottom color=blue,
      minimum width = 15pt, double arrow head extend=5pt,
      minimum height=12mm,
      rotate=90] {};
    \end{tikzpicture} \\[0.25cm]
    \begin{tikzpicture}
        \node [] (input) at (-3,0){$x(t)$};
        \node [] (output) at (+6,0){$y(t)$};
        \node [draw,
            fill=blue!30, 
            minimum width=2cm, 
            minimum height=1.2cm
        ] (system) at (1.5,0){$h_1(t)+h_2(t)$};
        \draw [-latex, line width=1.5pt] (input) -- (system);
        \draw [-latex, line width=1.5pt] (system) -- (output);
    \end{tikzpicture}
\end{center}

\begin{example}
    Consider the following interconnection of LTI systems. Find the equivalent impulse response for the interconnection. \\ \\
    \resizebox{\textwidth}{!}{
    \begin{tikzpicture}
        \node [] (input) at (-3,0){\LARGE $x(t)$};
        \node [] (output) at (+14.5,0){\LARGE $y(t)$};
        \node [draw,
            fill=yellow!50, 
            minimum width=2cm, 
            minimum height=1.2cm
        ] (sys1) at (0,0){\LARGE $h_1(t)$};
        \node [draw,
            fill=orange!50, 
            minimum width=2cm, 
            minimum height=1.2cm
        ] (sys2) at (3.5,2){\LARGE $h_2(t)$};
        \node [draw,
            fill=green!30, 
            minimum width=2cm, 
            minimum height=1.2cm
        ] (sys3) at (8,2){\LARGE $h_3(t)$};
        \node [draw,
            fill=blue!30, 
            minimum width=2cm, 
            minimum height=1.2cm
        ] (sys4) at (8,0){\LARGE $h_4(t)$};
        \node [draw,
            fill=violet!30, 
            minimum width=2cm, 
            minimum height=1.2cm
        ] (sys5) at (8,-2){\LARGE $h_5(t)$};
        \node[draw,
            circle,
            minimum size=0.6cm,
            fill=red!30
        ] (sum) at (12,0){\Huge +};
        \draw [-latex, line width=1.5pt] (input) -- (sys1);
        \draw [-, line width=1.5pt] (sys1) -- (0,2);
        \draw [-latex, line width=1.5pt] (0,2) -- (sys2);
        \draw [-, line width=1.5pt] (sys1) -- (0,-2);
        \draw [-latex, line width=1.5pt] (0,-2) -- (sys5);
        \draw [-latex, line width=1.5pt] (sys2) -- (sys3);
        \draw [-, line width=1.5pt] (5.75,2) -- (5.75,0);
        \draw [-latex, line width=1.5pt] (5.75,0) -- (sys4);
        \draw [-, line width=1.5pt] (sys3) -- (12,2);
        \draw [-, line width=1.5pt] (sys5) -- (12,-2);
        \draw [-latex, line width=1.5pt] (12,2) -- (sum);
        \draw [-latex, line width=1.5pt] (sys4) -- (sum);
        \draw [-latex, line width=1.5pt] (12,-2) -- (sum);
        \draw [-latex, line width=1.5pt] (sum) -- (output);
    \end{tikzpicture}}
\end{example}
\begin{solution}
    In the top-right corner of the interconnection, two systems are in parallel. This parallel set is itself in series with $h_2(t)$. 
    Therefore, the top branch can be rewritten as 
    \begin{align*}
        h_2(t)*[h_3(t)+h_4(t)].
    \end{align*}
    The top branch equivalent and the bottom branch are in parallel and can be rewritten as 
    \begin{align*}
        h_2(t)*[h_3(t)+h_4(t)]+h_5(t).
    \end{align*}
    Lastly, the two-branch equivalent and $h_1(t)$ are in series such that the overall equivalent impulse response is 
    \begin{align*}
        h(t) = h_1(t) * \{h_2(t)*[h_3(t)+h_4(t)]+h_5(t)\}.
    \end{align*}
\end{solution}

\end{document}
