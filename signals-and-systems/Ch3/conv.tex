\documentclass{report}
% PACKAGES
\usepackage{adjustbox}
\usepackage{amsmath}
\usepackage{amssymb}
\usepackage{bodegraph}
\usepackage{bbm}
\usepackage{circledsteps}
\usepackage{circuitikz}
\usepackage{enumerate}
\usepackage{mathtools}
\usepackage{nicematrix}
\usepackage{pgfplots}
\usepackage{polynom}
\usepackage{qtree}
\usepackage{rotating}
\usepackage[usestackEOL]{stackengine}
\usepackage[subpreambles=true]{standalone}
\usepackage{steinmetz}
\usepackage{subcaption}
\usepackage{tabularray}
\usepackage{tcolorbox}
\usepackage{tikz}
\usepackage{xcolor}

\usepackage[colorlinks=true,linkcolor=blue,urlcolor=black,bookmarksopen=true]{hyperref}
\usepackage{bookmark}

\usetikzlibrary{shapes.arrows}
\usetikzlibrary{shapes.misc}
\usetikzlibrary{backgrounds}
\tikzset{cross/.style={cross out, draw=black, minimum size=2*(#1-\pgflinewidth), inner sep=0pt, outer sep=0pt},
%default radius will be 1pt. 
cross/.default={1pt}}

\renewcommand{\Re}{\operatorname{Re}}
\renewcommand{\Im}{\operatorname{Im}}

\usepackage{pifont}
\newcommand{\cmark}{\text{\ding{51}}}
\newcommand{\xmark}{\text{\ding{55}}}

\newcommand{\circconv}[1]{\text{ \small\Circled{#1} }}

\newcommand{\tikzmark}[3][]{\tikz[remember picture,baseline] \node [anchor=base,#1](#2) {$#3$};}

\pgfplotsset{compat=1.18}
\pgfplotsset{
    dirac/.style={
        mark=triangle*,
        mark options={scale=1.5},
        ycomb,
        scatter,
        visualization depends on={y/abs(y)-1 \as \sign},
        scatter/@pre marker code/.code={\scope[rotate=90*\sign,yshift=-2pt]}
    }
}

\usepackage[letterpaper, portrait, margin=1.25in]{geometry}
\usepackage[font=bf]{caption}
\hbadness = 10000
\hfuzz=2pt

\newtheorem{theorem}{Theorem}[chapter]

\tikzset{every tree node/.style={anchor=north,align=center}}
\usetikzlibrary{decorations.markings}
\usetikzlibrary{arrows}
\definecolor{darkgreen}{rgb}{0.133,0.545,0.133}
\tikzstyle{n}= [circle, fill=blue, minimum size=4pt,inner sep=0pt, outer sep=0pt]

\patchcmd{\thebibliography}{\chapter*}{\section*}{}{}
\setcounter{secnumdepth}{5}

%%%%%%%%%%% EXAMPLE ENVIRONMENT %%%%%%%%%%
\usepackage{calc}
\usepackage{tabto}
\usepackage[framemethod=tikz]{mdframed} % for colored backgrounds
\newcommand{\halmos}{} % makes a box at the end

\newlength{\framedinnerleftmargin}
\newlength{\framedinnertopmargin}
\newlength{\framedreversedinnerleftmargin}
\setlength{\framedinnerleftmargin}{\widthof{Theoreme 10.10.10}+2em}
\setlength{\framedreversedinnerleftmargin}{\widthof{Theoreme 10.10.10}+1em}
\setlength{\framedinnertopmargin}{1em}
 
% first argument: label in upper left corner,
% second argument: background color
\newenvironment{boxedtext}[2]{\begin{mdframed}[%
hidealllines=true,%
backgroundcolor=#2,%
innertopmargin=\framedinnertopmargin,%
innerleftmargin=\framedinnerleftmargin,%
innerrightmargin=1em%
]%
\tabto{-\framedreversedinnerleftmargin}\textbf{#1}\tabto*{0em}%
}% begin code
{\hskip 0pt\\\hspace*{\fill}\halmos{}\end{mdframed}\vspace{1em}} % end code
 
\newenvironment{summary}[0]{\begin{center}\begin{minipage}[c]{\summarywidth}\begin{spacing}{0.9}\footnotesize} % begin code
{\end{spacing}\end{minipage}\end{center}} % end code

\newcounter{example}
 
% optional! if you want it to start at zero
% with every new chapter/section/etc.
\numberwithin{example}{section}
 
\newenvironment{example}[0]
{\refstepcounter{example}\vspace{1em plus 1em}\begin{boxedtext}{Example \theexample.}{blue!7}}%\setlength{\parskip}{0em}}
{\end{boxedtext}\vspace{-1em plus 1em}}
 
\newenvironment{example*}[0]
{\vspace{1em plus 1em}\begin{boxedtext}{Example.}{blue!7}}
{\end{boxedtext}\vspace{-1em plus 1em}}

\newmdenv[
  topline=false,
  bottomline=false,
  rightline=false,
  skipabove=\topsep,
  skipbelow=\topsep,
  linecolor=purple,
  frametitle={\noindent\textcolor{purple}{\textbf{SOLUTION }}},
  endinnercode={$\hfill\textcolor{purple}{\blacksquare}$}
]{solution}

%%%%%%%%%%%%%% COLOR BOXED %%%%%%%%%%%%%%%
% Syntax: \colorboxed[<color model>]{<color specification>}{<math formula>}
\newcommand*{\colorboxed}{}
\def\colorboxed#1#{%
  \colorboxedAux{#1}%
}
\newcommand*{\colorboxedAux}[3]{%
  % #1: optional argument for color model
  % #2: color specification
  % #3: formula
  \begingroup
    \colorlet{cb@saved}{.}%
    \color#1{#2}%
    \boxed{%
      \color{cb@saved}%
      #3%
    }%
  \endgroup
}
%%%%%%%%%%%%%%%%%%%%%%%%%%%%%%%%%%%%%%%%%

\begin{document}
\setcounter{chapter}{2}
\setcounter{page}{34}
\chapter{Convolution}
In the previous chapter, we were introduced to the concept of \emph{convolution}, an operation ($*$) defined by the \emph{convolution integral}
\begin{align*}
    x(t) * h(t) &= \int_{-\infty}^{+\infty} x(\tau)h(t-\tau) \,d\tau,
\end{align*}
We have seen that the operation is commutative since by change of variables, 
\begin{align*}
    x(t) * h(t) &= \int_{-\infty}^{+\infty} x(\tau)h(t-\tau) \,d\tau \\
    &= \int_{-\infty}^{+\infty} x(t-\tau)h(\tau) \,d\tau = h(t) * x(t).
\end{align*}
We have also already seen another convolution property in effect with the sampling property of the impulse function. 
\begin{align*}
    \int_{-\infty}^{+\infty} x(t)\delta(t-T) \,dt &= x(T) \\
    \Longrightarrow \int_{-\infty}^{+\infty} x(\tau)\delta(\tau-t) \,d\tau &= \int_{-\infty}^{+\infty} x(\tau)\delta(t-\tau) \,d\tau = x(t) \\
    x(t) * \delta(t) &= x(t).
\end{align*}
We also introduced the concept that any LTI system characterized by impulse response $h(t)$ will always have predictable outputs since the output is simply the 
convolution of the input signal and the impulse response, provided that all initial conditions of the system are zero. That is, 
\begin{align}
    y(t) = x(t) * h(t) = \int_{-\infty}^{+\infty} x(\tau)h(t-\tau) \,d\tau.
\end{align}
We can even take this one step further. Since the physical world can only have causal signals and causal systems (with $x(t)$ and $h(t)$ both being causal) the output $y(t)$ must be causal as well. 
This is reflected in a modified version of the convolution integral, 
\begin{align}
    y(t) = u(t) \int_{0}^{t} x(\tau)h(t-\tau) \,d\tau = u(t) \int_{0}^{t} h(\tau)x(t-\tau) \,d\tau.
\end{align}
In this chapter, different techniques to computing the convolution of two functions are explored. 

\section{Analytical Convolution}
One way to compute the convolution is to directly solve the convolution integral itself. Refer to Table \ref{conv_prop} for some 
convolution properties.
\begin{table}[hbt!]
    \centering
    \caption{Convolution properties.}
    \label{conv_prop}
    \begin{tabular}{|c|c|}
        \hline
        Property & Description \\[0.15cm]
        \hline
        Commutativity & $x(t)*h(t)=h(t)*x(t)$ \\[0.5cm]
        Associativity & $x(t)*[h(t)*g(t)]=[x(t)*h(t)]*g(t)$ \\[0.5cm]
        Distributivity & $x(t)*[h_1(t)+\cdots+h_N(t)]=[x(t)*h_1(t)]+\cdots+[x(t)*h_N(t)]$  \\[0.5cm]
        Time-shift & $x(t-T_1)*h(t-T_2)=y(t-T_1-T_2)$ \\[0.5cm]
        Convolution with impulse & $x(t)*\delta(t)=x(t)$ \\[0.5cm]
        Convolution with step & $x(t)*u(t)=\displaystyle\int_{-\infty}^{t} x(\tau) \,d\tau$ (ideal integrator) \\[0.5cm]
        Causal signals and systems & $y(t)=u(t)\displaystyle\int_{0}^{t} h(\tau)x(t-\tau) \,d\tau$ \\[0.5cm]
        Width & $\text{Width}[y(t)]=\text{Width}[x(t)]+\text{Width}[h(t)]$ \\[0.5cm]
        Area & $\text{Area}[y(t)]=\text{Area}[x(t)]+\text{Area}[h(t)]$ \\[0.5cm]
        Differentation & $\left(\dfrac{d^mx(t)}{dt^m}\right) * \left(\dfrac{d^nh(t)}{dt^n}\right) = \dfrac{d^{m+n}y(t)}{dt^{m+n}}$ \\[0.5cm]
        Integration & $\displaystyle\int_{-\infty}^{t} y(\tau) \,d\tau = x(t) * \left[\displaystyle\int_{-\infty}^{t} h(\tau) \,d\tau\right] = \left[\displaystyle\int_{-\infty}^{t} x(\tau) \,d\tau\right] * h(t)$ \\[0.5cm]
        \hline
    \end{tabular}
\end{table} 
\\
When integrating two causal functions, one of the functions will become time-reversed in the convolution integral. We can use this to our advantage in redefining the limits of integration. That is, for 
$f(\tau)$ representing the non-step factors, 
\begin{align}
    \int_{-\infty}^{+\infty} f(\tau) \cdot \underbrace{u(\tau-a)}_\textrm{$\tau-a>0$} \cdot \underbrace{u(b-\tau)}_\textrm{$b-\tau>0$} \,d\tau = u(b-a)\int_{a}^{b} f(\tau) \,d\tau.
\end{align}
The inequalities above are defined for when the step functions are ``switched on'', and combining the two inequalities gives us an interval of validity $a<\tau<b$, which is what is used to update the convolution integral.

\begin{example}
    Find $y(t)=x(t)*h(t)$ for 
    \begin{align*}
        x(t) &= t e^{-2t}u(t), \\
        h(t) &= e^{-3t}u(t).
    \end{align*}
\end{example}
\begin{solution}
    Using the convolution integral, it follows that 
    \begin{align*}
        y(t) = [t e^{-2t}u(t)] * [e^{-3t}u(t)] &= \int_{-\infty}^{+\infty} \tau e^{-2\tau} \underbrace{u(\tau)}_\textrm{$\tau>0$} e^{-3(t-\tau)}\underbrace{u(t-\tau)}_\textrm{$\tau<t$} \,d\tau \\
        &= u(t-0) \int_{0}^{t} \tau e^{-2\tau} e^{-3(t-\tau)} \,d\tau \\
        &= e^{-3t}u(t) \int_{0}^{t} \tau e^{\tau} \,d\tau = e^{-3t}[te^t-e^t+1]u(t).
    \end{align*}
\end{solution}

\begin{example}
    Find $y(t)=x(t)*h(t)$ for 
    \begin{align*}
        x(t) &= u(t-2)-u(t-4), \\
        h(t) &= u(t-3)-u(t-5).
    \end{align*}
\end{example}
\begin{solution}
    Using the convolution integral and the distributive property, it follows that 
    \begin{align*}
        y(t) &= [u(t-2)-u(t-4)] * [u(t-3)-u(t-5)] \\
        &= u(t-2) * u(t-3) - u(t-2) * u(t-5) - u(t-4) * u(t-3) + u(t-4) * u(t-5) \\
        &= \int_{-\infty}^{+\infty} \underbrace{u(\tau-2)}_\textrm{$\tau>2$} \underbrace{u((t-3)-\tau)}_\textrm{$\tau<t-3$} \,d\tau 
        -\int_{-\infty}^{+\infty} \underbrace{u(\tau-2)}_\textrm{$\tau>2$} \underbrace{u((t-5)-\tau)}_\textrm{$\tau<t-5$} \,d\tau \\
        & \indent -\int_{-\infty}^{+\infty} \underbrace{u(\tau-4)}_\textrm{$\tau>4$} \underbrace{u((t-3)-\tau)}_\textrm{$\tau<t-3$} \,d\tau 
        +\int_{-\infty}^{+\infty} \underbrace{u(\tau-4)}_\textrm{$\tau>4$} \underbrace{u((t-5)-\tau)}_\textrm{$\tau<t-5$} \,d\tau.
    \end{align*}
    Simplifying the integrals, we get
    \begin{align*}
        y(t) &= u((t-3)-2)\int_{2}^{t-3} \,d\tau -u((t-5)-2)\int_{2}^{t-5} \,d\tau \\
        & \indent -u((t-3)-4)\int_{4}^{t-3} \,d\tau +u((t-5)-4)\int_{4}^{t-5} \,d\tau \\
        &= (t-5)u(t-5) -2(t-7)u(t-7) +(t-9)u(t-9) \\
        &= r(t-5) - 2r(t-7) + r(t-9).
    \end{align*}
\end{solution}

While we are mostly interested in intervals of validity that follow causal signals and systems, there are other possible intervals of validity for different 
types of systems that may or may not be causal. Refer to Table \ref{int_redef} for these integral redefinitions. 
\begin{table}[hbt!]
    \centering
    \caption{Integral redefinitions based on unit step functions.}
    \label{int_redef}
    \begin{mdframed}
    \begin{align}
        \int_{-\infty}^{+\infty} f(\tau) u(\tau-a) \,d\tau &= \int_{a}^{+\infty} f(\tau) \,d\tau \\[0.25cm]
        \int_{-\infty}^{+\infty} f(\tau) u(b-\tau) \,d\tau &= \int_{-\infty}^{b} f(\tau) \,d\tau \\[0.25cm]
        \int_{-\infty}^{+\infty} f(\tau) u(\tau-a) u(b-\tau) \,d\tau &= u(b-a)\int_{a}^{b} f(\tau) \,d\tau \\[0.25cm]
        \int_{-\infty}^{+\infty} f(\tau) u(\tau-a) u(\tau-b) \,d\tau &= \int_{\max(a,b)}^{+\infty} f(\tau) \,d\tau \nonumber \\ 
        &= u(a-b) \int_{a}^{+\infty} f(\tau) \,d\tau + u(b-a) \int_{b}^{+\infty} f(\tau) \,d\tau \\[0.25cm] 
        \int_{-\infty}^{+\infty} f(\tau) u(a-\tau) u(b-\tau) \,d\tau &= \int_{-\infty}^{\min(a,b)} f(\tau) \,d\tau \nonumber \\ 
        &= u(b-a) \int_{-\infty}^{a} f(\tau) \,d\tau + u(a-b) \int_{-\infty}^{b} f(\tau) \,d\tau \\[0.15cm] \nonumber
    \end{align}
    \end{mdframed}
\end{table} 

\begin{example}
    Find $y(t)=x(t)*h(t)$ for 
    \begin{align*}
        x(t) &= e^{-t}u(t-3), \\
        h(t) &= u(-t).
    \end{align*} 
\end{example}
\begin{solution}
    Using the convolution integral, it follows that 
    \begin{align*}
        y(t) = [e^{-t}u(t-3)] * [u(-t)] &= \int_{-\infty}^{+\infty} e^{-\tau} \underbrace{u(\tau-3)}_\textrm{$\tau>3$} \underbrace{u(\tau-t)}_\textrm{$\tau>t$} \,d\tau \\
        &= \int_{\max(3,t)}^{+\infty} e^{-\tau} u(\tau-3)u(\tau-t) \,d\tau \\
        &= u(3-t)\int_{3}^{+\infty} e^{-\tau} \,d\tau + u(t-3)\int_{t}^{+\infty} e^{-\tau} \,d\tau \\
        &= e^{-3}u(3-t)+e^{-t}u(t-3).
    \end{align*}
\end{solution}

\begin{table}[hbt!]
    \centering
    \caption{Commonly encountered convolutions.}
    \label{common_conv}
    \begin{mdframed}
    \begin{align}
        u(t)*u(t) &= t\cdot u(t) \\[0.25cm]
        e^{at}u(t)*u(t) &= \left(\frac{e^{at}-1}{a}\right) u(t) \\[0.25cm]
        e^{at}u(t)*e^{bt}u(t) &= \left(\frac{e^{at}-e^{bt}}{a-b}\right) u(t) \\[0.25cm]
        e^{at}u(t)*e^{at}u(t) &= te^{at} u(t) \\[0.25cm]
        te^{at}u(t)*e^{bt}u(t) &= \left(\frac{e^{bt}-e^{at}+(a-b)te^{at}}{(a-b)^2}\right) u(t) \\[0.25cm]
        te^{at}u(t)*e^{at}u(t) &= \frac{1}{2} \cdot t^2 e^{at} u(t) \\[0.25cm]
        \delta(t-T_1)*\delta(t-T_2) &= \delta(t-T_1-T_2) \\[0.15cm] \nonumber
    \end{align}
    \end{mdframed}
\end{table} 

\section{Pointwise Graphical Convolution}
Another way to compute the convolution is to interpret the convolution graphically. Here, we will first introduce the pointwise method, 
best used when only a few samples of the output are of interest. Since 
\begin{align*}
    y(t) = x(t) * h(t) = \int_{-\infty}^{+\infty} x(\tau)h(t-\tau) \,d\tau,
\end{align*}
the output at a certain single time $t$ is given by the area under the resultant curve formed by multiplying $x(\tau)$ and $h(t-\tau)$, which is a time-reversed and translated 
version of $h(\tau)$. 

\begin{tcolorbox}[width=\textwidth,colback={white}, sharp corners]
    \noindent\emph{POINTWISE GRAPHICAL CONVOLUTION.}
    \begin{enumerate}[Step 1:]
        \item Determine the set of time values of interest $t_0, ..., t_N$.
        \item Plot $x(\tau)$ and $h(-\tau)$ along the $\tau$-axis.
        \item Let $t=t_0$ and shift $h(-\tau)$ to the right by $t$ units to represent $h(t-\tau)$.
        \item Using the region of overlap between $x(\tau)$ and $h(t-\tau)$, there are many options to solve:
        \begin{enumerate}
            \item if one curve is constant during the overlap and the other curve has a simple area under the curve, then $y(t)$ is the area under the curve but scaled by the constant.
            \item if both curves are linear during the overlap, then point-by-point multiplication can be performed, and the area under the new curve is taken to be $y(t)$.
            \item in general, integration during the overlap will always work.
        \end{enumerate}
        \item Repeat Steps 2--4 for the remaining time values of interest.
    \end{enumerate}
\end{tcolorbox}

\begin{example}
    Find $y(t)=x(t)*h(t)$, given the following plots. \\ \\
    \begin{tikzpicture}
        [declare function={
            func(\x)= (\x < 1) * (0) + and(\x >= 1, \x < 2) * (4) + (\x > 2) * (0); }]
        \begin{axis}[
            axis x line=center, axis y line=center,
            ymin=0, ymax=4.5, ytick={0,2,4}, ylabel={$x(t)$},
            xmin=0, xmax=4, xtick={0,...,2}, xlabel={$t$ [s]},
            domain=0:3,samples=200,
            width=5cm, height=4cm]
        \addplot [blue,thick]{func(x)};
        \end{axis}
    \end{tikzpicture}
    \begin{tikzpicture}
        [declare function={
            func(\x)= (\x < 0) * (0) + and(\x >= 0, \x < 3) * (\x) + (\x > 3) * (0); }]
        \begin{axis}[
            axis x line=center, axis y line=center,
            ymin=0, ymax=4.5, ytick={0,3}, ylabel={$h(t)$},
            xmin=0, xmax=4, xtick={0,...,3}, xlabel={$t$ [s]},
            domain=0:3.5,samples=200,
            width=5cm, height=4cm]
        \addplot [red,thick]{func(x)};
        \end{axis}
    \end{tikzpicture}
\end{example}
\begin{solution}
    From the width property of convolution, we know that the width of the output is the sum of the two widths, 
    which is 5 seconds. \\ \\ 
    We also know that from the time-shift property of the convolution, since $x(t)$ is excited at $t=0$ and 
    $h(t)$ is excited at $t=1$, the output $y(t)$ must be excited at $t=1$. \\ \\ 
    From this information alone, we can already deduce that 
    the output is only nonzero when $1<t<6$. Therefore, it is wise to choose values during that interval. 
    \\ \\
    Suppose we are interested in only finding $y(2)$. First, we plot $x(\tau)$ and $h(-\tau)$ on the $\tau$-axis.
    \begin{tikzpicture}
        [declare function={
            func1(\x) = (\x < 1) * (0) + and(\x >= 1, \x < 2) * (4) + (\x > 2) * (0); 
            func2(\x)= (\x < 0) * (0) + and(\x >= 0, \x < 3) * (\x) + (\x > 3) * (0); }]
        \begin{axis}[
            axis x line=center, axis y line=center,
            ymin=0, ymax=4.5, ytick={0,...,4}, ylabel={ },
            xmin=-4, xmax=4, xtick={-3,...,3}, xlabel={$\tau$ [s]},
            domain=-3.5:3.5,samples=200,
            width=8cm, height=4cm]
        \addplot [blue,thick]{func1(x)};
        \addplot [red,thick]{func2(-x)};
        \end{axis}
    \end{tikzpicture}
    \\ \\
    Since we want to observe at time $t=2$, we shift the red plot to the right by 2 units: \\ \\
    \begin{tikzpicture}
        [declare function={
            func1(\x) = (\x < 1) * (0) + and(\x >= 1, \x < 2) * (4) + (\x > 2) * (0); 
            func2(\x)= (\x < 0) * (0) + and(\x >= 0, \x < 3) * (\x) + (\x > 3) * (0); }]
        \begin{axis}[
            axis x line=center, axis y line=center,
            ymin=0, ymax=4.5, ytick={0,...,4}, ylabel={ },
            xmin=-4, xmax=4, xtick={-3,...,3}, xlabel={$\tau$ [s]},
            domain=-3.5:3.5,samples=200,
            width=8cm, height=4cm]
        \addplot [blue,thick]{func1(x)};
        \addplot [red,thick]{func2(-x+2)};
        \end{axis}
    \end{tikzpicture}
    \\ \\
    Here, we notice that the region of overlap occurs during $1<\tau<2$. Note that the area under the red curve during that time is $\frac{1}{2}\times 1 \times 1 = \frac{1}{2}$, and the constant value 
    of the blue plot during that time is 4. Therefore, $y(2)$ is the scaled area under the curve during the region of overlap. That is, $y(2)=\frac{1}{2}\times 4=2$. 
    \\ \\ 
    Alternatively, we could have performed point-by-point multiplication and find that the area under the green curve in the following plot is still 2. \\ \\
    \begin{tikzpicture}
        [declare function={
            func1(\x) = (\x < 1) * (0) + and(\x >= 1, \x < 2) * (4) + (\x > 2) * (0); 
            func2(\x)= (\x < 0) * (0) + and(\x >= 0, \x < 3) * (\x) + (\x > 3) * (0); }]
        \begin{axis}[
            axis x line=center, axis y line=center,
            ymin=0, ymax=4.5, ytick={0,...,4}, ylabel={ },
            xmin=-4, xmax=4, xtick={-3,...,3}, xlabel={$\tau$ [s]},
            domain=-3.5:3.5,samples=200,
            width=8cm, height=4cm]
        \addplot [blue,thick]{func1(x)};
        \addplot [red,thick]{func2(-x+2)};
        \addplot [darkgreen,thick]{func2(-x+2)*func1(x)};
        \end{axis}
    \end{tikzpicture}
    \\ \\ 
    Of course, integration generally always works as well. During the region of overlap, we can let $x(\tau)=4$ and $h(2-\tau)=2-\tau$ such that 
    \begin{align*}
        y(2) = \int_{1}^{2} x(\tau)h(2-\tau) \,d\tau = \int_{1}^{2} 4(2-\tau) \,d\tau = 2.
    \end{align*}
    We can repeat the process as many times as needed for other values of $t\in(1,6)$ until we get enough samples to sketch $y(t)$.
    \\ \\
\end{solution}

\section{Region-Based Graphical Convolution}
The pointwise method is impractical if we are interested in the output response over a larger, if not entire, time domain. Here, we introduce the region-based 
graphical convolution, which utilizes a signal's \emph{leading and trailing edges}, if any. 

\begin{tcolorbox}[width=\textwidth,colback={white}, sharp corners]
    \noindent\emph{REGION-BASED GRAPHICAL CONVOLUTION.}
    \begin{enumerate}[Step 1:]
        \item Plot $x(\tau)$ and $h(-\tau)$ along the $\tau$-axis, with $h(-\tau)$ being the simpler function to flip.
        \item Label the leading and trailing edges of $h(-\tau)$.
        \item Identify a region of overlap and use the edges to determine the interval of $t$-values. Then evaluate the convolution integral during that region of overlap. 
        \item Repeat Steps 2--3 for any other regions of overlap when shifting $h(-\tau)$ from left-to-right along the $\tau$-axis. Keep track of the width of $h(t-\tau)$. 
    \end{enumerate}
\end{tcolorbox}

\begin{example}
    Find $y(t)=x(t)*h(t)$, given the following plots. \\ \\
    \begin{tikzpicture}
        [declare function={
            func(\x)= (\x < 0) * (0) + and(\x >= 0, \x < 2) * (0.5 * \x) + (\x > 2) * (0); }]
        \begin{axis}[
            axis x line=center, axis y line=center,
            ymin=0, ymax=2.5, ytick={0,1,2}, ylabel={$x(t)$},
            xmin=0, xmax=3, xtick={0,...,2}, xlabel={$t$ [s]},
            domain=0:3,samples=200,
            width=5cm, height=3cm]
        \addplot [blue,thick]{func(x)};
        \end{axis}
    \end{tikzpicture}
    \begin{tikzpicture}
        [declare function={
            func(\x)= (\x < -2) * (0) + and(\x >= -2, \x < 2) * (2) + (\x > 2) * (0); }]
        \begin{axis}[
            axis x line=center, axis y line=center,
            ymin=0, ymax=2.5, ytick={0,1,2}, ylabel={$h(t)$},
            xmin=-2.5, xmax=2.5, xtick={-2,...,2}, xlabel={$t$ [s]},
            domain=-3:3,samples=200,
            width=6cm, height=3cm]
        \addplot [red,thick]{func(x)};
        \end{axis}
    \end{tikzpicture}
\end{example}
\begin{solution}
    First, plot $x(\tau)$ and $h(-\tau)$. Then identify the leading and trailing edges of $h(-\tau)$. \\ \\
    \begin{tikzpicture}
        [declare function={
            func(\x)= (\x < 0) * (0) + and(\x >= 0, \x < 2) * (0.5 * \x) + (\x > 2) * (0); }]
        \begin{axis}[
            axis x line=center, axis y line=center,
            ymin=0, ymax=3, ytick={0,1,2}, ylabel={$x(\tau)$},
            xmin=0, xmax=3, xtick={0,...,2}, xlabel={$\tau$ [s]},
            domain=0:3,samples=200,
            width=7cm, height=4cm]
        \addplot [blue,thick]{func(x)};
        \end{axis}
    \end{tikzpicture}
    \begin{tikzpicture}
        [declare function={
            func(\x)= (\x < -2) * (0) + and(\x >= -2, \x < 2) * (2) + (\x > 2) * (0); }]
        \begin{axis}[
            axis x line=center, axis y line=center,
            ymin=0, ymax=3, ytick={0,1,2}, ylabel={$\big[h(t-\tau)\big|_{t=0} = h(-\tau)$},
            xmin=-3, xmax=3, xtick={-2,...,2}, xlabel={ },
            domain=-3:3,samples=200,
            width=8cm, height=4cm]
        \addplot [red,thick]{func(-x)} node[above,pos=0.05] {\scriptsize \shortstack{Trailing \\ edge, \\ $t-2$}};
        \addplot [red,thick]{func(-x)} node[above,pos=0.95] {\scriptsize \shortstack{Leading \\ edge, \\ $t+2$}};
        \end{axis}
    \end{tikzpicture}
    \\
    From $h(-\tau)$ above, we see that the leading edge is $\tau=t+2$ and the trailing edge is $\tau=t-2$. Now we place $h(t-\tau)$ to the furthest left and gradually shift to the right. 
    \\ \\ \\
    \underline{Region 1: No overlap} \\ \\
    \begin{tabular}{ cc }
        \begin{tabular}{ c }
            \begin{tikzpicture}
                [declare function={
                    func1(\x)= (\x < 0) * (0) + and(\x >= 0, \x < 2) * (0.5 * \x) + (\x > 2) * (0);  }]
                \begin{axis}[
                    axis x line=center, axis y line=center,
                    ymin=0, ymax=2.5, ytick={0,1,2}, ylabel={$x(\tau)$},
                    xmin=-2.5, xmax=2.5, xtick={-2,...,2}, xlabel={$\tau$},
                    domain=-3:3,samples=200,
                    width=6cm, height=3cm]
                \addplot [blue,thick]{func1(x)};
                \end{axis}
            \end{tikzpicture} \\
            \begin{tikzpicture}
                [declare function={
                    func2(\x)= (\x < -2) * (0) + and(\x >= -2, \x < 2) * (2) + (\x > 2) * (0); }]
                \begin{axis}[
                    axis x line=center, axis y line=center,
                    ymin=0, ymax=3, ytick={0,2}, ylabel={$h(t-\tau)$},
                    xmin=-2.5, xmax=2.5, xtick={0}, xlabel={$\tau$},
                    extra x ticks={-5,-1}, extra x tick labels={$t-2$,$t+2$}, 
                    domain=-3:3,samples=200,
                    width=6cm, height=3cm]
                \addplot [red,thick]{func2(-x-3)};
                \end{axis}
            \end{tikzpicture}
        \end{tabular} & 
        \begin{tabular}{ c }
            Leading edge: $t+2 < 0 \Rightarrow t < -2$ \\ 
            \\
            No overlap, so $y(t)=0$ for $t<-2.$
        \end{tabular}
    \end{tabular} 
    \\ \\ 
    \noindent \underline{Region 2: Partial overlap} \\ \\
    \begin{tabular}{ cc }
        \begin{tabular}{ c }
            \begin{tikzpicture}
                [declare function={
                    func1(\x)= (\x < 0) * (0) + and(\x >= 0, \x < 2) * (0.5 * \x) + (\x > 2) * (0);  }]
                \begin{axis}[
                    axis x line=center, axis y line=center,
                    ymin=0, ymax=2.5, ytick={0,1,2}, ylabel={$x(\tau)$},
                    xmin=-2.5, xmax=2.5, xtick={-2,...,2}, xlabel={$\tau$},
                    domain=-3:3,samples=200,
                    width=6cm, height=3cm]
                \addplot [blue,thick]{func1(x)};
                \end{axis}
            \end{tikzpicture} \\
            \begin{tikzpicture}
                [declare function={
                    func2(\x)= (\x < -2) * (0) + and(\x >= -2, \x < 2) * (2) + (\x > 2) * (0); }]
                \begin{axis}[
                    axis x line=center, axis y line=center,
                    ymin=0, ymax=3, ytick={0,2}, ylabel={$h(t-\tau)$},
                    xmin=-2.5, xmax=2.5, xtick={0}, xlabel={$\tau$},
                    extra x ticks={-3,1}, extra x tick labels={$t-2$,$t+2$}, 
                    domain=-3:3,samples=200,
                    width=6cm, height=3cm]
                \addplot [red,thick]{func2(-x-1)};
                \end{axis}
            \end{tikzpicture} \\
            \begin{tikzpicture}
                [declare function={
                    func1(\x)= (\x < 0) * (0) + and(\x >= 0, \x < 2) * (0.5 * \x) + (\x > 2) * (0);
                    func2(\x)= (\x < -2) * (0) + and(\x >= -2, \x < 2) * (2) + (\x > 2) * (0); }]
                \begin{axis}[
                    axis x line=center, axis y line=center,
                    ymin=0, ymax=3, ytick={0,1,2}, ylabel={$x(\tau)h(t-\tau)$},
                    xmin=-2.5, xmax=2.5, xtick={0,2}, xlabel={$\tau$},
                    extra x ticks={1}, extra x tick labels={$t+2$}, 
                    domain=-3:3,samples=200,
                    width=6cm, height=3cm]
                \addplot [darkgreen,thick]{func1(x) * func2(-x-1)};
                \end{axis}
            \end{tikzpicture}
        \end{tabular} & 
        \begin{tabular}{ c }
            Leading edge: $0 < t+2 < 2 \Rightarrow -2 < t < 0$ \\ 
            \\
            $y(t) = \displaystyle\int_{-\infty}^{+\infty} x(\tau)h(t-\tau) \,d\tau$ \\ \\
            $\Longrightarrow y(t) = \displaystyle\int_{0}^{t+2} \left[\frac{1}{2}\tau\right][2] \,d\tau=\frac{1}{2}(t+2)^2$
        \end{tabular}
    \end{tabular} 
    \\[2.5cm]
    \underline{Region 3: Complete overlap} \\ \\
    \begin{tabular}{ cc }
        \begin{tabular}{ c }
            \begin{tikzpicture}
                [declare function={
                    func1(\x)= (\x < 0) * (0) + and(\x >= 0, \x < 2) * (0.5 * \x) + (\x > 2) * (0);  }]
                \begin{axis}[
                    axis x line=center, axis y line=center,
                    ymin=0, ymax=2.5, ytick={0,1,2}, ylabel={$x(\tau)$},
                    xmin=-1.5, xmax=3.5, xtick={-1,...,3}, xlabel={$\tau$},
                    domain=-3:3,samples=200,
                    width=6cm, height=3cm]
                \addplot [blue,thick]{func1(x)};
                \end{axis}
            \end{tikzpicture} \\
            \begin{tikzpicture}
                [declare function={
                    func2(\x)= (\x < -2) * (0) + and(\x >= -2, \x < 2) * (2) + (\x > 2) * (0); }]
                \begin{axis}[
                    axis x line=center, axis y line=center,
                    ymin=0, ymax=3, ytick={0,2}, ylabel={$h(t-\tau)$},
                    xmin=-1.5, xmax=3.5, xtick={0}, xlabel={$\tau$},
                    extra x ticks={-1,3}, extra x tick labels={$t-2$,$t+2$}, 
                    domain=-3:3,samples=200,
                    width=6cm, height=3cm]
                \addplot [red,thick]{func2(-x+1)};
                \end{axis}
            \end{tikzpicture} \\
            \begin{tikzpicture}
                [declare function={
                    func1(\x)= (\x < 0) * (0) + and(\x >= 0, \x < 2) * (0.5 * \x) + (\x > 2) * (0);
                    func2(\x)= (\x < -2) * (0) + and(\x >= -2, \x < 2) * (2) + (\x > 2) * (0); }]
                \begin{axis}[
                    axis x line=center, axis y line=center,
                    ymin=0, ymax=3, ytick={0,1,2}, ylabel={$x(\tau)h(t-\tau)$},
                    xmin=-1.5, xmax=3.5, xtick={0,2}, xlabel={$\tau$},
                    domain=-3:3,samples=200,
                    width=6cm, height=3cm]
                \addplot [darkgreen,thick]{func1(x) * func2(-x+1)};
                \end{axis}
            \end{tikzpicture}
        \end{tabular} & 
        \begin{tabular}{ c }
            Leading edge: $t+2 > 2 \Rightarrow t > 0$ \\
            Trailing edge: $t-2 < 0 \Rightarrow t < 2$ \\ \\
            $\Longrightarrow 0 < t < 2$  \\ \\
            $\Longrightarrow y(t) = \displaystyle\int_{0}^{2} \left[\frac{1}{2}\tau\right][2] \,d\tau=2$
        \end{tabular}
    \end{tabular} 
    \\ \newpage
    \noindent\underline{Region 4: Partial overlap} \\ \\
    \begin{tabular}{ cc }
        \begin{tabular}{ c }
            \begin{tikzpicture}
                [declare function={
                    func1(\x)= (\x < 0) * (0) + and(\x >= 0, \x < 2) * (0.5 * \x) + (\x > 2) * (0);  }]
                \begin{axis}[
                    axis x line=center, axis y line=center,
                    ymin=0, ymax=2.5, ytick={0,1,2}, ylabel={$x(\tau)$},
                    xmin=-1, xmax=4, xtick={0,...,2}, xlabel={$\tau$},
                    domain=-3:3,samples=200,
                    width=6cm, height=3cm]
                \addplot [blue,thick]{func1(x)};
                \end{axis}
            \end{tikzpicture} \\
            \begin{tikzpicture}
                [declare function={
                    func2(\x)= (\x < -2) * (0) + and(\x >= -2, \x < 2) * (2) + (\x > 2) * (0); }]
                \begin{axis}[
                    axis x line=center, axis y line=center,
                    ymin=0, ymax=3, ytick={0,2}, ylabel={$h(t-\tau)$},
                    xmin=-1, xmax=4, xtick={0}, xlabel={$\tau$},
                    extra x ticks={1}, extra x tick labels={$t-2$}, 
                    domain=-3:3,samples=200,
                    width=6cm, height=3cm]
                \addplot [red,thick]{func2(-x+3)};
                \end{axis}
            \end{tikzpicture} \\
            \begin{tikzpicture}
                [declare function={
                    func1(\x)= (\x < 0) * (0) + and(\x >= 0, \x < 2) * (0.5 * \x) + (\x > 2) * (0);
                    func2(\x)= (\x < -2) * (0) + and(\x >= -2, \x < 2) * (2) + (\x > 2) * (0); }]
                \begin{axis}[
                    axis x line=center, axis y line=center,
                    ymin=0, ymax=3, ytick={0,1,2}, ylabel={$x(\tau)h(t-\tau)$},
                    xmin=-1, xmax=4, xtick={0,2}, xlabel={$\tau$},
                    extra x ticks={1}, extra x tick labels={$t-2$}, 
                    domain=-3:3,samples=200,
                    width=6cm, height=3cm]
                \addplot [darkgreen,thick]{func1(x) * func2(-x+3)};
                \end{axis}
            \end{tikzpicture}
        \end{tabular} & 
        \begin{tabular}{ c }
            Trailing edge: $0 < t-2 < 2 \Rightarrow 2 < t < 4$ \\ \\
            $\Longrightarrow y(t) = \displaystyle\int_{t-2}^{2} \left[\frac{1}{2}\tau\right][2] \,d\tau=2-\frac{1}{2}(t-2)^2$
        \end{tabular}
    \end{tabular}
    \\[2.5cm]
    \underline{Region 5: No overlap} \\ \\
    \begin{tabular}{ cc }
        \begin{tabular}{ c }
            \begin{tikzpicture}
                [declare function={
                    func1(\x)= (\x < 0) * (0) + and(\x >= 0, \x < 2) * (0.5 * \x) + (\x > 2) * (0);  }]
                \begin{axis}[
                    axis x line=center, axis y line=center,
                    ymin=0, ymax=2.5, ytick={0,1,2}, ylabel={$x(\tau)$},
                    xmin=0, xmax=5, xtick={0,...,2}, xlabel={$\tau$},
                    domain=0:4,samples=200,
                    width=6cm, height=3cm]
                \addplot [blue,thick]{func1(x)};
                \end{axis}
            \end{tikzpicture} \\
            \begin{tikzpicture}
                [declare function={
                    func2(\x)= (\x < -2) * (0) + and(\x >= -2, \x < 2) * (2) + (\x > 2) * (0); }]
                \begin{axis}[
                    axis x line=center, axis y line=center,
                    ymin=0, ymax=3, ytick={0,2}, ylabel={$h(t-\tau)$},
                    xmin=0, xmax=5, xtick={0}, xlabel={$\tau$},
                    extra x ticks={3}, extra x tick labels={$t-2$}, 
                    domain=0:4,samples=200,
                    width=6cm, height=3cm]
                \addplot [red,thick]{func2(-x+5)};
                \end{axis}
            \end{tikzpicture}
        \end{tabular} & 
        \begin{tabular}{ c }
            Trailing edge: $t-2 > 2 \Rightarrow t > 4$ \\ \\
            No overlap, so $y(t)=0$ for $t>4$.
        \end{tabular}
    \end{tabular}
    \\ \\ \\
    Aggregating the outputs from the five regions, we can write a piecewise function for the overall output:
    \begin{align*}
        y(t) = 
        \begin{cases} 
            0, & t<-2 \\
            \frac{1}{2}(t+2)^2, & -2<t<0 \\
            2, & 0<t<2 \\
            2-\frac{1}{2}(t-2)^2, & 2<t<4 \\
            0, & t>4
        \end{cases}.
    \end{align*}
    This can also be written with unit step functions:
    \begin{align*}
        y(t) = \frac{1}{2}(t+2)^2[u(t+2)-u(t)]+2[u(t)-u(t-2)]+\left[2-\frac{1}{2}(t-2)^2\right][u(t-2)-u(t-4)].
    \end{align*}
\end{solution}

\end{document}
